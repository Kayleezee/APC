% PREAMBLE
%%%%%%%%%%
%%%%%%%%%%

\documentclass[DIV=12,oneside,a4paper]{scrartcl}

% PACKAGES
%%%%%%%%%%
\usepackage[english]{babel}
\usepackage{graphicx}
\usepackage{pgf}
\usepackage{placeins}
\usepackage{listings}


% DOCUMENT
%%%%%%%%%%
%%%%%%%%%%

\begin{document}

%%%%%%%
% TITLE
%%%%%%%

\title{Exercise Sheet I}
\subject{Advanced Parallel Computing}
\author{Klaus Naumann \& Christoph Klein}
\maketitle

%%%%%%%
% TABLE OF CONTENTS
%%%%%%%

%\newpage
%\tableofcontents
%\newpage

%%%%%%%
% PART 1
%%%%%%%%

\section{Review: The Landscape of Parallel Computing Research}
The paper 'The Landscape of Parallel Computing Research: A View from Berkeley' 
from Asanovic et al. published in December 2006 outlines the need for a naturally
parallel programming model, system software and underlying architecture for both,
embedded and high performance computing. Although the step to parallel microprocessors 
illustrates an milestone in computing history it poses new challenges for research and 
industry due to the fact that conventional wisdoms become outdated. With an increasing
number of processors on a chip in mind the Berkeley researchers defined a number of
"dwarfs" such as n-body methods, structured grids to discuss future hard- and software
requirements. \\ \\
The Berkeley research group assert that manycore architectures and human-centric 
programming models independent of the number of processors are necessary to increase
both, application efficiency and programmer productivity. In view of interconnected 
networks and cache coherence the researche group recommends a hybrid interconnect design
based on switch circuits with support for fine-grained synchronization and communication
constructs. With these requirements it'll be likely to develop efficient applications
for manycore architectures. \\ \\
In our opinion the report from the Berkele research group delivers urgently required 
designs to increase efficiency and productivity in the process of designing "portable"
parallel applications.
       


%%%%%%%
% PART 2
%%%%%%%%
\end{document}
