% PREAMBLE
%%%%%%%%%%
%%%%%%%%%%

\documentclass[DIV=12,oneside,a4paper]{scrartcl}

% PACKAGES
%%%%%%%%%%
\usepackage[english]{babel}
\usepackage{graphicx}
\usepackage{pgf}
\usepackage{placeins}
\usepackage{listings}


% DOCUMENT
%%%%%%%%%%
%%%%%%%%%%

\begin{document}

%%%%%%%
% TITLE
%%%%%%%

\title{Exercise Sheet II}
\subject{Advanced Parallel Computing}
\author{Klaus Naumann \& Christoph Klein}
\maketitle

%%%%%%%
% TABLE OF CONTENTS
%%%%%%%

%\newpage
%\tableofcontents
%\newpage

%%%%%%%
% PART 1
%%%%%%%%

\section{Review: The Future of Microprocessors}
The paper 'The Future of Microprocessors' from Kunle Ulukotun and Lance Hammond published
in September 2005 deals with the expected trend from Von-Neumann architectures to
Chip-Multiprocessor (CMP) design.

 In the first part the authors explain that the
uniprocessor developement reaches it's limits, because economic improvements
in power consumption, pipelining and superscalar issues are no longer realizable.
Therefore the only way to to keep performance increasing is to parallelize
the architectures. This seems to work well for server systems, as they have
to deal with parallel incoming user requests, but software developer still have to
adjust their code to the emerging parallel computer architectures.

The authors outline that a CMP design is beneficial, as inter thread communication
is faster and less power hungry. 

To our mind this paper describes the expected developement qualitatively in o good way.
As CMP designs are actually in our every day life the authors' prediction was correct,
which makes us accepting this paper.

\section{Review: Software and the Concurrency Revolution}
In their 2005 released paper "Software and the Concurrency Revolution" Sutter 
and Larus focus on concurrency and its importance for programmers and 
programming languages alike. Since the step in computer architecture from 
uniprocessors to multicore processors concurrency turned into a key to boost 
application performance on parallel architectures. Though the benefits of 
concurrency in software development are foreseeable it demands an advanced 
knowledge of the underlying hardware architecture and programmers to think in 
an unusual way.

Sutter and Luras outline the importance of separating applications in hundreds 
of tasks to gain performance in applications. The industry has to create new 
parallel-focused constructs as languages and tools to exploit the parallel 
hardware and make parallel applications understandable and transparent. 
Furthermore concurrency can increase stability and functionality of software.

We highly accept the opinion of the papers authors. Concurrency can highly 
increase the performance and stability of applications on a parallel architecture
as the difficulty without tools supporting the programmer in parallel application 
development.
      
%%%%%%%
% PART 2
%%%%%%%%

\section{Experiment: Pointer Chasing Benchmark}
In figure \ref{pointer_chasing_local} and \ref{pointer_chasing_remote}
you can see the pointer chasing benchmark's result on a home pc and on \emph{moore}
respectively. Some lines in figure \ref{pointer_chasing_remote} are truncated, because
the ssh connection was aborted. In the following we will discuss the graphs' behaviour for
the home pc. For low array sizes up to $2^{16}$ Bytes on the home pc the loading times
are very low, which shows us that the problem fits completely in the L1 cache. Thus the
L1 Cache seems to have 65 kB. The line with an array size of $2^{20}$ Bytes raises
up at a stride of $2^8$ Bytes, because at this point we have a saturation in
L1 and L2 caches. This fits well to the processor's data as it has four
L2 caches with 256 kB each ($2^{20}\approx 1048$ kB). When the lines up to
array sizes of $2^20$ Bytes drop down at a stride of $2^15$ Bytes we see
the L1 associativity. The bigger array sizes up to $2^{23}$ Bytes drop
down at a stride of $2^{20}$ Bytes due to L2 cache associativity.
For the three biggest array sizes we see a raise in loading time at a stride of
$2^16$ Bytes, because of increasing saturation of TLB misses. This indicates roughly
a page size of $2^{17}$ Bytes. The discussion of \emph{moore's} pointer chasing benchmark
as analog.

\begin{figure}
	%% Creator: Matplotlib, PGF backend
%%
%% To include the figure in your LaTeX document, write
%%   \input{<filename>.pgf}
%%
%% Make sure the required packages are loaded in your preamble
%%   \usepackage{pgf}
%%
%% Figures using additional raster images can only be included by \input if
%% they are in the same directory as the main LaTeX file. For loading figures
%% from other directories you can use the `import` package
%%   \usepackage{import}
%% and then include the figures with
%%   \import{<path to file>}{<filename>.pgf}
%%
%% Matplotlib used the following preamble
%%   \usepackage[T1]{fontenc}
%%   \usepackage{lmodern}
%%
\begingroup%
\makeatletter%
\begin{pgfpicture}%
\pgfpathrectangle{\pgfpointorigin}{\pgfqpoint{6.000000in}{4.500000in}}%
\pgfusepath{use as bounding box}%
\begin{pgfscope}%
\pgfsetbuttcap%
\pgfsetroundjoin%
\definecolor{currentfill}{rgb}{1.000000,1.000000,1.000000}%
\pgfsetfillcolor{currentfill}%
\pgfsetlinewidth{0.000000pt}%
\definecolor{currentstroke}{rgb}{1.000000,1.000000,1.000000}%
\pgfsetstrokecolor{currentstroke}%
\pgfsetdash{}{0pt}%
\pgfpathmoveto{\pgfqpoint{0.000000in}{0.000000in}}%
\pgfpathlineto{\pgfqpoint{6.000000in}{0.000000in}}%
\pgfpathlineto{\pgfqpoint{6.000000in}{4.500000in}}%
\pgfpathlineto{\pgfqpoint{0.000000in}{4.500000in}}%
\pgfpathclose%
\pgfusepath{fill}%
\end{pgfscope}%
\begin{pgfscope}%
\pgfsetbuttcap%
\pgfsetroundjoin%
\definecolor{currentfill}{rgb}{1.000000,1.000000,1.000000}%
\pgfsetfillcolor{currentfill}%
\pgfsetlinewidth{0.000000pt}%
\definecolor{currentstroke}{rgb}{0.000000,0.000000,0.000000}%
\pgfsetstrokecolor{currentstroke}%
\pgfsetstrokeopacity{0.000000}%
\pgfsetdash{}{0pt}%
\pgfpathmoveto{\pgfqpoint{0.750000in}{0.450000in}}%
\pgfpathlineto{\pgfqpoint{4.800000in}{0.450000in}}%
\pgfpathlineto{\pgfqpoint{4.800000in}{4.050000in}}%
\pgfpathlineto{\pgfqpoint{0.750000in}{4.050000in}}%
\pgfpathclose%
\pgfusepath{fill}%
\end{pgfscope}%
\begin{pgfscope}%
\pgfpathrectangle{\pgfqpoint{0.750000in}{0.450000in}}{\pgfqpoint{4.050000in}{3.600000in}} %
\pgfusepath{clip}%
\pgfsetrectcap%
\pgfsetroundjoin%
\pgfsetlinewidth{2.007500pt}%
\definecolor{currentstroke}{rgb}{0.000000,0.000000,0.000000}%
\pgfsetstrokecolor{currentstroke}%
\pgfsetdash{}{0pt}%
\pgfpathmoveto{\pgfqpoint{0.750000in}{0.903243in}}%
\pgfpathlineto{\pgfqpoint{0.926087in}{0.903314in}}%
\pgfpathlineto{\pgfqpoint{1.102174in}{0.903445in}}%
\pgfpathlineto{\pgfqpoint{1.278261in}{0.904426in}}%
\pgfpathlineto{\pgfqpoint{1.454348in}{0.913024in}}%
\pgfpathlineto{\pgfqpoint{1.630435in}{0.903555in}}%
\pgfpathlineto{\pgfqpoint{1.806522in}{0.889373in}}%
\pgfpathlineto{\pgfqpoint{1.982609in}{0.857152in}}%
\pgfpathlineto{\pgfqpoint{2.158696in}{0.846701in}}%
\pgfpathlineto{\pgfqpoint{2.334783in}{0.846365in}}%
\pgfusepath{stroke}%
\end{pgfscope}%
\begin{pgfscope}%
\pgfpathrectangle{\pgfqpoint{0.750000in}{0.450000in}}{\pgfqpoint{4.050000in}{3.600000in}} %
\pgfusepath{clip}%
\pgfsetrectcap%
\pgfsetroundjoin%
\pgfsetlinewidth{2.007500pt}%
\definecolor{currentstroke}{rgb}{0.066667,0.066667,0.066667}%
\pgfsetstrokecolor{currentstroke}%
\pgfsetdash{}{0pt}%
\pgfpathmoveto{\pgfqpoint{0.750000in}{0.903482in}}%
\pgfpathlineto{\pgfqpoint{0.926087in}{0.903200in}}%
\pgfpathlineto{\pgfqpoint{1.102174in}{0.903402in}}%
\pgfpathlineto{\pgfqpoint{1.278261in}{0.903667in}}%
\pgfpathlineto{\pgfqpoint{1.454348in}{0.903826in}}%
\pgfpathlineto{\pgfqpoint{1.630435in}{0.912473in}}%
\pgfpathlineto{\pgfqpoint{1.806522in}{0.907860in}}%
\pgfpathlineto{\pgfqpoint{1.982609in}{0.883601in}}%
\pgfpathlineto{\pgfqpoint{2.158696in}{0.857055in}}%
\pgfpathlineto{\pgfqpoint{2.334783in}{0.850000in}}%
\pgfpathlineto{\pgfqpoint{2.510870in}{0.846339in}}%
\pgfusepath{stroke}%
\end{pgfscope}%
\begin{pgfscope}%
\pgfpathrectangle{\pgfqpoint{0.750000in}{0.450000in}}{\pgfqpoint{4.050000in}{3.600000in}} %
\pgfusepath{clip}%
\pgfsetrectcap%
\pgfsetroundjoin%
\pgfsetlinewidth{2.007500pt}%
\definecolor{currentstroke}{rgb}{0.133333,0.133333,0.133333}%
\pgfsetstrokecolor{currentstroke}%
\pgfsetdash{}{0pt}%
\pgfpathmoveto{\pgfqpoint{0.750000in}{0.903854in}}%
\pgfpathlineto{\pgfqpoint{0.926087in}{0.903021in}}%
\pgfpathlineto{\pgfqpoint{1.102174in}{0.903077in}}%
\pgfpathlineto{\pgfqpoint{1.278261in}{0.903276in}}%
\pgfpathlineto{\pgfqpoint{1.454348in}{0.903676in}}%
\pgfpathlineto{\pgfqpoint{1.630435in}{0.904183in}}%
\pgfpathlineto{\pgfqpoint{1.806522in}{0.906048in}}%
\pgfpathlineto{\pgfqpoint{1.982609in}{0.907477in}}%
\pgfpathlineto{\pgfqpoint{2.158696in}{0.883611in}}%
\pgfpathlineto{\pgfqpoint{2.334783in}{0.857124in}}%
\pgfpathlineto{\pgfqpoint{2.510870in}{0.846710in}}%
\pgfpathlineto{\pgfqpoint{2.686957in}{0.846336in}}%
\pgfusepath{stroke}%
\end{pgfscope}%
\begin{pgfscope}%
\pgfpathrectangle{\pgfqpoint{0.750000in}{0.450000in}}{\pgfqpoint{4.050000in}{3.600000in}} %
\pgfusepath{clip}%
\pgfsetrectcap%
\pgfsetroundjoin%
\pgfsetlinewidth{2.007500pt}%
\definecolor{currentstroke}{rgb}{0.200000,0.200000,0.200000}%
\pgfsetstrokecolor{currentstroke}%
\pgfsetdash{}{0pt}%
\pgfpathmoveto{\pgfqpoint{0.750000in}{0.903546in}}%
\pgfpathlineto{\pgfqpoint{0.926087in}{0.909778in}}%
\pgfpathlineto{\pgfqpoint{1.102174in}{0.902998in}}%
\pgfpathlineto{\pgfqpoint{1.278261in}{0.903254in}}%
\pgfpathlineto{\pgfqpoint{1.454348in}{0.903931in}}%
\pgfpathlineto{\pgfqpoint{1.630435in}{0.903760in}}%
\pgfpathlineto{\pgfqpoint{1.806522in}{0.904556in}}%
\pgfpathlineto{\pgfqpoint{1.982609in}{0.905670in}}%
\pgfpathlineto{\pgfqpoint{2.158696in}{0.907361in}}%
\pgfpathlineto{\pgfqpoint{2.334783in}{0.883138in}}%
\pgfpathlineto{\pgfqpoint{2.510870in}{0.857165in}}%
\pgfpathlineto{\pgfqpoint{2.686957in}{0.846727in}}%
\pgfpathlineto{\pgfqpoint{2.863043in}{0.846389in}}%
\pgfusepath{stroke}%
\end{pgfscope}%
\begin{pgfscope}%
\pgfpathrectangle{\pgfqpoint{0.750000in}{0.450000in}}{\pgfqpoint{4.050000in}{3.600000in}} %
\pgfusepath{clip}%
\pgfsetrectcap%
\pgfsetroundjoin%
\pgfsetlinewidth{2.007500pt}%
\definecolor{currentstroke}{rgb}{0.266667,0.266667,0.266667}%
\pgfsetstrokecolor{currentstroke}%
\pgfsetdash{}{0pt}%
\pgfpathmoveto{\pgfqpoint{0.750000in}{0.902958in}}%
\pgfpathlineto{\pgfqpoint{0.926087in}{0.909603in}}%
\pgfpathlineto{\pgfqpoint{1.102174in}{0.903052in}}%
\pgfpathlineto{\pgfqpoint{1.278261in}{0.903231in}}%
\pgfpathlineto{\pgfqpoint{1.454348in}{0.931900in}}%
\pgfpathlineto{\pgfqpoint{1.630435in}{0.946991in}}%
\pgfpathlineto{\pgfqpoint{1.806522in}{0.932166in}}%
\pgfpathlineto{\pgfqpoint{1.982609in}{0.932882in}}%
\pgfpathlineto{\pgfqpoint{2.158696in}{0.944591in}}%
\pgfpathlineto{\pgfqpoint{2.334783in}{0.958645in}}%
\pgfpathlineto{\pgfqpoint{2.510870in}{0.934435in}}%
\pgfpathlineto{\pgfqpoint{2.686957in}{0.857113in}}%
\pgfpathlineto{\pgfqpoint{2.863043in}{0.846729in}}%
\pgfpathlineto{\pgfqpoint{3.039130in}{0.846390in}}%
\pgfusepath{stroke}%
\end{pgfscope}%
\begin{pgfscope}%
\pgfpathrectangle{\pgfqpoint{0.750000in}{0.450000in}}{\pgfqpoint{4.050000in}{3.600000in}} %
\pgfusepath{clip}%
\pgfsetrectcap%
\pgfsetroundjoin%
\pgfsetlinewidth{2.007500pt}%
\definecolor{currentstroke}{rgb}{0.333333,0.333333,0.333333}%
\pgfsetstrokecolor{currentstroke}%
\pgfsetdash{}{0pt}%
\pgfpathmoveto{\pgfqpoint{0.750000in}{0.902952in}}%
\pgfpathlineto{\pgfqpoint{0.926087in}{0.902975in}}%
\pgfpathlineto{\pgfqpoint{1.102174in}{0.903047in}}%
\pgfpathlineto{\pgfqpoint{1.278261in}{0.903184in}}%
\pgfpathlineto{\pgfqpoint{1.454348in}{0.931844in}}%
\pgfpathlineto{\pgfqpoint{1.630435in}{0.952020in}}%
\pgfpathlineto{\pgfqpoint{1.806522in}{0.932350in}}%
\pgfpathlineto{\pgfqpoint{1.982609in}{0.933246in}}%
\pgfpathlineto{\pgfqpoint{2.158696in}{0.934048in}}%
\pgfpathlineto{\pgfqpoint{2.334783in}{0.966859in}}%
\pgfpathlineto{\pgfqpoint{2.510870in}{0.958574in}}%
\pgfpathlineto{\pgfqpoint{2.686957in}{0.934384in}}%
\pgfpathlineto{\pgfqpoint{2.863043in}{0.857042in}}%
\pgfpathlineto{\pgfqpoint{3.039130in}{0.846730in}}%
\pgfpathlineto{\pgfqpoint{3.215217in}{0.846391in}}%
\pgfusepath{stroke}%
\end{pgfscope}%
\begin{pgfscope}%
\pgfpathrectangle{\pgfqpoint{0.750000in}{0.450000in}}{\pgfqpoint{4.050000in}{3.600000in}} %
\pgfusepath{clip}%
\pgfsetrectcap%
\pgfsetroundjoin%
\pgfsetlinewidth{2.007500pt}%
\definecolor{currentstroke}{rgb}{0.400000,0.400000,0.400000}%
\pgfsetstrokecolor{currentstroke}%
\pgfsetdash{}{0pt}%
\pgfpathmoveto{\pgfqpoint{0.750000in}{0.903064in}}%
\pgfpathlineto{\pgfqpoint{0.926087in}{0.903198in}}%
\pgfpathlineto{\pgfqpoint{1.102174in}{0.903479in}}%
\pgfpathlineto{\pgfqpoint{1.278261in}{0.904134in}}%
\pgfpathlineto{\pgfqpoint{1.454348in}{0.945578in}}%
\pgfpathlineto{\pgfqpoint{1.630435in}{0.995604in}}%
\pgfpathlineto{\pgfqpoint{1.806522in}{1.081744in}}%
\pgfpathlineto{\pgfqpoint{1.982609in}{1.073338in}}%
\pgfpathlineto{\pgfqpoint{2.158696in}{1.079059in}}%
\pgfpathlineto{\pgfqpoint{2.334783in}{1.231891in}}%
\pgfpathlineto{\pgfqpoint{2.510870in}{1.203369in}}%
\pgfpathlineto{\pgfqpoint{2.686957in}{0.958535in}}%
\pgfpathlineto{\pgfqpoint{2.863043in}{0.934385in}}%
\pgfpathlineto{\pgfqpoint{3.039130in}{0.857085in}}%
\pgfpathlineto{\pgfqpoint{3.215217in}{0.846730in}}%
\pgfpathlineto{\pgfqpoint{3.391304in}{0.846392in}}%
\pgfusepath{stroke}%
\end{pgfscope}%
\begin{pgfscope}%
\pgfpathrectangle{\pgfqpoint{0.750000in}{0.450000in}}{\pgfqpoint{4.050000in}{3.600000in}} %
\pgfusepath{clip}%
\pgfsetrectcap%
\pgfsetroundjoin%
\pgfsetlinewidth{2.007500pt}%
\definecolor{currentstroke}{rgb}{0.466667,0.466667,0.466667}%
\pgfsetstrokecolor{currentstroke}%
\pgfsetdash{}{0pt}%
\pgfpathmoveto{\pgfqpoint{0.750000in}{0.903172in}}%
\pgfpathlineto{\pgfqpoint{0.926087in}{0.910088in}}%
\pgfpathlineto{\pgfqpoint{1.102174in}{0.903905in}}%
\pgfpathlineto{\pgfqpoint{1.278261in}{0.905114in}}%
\pgfpathlineto{\pgfqpoint{1.454348in}{0.951097in}}%
\pgfpathlineto{\pgfqpoint{1.630435in}{1.044238in}}%
\pgfpathlineto{\pgfqpoint{1.806522in}{1.238352in}}%
\pgfpathlineto{\pgfqpoint{1.982609in}{1.251360in}}%
\pgfpathlineto{\pgfqpoint{2.158696in}{1.329480in}}%
\pgfpathlineto{\pgfqpoint{2.334783in}{1.481980in}}%
\pgfpathlineto{\pgfqpoint{2.510870in}{1.513264in}}%
\pgfpathlineto{\pgfqpoint{2.686957in}{1.331520in}}%
\pgfpathlineto{\pgfqpoint{2.863043in}{1.023863in}}%
\pgfpathlineto{\pgfqpoint{3.039130in}{0.980373in}}%
\pgfpathlineto{\pgfqpoint{3.215217in}{0.892405in}}%
\pgfpathlineto{\pgfqpoint{3.391304in}{0.846730in}}%
\pgfpathlineto{\pgfqpoint{3.567391in}{0.846391in}}%
\pgfusepath{stroke}%
\end{pgfscope}%
\begin{pgfscope}%
\pgfpathrectangle{\pgfqpoint{0.750000in}{0.450000in}}{\pgfqpoint{4.050000in}{3.600000in}} %
\pgfusepath{clip}%
\pgfsetrectcap%
\pgfsetroundjoin%
\pgfsetlinewidth{2.007500pt}%
\definecolor{currentstroke}{rgb}{0.533333,0.533333,0.533333}%
\pgfsetstrokecolor{currentstroke}%
\pgfsetdash{}{0pt}%
\pgfpathmoveto{\pgfqpoint{0.750000in}{0.903176in}}%
\pgfpathlineto{\pgfqpoint{0.926087in}{0.910087in}}%
\pgfpathlineto{\pgfqpoint{1.102174in}{0.903894in}}%
\pgfpathlineto{\pgfqpoint{1.278261in}{0.905096in}}%
\pgfpathlineto{\pgfqpoint{1.454348in}{0.951046in}}%
\pgfpathlineto{\pgfqpoint{1.630435in}{1.043762in}}%
\pgfpathlineto{\pgfqpoint{1.806522in}{1.238439in}}%
\pgfpathlineto{\pgfqpoint{1.982609in}{1.251236in}}%
\pgfpathlineto{\pgfqpoint{2.158696in}{1.357404in}}%
\pgfpathlineto{\pgfqpoint{2.334783in}{1.481544in}}%
\pgfpathlineto{\pgfqpoint{2.510870in}{1.512148in}}%
\pgfpathlineto{\pgfqpoint{2.686957in}{1.490976in}}%
\pgfpathlineto{\pgfqpoint{2.863043in}{1.344524in}}%
\pgfpathlineto{\pgfqpoint{3.039130in}{1.023817in}}%
\pgfpathlineto{\pgfqpoint{3.215217in}{0.980353in}}%
\pgfpathlineto{\pgfqpoint{3.391304in}{0.892343in}}%
\pgfpathlineto{\pgfqpoint{3.567391in}{0.846730in}}%
\pgfpathlineto{\pgfqpoint{3.743478in}{0.846392in}}%
\pgfusepath{stroke}%
\end{pgfscope}%
\begin{pgfscope}%
\pgfpathrectangle{\pgfqpoint{0.750000in}{0.450000in}}{\pgfqpoint{4.050000in}{3.600000in}} %
\pgfusepath{clip}%
\pgfsetrectcap%
\pgfsetroundjoin%
\pgfsetlinewidth{2.007500pt}%
\definecolor{currentstroke}{rgb}{0.600000,0.600000,0.600000}%
\pgfsetstrokecolor{currentstroke}%
\pgfsetdash{}{0pt}%
\pgfpathmoveto{\pgfqpoint{0.750000in}{0.903180in}}%
\pgfpathlineto{\pgfqpoint{0.926087in}{0.903432in}}%
\pgfpathlineto{\pgfqpoint{1.102174in}{0.903899in}}%
\pgfpathlineto{\pgfqpoint{1.278261in}{0.905106in}}%
\pgfpathlineto{\pgfqpoint{1.454348in}{0.951040in}}%
\pgfpathlineto{\pgfqpoint{1.630435in}{1.043359in}}%
\pgfpathlineto{\pgfqpoint{1.806522in}{1.238439in}}%
\pgfpathlineto{\pgfqpoint{1.982609in}{1.251052in}}%
\pgfpathlineto{\pgfqpoint{2.158696in}{1.356748in}}%
\pgfpathlineto{\pgfqpoint{2.334783in}{1.481108in}}%
\pgfpathlineto{\pgfqpoint{2.510870in}{1.511508in}}%
\pgfpathlineto{\pgfqpoint{2.686957in}{1.511828in}}%
\pgfpathlineto{\pgfqpoint{2.863043in}{1.513424in}}%
\pgfpathlineto{\pgfqpoint{3.039130in}{1.244584in}}%
\pgfpathlineto{\pgfqpoint{3.215217in}{1.064535in}}%
\pgfpathlineto{\pgfqpoint{3.391304in}{0.980369in}}%
\pgfpathlineto{\pgfqpoint{3.567391in}{0.892280in}}%
\pgfpathlineto{\pgfqpoint{3.743478in}{0.846729in}}%
\pgfpathlineto{\pgfqpoint{3.919565in}{0.846393in}}%
\pgfusepath{stroke}%
\end{pgfscope}%
\begin{pgfscope}%
\pgfpathrectangle{\pgfqpoint{0.750000in}{0.450000in}}{\pgfqpoint{4.050000in}{3.600000in}} %
\pgfusepath{clip}%
\pgfsetrectcap%
\pgfsetroundjoin%
\pgfsetlinewidth{2.007500pt}%
\definecolor{currentstroke}{rgb}{0.666667,0.666667,0.666667}%
\pgfsetstrokecolor{currentstroke}%
\pgfsetdash{}{0pt}%
\pgfpathmoveto{\pgfqpoint{0.750000in}{0.903157in}}%
\pgfpathlineto{\pgfqpoint{0.926087in}{0.903474in}}%
\pgfpathlineto{\pgfqpoint{1.102174in}{0.911826in}}%
\pgfpathlineto{\pgfqpoint{1.278261in}{0.905166in}}%
\pgfpathlineto{\pgfqpoint{1.454348in}{0.951283in}}%
\pgfpathlineto{\pgfqpoint{1.630435in}{1.061458in}}%
\pgfpathlineto{\pgfqpoint{1.806522in}{1.224206in}}%
\pgfpathlineto{\pgfqpoint{1.982609in}{1.222059in}}%
\pgfpathlineto{\pgfqpoint{2.158696in}{1.266236in}}%
\pgfpathlineto{\pgfqpoint{2.334783in}{1.461984in}}%
\pgfpathlineto{\pgfqpoint{2.510870in}{1.478220in}}%
\pgfpathlineto{\pgfqpoint{2.686957in}{1.478796in}}%
\pgfpathlineto{\pgfqpoint{2.863043in}{1.479952in}}%
\pgfpathlineto{\pgfqpoint{3.039130in}{1.383108in}}%
\pgfpathlineto{\pgfqpoint{3.215217in}{1.305636in}}%
\pgfpathlineto{\pgfqpoint{3.391304in}{1.179689in}}%
\pgfpathlineto{\pgfqpoint{3.567391in}{1.017214in}}%
\pgfpathlineto{\pgfqpoint{3.743478in}{0.857045in}}%
\pgfpathlineto{\pgfqpoint{3.919565in}{0.846730in}}%
\pgfpathlineto{\pgfqpoint{4.095652in}{0.846363in}}%
\pgfusepath{stroke}%
\end{pgfscope}%
\begin{pgfscope}%
\pgfpathrectangle{\pgfqpoint{0.750000in}{0.450000in}}{\pgfqpoint{4.050000in}{3.600000in}} %
\pgfusepath{clip}%
\pgfsetrectcap%
\pgfsetroundjoin%
\pgfsetlinewidth{2.007500pt}%
\definecolor{currentstroke}{rgb}{0.733333,0.733333,0.733333}%
\pgfsetstrokecolor{currentstroke}%
\pgfsetdash{}{0pt}%
\pgfpathmoveto{\pgfqpoint{0.750000in}{0.910615in}}%
\pgfpathlineto{\pgfqpoint{0.926087in}{0.911395in}}%
\pgfpathlineto{\pgfqpoint{1.102174in}{0.913064in}}%
\pgfpathlineto{\pgfqpoint{1.278261in}{0.916355in}}%
\pgfpathlineto{\pgfqpoint{1.454348in}{0.974596in}}%
\pgfpathlineto{\pgfqpoint{1.630435in}{1.268388in}}%
\pgfpathlineto{\pgfqpoint{1.806522in}{1.489336in}}%
\pgfpathlineto{\pgfqpoint{1.982609in}{1.496560in}}%
\pgfpathlineto{\pgfqpoint{2.158696in}{1.565256in}}%
\pgfpathlineto{\pgfqpoint{2.334783in}{1.948172in}}%
\pgfpathlineto{\pgfqpoint{2.510870in}{2.022804in}}%
\pgfpathlineto{\pgfqpoint{2.686957in}{1.572784in}}%
\pgfpathlineto{\pgfqpoint{2.863043in}{1.461984in}}%
\pgfpathlineto{\pgfqpoint{3.039130in}{1.412824in}}%
\pgfpathlineto{\pgfqpoint{3.215217in}{1.401964in}}%
\pgfpathlineto{\pgfqpoint{3.391304in}{1.320952in}}%
\pgfpathlineto{\pgfqpoint{3.567391in}{1.268688in}}%
\pgfpathlineto{\pgfqpoint{3.743478in}{1.099530in}}%
\pgfpathlineto{\pgfqpoint{3.919565in}{0.857083in}}%
\pgfpathlineto{\pgfqpoint{4.095652in}{0.846730in}}%
\pgfpathlineto{\pgfqpoint{4.271739in}{0.846386in}}%
\pgfusepath{stroke}%
\end{pgfscope}%
\begin{pgfscope}%
\pgfpathrectangle{\pgfqpoint{0.750000in}{0.450000in}}{\pgfqpoint{4.050000in}{3.600000in}} %
\pgfusepath{clip}%
\pgfsetrectcap%
\pgfsetroundjoin%
\pgfsetlinewidth{2.007500pt}%
\definecolor{currentstroke}{rgb}{0.800000,0.800000,0.800000}%
\pgfsetstrokecolor{currentstroke}%
\pgfsetdash{}{0pt}%
\pgfpathmoveto{\pgfqpoint{0.750000in}{0.910787in}}%
\pgfpathlineto{\pgfqpoint{0.926087in}{0.912241in}}%
\pgfpathlineto{\pgfqpoint{1.102174in}{0.922303in}}%
\pgfpathlineto{\pgfqpoint{1.278261in}{0.940201in}}%
\pgfpathlineto{\pgfqpoint{1.454348in}{1.027782in}}%
\pgfpathlineto{\pgfqpoint{1.630435in}{1.688880in}}%
\pgfpathlineto{\pgfqpoint{1.806522in}{2.034300in}}%
\pgfpathlineto{\pgfqpoint{1.982609in}{2.066096in}}%
\pgfpathlineto{\pgfqpoint{2.158696in}{2.202224in}}%
\pgfpathlineto{\pgfqpoint{2.334783in}{3.120652in}}%
\pgfpathlineto{\pgfqpoint{2.510870in}{3.295320in}}%
\pgfpathlineto{\pgfqpoint{2.686957in}{3.393132in}}%
\pgfpathlineto{\pgfqpoint{2.863043in}{3.042356in}}%
\pgfpathlineto{\pgfqpoint{3.039130in}{3.093940in}}%
\pgfpathlineto{\pgfqpoint{3.215217in}{3.536408in}}%
\pgfpathlineto{\pgfqpoint{3.391304in}{3.654924in}}%
\pgfpathlineto{\pgfqpoint{3.567391in}{3.359668in}}%
\pgfpathlineto{\pgfqpoint{3.743478in}{3.017440in}}%
\pgfpathlineto{\pgfqpoint{3.919565in}{1.171676in}}%
\pgfpathlineto{\pgfqpoint{4.095652in}{0.857051in}}%
\pgfpathlineto{\pgfqpoint{4.271739in}{0.846726in}}%
\pgfpathlineto{\pgfqpoint{4.447826in}{0.846389in}}%
\pgfusepath{stroke}%
\end{pgfscope}%
\begin{pgfscope}%
\pgfpathrectangle{\pgfqpoint{0.750000in}{0.450000in}}{\pgfqpoint{4.050000in}{3.600000in}} %
\pgfusepath{clip}%
\pgfsetrectcap%
\pgfsetroundjoin%
\pgfsetlinewidth{2.007500pt}%
\definecolor{currentstroke}{rgb}{0.866667,0.866667,0.866667}%
\pgfsetstrokecolor{currentstroke}%
\pgfsetdash{}{0pt}%
\pgfpathmoveto{\pgfqpoint{0.750000in}{0.910873in}}%
\pgfpathlineto{\pgfqpoint{0.926087in}{0.919107in}}%
\pgfpathlineto{\pgfqpoint{1.102174in}{0.922911in}}%
\pgfpathlineto{\pgfqpoint{1.278261in}{0.941221in}}%
\pgfpathlineto{\pgfqpoint{1.454348in}{1.032397in}}%
\pgfpathlineto{\pgfqpoint{1.630435in}{1.644728in}}%
\pgfpathlineto{\pgfqpoint{1.806522in}{2.042092in}}%
\pgfpathlineto{\pgfqpoint{1.982609in}{2.057988in}}%
\pgfpathlineto{\pgfqpoint{2.158696in}{2.182340in}}%
\pgfpathlineto{\pgfqpoint{2.334783in}{3.007120in}}%
\pgfpathlineto{\pgfqpoint{2.510870in}{3.173052in}}%
\pgfpathlineto{\pgfqpoint{2.686957in}{3.366640in}}%
\pgfpathlineto{\pgfqpoint{2.863043in}{3.176036in}}%
\pgfpathlineto{\pgfqpoint{3.039130in}{3.295320in}}%
\pgfpathlineto{\pgfqpoint{3.215217in}{3.681220in}}%
\pgfpathlineto{\pgfqpoint{3.391304in}{3.681220in}}%
\pgfpathlineto{\pgfqpoint{3.567391in}{3.926368in}}%
\pgfpathlineto{\pgfqpoint{3.743478in}{3.574784in}}%
\pgfpathlineto{\pgfqpoint{3.919565in}{3.120652in}}%
\pgfpathlineto{\pgfqpoint{4.095652in}{1.190598in}}%
\pgfpathlineto{\pgfqpoint{4.271739in}{0.857101in}}%
\pgfpathlineto{\pgfqpoint{4.447826in}{0.846730in}}%
\pgfpathlineto{\pgfqpoint{4.623913in}{0.846393in}}%
\pgfusepath{stroke}%
\end{pgfscope}%
\begin{pgfscope}%
\pgfpathrectangle{\pgfqpoint{0.750000in}{0.450000in}}{\pgfqpoint{4.050000in}{3.600000in}} %
\pgfusepath{clip}%
\pgfsetrectcap%
\pgfsetroundjoin%
\pgfsetlinewidth{2.007500pt}%
\definecolor{currentstroke}{rgb}{0.933333,0.933333,0.933333}%
\pgfsetstrokecolor{currentstroke}%
\pgfsetdash{}{0pt}%
\pgfpathmoveto{\pgfqpoint{0.750000in}{0.910959in}}%
\pgfpathlineto{\pgfqpoint{0.926087in}{0.919107in}}%
\pgfpathlineto{\pgfqpoint{1.102174in}{0.922468in}}%
\pgfpathlineto{\pgfqpoint{1.278261in}{0.940435in}}%
\pgfpathlineto{\pgfqpoint{1.454348in}{1.029455in}}%
\pgfpathlineto{\pgfqpoint{1.630435in}{1.630280in}}%
\pgfpathlineto{\pgfqpoint{1.806522in}{2.011528in}}%
\pgfpathlineto{\pgfqpoint{1.982609in}{2.042092in}}%
\pgfpathlineto{\pgfqpoint{2.158696in}{2.144272in}}%
\pgfpathlineto{\pgfqpoint{2.334783in}{3.120652in}}%
\pgfpathlineto{\pgfqpoint{2.510870in}{3.114976in}}%
\pgfpathlineto{\pgfqpoint{2.686957in}{3.366640in}}%
\pgfpathlineto{\pgfqpoint{2.863043in}{3.007120in}}%
\pgfpathlineto{\pgfqpoint{3.039130in}{3.114976in}}%
\pgfpathlineto{\pgfqpoint{3.215217in}{3.681220in}}%
\pgfpathlineto{\pgfqpoint{3.391304in}{3.681220in}}%
\pgfpathlineto{\pgfqpoint{3.567391in}{3.681220in}}%
\pgfpathlineto{\pgfqpoint{3.743478in}{3.681220in}}%
\pgfpathlineto{\pgfqpoint{3.919565in}{3.499096in}}%
\pgfpathlineto{\pgfqpoint{4.095652in}{3.499096in}}%
\pgfpathlineto{\pgfqpoint{4.271739in}{1.184968in}}%
\pgfpathlineto{\pgfqpoint{4.447826in}{0.857043in}}%
\pgfpathlineto{\pgfqpoint{4.623913in}{0.846725in}}%
\pgfpathlineto{\pgfqpoint{4.800000in}{0.846393in}}%
\pgfusepath{stroke}%
\end{pgfscope}%
\begin{pgfscope}%
\pgfpathrectangle{\pgfqpoint{0.750000in}{0.450000in}}{\pgfqpoint{4.050000in}{3.600000in}} %
\pgfusepath{clip}%
\pgfsetbuttcap%
\pgfsetroundjoin%
\pgfsetlinewidth{0.501875pt}%
\definecolor{currentstroke}{rgb}{0.000000,0.000000,0.000000}%
\pgfsetstrokecolor{currentstroke}%
\pgfsetdash{{1.000000pt}{3.000000pt}}{0.000000pt}%
\pgfpathmoveto{\pgfqpoint{0.750000in}{0.450000in}}%
\pgfpathlineto{\pgfqpoint{0.750000in}{4.050000in}}%
\pgfusepath{stroke}%
\end{pgfscope}%
\begin{pgfscope}%
\pgfsetbuttcap%
\pgfsetroundjoin%
\definecolor{currentfill}{rgb}{0.000000,0.000000,0.000000}%
\pgfsetfillcolor{currentfill}%
\pgfsetlinewidth{0.501875pt}%
\definecolor{currentstroke}{rgb}{0.000000,0.000000,0.000000}%
\pgfsetstrokecolor{currentstroke}%
\pgfsetdash{}{0pt}%
\pgfsys@defobject{currentmarker}{\pgfqpoint{0.000000in}{0.000000in}}{\pgfqpoint{0.000000in}{0.055556in}}{%
\pgfpathmoveto{\pgfqpoint{0.000000in}{0.000000in}}%
\pgfpathlineto{\pgfqpoint{0.000000in}{0.055556in}}%
\pgfusepath{stroke,fill}%
}%
\begin{pgfscope}%
\pgfsys@transformshift{0.750000in}{0.450000in}%
\pgfsys@useobject{currentmarker}{}%
\end{pgfscope}%
\end{pgfscope}%
\begin{pgfscope}%
\pgfsetbuttcap%
\pgfsetroundjoin%
\definecolor{currentfill}{rgb}{0.000000,0.000000,0.000000}%
\pgfsetfillcolor{currentfill}%
\pgfsetlinewidth{0.501875pt}%
\definecolor{currentstroke}{rgb}{0.000000,0.000000,0.000000}%
\pgfsetstrokecolor{currentstroke}%
\pgfsetdash{}{0pt}%
\pgfsys@defobject{currentmarker}{\pgfqpoint{0.000000in}{-0.055556in}}{\pgfqpoint{0.000000in}{0.000000in}}{%
\pgfpathmoveto{\pgfqpoint{0.000000in}{0.000000in}}%
\pgfpathlineto{\pgfqpoint{0.000000in}{-0.055556in}}%
\pgfusepath{stroke,fill}%
}%
\begin{pgfscope}%
\pgfsys@transformshift{0.750000in}{4.050000in}%
\pgfsys@useobject{currentmarker}{}%
\end{pgfscope}%
\end{pgfscope}%
\begin{pgfscope}%
\pgftext[x=0.750000in,y=0.394444in,,top]{{\rmfamily\fontsize{11.000000}{13.200000}\selectfont \(\displaystyle 2^{2}\)}}%
\end{pgfscope}%
\begin{pgfscope}%
\pgfpathrectangle{\pgfqpoint{0.750000in}{0.450000in}}{\pgfqpoint{4.050000in}{3.600000in}} %
\pgfusepath{clip}%
\pgfsetbuttcap%
\pgfsetroundjoin%
\pgfsetlinewidth{0.501875pt}%
\definecolor{currentstroke}{rgb}{0.000000,0.000000,0.000000}%
\pgfsetstrokecolor{currentstroke}%
\pgfsetdash{{1.000000pt}{3.000000pt}}{0.000000pt}%
\pgfpathmoveto{\pgfqpoint{1.102174in}{0.450000in}}%
\pgfpathlineto{\pgfqpoint{1.102174in}{4.050000in}}%
\pgfusepath{stroke}%
\end{pgfscope}%
\begin{pgfscope}%
\pgfsetbuttcap%
\pgfsetroundjoin%
\definecolor{currentfill}{rgb}{0.000000,0.000000,0.000000}%
\pgfsetfillcolor{currentfill}%
\pgfsetlinewidth{0.501875pt}%
\definecolor{currentstroke}{rgb}{0.000000,0.000000,0.000000}%
\pgfsetstrokecolor{currentstroke}%
\pgfsetdash{}{0pt}%
\pgfsys@defobject{currentmarker}{\pgfqpoint{0.000000in}{0.000000in}}{\pgfqpoint{0.000000in}{0.055556in}}{%
\pgfpathmoveto{\pgfqpoint{0.000000in}{0.000000in}}%
\pgfpathlineto{\pgfqpoint{0.000000in}{0.055556in}}%
\pgfusepath{stroke,fill}%
}%
\begin{pgfscope}%
\pgfsys@transformshift{1.102174in}{0.450000in}%
\pgfsys@useobject{currentmarker}{}%
\end{pgfscope}%
\end{pgfscope}%
\begin{pgfscope}%
\pgfsetbuttcap%
\pgfsetroundjoin%
\definecolor{currentfill}{rgb}{0.000000,0.000000,0.000000}%
\pgfsetfillcolor{currentfill}%
\pgfsetlinewidth{0.501875pt}%
\definecolor{currentstroke}{rgb}{0.000000,0.000000,0.000000}%
\pgfsetstrokecolor{currentstroke}%
\pgfsetdash{}{0pt}%
\pgfsys@defobject{currentmarker}{\pgfqpoint{0.000000in}{-0.055556in}}{\pgfqpoint{0.000000in}{0.000000in}}{%
\pgfpathmoveto{\pgfqpoint{0.000000in}{0.000000in}}%
\pgfpathlineto{\pgfqpoint{0.000000in}{-0.055556in}}%
\pgfusepath{stroke,fill}%
}%
\begin{pgfscope}%
\pgfsys@transformshift{1.102174in}{4.050000in}%
\pgfsys@useobject{currentmarker}{}%
\end{pgfscope}%
\end{pgfscope}%
\begin{pgfscope}%
\pgftext[x=1.102174in,y=0.394444in,,top]{{\rmfamily\fontsize{11.000000}{13.200000}\selectfont \(\displaystyle 2^{4}\)}}%
\end{pgfscope}%
\begin{pgfscope}%
\pgfpathrectangle{\pgfqpoint{0.750000in}{0.450000in}}{\pgfqpoint{4.050000in}{3.600000in}} %
\pgfusepath{clip}%
\pgfsetbuttcap%
\pgfsetroundjoin%
\pgfsetlinewidth{0.501875pt}%
\definecolor{currentstroke}{rgb}{0.000000,0.000000,0.000000}%
\pgfsetstrokecolor{currentstroke}%
\pgfsetdash{{1.000000pt}{3.000000pt}}{0.000000pt}%
\pgfpathmoveto{\pgfqpoint{1.454348in}{0.450000in}}%
\pgfpathlineto{\pgfqpoint{1.454348in}{4.050000in}}%
\pgfusepath{stroke}%
\end{pgfscope}%
\begin{pgfscope}%
\pgfsetbuttcap%
\pgfsetroundjoin%
\definecolor{currentfill}{rgb}{0.000000,0.000000,0.000000}%
\pgfsetfillcolor{currentfill}%
\pgfsetlinewidth{0.501875pt}%
\definecolor{currentstroke}{rgb}{0.000000,0.000000,0.000000}%
\pgfsetstrokecolor{currentstroke}%
\pgfsetdash{}{0pt}%
\pgfsys@defobject{currentmarker}{\pgfqpoint{0.000000in}{0.000000in}}{\pgfqpoint{0.000000in}{0.055556in}}{%
\pgfpathmoveto{\pgfqpoint{0.000000in}{0.000000in}}%
\pgfpathlineto{\pgfqpoint{0.000000in}{0.055556in}}%
\pgfusepath{stroke,fill}%
}%
\begin{pgfscope}%
\pgfsys@transformshift{1.454348in}{0.450000in}%
\pgfsys@useobject{currentmarker}{}%
\end{pgfscope}%
\end{pgfscope}%
\begin{pgfscope}%
\pgfsetbuttcap%
\pgfsetroundjoin%
\definecolor{currentfill}{rgb}{0.000000,0.000000,0.000000}%
\pgfsetfillcolor{currentfill}%
\pgfsetlinewidth{0.501875pt}%
\definecolor{currentstroke}{rgb}{0.000000,0.000000,0.000000}%
\pgfsetstrokecolor{currentstroke}%
\pgfsetdash{}{0pt}%
\pgfsys@defobject{currentmarker}{\pgfqpoint{0.000000in}{-0.055556in}}{\pgfqpoint{0.000000in}{0.000000in}}{%
\pgfpathmoveto{\pgfqpoint{0.000000in}{0.000000in}}%
\pgfpathlineto{\pgfqpoint{0.000000in}{-0.055556in}}%
\pgfusepath{stroke,fill}%
}%
\begin{pgfscope}%
\pgfsys@transformshift{1.454348in}{4.050000in}%
\pgfsys@useobject{currentmarker}{}%
\end{pgfscope}%
\end{pgfscope}%
\begin{pgfscope}%
\pgftext[x=1.454348in,y=0.394444in,,top]{{\rmfamily\fontsize{11.000000}{13.200000}\selectfont \(\displaystyle 2^{6}\)}}%
\end{pgfscope}%
\begin{pgfscope}%
\pgfpathrectangle{\pgfqpoint{0.750000in}{0.450000in}}{\pgfqpoint{4.050000in}{3.600000in}} %
\pgfusepath{clip}%
\pgfsetbuttcap%
\pgfsetroundjoin%
\pgfsetlinewidth{0.501875pt}%
\definecolor{currentstroke}{rgb}{0.000000,0.000000,0.000000}%
\pgfsetstrokecolor{currentstroke}%
\pgfsetdash{{1.000000pt}{3.000000pt}}{0.000000pt}%
\pgfpathmoveto{\pgfqpoint{1.806522in}{0.450000in}}%
\pgfpathlineto{\pgfqpoint{1.806522in}{4.050000in}}%
\pgfusepath{stroke}%
\end{pgfscope}%
\begin{pgfscope}%
\pgfsetbuttcap%
\pgfsetroundjoin%
\definecolor{currentfill}{rgb}{0.000000,0.000000,0.000000}%
\pgfsetfillcolor{currentfill}%
\pgfsetlinewidth{0.501875pt}%
\definecolor{currentstroke}{rgb}{0.000000,0.000000,0.000000}%
\pgfsetstrokecolor{currentstroke}%
\pgfsetdash{}{0pt}%
\pgfsys@defobject{currentmarker}{\pgfqpoint{0.000000in}{0.000000in}}{\pgfqpoint{0.000000in}{0.055556in}}{%
\pgfpathmoveto{\pgfqpoint{0.000000in}{0.000000in}}%
\pgfpathlineto{\pgfqpoint{0.000000in}{0.055556in}}%
\pgfusepath{stroke,fill}%
}%
\begin{pgfscope}%
\pgfsys@transformshift{1.806522in}{0.450000in}%
\pgfsys@useobject{currentmarker}{}%
\end{pgfscope}%
\end{pgfscope}%
\begin{pgfscope}%
\pgfsetbuttcap%
\pgfsetroundjoin%
\definecolor{currentfill}{rgb}{0.000000,0.000000,0.000000}%
\pgfsetfillcolor{currentfill}%
\pgfsetlinewidth{0.501875pt}%
\definecolor{currentstroke}{rgb}{0.000000,0.000000,0.000000}%
\pgfsetstrokecolor{currentstroke}%
\pgfsetdash{}{0pt}%
\pgfsys@defobject{currentmarker}{\pgfqpoint{0.000000in}{-0.055556in}}{\pgfqpoint{0.000000in}{0.000000in}}{%
\pgfpathmoveto{\pgfqpoint{0.000000in}{0.000000in}}%
\pgfpathlineto{\pgfqpoint{0.000000in}{-0.055556in}}%
\pgfusepath{stroke,fill}%
}%
\begin{pgfscope}%
\pgfsys@transformshift{1.806522in}{4.050000in}%
\pgfsys@useobject{currentmarker}{}%
\end{pgfscope}%
\end{pgfscope}%
\begin{pgfscope}%
\pgftext[x=1.806522in,y=0.394444in,,top]{{\rmfamily\fontsize{11.000000}{13.200000}\selectfont \(\displaystyle 2^{8}\)}}%
\end{pgfscope}%
\begin{pgfscope}%
\pgfpathrectangle{\pgfqpoint{0.750000in}{0.450000in}}{\pgfqpoint{4.050000in}{3.600000in}} %
\pgfusepath{clip}%
\pgfsetbuttcap%
\pgfsetroundjoin%
\pgfsetlinewidth{0.501875pt}%
\definecolor{currentstroke}{rgb}{0.000000,0.000000,0.000000}%
\pgfsetstrokecolor{currentstroke}%
\pgfsetdash{{1.000000pt}{3.000000pt}}{0.000000pt}%
\pgfpathmoveto{\pgfqpoint{2.158696in}{0.450000in}}%
\pgfpathlineto{\pgfqpoint{2.158696in}{4.050000in}}%
\pgfusepath{stroke}%
\end{pgfscope}%
\begin{pgfscope}%
\pgfsetbuttcap%
\pgfsetroundjoin%
\definecolor{currentfill}{rgb}{0.000000,0.000000,0.000000}%
\pgfsetfillcolor{currentfill}%
\pgfsetlinewidth{0.501875pt}%
\definecolor{currentstroke}{rgb}{0.000000,0.000000,0.000000}%
\pgfsetstrokecolor{currentstroke}%
\pgfsetdash{}{0pt}%
\pgfsys@defobject{currentmarker}{\pgfqpoint{0.000000in}{0.000000in}}{\pgfqpoint{0.000000in}{0.055556in}}{%
\pgfpathmoveto{\pgfqpoint{0.000000in}{0.000000in}}%
\pgfpathlineto{\pgfqpoint{0.000000in}{0.055556in}}%
\pgfusepath{stroke,fill}%
}%
\begin{pgfscope}%
\pgfsys@transformshift{2.158696in}{0.450000in}%
\pgfsys@useobject{currentmarker}{}%
\end{pgfscope}%
\end{pgfscope}%
\begin{pgfscope}%
\pgfsetbuttcap%
\pgfsetroundjoin%
\definecolor{currentfill}{rgb}{0.000000,0.000000,0.000000}%
\pgfsetfillcolor{currentfill}%
\pgfsetlinewidth{0.501875pt}%
\definecolor{currentstroke}{rgb}{0.000000,0.000000,0.000000}%
\pgfsetstrokecolor{currentstroke}%
\pgfsetdash{}{0pt}%
\pgfsys@defobject{currentmarker}{\pgfqpoint{0.000000in}{-0.055556in}}{\pgfqpoint{0.000000in}{0.000000in}}{%
\pgfpathmoveto{\pgfqpoint{0.000000in}{0.000000in}}%
\pgfpathlineto{\pgfqpoint{0.000000in}{-0.055556in}}%
\pgfusepath{stroke,fill}%
}%
\begin{pgfscope}%
\pgfsys@transformshift{2.158696in}{4.050000in}%
\pgfsys@useobject{currentmarker}{}%
\end{pgfscope}%
\end{pgfscope}%
\begin{pgfscope}%
\pgftext[x=2.158696in,y=0.394444in,,top]{{\rmfamily\fontsize{11.000000}{13.200000}\selectfont \(\displaystyle 2^{10}\)}}%
\end{pgfscope}%
\begin{pgfscope}%
\pgfpathrectangle{\pgfqpoint{0.750000in}{0.450000in}}{\pgfqpoint{4.050000in}{3.600000in}} %
\pgfusepath{clip}%
\pgfsetbuttcap%
\pgfsetroundjoin%
\pgfsetlinewidth{0.501875pt}%
\definecolor{currentstroke}{rgb}{0.000000,0.000000,0.000000}%
\pgfsetstrokecolor{currentstroke}%
\pgfsetdash{{1.000000pt}{3.000000pt}}{0.000000pt}%
\pgfpathmoveto{\pgfqpoint{2.510870in}{0.450000in}}%
\pgfpathlineto{\pgfqpoint{2.510870in}{4.050000in}}%
\pgfusepath{stroke}%
\end{pgfscope}%
\begin{pgfscope}%
\pgfsetbuttcap%
\pgfsetroundjoin%
\definecolor{currentfill}{rgb}{0.000000,0.000000,0.000000}%
\pgfsetfillcolor{currentfill}%
\pgfsetlinewidth{0.501875pt}%
\definecolor{currentstroke}{rgb}{0.000000,0.000000,0.000000}%
\pgfsetstrokecolor{currentstroke}%
\pgfsetdash{}{0pt}%
\pgfsys@defobject{currentmarker}{\pgfqpoint{0.000000in}{0.000000in}}{\pgfqpoint{0.000000in}{0.055556in}}{%
\pgfpathmoveto{\pgfqpoint{0.000000in}{0.000000in}}%
\pgfpathlineto{\pgfqpoint{0.000000in}{0.055556in}}%
\pgfusepath{stroke,fill}%
}%
\begin{pgfscope}%
\pgfsys@transformshift{2.510870in}{0.450000in}%
\pgfsys@useobject{currentmarker}{}%
\end{pgfscope}%
\end{pgfscope}%
\begin{pgfscope}%
\pgfsetbuttcap%
\pgfsetroundjoin%
\definecolor{currentfill}{rgb}{0.000000,0.000000,0.000000}%
\pgfsetfillcolor{currentfill}%
\pgfsetlinewidth{0.501875pt}%
\definecolor{currentstroke}{rgb}{0.000000,0.000000,0.000000}%
\pgfsetstrokecolor{currentstroke}%
\pgfsetdash{}{0pt}%
\pgfsys@defobject{currentmarker}{\pgfqpoint{0.000000in}{-0.055556in}}{\pgfqpoint{0.000000in}{0.000000in}}{%
\pgfpathmoveto{\pgfqpoint{0.000000in}{0.000000in}}%
\pgfpathlineto{\pgfqpoint{0.000000in}{-0.055556in}}%
\pgfusepath{stroke,fill}%
}%
\begin{pgfscope}%
\pgfsys@transformshift{2.510870in}{4.050000in}%
\pgfsys@useobject{currentmarker}{}%
\end{pgfscope}%
\end{pgfscope}%
\begin{pgfscope}%
\pgftext[x=2.510870in,y=0.394444in,,top]{{\rmfamily\fontsize{11.000000}{13.200000}\selectfont \(\displaystyle 2^{12}\)}}%
\end{pgfscope}%
\begin{pgfscope}%
\pgfpathrectangle{\pgfqpoint{0.750000in}{0.450000in}}{\pgfqpoint{4.050000in}{3.600000in}} %
\pgfusepath{clip}%
\pgfsetbuttcap%
\pgfsetroundjoin%
\pgfsetlinewidth{0.501875pt}%
\definecolor{currentstroke}{rgb}{0.000000,0.000000,0.000000}%
\pgfsetstrokecolor{currentstroke}%
\pgfsetdash{{1.000000pt}{3.000000pt}}{0.000000pt}%
\pgfpathmoveto{\pgfqpoint{2.863043in}{0.450000in}}%
\pgfpathlineto{\pgfqpoint{2.863043in}{4.050000in}}%
\pgfusepath{stroke}%
\end{pgfscope}%
\begin{pgfscope}%
\pgfsetbuttcap%
\pgfsetroundjoin%
\definecolor{currentfill}{rgb}{0.000000,0.000000,0.000000}%
\pgfsetfillcolor{currentfill}%
\pgfsetlinewidth{0.501875pt}%
\definecolor{currentstroke}{rgb}{0.000000,0.000000,0.000000}%
\pgfsetstrokecolor{currentstroke}%
\pgfsetdash{}{0pt}%
\pgfsys@defobject{currentmarker}{\pgfqpoint{0.000000in}{0.000000in}}{\pgfqpoint{0.000000in}{0.055556in}}{%
\pgfpathmoveto{\pgfqpoint{0.000000in}{0.000000in}}%
\pgfpathlineto{\pgfqpoint{0.000000in}{0.055556in}}%
\pgfusepath{stroke,fill}%
}%
\begin{pgfscope}%
\pgfsys@transformshift{2.863043in}{0.450000in}%
\pgfsys@useobject{currentmarker}{}%
\end{pgfscope}%
\end{pgfscope}%
\begin{pgfscope}%
\pgfsetbuttcap%
\pgfsetroundjoin%
\definecolor{currentfill}{rgb}{0.000000,0.000000,0.000000}%
\pgfsetfillcolor{currentfill}%
\pgfsetlinewidth{0.501875pt}%
\definecolor{currentstroke}{rgb}{0.000000,0.000000,0.000000}%
\pgfsetstrokecolor{currentstroke}%
\pgfsetdash{}{0pt}%
\pgfsys@defobject{currentmarker}{\pgfqpoint{0.000000in}{-0.055556in}}{\pgfqpoint{0.000000in}{0.000000in}}{%
\pgfpathmoveto{\pgfqpoint{0.000000in}{0.000000in}}%
\pgfpathlineto{\pgfqpoint{0.000000in}{-0.055556in}}%
\pgfusepath{stroke,fill}%
}%
\begin{pgfscope}%
\pgfsys@transformshift{2.863043in}{4.050000in}%
\pgfsys@useobject{currentmarker}{}%
\end{pgfscope}%
\end{pgfscope}%
\begin{pgfscope}%
\pgftext[x=2.863043in,y=0.394444in,,top]{{\rmfamily\fontsize{11.000000}{13.200000}\selectfont \(\displaystyle 2^{14}\)}}%
\end{pgfscope}%
\begin{pgfscope}%
\pgfpathrectangle{\pgfqpoint{0.750000in}{0.450000in}}{\pgfqpoint{4.050000in}{3.600000in}} %
\pgfusepath{clip}%
\pgfsetbuttcap%
\pgfsetroundjoin%
\pgfsetlinewidth{0.501875pt}%
\definecolor{currentstroke}{rgb}{0.000000,0.000000,0.000000}%
\pgfsetstrokecolor{currentstroke}%
\pgfsetdash{{1.000000pt}{3.000000pt}}{0.000000pt}%
\pgfpathmoveto{\pgfqpoint{3.215217in}{0.450000in}}%
\pgfpathlineto{\pgfqpoint{3.215217in}{4.050000in}}%
\pgfusepath{stroke}%
\end{pgfscope}%
\begin{pgfscope}%
\pgfsetbuttcap%
\pgfsetroundjoin%
\definecolor{currentfill}{rgb}{0.000000,0.000000,0.000000}%
\pgfsetfillcolor{currentfill}%
\pgfsetlinewidth{0.501875pt}%
\definecolor{currentstroke}{rgb}{0.000000,0.000000,0.000000}%
\pgfsetstrokecolor{currentstroke}%
\pgfsetdash{}{0pt}%
\pgfsys@defobject{currentmarker}{\pgfqpoint{0.000000in}{0.000000in}}{\pgfqpoint{0.000000in}{0.055556in}}{%
\pgfpathmoveto{\pgfqpoint{0.000000in}{0.000000in}}%
\pgfpathlineto{\pgfqpoint{0.000000in}{0.055556in}}%
\pgfusepath{stroke,fill}%
}%
\begin{pgfscope}%
\pgfsys@transformshift{3.215217in}{0.450000in}%
\pgfsys@useobject{currentmarker}{}%
\end{pgfscope}%
\end{pgfscope}%
\begin{pgfscope}%
\pgfsetbuttcap%
\pgfsetroundjoin%
\definecolor{currentfill}{rgb}{0.000000,0.000000,0.000000}%
\pgfsetfillcolor{currentfill}%
\pgfsetlinewidth{0.501875pt}%
\definecolor{currentstroke}{rgb}{0.000000,0.000000,0.000000}%
\pgfsetstrokecolor{currentstroke}%
\pgfsetdash{}{0pt}%
\pgfsys@defobject{currentmarker}{\pgfqpoint{0.000000in}{-0.055556in}}{\pgfqpoint{0.000000in}{0.000000in}}{%
\pgfpathmoveto{\pgfqpoint{0.000000in}{0.000000in}}%
\pgfpathlineto{\pgfqpoint{0.000000in}{-0.055556in}}%
\pgfusepath{stroke,fill}%
}%
\begin{pgfscope}%
\pgfsys@transformshift{3.215217in}{4.050000in}%
\pgfsys@useobject{currentmarker}{}%
\end{pgfscope}%
\end{pgfscope}%
\begin{pgfscope}%
\pgftext[x=3.215217in,y=0.394444in,,top]{{\rmfamily\fontsize{11.000000}{13.200000}\selectfont \(\displaystyle 2^{16}\)}}%
\end{pgfscope}%
\begin{pgfscope}%
\pgfpathrectangle{\pgfqpoint{0.750000in}{0.450000in}}{\pgfqpoint{4.050000in}{3.600000in}} %
\pgfusepath{clip}%
\pgfsetbuttcap%
\pgfsetroundjoin%
\pgfsetlinewidth{0.501875pt}%
\definecolor{currentstroke}{rgb}{0.000000,0.000000,0.000000}%
\pgfsetstrokecolor{currentstroke}%
\pgfsetdash{{1.000000pt}{3.000000pt}}{0.000000pt}%
\pgfpathmoveto{\pgfqpoint{3.567391in}{0.450000in}}%
\pgfpathlineto{\pgfqpoint{3.567391in}{4.050000in}}%
\pgfusepath{stroke}%
\end{pgfscope}%
\begin{pgfscope}%
\pgfsetbuttcap%
\pgfsetroundjoin%
\definecolor{currentfill}{rgb}{0.000000,0.000000,0.000000}%
\pgfsetfillcolor{currentfill}%
\pgfsetlinewidth{0.501875pt}%
\definecolor{currentstroke}{rgb}{0.000000,0.000000,0.000000}%
\pgfsetstrokecolor{currentstroke}%
\pgfsetdash{}{0pt}%
\pgfsys@defobject{currentmarker}{\pgfqpoint{0.000000in}{0.000000in}}{\pgfqpoint{0.000000in}{0.055556in}}{%
\pgfpathmoveto{\pgfqpoint{0.000000in}{0.000000in}}%
\pgfpathlineto{\pgfqpoint{0.000000in}{0.055556in}}%
\pgfusepath{stroke,fill}%
}%
\begin{pgfscope}%
\pgfsys@transformshift{3.567391in}{0.450000in}%
\pgfsys@useobject{currentmarker}{}%
\end{pgfscope}%
\end{pgfscope}%
\begin{pgfscope}%
\pgfsetbuttcap%
\pgfsetroundjoin%
\definecolor{currentfill}{rgb}{0.000000,0.000000,0.000000}%
\pgfsetfillcolor{currentfill}%
\pgfsetlinewidth{0.501875pt}%
\definecolor{currentstroke}{rgb}{0.000000,0.000000,0.000000}%
\pgfsetstrokecolor{currentstroke}%
\pgfsetdash{}{0pt}%
\pgfsys@defobject{currentmarker}{\pgfqpoint{0.000000in}{-0.055556in}}{\pgfqpoint{0.000000in}{0.000000in}}{%
\pgfpathmoveto{\pgfqpoint{0.000000in}{0.000000in}}%
\pgfpathlineto{\pgfqpoint{0.000000in}{-0.055556in}}%
\pgfusepath{stroke,fill}%
}%
\begin{pgfscope}%
\pgfsys@transformshift{3.567391in}{4.050000in}%
\pgfsys@useobject{currentmarker}{}%
\end{pgfscope}%
\end{pgfscope}%
\begin{pgfscope}%
\pgftext[x=3.567391in,y=0.394444in,,top]{{\rmfamily\fontsize{11.000000}{13.200000}\selectfont \(\displaystyle 2^{18}\)}}%
\end{pgfscope}%
\begin{pgfscope}%
\pgfpathrectangle{\pgfqpoint{0.750000in}{0.450000in}}{\pgfqpoint{4.050000in}{3.600000in}} %
\pgfusepath{clip}%
\pgfsetbuttcap%
\pgfsetroundjoin%
\pgfsetlinewidth{0.501875pt}%
\definecolor{currentstroke}{rgb}{0.000000,0.000000,0.000000}%
\pgfsetstrokecolor{currentstroke}%
\pgfsetdash{{1.000000pt}{3.000000pt}}{0.000000pt}%
\pgfpathmoveto{\pgfqpoint{3.919565in}{0.450000in}}%
\pgfpathlineto{\pgfqpoint{3.919565in}{4.050000in}}%
\pgfusepath{stroke}%
\end{pgfscope}%
\begin{pgfscope}%
\pgfsetbuttcap%
\pgfsetroundjoin%
\definecolor{currentfill}{rgb}{0.000000,0.000000,0.000000}%
\pgfsetfillcolor{currentfill}%
\pgfsetlinewidth{0.501875pt}%
\definecolor{currentstroke}{rgb}{0.000000,0.000000,0.000000}%
\pgfsetstrokecolor{currentstroke}%
\pgfsetdash{}{0pt}%
\pgfsys@defobject{currentmarker}{\pgfqpoint{0.000000in}{0.000000in}}{\pgfqpoint{0.000000in}{0.055556in}}{%
\pgfpathmoveto{\pgfqpoint{0.000000in}{0.000000in}}%
\pgfpathlineto{\pgfqpoint{0.000000in}{0.055556in}}%
\pgfusepath{stroke,fill}%
}%
\begin{pgfscope}%
\pgfsys@transformshift{3.919565in}{0.450000in}%
\pgfsys@useobject{currentmarker}{}%
\end{pgfscope}%
\end{pgfscope}%
\begin{pgfscope}%
\pgfsetbuttcap%
\pgfsetroundjoin%
\definecolor{currentfill}{rgb}{0.000000,0.000000,0.000000}%
\pgfsetfillcolor{currentfill}%
\pgfsetlinewidth{0.501875pt}%
\definecolor{currentstroke}{rgb}{0.000000,0.000000,0.000000}%
\pgfsetstrokecolor{currentstroke}%
\pgfsetdash{}{0pt}%
\pgfsys@defobject{currentmarker}{\pgfqpoint{0.000000in}{-0.055556in}}{\pgfqpoint{0.000000in}{0.000000in}}{%
\pgfpathmoveto{\pgfqpoint{0.000000in}{0.000000in}}%
\pgfpathlineto{\pgfqpoint{0.000000in}{-0.055556in}}%
\pgfusepath{stroke,fill}%
}%
\begin{pgfscope}%
\pgfsys@transformshift{3.919565in}{4.050000in}%
\pgfsys@useobject{currentmarker}{}%
\end{pgfscope}%
\end{pgfscope}%
\begin{pgfscope}%
\pgftext[x=3.919565in,y=0.394444in,,top]{{\rmfamily\fontsize{11.000000}{13.200000}\selectfont \(\displaystyle 2^{20}\)}}%
\end{pgfscope}%
\begin{pgfscope}%
\pgfpathrectangle{\pgfqpoint{0.750000in}{0.450000in}}{\pgfqpoint{4.050000in}{3.600000in}} %
\pgfusepath{clip}%
\pgfsetbuttcap%
\pgfsetroundjoin%
\pgfsetlinewidth{0.501875pt}%
\definecolor{currentstroke}{rgb}{0.000000,0.000000,0.000000}%
\pgfsetstrokecolor{currentstroke}%
\pgfsetdash{{1.000000pt}{3.000000pt}}{0.000000pt}%
\pgfpathmoveto{\pgfqpoint{4.271739in}{0.450000in}}%
\pgfpathlineto{\pgfqpoint{4.271739in}{4.050000in}}%
\pgfusepath{stroke}%
\end{pgfscope}%
\begin{pgfscope}%
\pgfsetbuttcap%
\pgfsetroundjoin%
\definecolor{currentfill}{rgb}{0.000000,0.000000,0.000000}%
\pgfsetfillcolor{currentfill}%
\pgfsetlinewidth{0.501875pt}%
\definecolor{currentstroke}{rgb}{0.000000,0.000000,0.000000}%
\pgfsetstrokecolor{currentstroke}%
\pgfsetdash{}{0pt}%
\pgfsys@defobject{currentmarker}{\pgfqpoint{0.000000in}{0.000000in}}{\pgfqpoint{0.000000in}{0.055556in}}{%
\pgfpathmoveto{\pgfqpoint{0.000000in}{0.000000in}}%
\pgfpathlineto{\pgfqpoint{0.000000in}{0.055556in}}%
\pgfusepath{stroke,fill}%
}%
\begin{pgfscope}%
\pgfsys@transformshift{4.271739in}{0.450000in}%
\pgfsys@useobject{currentmarker}{}%
\end{pgfscope}%
\end{pgfscope}%
\begin{pgfscope}%
\pgfsetbuttcap%
\pgfsetroundjoin%
\definecolor{currentfill}{rgb}{0.000000,0.000000,0.000000}%
\pgfsetfillcolor{currentfill}%
\pgfsetlinewidth{0.501875pt}%
\definecolor{currentstroke}{rgb}{0.000000,0.000000,0.000000}%
\pgfsetstrokecolor{currentstroke}%
\pgfsetdash{}{0pt}%
\pgfsys@defobject{currentmarker}{\pgfqpoint{0.000000in}{-0.055556in}}{\pgfqpoint{0.000000in}{0.000000in}}{%
\pgfpathmoveto{\pgfqpoint{0.000000in}{0.000000in}}%
\pgfpathlineto{\pgfqpoint{0.000000in}{-0.055556in}}%
\pgfusepath{stroke,fill}%
}%
\begin{pgfscope}%
\pgfsys@transformshift{4.271739in}{4.050000in}%
\pgfsys@useobject{currentmarker}{}%
\end{pgfscope}%
\end{pgfscope}%
\begin{pgfscope}%
\pgftext[x=4.271739in,y=0.394444in,,top]{{\rmfamily\fontsize{11.000000}{13.200000}\selectfont \(\displaystyle 2^{22}\)}}%
\end{pgfscope}%
\begin{pgfscope}%
\pgfpathrectangle{\pgfqpoint{0.750000in}{0.450000in}}{\pgfqpoint{4.050000in}{3.600000in}} %
\pgfusepath{clip}%
\pgfsetbuttcap%
\pgfsetroundjoin%
\pgfsetlinewidth{0.501875pt}%
\definecolor{currentstroke}{rgb}{0.000000,0.000000,0.000000}%
\pgfsetstrokecolor{currentstroke}%
\pgfsetdash{{1.000000pt}{3.000000pt}}{0.000000pt}%
\pgfpathmoveto{\pgfqpoint{4.623913in}{0.450000in}}%
\pgfpathlineto{\pgfqpoint{4.623913in}{4.050000in}}%
\pgfusepath{stroke}%
\end{pgfscope}%
\begin{pgfscope}%
\pgfsetbuttcap%
\pgfsetroundjoin%
\definecolor{currentfill}{rgb}{0.000000,0.000000,0.000000}%
\pgfsetfillcolor{currentfill}%
\pgfsetlinewidth{0.501875pt}%
\definecolor{currentstroke}{rgb}{0.000000,0.000000,0.000000}%
\pgfsetstrokecolor{currentstroke}%
\pgfsetdash{}{0pt}%
\pgfsys@defobject{currentmarker}{\pgfqpoint{0.000000in}{0.000000in}}{\pgfqpoint{0.000000in}{0.055556in}}{%
\pgfpathmoveto{\pgfqpoint{0.000000in}{0.000000in}}%
\pgfpathlineto{\pgfqpoint{0.000000in}{0.055556in}}%
\pgfusepath{stroke,fill}%
}%
\begin{pgfscope}%
\pgfsys@transformshift{4.623913in}{0.450000in}%
\pgfsys@useobject{currentmarker}{}%
\end{pgfscope}%
\end{pgfscope}%
\begin{pgfscope}%
\pgfsetbuttcap%
\pgfsetroundjoin%
\definecolor{currentfill}{rgb}{0.000000,0.000000,0.000000}%
\pgfsetfillcolor{currentfill}%
\pgfsetlinewidth{0.501875pt}%
\definecolor{currentstroke}{rgb}{0.000000,0.000000,0.000000}%
\pgfsetstrokecolor{currentstroke}%
\pgfsetdash{}{0pt}%
\pgfsys@defobject{currentmarker}{\pgfqpoint{0.000000in}{-0.055556in}}{\pgfqpoint{0.000000in}{0.000000in}}{%
\pgfpathmoveto{\pgfqpoint{0.000000in}{0.000000in}}%
\pgfpathlineto{\pgfqpoint{0.000000in}{-0.055556in}}%
\pgfusepath{stroke,fill}%
}%
\begin{pgfscope}%
\pgfsys@transformshift{4.623913in}{4.050000in}%
\pgfsys@useobject{currentmarker}{}%
\end{pgfscope}%
\end{pgfscope}%
\begin{pgfscope}%
\pgftext[x=4.623913in,y=0.394444in,,top]{{\rmfamily\fontsize{11.000000}{13.200000}\selectfont \(\displaystyle 2^{24}\)}}%
\end{pgfscope}%
\begin{pgfscope}%
\pgfsetbuttcap%
\pgfsetroundjoin%
\definecolor{currentfill}{rgb}{0.000000,0.000000,0.000000}%
\pgfsetfillcolor{currentfill}%
\pgfsetlinewidth{0.501875pt}%
\definecolor{currentstroke}{rgb}{0.000000,0.000000,0.000000}%
\pgfsetstrokecolor{currentstroke}%
\pgfsetdash{}{0pt}%
\pgfsys@defobject{currentmarker}{\pgfqpoint{0.000000in}{0.000000in}}{\pgfqpoint{0.000000in}{0.027778in}}{%
\pgfpathmoveto{\pgfqpoint{0.000000in}{0.000000in}}%
\pgfpathlineto{\pgfqpoint{0.000000in}{0.027778in}}%
\pgfusepath{stroke,fill}%
}%
\begin{pgfscope}%
\pgfsys@transformshift{0.750000in}{0.450000in}%
\pgfsys@useobject{currentmarker}{}%
\end{pgfscope}%
\end{pgfscope}%
\begin{pgfscope}%
\pgfsetbuttcap%
\pgfsetroundjoin%
\definecolor{currentfill}{rgb}{0.000000,0.000000,0.000000}%
\pgfsetfillcolor{currentfill}%
\pgfsetlinewidth{0.501875pt}%
\definecolor{currentstroke}{rgb}{0.000000,0.000000,0.000000}%
\pgfsetstrokecolor{currentstroke}%
\pgfsetdash{}{0pt}%
\pgfsys@defobject{currentmarker}{\pgfqpoint{0.000000in}{-0.027778in}}{\pgfqpoint{0.000000in}{0.000000in}}{%
\pgfpathmoveto{\pgfqpoint{0.000000in}{0.000000in}}%
\pgfpathlineto{\pgfqpoint{0.000000in}{-0.027778in}}%
\pgfusepath{stroke,fill}%
}%
\begin{pgfscope}%
\pgfsys@transformshift{0.750000in}{4.050000in}%
\pgfsys@useobject{currentmarker}{}%
\end{pgfscope}%
\end{pgfscope}%
\begin{pgfscope}%
\pgfsetbuttcap%
\pgfsetroundjoin%
\definecolor{currentfill}{rgb}{0.000000,0.000000,0.000000}%
\pgfsetfillcolor{currentfill}%
\pgfsetlinewidth{0.501875pt}%
\definecolor{currentstroke}{rgb}{0.000000,0.000000,0.000000}%
\pgfsetstrokecolor{currentstroke}%
\pgfsetdash{}{0pt}%
\pgfsys@defobject{currentmarker}{\pgfqpoint{0.000000in}{0.000000in}}{\pgfqpoint{0.000000in}{0.027778in}}{%
\pgfpathmoveto{\pgfqpoint{0.000000in}{0.000000in}}%
\pgfpathlineto{\pgfqpoint{0.000000in}{0.027778in}}%
\pgfusepath{stroke,fill}%
}%
\begin{pgfscope}%
\pgfsys@transformshift{1.102174in}{0.450000in}%
\pgfsys@useobject{currentmarker}{}%
\end{pgfscope}%
\end{pgfscope}%
\begin{pgfscope}%
\pgfsetbuttcap%
\pgfsetroundjoin%
\definecolor{currentfill}{rgb}{0.000000,0.000000,0.000000}%
\pgfsetfillcolor{currentfill}%
\pgfsetlinewidth{0.501875pt}%
\definecolor{currentstroke}{rgb}{0.000000,0.000000,0.000000}%
\pgfsetstrokecolor{currentstroke}%
\pgfsetdash{}{0pt}%
\pgfsys@defobject{currentmarker}{\pgfqpoint{0.000000in}{-0.027778in}}{\pgfqpoint{0.000000in}{0.000000in}}{%
\pgfpathmoveto{\pgfqpoint{0.000000in}{0.000000in}}%
\pgfpathlineto{\pgfqpoint{0.000000in}{-0.027778in}}%
\pgfusepath{stroke,fill}%
}%
\begin{pgfscope}%
\pgfsys@transformshift{1.102174in}{4.050000in}%
\pgfsys@useobject{currentmarker}{}%
\end{pgfscope}%
\end{pgfscope}%
\begin{pgfscope}%
\pgfsetbuttcap%
\pgfsetroundjoin%
\definecolor{currentfill}{rgb}{0.000000,0.000000,0.000000}%
\pgfsetfillcolor{currentfill}%
\pgfsetlinewidth{0.501875pt}%
\definecolor{currentstroke}{rgb}{0.000000,0.000000,0.000000}%
\pgfsetstrokecolor{currentstroke}%
\pgfsetdash{}{0pt}%
\pgfsys@defobject{currentmarker}{\pgfqpoint{0.000000in}{0.000000in}}{\pgfqpoint{0.000000in}{0.027778in}}{%
\pgfpathmoveto{\pgfqpoint{0.000000in}{0.000000in}}%
\pgfpathlineto{\pgfqpoint{0.000000in}{0.027778in}}%
\pgfusepath{stroke,fill}%
}%
\begin{pgfscope}%
\pgfsys@transformshift{1.454348in}{0.450000in}%
\pgfsys@useobject{currentmarker}{}%
\end{pgfscope}%
\end{pgfscope}%
\begin{pgfscope}%
\pgfsetbuttcap%
\pgfsetroundjoin%
\definecolor{currentfill}{rgb}{0.000000,0.000000,0.000000}%
\pgfsetfillcolor{currentfill}%
\pgfsetlinewidth{0.501875pt}%
\definecolor{currentstroke}{rgb}{0.000000,0.000000,0.000000}%
\pgfsetstrokecolor{currentstroke}%
\pgfsetdash{}{0pt}%
\pgfsys@defobject{currentmarker}{\pgfqpoint{0.000000in}{-0.027778in}}{\pgfqpoint{0.000000in}{0.000000in}}{%
\pgfpathmoveto{\pgfqpoint{0.000000in}{0.000000in}}%
\pgfpathlineto{\pgfqpoint{0.000000in}{-0.027778in}}%
\pgfusepath{stroke,fill}%
}%
\begin{pgfscope}%
\pgfsys@transformshift{1.454348in}{4.050000in}%
\pgfsys@useobject{currentmarker}{}%
\end{pgfscope}%
\end{pgfscope}%
\begin{pgfscope}%
\pgfsetbuttcap%
\pgfsetroundjoin%
\definecolor{currentfill}{rgb}{0.000000,0.000000,0.000000}%
\pgfsetfillcolor{currentfill}%
\pgfsetlinewidth{0.501875pt}%
\definecolor{currentstroke}{rgb}{0.000000,0.000000,0.000000}%
\pgfsetstrokecolor{currentstroke}%
\pgfsetdash{}{0pt}%
\pgfsys@defobject{currentmarker}{\pgfqpoint{0.000000in}{0.000000in}}{\pgfqpoint{0.000000in}{0.027778in}}{%
\pgfpathmoveto{\pgfqpoint{0.000000in}{0.000000in}}%
\pgfpathlineto{\pgfqpoint{0.000000in}{0.027778in}}%
\pgfusepath{stroke,fill}%
}%
\begin{pgfscope}%
\pgfsys@transformshift{1.806522in}{0.450000in}%
\pgfsys@useobject{currentmarker}{}%
\end{pgfscope}%
\end{pgfscope}%
\begin{pgfscope}%
\pgfsetbuttcap%
\pgfsetroundjoin%
\definecolor{currentfill}{rgb}{0.000000,0.000000,0.000000}%
\pgfsetfillcolor{currentfill}%
\pgfsetlinewidth{0.501875pt}%
\definecolor{currentstroke}{rgb}{0.000000,0.000000,0.000000}%
\pgfsetstrokecolor{currentstroke}%
\pgfsetdash{}{0pt}%
\pgfsys@defobject{currentmarker}{\pgfqpoint{0.000000in}{-0.027778in}}{\pgfqpoint{0.000000in}{0.000000in}}{%
\pgfpathmoveto{\pgfqpoint{0.000000in}{0.000000in}}%
\pgfpathlineto{\pgfqpoint{0.000000in}{-0.027778in}}%
\pgfusepath{stroke,fill}%
}%
\begin{pgfscope}%
\pgfsys@transformshift{1.806522in}{4.050000in}%
\pgfsys@useobject{currentmarker}{}%
\end{pgfscope}%
\end{pgfscope}%
\begin{pgfscope}%
\pgfsetbuttcap%
\pgfsetroundjoin%
\definecolor{currentfill}{rgb}{0.000000,0.000000,0.000000}%
\pgfsetfillcolor{currentfill}%
\pgfsetlinewidth{0.501875pt}%
\definecolor{currentstroke}{rgb}{0.000000,0.000000,0.000000}%
\pgfsetstrokecolor{currentstroke}%
\pgfsetdash{}{0pt}%
\pgfsys@defobject{currentmarker}{\pgfqpoint{0.000000in}{0.000000in}}{\pgfqpoint{0.000000in}{0.027778in}}{%
\pgfpathmoveto{\pgfqpoint{0.000000in}{0.000000in}}%
\pgfpathlineto{\pgfqpoint{0.000000in}{0.027778in}}%
\pgfusepath{stroke,fill}%
}%
\begin{pgfscope}%
\pgfsys@transformshift{2.158696in}{0.450000in}%
\pgfsys@useobject{currentmarker}{}%
\end{pgfscope}%
\end{pgfscope}%
\begin{pgfscope}%
\pgfsetbuttcap%
\pgfsetroundjoin%
\definecolor{currentfill}{rgb}{0.000000,0.000000,0.000000}%
\pgfsetfillcolor{currentfill}%
\pgfsetlinewidth{0.501875pt}%
\definecolor{currentstroke}{rgb}{0.000000,0.000000,0.000000}%
\pgfsetstrokecolor{currentstroke}%
\pgfsetdash{}{0pt}%
\pgfsys@defobject{currentmarker}{\pgfqpoint{0.000000in}{-0.027778in}}{\pgfqpoint{0.000000in}{0.000000in}}{%
\pgfpathmoveto{\pgfqpoint{0.000000in}{0.000000in}}%
\pgfpathlineto{\pgfqpoint{0.000000in}{-0.027778in}}%
\pgfusepath{stroke,fill}%
}%
\begin{pgfscope}%
\pgfsys@transformshift{2.158696in}{4.050000in}%
\pgfsys@useobject{currentmarker}{}%
\end{pgfscope}%
\end{pgfscope}%
\begin{pgfscope}%
\pgfsetbuttcap%
\pgfsetroundjoin%
\definecolor{currentfill}{rgb}{0.000000,0.000000,0.000000}%
\pgfsetfillcolor{currentfill}%
\pgfsetlinewidth{0.501875pt}%
\definecolor{currentstroke}{rgb}{0.000000,0.000000,0.000000}%
\pgfsetstrokecolor{currentstroke}%
\pgfsetdash{}{0pt}%
\pgfsys@defobject{currentmarker}{\pgfqpoint{0.000000in}{0.000000in}}{\pgfqpoint{0.000000in}{0.027778in}}{%
\pgfpathmoveto{\pgfqpoint{0.000000in}{0.000000in}}%
\pgfpathlineto{\pgfqpoint{0.000000in}{0.027778in}}%
\pgfusepath{stroke,fill}%
}%
\begin{pgfscope}%
\pgfsys@transformshift{2.510870in}{0.450000in}%
\pgfsys@useobject{currentmarker}{}%
\end{pgfscope}%
\end{pgfscope}%
\begin{pgfscope}%
\pgfsetbuttcap%
\pgfsetroundjoin%
\definecolor{currentfill}{rgb}{0.000000,0.000000,0.000000}%
\pgfsetfillcolor{currentfill}%
\pgfsetlinewidth{0.501875pt}%
\definecolor{currentstroke}{rgb}{0.000000,0.000000,0.000000}%
\pgfsetstrokecolor{currentstroke}%
\pgfsetdash{}{0pt}%
\pgfsys@defobject{currentmarker}{\pgfqpoint{0.000000in}{-0.027778in}}{\pgfqpoint{0.000000in}{0.000000in}}{%
\pgfpathmoveto{\pgfqpoint{0.000000in}{0.000000in}}%
\pgfpathlineto{\pgfqpoint{0.000000in}{-0.027778in}}%
\pgfusepath{stroke,fill}%
}%
\begin{pgfscope}%
\pgfsys@transformshift{2.510870in}{4.050000in}%
\pgfsys@useobject{currentmarker}{}%
\end{pgfscope}%
\end{pgfscope}%
\begin{pgfscope}%
\pgfsetbuttcap%
\pgfsetroundjoin%
\definecolor{currentfill}{rgb}{0.000000,0.000000,0.000000}%
\pgfsetfillcolor{currentfill}%
\pgfsetlinewidth{0.501875pt}%
\definecolor{currentstroke}{rgb}{0.000000,0.000000,0.000000}%
\pgfsetstrokecolor{currentstroke}%
\pgfsetdash{}{0pt}%
\pgfsys@defobject{currentmarker}{\pgfqpoint{0.000000in}{0.000000in}}{\pgfqpoint{0.000000in}{0.027778in}}{%
\pgfpathmoveto{\pgfqpoint{0.000000in}{0.000000in}}%
\pgfpathlineto{\pgfqpoint{0.000000in}{0.027778in}}%
\pgfusepath{stroke,fill}%
}%
\begin{pgfscope}%
\pgfsys@transformshift{2.863043in}{0.450000in}%
\pgfsys@useobject{currentmarker}{}%
\end{pgfscope}%
\end{pgfscope}%
\begin{pgfscope}%
\pgfsetbuttcap%
\pgfsetroundjoin%
\definecolor{currentfill}{rgb}{0.000000,0.000000,0.000000}%
\pgfsetfillcolor{currentfill}%
\pgfsetlinewidth{0.501875pt}%
\definecolor{currentstroke}{rgb}{0.000000,0.000000,0.000000}%
\pgfsetstrokecolor{currentstroke}%
\pgfsetdash{}{0pt}%
\pgfsys@defobject{currentmarker}{\pgfqpoint{0.000000in}{-0.027778in}}{\pgfqpoint{0.000000in}{0.000000in}}{%
\pgfpathmoveto{\pgfqpoint{0.000000in}{0.000000in}}%
\pgfpathlineto{\pgfqpoint{0.000000in}{-0.027778in}}%
\pgfusepath{stroke,fill}%
}%
\begin{pgfscope}%
\pgfsys@transformshift{2.863043in}{4.050000in}%
\pgfsys@useobject{currentmarker}{}%
\end{pgfscope}%
\end{pgfscope}%
\begin{pgfscope}%
\pgfsetbuttcap%
\pgfsetroundjoin%
\definecolor{currentfill}{rgb}{0.000000,0.000000,0.000000}%
\pgfsetfillcolor{currentfill}%
\pgfsetlinewidth{0.501875pt}%
\definecolor{currentstroke}{rgb}{0.000000,0.000000,0.000000}%
\pgfsetstrokecolor{currentstroke}%
\pgfsetdash{}{0pt}%
\pgfsys@defobject{currentmarker}{\pgfqpoint{0.000000in}{0.000000in}}{\pgfqpoint{0.000000in}{0.027778in}}{%
\pgfpathmoveto{\pgfqpoint{0.000000in}{0.000000in}}%
\pgfpathlineto{\pgfqpoint{0.000000in}{0.027778in}}%
\pgfusepath{stroke,fill}%
}%
\begin{pgfscope}%
\pgfsys@transformshift{3.215217in}{0.450000in}%
\pgfsys@useobject{currentmarker}{}%
\end{pgfscope}%
\end{pgfscope}%
\begin{pgfscope}%
\pgfsetbuttcap%
\pgfsetroundjoin%
\definecolor{currentfill}{rgb}{0.000000,0.000000,0.000000}%
\pgfsetfillcolor{currentfill}%
\pgfsetlinewidth{0.501875pt}%
\definecolor{currentstroke}{rgb}{0.000000,0.000000,0.000000}%
\pgfsetstrokecolor{currentstroke}%
\pgfsetdash{}{0pt}%
\pgfsys@defobject{currentmarker}{\pgfqpoint{0.000000in}{-0.027778in}}{\pgfqpoint{0.000000in}{0.000000in}}{%
\pgfpathmoveto{\pgfqpoint{0.000000in}{0.000000in}}%
\pgfpathlineto{\pgfqpoint{0.000000in}{-0.027778in}}%
\pgfusepath{stroke,fill}%
}%
\begin{pgfscope}%
\pgfsys@transformshift{3.215217in}{4.050000in}%
\pgfsys@useobject{currentmarker}{}%
\end{pgfscope}%
\end{pgfscope}%
\begin{pgfscope}%
\pgfsetbuttcap%
\pgfsetroundjoin%
\definecolor{currentfill}{rgb}{0.000000,0.000000,0.000000}%
\pgfsetfillcolor{currentfill}%
\pgfsetlinewidth{0.501875pt}%
\definecolor{currentstroke}{rgb}{0.000000,0.000000,0.000000}%
\pgfsetstrokecolor{currentstroke}%
\pgfsetdash{}{0pt}%
\pgfsys@defobject{currentmarker}{\pgfqpoint{0.000000in}{0.000000in}}{\pgfqpoint{0.000000in}{0.027778in}}{%
\pgfpathmoveto{\pgfqpoint{0.000000in}{0.000000in}}%
\pgfpathlineto{\pgfqpoint{0.000000in}{0.027778in}}%
\pgfusepath{stroke,fill}%
}%
\begin{pgfscope}%
\pgfsys@transformshift{3.567391in}{0.450000in}%
\pgfsys@useobject{currentmarker}{}%
\end{pgfscope}%
\end{pgfscope}%
\begin{pgfscope}%
\pgfsetbuttcap%
\pgfsetroundjoin%
\definecolor{currentfill}{rgb}{0.000000,0.000000,0.000000}%
\pgfsetfillcolor{currentfill}%
\pgfsetlinewidth{0.501875pt}%
\definecolor{currentstroke}{rgb}{0.000000,0.000000,0.000000}%
\pgfsetstrokecolor{currentstroke}%
\pgfsetdash{}{0pt}%
\pgfsys@defobject{currentmarker}{\pgfqpoint{0.000000in}{-0.027778in}}{\pgfqpoint{0.000000in}{0.000000in}}{%
\pgfpathmoveto{\pgfqpoint{0.000000in}{0.000000in}}%
\pgfpathlineto{\pgfqpoint{0.000000in}{-0.027778in}}%
\pgfusepath{stroke,fill}%
}%
\begin{pgfscope}%
\pgfsys@transformshift{3.567391in}{4.050000in}%
\pgfsys@useobject{currentmarker}{}%
\end{pgfscope}%
\end{pgfscope}%
\begin{pgfscope}%
\pgfsetbuttcap%
\pgfsetroundjoin%
\definecolor{currentfill}{rgb}{0.000000,0.000000,0.000000}%
\pgfsetfillcolor{currentfill}%
\pgfsetlinewidth{0.501875pt}%
\definecolor{currentstroke}{rgb}{0.000000,0.000000,0.000000}%
\pgfsetstrokecolor{currentstroke}%
\pgfsetdash{}{0pt}%
\pgfsys@defobject{currentmarker}{\pgfqpoint{0.000000in}{0.000000in}}{\pgfqpoint{0.000000in}{0.027778in}}{%
\pgfpathmoveto{\pgfqpoint{0.000000in}{0.000000in}}%
\pgfpathlineto{\pgfqpoint{0.000000in}{0.027778in}}%
\pgfusepath{stroke,fill}%
}%
\begin{pgfscope}%
\pgfsys@transformshift{3.919565in}{0.450000in}%
\pgfsys@useobject{currentmarker}{}%
\end{pgfscope}%
\end{pgfscope}%
\begin{pgfscope}%
\pgfsetbuttcap%
\pgfsetroundjoin%
\definecolor{currentfill}{rgb}{0.000000,0.000000,0.000000}%
\pgfsetfillcolor{currentfill}%
\pgfsetlinewidth{0.501875pt}%
\definecolor{currentstroke}{rgb}{0.000000,0.000000,0.000000}%
\pgfsetstrokecolor{currentstroke}%
\pgfsetdash{}{0pt}%
\pgfsys@defobject{currentmarker}{\pgfqpoint{0.000000in}{-0.027778in}}{\pgfqpoint{0.000000in}{0.000000in}}{%
\pgfpathmoveto{\pgfqpoint{0.000000in}{0.000000in}}%
\pgfpathlineto{\pgfqpoint{0.000000in}{-0.027778in}}%
\pgfusepath{stroke,fill}%
}%
\begin{pgfscope}%
\pgfsys@transformshift{3.919565in}{4.050000in}%
\pgfsys@useobject{currentmarker}{}%
\end{pgfscope}%
\end{pgfscope}%
\begin{pgfscope}%
\pgfsetbuttcap%
\pgfsetroundjoin%
\definecolor{currentfill}{rgb}{0.000000,0.000000,0.000000}%
\pgfsetfillcolor{currentfill}%
\pgfsetlinewidth{0.501875pt}%
\definecolor{currentstroke}{rgb}{0.000000,0.000000,0.000000}%
\pgfsetstrokecolor{currentstroke}%
\pgfsetdash{}{0pt}%
\pgfsys@defobject{currentmarker}{\pgfqpoint{0.000000in}{0.000000in}}{\pgfqpoint{0.000000in}{0.027778in}}{%
\pgfpathmoveto{\pgfqpoint{0.000000in}{0.000000in}}%
\pgfpathlineto{\pgfqpoint{0.000000in}{0.027778in}}%
\pgfusepath{stroke,fill}%
}%
\begin{pgfscope}%
\pgfsys@transformshift{4.271739in}{0.450000in}%
\pgfsys@useobject{currentmarker}{}%
\end{pgfscope}%
\end{pgfscope}%
\begin{pgfscope}%
\pgfsetbuttcap%
\pgfsetroundjoin%
\definecolor{currentfill}{rgb}{0.000000,0.000000,0.000000}%
\pgfsetfillcolor{currentfill}%
\pgfsetlinewidth{0.501875pt}%
\definecolor{currentstroke}{rgb}{0.000000,0.000000,0.000000}%
\pgfsetstrokecolor{currentstroke}%
\pgfsetdash{}{0pt}%
\pgfsys@defobject{currentmarker}{\pgfqpoint{0.000000in}{-0.027778in}}{\pgfqpoint{0.000000in}{0.000000in}}{%
\pgfpathmoveto{\pgfqpoint{0.000000in}{0.000000in}}%
\pgfpathlineto{\pgfqpoint{0.000000in}{-0.027778in}}%
\pgfusepath{stroke,fill}%
}%
\begin{pgfscope}%
\pgfsys@transformshift{4.271739in}{4.050000in}%
\pgfsys@useobject{currentmarker}{}%
\end{pgfscope}%
\end{pgfscope}%
\begin{pgfscope}%
\pgfsetbuttcap%
\pgfsetroundjoin%
\definecolor{currentfill}{rgb}{0.000000,0.000000,0.000000}%
\pgfsetfillcolor{currentfill}%
\pgfsetlinewidth{0.501875pt}%
\definecolor{currentstroke}{rgb}{0.000000,0.000000,0.000000}%
\pgfsetstrokecolor{currentstroke}%
\pgfsetdash{}{0pt}%
\pgfsys@defobject{currentmarker}{\pgfqpoint{0.000000in}{0.000000in}}{\pgfqpoint{0.000000in}{0.027778in}}{%
\pgfpathmoveto{\pgfqpoint{0.000000in}{0.000000in}}%
\pgfpathlineto{\pgfqpoint{0.000000in}{0.027778in}}%
\pgfusepath{stroke,fill}%
}%
\begin{pgfscope}%
\pgfsys@transformshift{4.623913in}{0.450000in}%
\pgfsys@useobject{currentmarker}{}%
\end{pgfscope}%
\end{pgfscope}%
\begin{pgfscope}%
\pgfsetbuttcap%
\pgfsetroundjoin%
\definecolor{currentfill}{rgb}{0.000000,0.000000,0.000000}%
\pgfsetfillcolor{currentfill}%
\pgfsetlinewidth{0.501875pt}%
\definecolor{currentstroke}{rgb}{0.000000,0.000000,0.000000}%
\pgfsetstrokecolor{currentstroke}%
\pgfsetdash{}{0pt}%
\pgfsys@defobject{currentmarker}{\pgfqpoint{0.000000in}{-0.027778in}}{\pgfqpoint{0.000000in}{0.000000in}}{%
\pgfpathmoveto{\pgfqpoint{0.000000in}{0.000000in}}%
\pgfpathlineto{\pgfqpoint{0.000000in}{-0.027778in}}%
\pgfusepath{stroke,fill}%
}%
\begin{pgfscope}%
\pgfsys@transformshift{4.623913in}{4.050000in}%
\pgfsys@useobject{currentmarker}{}%
\end{pgfscope}%
\end{pgfscope}%
\begin{pgfscope}%
\pgftext[x=2.775000in,y=0.190049in,,top]{{\rmfamily\fontsize{11.000000}{13.200000}\selectfont stride [Bytes]}}%
\end{pgfscope}%
\begin{pgfscope}%
\pgfpathrectangle{\pgfqpoint{0.750000in}{0.450000in}}{\pgfqpoint{4.050000in}{3.600000in}} %
\pgfusepath{clip}%
\pgfsetbuttcap%
\pgfsetroundjoin%
\pgfsetlinewidth{0.501875pt}%
\definecolor{currentstroke}{rgb}{0.000000,0.000000,0.000000}%
\pgfsetstrokecolor{currentstroke}%
\pgfsetdash{{1.000000pt}{3.000000pt}}{0.000000pt}%
\pgfpathmoveto{\pgfqpoint{0.750000in}{0.450000in}}%
\pgfpathlineto{\pgfqpoint{4.800000in}{0.450000in}}%
\pgfusepath{stroke}%
\end{pgfscope}%
\begin{pgfscope}%
\pgfsetbuttcap%
\pgfsetroundjoin%
\definecolor{currentfill}{rgb}{0.000000,0.000000,0.000000}%
\pgfsetfillcolor{currentfill}%
\pgfsetlinewidth{0.501875pt}%
\definecolor{currentstroke}{rgb}{0.000000,0.000000,0.000000}%
\pgfsetstrokecolor{currentstroke}%
\pgfsetdash{}{0pt}%
\pgfsys@defobject{currentmarker}{\pgfqpoint{0.000000in}{0.000000in}}{\pgfqpoint{0.055556in}{0.000000in}}{%
\pgfpathmoveto{\pgfqpoint{0.000000in}{0.000000in}}%
\pgfpathlineto{\pgfqpoint{0.055556in}{0.000000in}}%
\pgfusepath{stroke,fill}%
}%
\begin{pgfscope}%
\pgfsys@transformshift{0.750000in}{0.450000in}%
\pgfsys@useobject{currentmarker}{}%
\end{pgfscope}%
\end{pgfscope}%
\begin{pgfscope}%
\pgfsetbuttcap%
\pgfsetroundjoin%
\definecolor{currentfill}{rgb}{0.000000,0.000000,0.000000}%
\pgfsetfillcolor{currentfill}%
\pgfsetlinewidth{0.501875pt}%
\definecolor{currentstroke}{rgb}{0.000000,0.000000,0.000000}%
\pgfsetstrokecolor{currentstroke}%
\pgfsetdash{}{0pt}%
\pgfsys@defobject{currentmarker}{\pgfqpoint{-0.055556in}{0.000000in}}{\pgfqpoint{0.000000in}{0.000000in}}{%
\pgfpathmoveto{\pgfqpoint{0.000000in}{0.000000in}}%
\pgfpathlineto{\pgfqpoint{-0.055556in}{0.000000in}}%
\pgfusepath{stroke,fill}%
}%
\begin{pgfscope}%
\pgfsys@transformshift{4.800000in}{0.450000in}%
\pgfsys@useobject{currentmarker}{}%
\end{pgfscope}%
\end{pgfscope}%
\begin{pgfscope}%
\pgftext[x=0.694444in,y=0.450000in,right,]{{\rmfamily\fontsize{11.000000}{13.200000}\selectfont \(\displaystyle -10\)}}%
\end{pgfscope}%
\begin{pgfscope}%
\pgfpathrectangle{\pgfqpoint{0.750000in}{0.450000in}}{\pgfqpoint{4.050000in}{3.600000in}} %
\pgfusepath{clip}%
\pgfsetbuttcap%
\pgfsetroundjoin%
\pgfsetlinewidth{0.501875pt}%
\definecolor{currentstroke}{rgb}{0.000000,0.000000,0.000000}%
\pgfsetstrokecolor{currentstroke}%
\pgfsetdash{{1.000000pt}{3.000000pt}}{0.000000pt}%
\pgfpathmoveto{\pgfqpoint{0.750000in}{0.850000in}}%
\pgfpathlineto{\pgfqpoint{4.800000in}{0.850000in}}%
\pgfusepath{stroke}%
\end{pgfscope}%
\begin{pgfscope}%
\pgfsetbuttcap%
\pgfsetroundjoin%
\definecolor{currentfill}{rgb}{0.000000,0.000000,0.000000}%
\pgfsetfillcolor{currentfill}%
\pgfsetlinewidth{0.501875pt}%
\definecolor{currentstroke}{rgb}{0.000000,0.000000,0.000000}%
\pgfsetstrokecolor{currentstroke}%
\pgfsetdash{}{0pt}%
\pgfsys@defobject{currentmarker}{\pgfqpoint{0.000000in}{0.000000in}}{\pgfqpoint{0.055556in}{0.000000in}}{%
\pgfpathmoveto{\pgfqpoint{0.000000in}{0.000000in}}%
\pgfpathlineto{\pgfqpoint{0.055556in}{0.000000in}}%
\pgfusepath{stroke,fill}%
}%
\begin{pgfscope}%
\pgfsys@transformshift{0.750000in}{0.850000in}%
\pgfsys@useobject{currentmarker}{}%
\end{pgfscope}%
\end{pgfscope}%
\begin{pgfscope}%
\pgfsetbuttcap%
\pgfsetroundjoin%
\definecolor{currentfill}{rgb}{0.000000,0.000000,0.000000}%
\pgfsetfillcolor{currentfill}%
\pgfsetlinewidth{0.501875pt}%
\definecolor{currentstroke}{rgb}{0.000000,0.000000,0.000000}%
\pgfsetstrokecolor{currentstroke}%
\pgfsetdash{}{0pt}%
\pgfsys@defobject{currentmarker}{\pgfqpoint{-0.055556in}{0.000000in}}{\pgfqpoint{0.000000in}{0.000000in}}{%
\pgfpathmoveto{\pgfqpoint{0.000000in}{0.000000in}}%
\pgfpathlineto{\pgfqpoint{-0.055556in}{0.000000in}}%
\pgfusepath{stroke,fill}%
}%
\begin{pgfscope}%
\pgfsys@transformshift{4.800000in}{0.850000in}%
\pgfsys@useobject{currentmarker}{}%
\end{pgfscope}%
\end{pgfscope}%
\begin{pgfscope}%
\pgftext[x=0.694444in,y=0.850000in,right,]{{\rmfamily\fontsize{11.000000}{13.200000}\selectfont \(\displaystyle 0\)}}%
\end{pgfscope}%
\begin{pgfscope}%
\pgfpathrectangle{\pgfqpoint{0.750000in}{0.450000in}}{\pgfqpoint{4.050000in}{3.600000in}} %
\pgfusepath{clip}%
\pgfsetbuttcap%
\pgfsetroundjoin%
\pgfsetlinewidth{0.501875pt}%
\definecolor{currentstroke}{rgb}{0.000000,0.000000,0.000000}%
\pgfsetstrokecolor{currentstroke}%
\pgfsetdash{{1.000000pt}{3.000000pt}}{0.000000pt}%
\pgfpathmoveto{\pgfqpoint{0.750000in}{1.250000in}}%
\pgfpathlineto{\pgfqpoint{4.800000in}{1.250000in}}%
\pgfusepath{stroke}%
\end{pgfscope}%
\begin{pgfscope}%
\pgfsetbuttcap%
\pgfsetroundjoin%
\definecolor{currentfill}{rgb}{0.000000,0.000000,0.000000}%
\pgfsetfillcolor{currentfill}%
\pgfsetlinewidth{0.501875pt}%
\definecolor{currentstroke}{rgb}{0.000000,0.000000,0.000000}%
\pgfsetstrokecolor{currentstroke}%
\pgfsetdash{}{0pt}%
\pgfsys@defobject{currentmarker}{\pgfqpoint{0.000000in}{0.000000in}}{\pgfqpoint{0.055556in}{0.000000in}}{%
\pgfpathmoveto{\pgfqpoint{0.000000in}{0.000000in}}%
\pgfpathlineto{\pgfqpoint{0.055556in}{0.000000in}}%
\pgfusepath{stroke,fill}%
}%
\begin{pgfscope}%
\pgfsys@transformshift{0.750000in}{1.250000in}%
\pgfsys@useobject{currentmarker}{}%
\end{pgfscope}%
\end{pgfscope}%
\begin{pgfscope}%
\pgfsetbuttcap%
\pgfsetroundjoin%
\definecolor{currentfill}{rgb}{0.000000,0.000000,0.000000}%
\pgfsetfillcolor{currentfill}%
\pgfsetlinewidth{0.501875pt}%
\definecolor{currentstroke}{rgb}{0.000000,0.000000,0.000000}%
\pgfsetstrokecolor{currentstroke}%
\pgfsetdash{}{0pt}%
\pgfsys@defobject{currentmarker}{\pgfqpoint{-0.055556in}{0.000000in}}{\pgfqpoint{0.000000in}{0.000000in}}{%
\pgfpathmoveto{\pgfqpoint{0.000000in}{0.000000in}}%
\pgfpathlineto{\pgfqpoint{-0.055556in}{0.000000in}}%
\pgfusepath{stroke,fill}%
}%
\begin{pgfscope}%
\pgfsys@transformshift{4.800000in}{1.250000in}%
\pgfsys@useobject{currentmarker}{}%
\end{pgfscope}%
\end{pgfscope}%
\begin{pgfscope}%
\pgftext[x=0.694444in,y=1.250000in,right,]{{\rmfamily\fontsize{11.000000}{13.200000}\selectfont \(\displaystyle 10\)}}%
\end{pgfscope}%
\begin{pgfscope}%
\pgfpathrectangle{\pgfqpoint{0.750000in}{0.450000in}}{\pgfqpoint{4.050000in}{3.600000in}} %
\pgfusepath{clip}%
\pgfsetbuttcap%
\pgfsetroundjoin%
\pgfsetlinewidth{0.501875pt}%
\definecolor{currentstroke}{rgb}{0.000000,0.000000,0.000000}%
\pgfsetstrokecolor{currentstroke}%
\pgfsetdash{{1.000000pt}{3.000000pt}}{0.000000pt}%
\pgfpathmoveto{\pgfqpoint{0.750000in}{1.650000in}}%
\pgfpathlineto{\pgfqpoint{4.800000in}{1.650000in}}%
\pgfusepath{stroke}%
\end{pgfscope}%
\begin{pgfscope}%
\pgfsetbuttcap%
\pgfsetroundjoin%
\definecolor{currentfill}{rgb}{0.000000,0.000000,0.000000}%
\pgfsetfillcolor{currentfill}%
\pgfsetlinewidth{0.501875pt}%
\definecolor{currentstroke}{rgb}{0.000000,0.000000,0.000000}%
\pgfsetstrokecolor{currentstroke}%
\pgfsetdash{}{0pt}%
\pgfsys@defobject{currentmarker}{\pgfqpoint{0.000000in}{0.000000in}}{\pgfqpoint{0.055556in}{0.000000in}}{%
\pgfpathmoveto{\pgfqpoint{0.000000in}{0.000000in}}%
\pgfpathlineto{\pgfqpoint{0.055556in}{0.000000in}}%
\pgfusepath{stroke,fill}%
}%
\begin{pgfscope}%
\pgfsys@transformshift{0.750000in}{1.650000in}%
\pgfsys@useobject{currentmarker}{}%
\end{pgfscope}%
\end{pgfscope}%
\begin{pgfscope}%
\pgfsetbuttcap%
\pgfsetroundjoin%
\definecolor{currentfill}{rgb}{0.000000,0.000000,0.000000}%
\pgfsetfillcolor{currentfill}%
\pgfsetlinewidth{0.501875pt}%
\definecolor{currentstroke}{rgb}{0.000000,0.000000,0.000000}%
\pgfsetstrokecolor{currentstroke}%
\pgfsetdash{}{0pt}%
\pgfsys@defobject{currentmarker}{\pgfqpoint{-0.055556in}{0.000000in}}{\pgfqpoint{0.000000in}{0.000000in}}{%
\pgfpathmoveto{\pgfqpoint{0.000000in}{0.000000in}}%
\pgfpathlineto{\pgfqpoint{-0.055556in}{0.000000in}}%
\pgfusepath{stroke,fill}%
}%
\begin{pgfscope}%
\pgfsys@transformshift{4.800000in}{1.650000in}%
\pgfsys@useobject{currentmarker}{}%
\end{pgfscope}%
\end{pgfscope}%
\begin{pgfscope}%
\pgftext[x=0.694444in,y=1.650000in,right,]{{\rmfamily\fontsize{11.000000}{13.200000}\selectfont \(\displaystyle 20\)}}%
\end{pgfscope}%
\begin{pgfscope}%
\pgfpathrectangle{\pgfqpoint{0.750000in}{0.450000in}}{\pgfqpoint{4.050000in}{3.600000in}} %
\pgfusepath{clip}%
\pgfsetbuttcap%
\pgfsetroundjoin%
\pgfsetlinewidth{0.501875pt}%
\definecolor{currentstroke}{rgb}{0.000000,0.000000,0.000000}%
\pgfsetstrokecolor{currentstroke}%
\pgfsetdash{{1.000000pt}{3.000000pt}}{0.000000pt}%
\pgfpathmoveto{\pgfqpoint{0.750000in}{2.050000in}}%
\pgfpathlineto{\pgfqpoint{4.800000in}{2.050000in}}%
\pgfusepath{stroke}%
\end{pgfscope}%
\begin{pgfscope}%
\pgfsetbuttcap%
\pgfsetroundjoin%
\definecolor{currentfill}{rgb}{0.000000,0.000000,0.000000}%
\pgfsetfillcolor{currentfill}%
\pgfsetlinewidth{0.501875pt}%
\definecolor{currentstroke}{rgb}{0.000000,0.000000,0.000000}%
\pgfsetstrokecolor{currentstroke}%
\pgfsetdash{}{0pt}%
\pgfsys@defobject{currentmarker}{\pgfqpoint{0.000000in}{0.000000in}}{\pgfqpoint{0.055556in}{0.000000in}}{%
\pgfpathmoveto{\pgfqpoint{0.000000in}{0.000000in}}%
\pgfpathlineto{\pgfqpoint{0.055556in}{0.000000in}}%
\pgfusepath{stroke,fill}%
}%
\begin{pgfscope}%
\pgfsys@transformshift{0.750000in}{2.050000in}%
\pgfsys@useobject{currentmarker}{}%
\end{pgfscope}%
\end{pgfscope}%
\begin{pgfscope}%
\pgfsetbuttcap%
\pgfsetroundjoin%
\definecolor{currentfill}{rgb}{0.000000,0.000000,0.000000}%
\pgfsetfillcolor{currentfill}%
\pgfsetlinewidth{0.501875pt}%
\definecolor{currentstroke}{rgb}{0.000000,0.000000,0.000000}%
\pgfsetstrokecolor{currentstroke}%
\pgfsetdash{}{0pt}%
\pgfsys@defobject{currentmarker}{\pgfqpoint{-0.055556in}{0.000000in}}{\pgfqpoint{0.000000in}{0.000000in}}{%
\pgfpathmoveto{\pgfqpoint{0.000000in}{0.000000in}}%
\pgfpathlineto{\pgfqpoint{-0.055556in}{0.000000in}}%
\pgfusepath{stroke,fill}%
}%
\begin{pgfscope}%
\pgfsys@transformshift{4.800000in}{2.050000in}%
\pgfsys@useobject{currentmarker}{}%
\end{pgfscope}%
\end{pgfscope}%
\begin{pgfscope}%
\pgftext[x=0.694444in,y=2.050000in,right,]{{\rmfamily\fontsize{11.000000}{13.200000}\selectfont \(\displaystyle 30\)}}%
\end{pgfscope}%
\begin{pgfscope}%
\pgfpathrectangle{\pgfqpoint{0.750000in}{0.450000in}}{\pgfqpoint{4.050000in}{3.600000in}} %
\pgfusepath{clip}%
\pgfsetbuttcap%
\pgfsetroundjoin%
\pgfsetlinewidth{0.501875pt}%
\definecolor{currentstroke}{rgb}{0.000000,0.000000,0.000000}%
\pgfsetstrokecolor{currentstroke}%
\pgfsetdash{{1.000000pt}{3.000000pt}}{0.000000pt}%
\pgfpathmoveto{\pgfqpoint{0.750000in}{2.450000in}}%
\pgfpathlineto{\pgfqpoint{4.800000in}{2.450000in}}%
\pgfusepath{stroke}%
\end{pgfscope}%
\begin{pgfscope}%
\pgfsetbuttcap%
\pgfsetroundjoin%
\definecolor{currentfill}{rgb}{0.000000,0.000000,0.000000}%
\pgfsetfillcolor{currentfill}%
\pgfsetlinewidth{0.501875pt}%
\definecolor{currentstroke}{rgb}{0.000000,0.000000,0.000000}%
\pgfsetstrokecolor{currentstroke}%
\pgfsetdash{}{0pt}%
\pgfsys@defobject{currentmarker}{\pgfqpoint{0.000000in}{0.000000in}}{\pgfqpoint{0.055556in}{0.000000in}}{%
\pgfpathmoveto{\pgfqpoint{0.000000in}{0.000000in}}%
\pgfpathlineto{\pgfqpoint{0.055556in}{0.000000in}}%
\pgfusepath{stroke,fill}%
}%
\begin{pgfscope}%
\pgfsys@transformshift{0.750000in}{2.450000in}%
\pgfsys@useobject{currentmarker}{}%
\end{pgfscope}%
\end{pgfscope}%
\begin{pgfscope}%
\pgfsetbuttcap%
\pgfsetroundjoin%
\definecolor{currentfill}{rgb}{0.000000,0.000000,0.000000}%
\pgfsetfillcolor{currentfill}%
\pgfsetlinewidth{0.501875pt}%
\definecolor{currentstroke}{rgb}{0.000000,0.000000,0.000000}%
\pgfsetstrokecolor{currentstroke}%
\pgfsetdash{}{0pt}%
\pgfsys@defobject{currentmarker}{\pgfqpoint{-0.055556in}{0.000000in}}{\pgfqpoint{0.000000in}{0.000000in}}{%
\pgfpathmoveto{\pgfqpoint{0.000000in}{0.000000in}}%
\pgfpathlineto{\pgfqpoint{-0.055556in}{0.000000in}}%
\pgfusepath{stroke,fill}%
}%
\begin{pgfscope}%
\pgfsys@transformshift{4.800000in}{2.450000in}%
\pgfsys@useobject{currentmarker}{}%
\end{pgfscope}%
\end{pgfscope}%
\begin{pgfscope}%
\pgftext[x=0.694444in,y=2.450000in,right,]{{\rmfamily\fontsize{11.000000}{13.200000}\selectfont \(\displaystyle 40\)}}%
\end{pgfscope}%
\begin{pgfscope}%
\pgfpathrectangle{\pgfqpoint{0.750000in}{0.450000in}}{\pgfqpoint{4.050000in}{3.600000in}} %
\pgfusepath{clip}%
\pgfsetbuttcap%
\pgfsetroundjoin%
\pgfsetlinewidth{0.501875pt}%
\definecolor{currentstroke}{rgb}{0.000000,0.000000,0.000000}%
\pgfsetstrokecolor{currentstroke}%
\pgfsetdash{{1.000000pt}{3.000000pt}}{0.000000pt}%
\pgfpathmoveto{\pgfqpoint{0.750000in}{2.850000in}}%
\pgfpathlineto{\pgfqpoint{4.800000in}{2.850000in}}%
\pgfusepath{stroke}%
\end{pgfscope}%
\begin{pgfscope}%
\pgfsetbuttcap%
\pgfsetroundjoin%
\definecolor{currentfill}{rgb}{0.000000,0.000000,0.000000}%
\pgfsetfillcolor{currentfill}%
\pgfsetlinewidth{0.501875pt}%
\definecolor{currentstroke}{rgb}{0.000000,0.000000,0.000000}%
\pgfsetstrokecolor{currentstroke}%
\pgfsetdash{}{0pt}%
\pgfsys@defobject{currentmarker}{\pgfqpoint{0.000000in}{0.000000in}}{\pgfqpoint{0.055556in}{0.000000in}}{%
\pgfpathmoveto{\pgfqpoint{0.000000in}{0.000000in}}%
\pgfpathlineto{\pgfqpoint{0.055556in}{0.000000in}}%
\pgfusepath{stroke,fill}%
}%
\begin{pgfscope}%
\pgfsys@transformshift{0.750000in}{2.850000in}%
\pgfsys@useobject{currentmarker}{}%
\end{pgfscope}%
\end{pgfscope}%
\begin{pgfscope}%
\pgfsetbuttcap%
\pgfsetroundjoin%
\definecolor{currentfill}{rgb}{0.000000,0.000000,0.000000}%
\pgfsetfillcolor{currentfill}%
\pgfsetlinewidth{0.501875pt}%
\definecolor{currentstroke}{rgb}{0.000000,0.000000,0.000000}%
\pgfsetstrokecolor{currentstroke}%
\pgfsetdash{}{0pt}%
\pgfsys@defobject{currentmarker}{\pgfqpoint{-0.055556in}{0.000000in}}{\pgfqpoint{0.000000in}{0.000000in}}{%
\pgfpathmoveto{\pgfqpoint{0.000000in}{0.000000in}}%
\pgfpathlineto{\pgfqpoint{-0.055556in}{0.000000in}}%
\pgfusepath{stroke,fill}%
}%
\begin{pgfscope}%
\pgfsys@transformshift{4.800000in}{2.850000in}%
\pgfsys@useobject{currentmarker}{}%
\end{pgfscope}%
\end{pgfscope}%
\begin{pgfscope}%
\pgftext[x=0.694444in,y=2.850000in,right,]{{\rmfamily\fontsize{11.000000}{13.200000}\selectfont \(\displaystyle 50\)}}%
\end{pgfscope}%
\begin{pgfscope}%
\pgfpathrectangle{\pgfqpoint{0.750000in}{0.450000in}}{\pgfqpoint{4.050000in}{3.600000in}} %
\pgfusepath{clip}%
\pgfsetbuttcap%
\pgfsetroundjoin%
\pgfsetlinewidth{0.501875pt}%
\definecolor{currentstroke}{rgb}{0.000000,0.000000,0.000000}%
\pgfsetstrokecolor{currentstroke}%
\pgfsetdash{{1.000000pt}{3.000000pt}}{0.000000pt}%
\pgfpathmoveto{\pgfqpoint{0.750000in}{3.250000in}}%
\pgfpathlineto{\pgfqpoint{4.800000in}{3.250000in}}%
\pgfusepath{stroke}%
\end{pgfscope}%
\begin{pgfscope}%
\pgfsetbuttcap%
\pgfsetroundjoin%
\definecolor{currentfill}{rgb}{0.000000,0.000000,0.000000}%
\pgfsetfillcolor{currentfill}%
\pgfsetlinewidth{0.501875pt}%
\definecolor{currentstroke}{rgb}{0.000000,0.000000,0.000000}%
\pgfsetstrokecolor{currentstroke}%
\pgfsetdash{}{0pt}%
\pgfsys@defobject{currentmarker}{\pgfqpoint{0.000000in}{0.000000in}}{\pgfqpoint{0.055556in}{0.000000in}}{%
\pgfpathmoveto{\pgfqpoint{0.000000in}{0.000000in}}%
\pgfpathlineto{\pgfqpoint{0.055556in}{0.000000in}}%
\pgfusepath{stroke,fill}%
}%
\begin{pgfscope}%
\pgfsys@transformshift{0.750000in}{3.250000in}%
\pgfsys@useobject{currentmarker}{}%
\end{pgfscope}%
\end{pgfscope}%
\begin{pgfscope}%
\pgfsetbuttcap%
\pgfsetroundjoin%
\definecolor{currentfill}{rgb}{0.000000,0.000000,0.000000}%
\pgfsetfillcolor{currentfill}%
\pgfsetlinewidth{0.501875pt}%
\definecolor{currentstroke}{rgb}{0.000000,0.000000,0.000000}%
\pgfsetstrokecolor{currentstroke}%
\pgfsetdash{}{0pt}%
\pgfsys@defobject{currentmarker}{\pgfqpoint{-0.055556in}{0.000000in}}{\pgfqpoint{0.000000in}{0.000000in}}{%
\pgfpathmoveto{\pgfqpoint{0.000000in}{0.000000in}}%
\pgfpathlineto{\pgfqpoint{-0.055556in}{0.000000in}}%
\pgfusepath{stroke,fill}%
}%
\begin{pgfscope}%
\pgfsys@transformshift{4.800000in}{3.250000in}%
\pgfsys@useobject{currentmarker}{}%
\end{pgfscope}%
\end{pgfscope}%
\begin{pgfscope}%
\pgftext[x=0.694444in,y=3.250000in,right,]{{\rmfamily\fontsize{11.000000}{13.200000}\selectfont \(\displaystyle 60\)}}%
\end{pgfscope}%
\begin{pgfscope}%
\pgfpathrectangle{\pgfqpoint{0.750000in}{0.450000in}}{\pgfqpoint{4.050000in}{3.600000in}} %
\pgfusepath{clip}%
\pgfsetbuttcap%
\pgfsetroundjoin%
\pgfsetlinewidth{0.501875pt}%
\definecolor{currentstroke}{rgb}{0.000000,0.000000,0.000000}%
\pgfsetstrokecolor{currentstroke}%
\pgfsetdash{{1.000000pt}{3.000000pt}}{0.000000pt}%
\pgfpathmoveto{\pgfqpoint{0.750000in}{3.650000in}}%
\pgfpathlineto{\pgfqpoint{4.800000in}{3.650000in}}%
\pgfusepath{stroke}%
\end{pgfscope}%
\begin{pgfscope}%
\pgfsetbuttcap%
\pgfsetroundjoin%
\definecolor{currentfill}{rgb}{0.000000,0.000000,0.000000}%
\pgfsetfillcolor{currentfill}%
\pgfsetlinewidth{0.501875pt}%
\definecolor{currentstroke}{rgb}{0.000000,0.000000,0.000000}%
\pgfsetstrokecolor{currentstroke}%
\pgfsetdash{}{0pt}%
\pgfsys@defobject{currentmarker}{\pgfqpoint{0.000000in}{0.000000in}}{\pgfqpoint{0.055556in}{0.000000in}}{%
\pgfpathmoveto{\pgfqpoint{0.000000in}{0.000000in}}%
\pgfpathlineto{\pgfqpoint{0.055556in}{0.000000in}}%
\pgfusepath{stroke,fill}%
}%
\begin{pgfscope}%
\pgfsys@transformshift{0.750000in}{3.650000in}%
\pgfsys@useobject{currentmarker}{}%
\end{pgfscope}%
\end{pgfscope}%
\begin{pgfscope}%
\pgfsetbuttcap%
\pgfsetroundjoin%
\definecolor{currentfill}{rgb}{0.000000,0.000000,0.000000}%
\pgfsetfillcolor{currentfill}%
\pgfsetlinewidth{0.501875pt}%
\definecolor{currentstroke}{rgb}{0.000000,0.000000,0.000000}%
\pgfsetstrokecolor{currentstroke}%
\pgfsetdash{}{0pt}%
\pgfsys@defobject{currentmarker}{\pgfqpoint{-0.055556in}{0.000000in}}{\pgfqpoint{0.000000in}{0.000000in}}{%
\pgfpathmoveto{\pgfqpoint{0.000000in}{0.000000in}}%
\pgfpathlineto{\pgfqpoint{-0.055556in}{0.000000in}}%
\pgfusepath{stroke,fill}%
}%
\begin{pgfscope}%
\pgfsys@transformshift{4.800000in}{3.650000in}%
\pgfsys@useobject{currentmarker}{}%
\end{pgfscope}%
\end{pgfscope}%
\begin{pgfscope}%
\pgftext[x=0.694444in,y=3.650000in,right,]{{\rmfamily\fontsize{11.000000}{13.200000}\selectfont \(\displaystyle 70\)}}%
\end{pgfscope}%
\begin{pgfscope}%
\pgfpathrectangle{\pgfqpoint{0.750000in}{0.450000in}}{\pgfqpoint{4.050000in}{3.600000in}} %
\pgfusepath{clip}%
\pgfsetbuttcap%
\pgfsetroundjoin%
\pgfsetlinewidth{0.501875pt}%
\definecolor{currentstroke}{rgb}{0.000000,0.000000,0.000000}%
\pgfsetstrokecolor{currentstroke}%
\pgfsetdash{{1.000000pt}{3.000000pt}}{0.000000pt}%
\pgfpathmoveto{\pgfqpoint{0.750000in}{4.050000in}}%
\pgfpathlineto{\pgfqpoint{4.800000in}{4.050000in}}%
\pgfusepath{stroke}%
\end{pgfscope}%
\begin{pgfscope}%
\pgfsetbuttcap%
\pgfsetroundjoin%
\definecolor{currentfill}{rgb}{0.000000,0.000000,0.000000}%
\pgfsetfillcolor{currentfill}%
\pgfsetlinewidth{0.501875pt}%
\definecolor{currentstroke}{rgb}{0.000000,0.000000,0.000000}%
\pgfsetstrokecolor{currentstroke}%
\pgfsetdash{}{0pt}%
\pgfsys@defobject{currentmarker}{\pgfqpoint{0.000000in}{0.000000in}}{\pgfqpoint{0.055556in}{0.000000in}}{%
\pgfpathmoveto{\pgfqpoint{0.000000in}{0.000000in}}%
\pgfpathlineto{\pgfqpoint{0.055556in}{0.000000in}}%
\pgfusepath{stroke,fill}%
}%
\begin{pgfscope}%
\pgfsys@transformshift{0.750000in}{4.050000in}%
\pgfsys@useobject{currentmarker}{}%
\end{pgfscope}%
\end{pgfscope}%
\begin{pgfscope}%
\pgfsetbuttcap%
\pgfsetroundjoin%
\definecolor{currentfill}{rgb}{0.000000,0.000000,0.000000}%
\pgfsetfillcolor{currentfill}%
\pgfsetlinewidth{0.501875pt}%
\definecolor{currentstroke}{rgb}{0.000000,0.000000,0.000000}%
\pgfsetstrokecolor{currentstroke}%
\pgfsetdash{}{0pt}%
\pgfsys@defobject{currentmarker}{\pgfqpoint{-0.055556in}{0.000000in}}{\pgfqpoint{0.000000in}{0.000000in}}{%
\pgfpathmoveto{\pgfqpoint{0.000000in}{0.000000in}}%
\pgfpathlineto{\pgfqpoint{-0.055556in}{0.000000in}}%
\pgfusepath{stroke,fill}%
}%
\begin{pgfscope}%
\pgfsys@transformshift{4.800000in}{4.050000in}%
\pgfsys@useobject{currentmarker}{}%
\end{pgfscope}%
\end{pgfscope}%
\begin{pgfscope}%
\pgftext[x=0.694444in,y=4.050000in,right,]{{\rmfamily\fontsize{11.000000}{13.200000}\selectfont \(\displaystyle 80\)}}%
\end{pgfscope}%
\begin{pgfscope}%
\pgftext[x=0.356578in,y=2.250000in,,bottom,rotate=90.000000]{{\rmfamily\fontsize{11.000000}{13.200000}\selectfont loadtime [ns]}}%
\end{pgfscope}%
\begin{pgfscope}%
\pgfsetbuttcap%
\pgfsetroundjoin%
\pgfsetlinewidth{1.003750pt}%
\definecolor{currentstroke}{rgb}{0.000000,0.000000,0.000000}%
\pgfsetstrokecolor{currentstroke}%
\pgfsetdash{}{0pt}%
\pgfpathmoveto{\pgfqpoint{0.750000in}{4.050000in}}%
\pgfpathlineto{\pgfqpoint{4.800000in}{4.050000in}}%
\pgfusepath{stroke}%
\end{pgfscope}%
\begin{pgfscope}%
\pgfsetbuttcap%
\pgfsetroundjoin%
\pgfsetlinewidth{1.003750pt}%
\definecolor{currentstroke}{rgb}{0.000000,0.000000,0.000000}%
\pgfsetstrokecolor{currentstroke}%
\pgfsetdash{}{0pt}%
\pgfpathmoveto{\pgfqpoint{4.800000in}{0.450000in}}%
\pgfpathlineto{\pgfqpoint{4.800000in}{4.050000in}}%
\pgfusepath{stroke}%
\end{pgfscope}%
\begin{pgfscope}%
\pgfsetbuttcap%
\pgfsetroundjoin%
\pgfsetlinewidth{1.003750pt}%
\definecolor{currentstroke}{rgb}{0.000000,0.000000,0.000000}%
\pgfsetstrokecolor{currentstroke}%
\pgfsetdash{}{0pt}%
\pgfpathmoveto{\pgfqpoint{0.750000in}{0.450000in}}%
\pgfpathlineto{\pgfqpoint{4.800000in}{0.450000in}}%
\pgfusepath{stroke}%
\end{pgfscope}%
\begin{pgfscope}%
\pgfsetbuttcap%
\pgfsetroundjoin%
\pgfsetlinewidth{1.003750pt}%
\definecolor{currentstroke}{rgb}{0.000000,0.000000,0.000000}%
\pgfsetstrokecolor{currentstroke}%
\pgfsetdash{}{0pt}%
\pgfpathmoveto{\pgfqpoint{0.750000in}{0.450000in}}%
\pgfpathlineto{\pgfqpoint{0.750000in}{4.050000in}}%
\pgfusepath{stroke}%
\end{pgfscope}%
\begin{pgfscope}%
\pgfsetbuttcap%
\pgfsetroundjoin%
\definecolor{currentfill}{rgb}{1.000000,1.000000,1.000000}%
\pgfsetfillcolor{currentfill}%
\pgfsetlinewidth{1.003750pt}%
\definecolor{currentstroke}{rgb}{0.000000,0.000000,0.000000}%
\pgfsetstrokecolor{currentstroke}%
\pgfsetdash{}{0pt}%
\pgfpathmoveto{\pgfqpoint{5.002500in}{0.091848in}}%
\pgfpathlineto{\pgfqpoint{6.393204in}{0.091848in}}%
\pgfpathlineto{\pgfqpoint{6.393204in}{4.410000in}}%
\pgfpathlineto{\pgfqpoint{5.002500in}{4.410000in}}%
\pgfpathlineto{\pgfqpoint{5.002500in}{0.091848in}}%
\pgfpathclose%
\pgfusepath{stroke,fill}%
\end{pgfscope}%
\begin{pgfscope}%
\pgfsetrectcap%
\pgfsetroundjoin%
\pgfsetlinewidth{2.007500pt}%
\definecolor{currentstroke}{rgb}{0.000000,0.000000,0.000000}%
\pgfsetstrokecolor{currentstroke}%
\pgfsetdash{}{0pt}%
\pgfpathmoveto{\pgfqpoint{5.130833in}{4.243938in}}%
\pgfpathlineto{\pgfqpoint{5.387500in}{4.243938in}}%
\pgfusepath{stroke}%
\end{pgfscope}%
\begin{pgfscope}%
\pgftext[x=5.589167in,y=4.179771in,left,base]{{\rmfamily\fontsize{13.200000}{15.840000}\selectfont \(\displaystyle 2^{12}\) Bytes}}%
\end{pgfscope}%
\begin{pgfscope}%
\pgfsetrectcap%
\pgfsetroundjoin%
\pgfsetlinewidth{2.007500pt}%
\definecolor{currentstroke}{rgb}{0.066667,0.066667,0.066667}%
\pgfsetstrokecolor{currentstroke}%
\pgfsetdash{}{0pt}%
\pgfpathmoveto{\pgfqpoint{5.130833in}{3.959728in}}%
\pgfpathlineto{\pgfqpoint{5.387500in}{3.959728in}}%
\pgfusepath{stroke}%
\end{pgfscope}%
\begin{pgfscope}%
\pgftext[x=5.589167in,y=3.895561in,left,base]{{\rmfamily\fontsize{13.200000}{15.840000}\selectfont \(\displaystyle 2^{13}\) Bytes}}%
\end{pgfscope}%
\begin{pgfscope}%
\pgfsetrectcap%
\pgfsetroundjoin%
\pgfsetlinewidth{2.007500pt}%
\definecolor{currentstroke}{rgb}{0.133333,0.133333,0.133333}%
\pgfsetstrokecolor{currentstroke}%
\pgfsetdash{}{0pt}%
\pgfpathmoveto{\pgfqpoint{5.130833in}{3.675518in}}%
\pgfpathlineto{\pgfqpoint{5.387500in}{3.675518in}}%
\pgfusepath{stroke}%
\end{pgfscope}%
\begin{pgfscope}%
\pgftext[x=5.589167in,y=3.611351in,left,base]{{\rmfamily\fontsize{13.200000}{15.840000}\selectfont \(\displaystyle 2^{14}\) Bytes}}%
\end{pgfscope}%
\begin{pgfscope}%
\pgfsetrectcap%
\pgfsetroundjoin%
\pgfsetlinewidth{2.007500pt}%
\definecolor{currentstroke}{rgb}{0.200000,0.200000,0.200000}%
\pgfsetstrokecolor{currentstroke}%
\pgfsetdash{}{0pt}%
\pgfpathmoveto{\pgfqpoint{5.130833in}{3.391307in}}%
\pgfpathlineto{\pgfqpoint{5.387500in}{3.391307in}}%
\pgfusepath{stroke}%
\end{pgfscope}%
\begin{pgfscope}%
\pgftext[x=5.589167in,y=3.327141in,left,base]{{\rmfamily\fontsize{13.200000}{15.840000}\selectfont \(\displaystyle 2^{15}\) Bytes}}%
\end{pgfscope}%
\begin{pgfscope}%
\pgfsetrectcap%
\pgfsetroundjoin%
\pgfsetlinewidth{2.007500pt}%
\definecolor{currentstroke}{rgb}{0.266667,0.266667,0.266667}%
\pgfsetstrokecolor{currentstroke}%
\pgfsetdash{}{0pt}%
\pgfpathmoveto{\pgfqpoint{5.130833in}{3.107097in}}%
\pgfpathlineto{\pgfqpoint{5.387500in}{3.107097in}}%
\pgfusepath{stroke}%
\end{pgfscope}%
\begin{pgfscope}%
\pgftext[x=5.589167in,y=3.042931in,left,base]{{\rmfamily\fontsize{13.200000}{15.840000}\selectfont \(\displaystyle 2^{16}\) Bytes}}%
\end{pgfscope}%
\begin{pgfscope}%
\pgfsetrectcap%
\pgfsetroundjoin%
\pgfsetlinewidth{2.007500pt}%
\definecolor{currentstroke}{rgb}{0.333333,0.333333,0.333333}%
\pgfsetstrokecolor{currentstroke}%
\pgfsetdash{}{0pt}%
\pgfpathmoveto{\pgfqpoint{5.130833in}{2.822887in}}%
\pgfpathlineto{\pgfqpoint{5.387500in}{2.822887in}}%
\pgfusepath{stroke}%
\end{pgfscope}%
\begin{pgfscope}%
\pgftext[x=5.589167in,y=2.758720in,left,base]{{\rmfamily\fontsize{13.200000}{15.840000}\selectfont \(\displaystyle 2^{17}\) Bytes}}%
\end{pgfscope}%
\begin{pgfscope}%
\pgfsetrectcap%
\pgfsetroundjoin%
\pgfsetlinewidth{2.007500pt}%
\definecolor{currentstroke}{rgb}{0.400000,0.400000,0.400000}%
\pgfsetstrokecolor{currentstroke}%
\pgfsetdash{}{0pt}%
\pgfpathmoveto{\pgfqpoint{5.130833in}{2.538677in}}%
\pgfpathlineto{\pgfqpoint{5.387500in}{2.538677in}}%
\pgfusepath{stroke}%
\end{pgfscope}%
\begin{pgfscope}%
\pgftext[x=5.589167in,y=2.474510in,left,base]{{\rmfamily\fontsize{13.200000}{15.840000}\selectfont \(\displaystyle 2^{18}\) Bytes}}%
\end{pgfscope}%
\begin{pgfscope}%
\pgfsetrectcap%
\pgfsetroundjoin%
\pgfsetlinewidth{2.007500pt}%
\definecolor{currentstroke}{rgb}{0.466667,0.466667,0.466667}%
\pgfsetstrokecolor{currentstroke}%
\pgfsetdash{}{0pt}%
\pgfpathmoveto{\pgfqpoint{5.130833in}{2.254467in}}%
\pgfpathlineto{\pgfqpoint{5.387500in}{2.254467in}}%
\pgfusepath{stroke}%
\end{pgfscope}%
\begin{pgfscope}%
\pgftext[x=5.589167in,y=2.190300in,left,base]{{\rmfamily\fontsize{13.200000}{15.840000}\selectfont \(\displaystyle 2^{19}\) Bytes}}%
\end{pgfscope}%
\begin{pgfscope}%
\pgfsetrectcap%
\pgfsetroundjoin%
\pgfsetlinewidth{2.007500pt}%
\definecolor{currentstroke}{rgb}{0.533333,0.533333,0.533333}%
\pgfsetstrokecolor{currentstroke}%
\pgfsetdash{}{0pt}%
\pgfpathmoveto{\pgfqpoint{5.130833in}{1.970257in}}%
\pgfpathlineto{\pgfqpoint{5.387500in}{1.970257in}}%
\pgfusepath{stroke}%
\end{pgfscope}%
\begin{pgfscope}%
\pgftext[x=5.589167in,y=1.906090in,left,base]{{\rmfamily\fontsize{13.200000}{15.840000}\selectfont \(\displaystyle 2^{20}\) Bytes}}%
\end{pgfscope}%
\begin{pgfscope}%
\pgfsetrectcap%
\pgfsetroundjoin%
\pgfsetlinewidth{2.007500pt}%
\definecolor{currentstroke}{rgb}{0.600000,0.600000,0.600000}%
\pgfsetstrokecolor{currentstroke}%
\pgfsetdash{}{0pt}%
\pgfpathmoveto{\pgfqpoint{5.130833in}{1.686047in}}%
\pgfpathlineto{\pgfqpoint{5.387500in}{1.686047in}}%
\pgfusepath{stroke}%
\end{pgfscope}%
\begin{pgfscope}%
\pgftext[x=5.589167in,y=1.621880in,left,base]{{\rmfamily\fontsize{13.200000}{15.840000}\selectfont \(\displaystyle 2^{21}\) Bytes}}%
\end{pgfscope}%
\begin{pgfscope}%
\pgfsetrectcap%
\pgfsetroundjoin%
\pgfsetlinewidth{2.007500pt}%
\definecolor{currentstroke}{rgb}{0.666667,0.666667,0.666667}%
\pgfsetstrokecolor{currentstroke}%
\pgfsetdash{}{0pt}%
\pgfpathmoveto{\pgfqpoint{5.130833in}{1.401836in}}%
\pgfpathlineto{\pgfqpoint{5.387500in}{1.401836in}}%
\pgfusepath{stroke}%
\end{pgfscope}%
\begin{pgfscope}%
\pgftext[x=5.589167in,y=1.337670in,left,base]{{\rmfamily\fontsize{13.200000}{15.840000}\selectfont \(\displaystyle 2^{22}\) Bytes}}%
\end{pgfscope}%
\begin{pgfscope}%
\pgfsetrectcap%
\pgfsetroundjoin%
\pgfsetlinewidth{2.007500pt}%
\definecolor{currentstroke}{rgb}{0.733333,0.733333,0.733333}%
\pgfsetstrokecolor{currentstroke}%
\pgfsetdash{}{0pt}%
\pgfpathmoveto{\pgfqpoint{5.130833in}{1.117626in}}%
\pgfpathlineto{\pgfqpoint{5.387500in}{1.117626in}}%
\pgfusepath{stroke}%
\end{pgfscope}%
\begin{pgfscope}%
\pgftext[x=5.589167in,y=1.053460in,left,base]{{\rmfamily\fontsize{13.200000}{15.840000}\selectfont \(\displaystyle 2^{23}\) Bytes}}%
\end{pgfscope}%
\begin{pgfscope}%
\pgfsetrectcap%
\pgfsetroundjoin%
\pgfsetlinewidth{2.007500pt}%
\definecolor{currentstroke}{rgb}{0.800000,0.800000,0.800000}%
\pgfsetstrokecolor{currentstroke}%
\pgfsetdash{}{0pt}%
\pgfpathmoveto{\pgfqpoint{5.130833in}{0.833416in}}%
\pgfpathlineto{\pgfqpoint{5.387500in}{0.833416in}}%
\pgfusepath{stroke}%
\end{pgfscope}%
\begin{pgfscope}%
\pgftext[x=5.589167in,y=0.769249in,left,base]{{\rmfamily\fontsize{13.200000}{15.840000}\selectfont \(\displaystyle 2^{24}\) Bytes}}%
\end{pgfscope}%
\begin{pgfscope}%
\pgfsetrectcap%
\pgfsetroundjoin%
\pgfsetlinewidth{2.007500pt}%
\definecolor{currentstroke}{rgb}{0.866667,0.866667,0.866667}%
\pgfsetstrokecolor{currentstroke}%
\pgfsetdash{}{0pt}%
\pgfpathmoveto{\pgfqpoint{5.130833in}{0.549206in}}%
\pgfpathlineto{\pgfqpoint{5.387500in}{0.549206in}}%
\pgfusepath{stroke}%
\end{pgfscope}%
\begin{pgfscope}%
\pgftext[x=5.589167in,y=0.485039in,left,base]{{\rmfamily\fontsize{13.200000}{15.840000}\selectfont \(\displaystyle 2^{25}\) Bytes}}%
\end{pgfscope}%
\begin{pgfscope}%
\pgfsetrectcap%
\pgfsetroundjoin%
\pgfsetlinewidth{2.007500pt}%
\definecolor{currentstroke}{rgb}{0.933333,0.933333,0.933333}%
\pgfsetstrokecolor{currentstroke}%
\pgfsetdash{}{0pt}%
\pgfpathmoveto{\pgfqpoint{5.130833in}{0.264996in}}%
\pgfpathlineto{\pgfqpoint{5.387500in}{0.264996in}}%
\pgfusepath{stroke}%
\end{pgfscope}%
\begin{pgfscope}%
\pgftext[x=5.589167in,y=0.200829in,left,base]{{\rmfamily\fontsize{13.200000}{15.840000}\selectfont \(\displaystyle 2^{26}\) Bytes}}%
\end{pgfscope}%
\end{pgfpicture}%
\makeatother%
\endgroup%

	\label{pointer_chasing_local}
	\caption{Local pointer chasing Benchmark on home computer. The lines' colors indicate the
	         used arrays' sizes in Bytes. The system's processor was an Intel i5-760 with 
			 4 cores at 2.8 GHz and a L3 cache size of 8792 kB. Due to rounding errors
			 some loading times are negative.}
\end{figure}
\begin{figure}
	%% Creator: Matplotlib, PGF backend
%%
%% To include the figure in your LaTeX document, write
%%   \input{<filename>.pgf}
%%
%% Make sure the required packages are loaded in your preamble
%%   \usepackage{pgf}
%%
%% Figures using additional raster images can only be included by \input if
%% they are in the same directory as the main LaTeX file. For loading figures
%% from other directories you can use the `import` package
%%   \usepackage{import}
%% and then include the figures with
%%   \import{<path to file>}{<filename>.pgf}
%%
%% Matplotlib used the following preamble
%%   \usepackage[T1]{fontenc}
%%   \usepackage{lmodern}
%%
\begingroup%
\makeatletter%
\begin{pgfpicture}%
\pgfpathrectangle{\pgfpointorigin}{\pgfqpoint{6.000000in}{4.500000in}}%
\pgfusepath{use as bounding box}%
\begin{pgfscope}%
\pgfsetbuttcap%
\pgfsetroundjoin%
\definecolor{currentfill}{rgb}{1.000000,1.000000,1.000000}%
\pgfsetfillcolor{currentfill}%
\pgfsetlinewidth{0.000000pt}%
\definecolor{currentstroke}{rgb}{1.000000,1.000000,1.000000}%
\pgfsetstrokecolor{currentstroke}%
\pgfsetdash{}{0pt}%
\pgfpathmoveto{\pgfqpoint{0.000000in}{0.000000in}}%
\pgfpathlineto{\pgfqpoint{6.000000in}{0.000000in}}%
\pgfpathlineto{\pgfqpoint{6.000000in}{4.500000in}}%
\pgfpathlineto{\pgfqpoint{0.000000in}{4.500000in}}%
\pgfpathclose%
\pgfusepath{fill}%
\end{pgfscope}%
\begin{pgfscope}%
\pgfsetbuttcap%
\pgfsetroundjoin%
\definecolor{currentfill}{rgb}{1.000000,1.000000,1.000000}%
\pgfsetfillcolor{currentfill}%
\pgfsetlinewidth{0.000000pt}%
\definecolor{currentstroke}{rgb}{0.000000,0.000000,0.000000}%
\pgfsetstrokecolor{currentstroke}%
\pgfsetstrokeopacity{0.000000}%
\pgfsetdash{}{0pt}%
\pgfpathmoveto{\pgfqpoint{0.750000in}{0.450000in}}%
\pgfpathlineto{\pgfqpoint{4.800000in}{0.450000in}}%
\pgfpathlineto{\pgfqpoint{4.800000in}{4.050000in}}%
\pgfpathlineto{\pgfqpoint{0.750000in}{4.050000in}}%
\pgfpathclose%
\pgfusepath{fill}%
\end{pgfscope}%
\begin{pgfscope}%
\pgfpathrectangle{\pgfqpoint{0.750000in}{0.450000in}}{\pgfqpoint{4.050000in}{3.600000in}} %
\pgfusepath{clip}%
\pgfsetrectcap%
\pgfsetroundjoin%
\pgfsetlinewidth{2.007500pt}%
\definecolor{currentstroke}{rgb}{0.000000,0.000000,0.000000}%
\pgfsetstrokecolor{currentstroke}%
\pgfsetdash{}{0pt}%
\pgfpathmoveto{\pgfqpoint{0.750000in}{0.745698in}}%
\pgfpathlineto{\pgfqpoint{0.975000in}{0.746914in}}%
\pgfpathlineto{\pgfqpoint{1.200000in}{0.746811in}}%
\pgfpathlineto{\pgfqpoint{1.425000in}{0.746915in}}%
\pgfpathlineto{\pgfqpoint{1.650000in}{0.711768in}}%
\pgfpathlineto{\pgfqpoint{1.875000in}{0.715712in}}%
\pgfpathlineto{\pgfqpoint{2.100000in}{0.606345in}}%
\pgfpathlineto{\pgfqpoint{2.325000in}{0.801778in}}%
\pgfpathlineto{\pgfqpoint{2.550000in}{0.881904in}}%
\pgfpathlineto{\pgfqpoint{2.775000in}{0.810819in}}%
\pgfusepath{stroke}%
\end{pgfscope}%
\begin{pgfscope}%
\pgfpathrectangle{\pgfqpoint{0.750000in}{0.450000in}}{\pgfqpoint{4.050000in}{3.600000in}} %
\pgfusepath{clip}%
\pgfsetrectcap%
\pgfsetroundjoin%
\pgfsetlinewidth{2.007500pt}%
\definecolor{currentstroke}{rgb}{0.090909,0.090909,0.090909}%
\pgfsetstrokecolor{currentstroke}%
\pgfsetdash{}{0pt}%
\pgfpathmoveto{\pgfqpoint{0.750000in}{0.742565in}}%
\pgfpathlineto{\pgfqpoint{0.975000in}{0.743289in}}%
\pgfpathlineto{\pgfqpoint{1.200000in}{0.743295in}}%
\pgfpathlineto{\pgfqpoint{1.425000in}{0.742775in}}%
\pgfpathlineto{\pgfqpoint{1.650000in}{0.743913in}}%
\pgfpathlineto{\pgfqpoint{1.875000in}{0.709140in}}%
\pgfpathlineto{\pgfqpoint{2.100000in}{0.713081in}}%
\pgfpathlineto{\pgfqpoint{2.325000in}{0.604840in}}%
\pgfpathlineto{\pgfqpoint{2.550000in}{0.798100in}}%
\pgfpathlineto{\pgfqpoint{2.775000in}{0.877926in}}%
\pgfpathlineto{\pgfqpoint{3.000000in}{0.807536in}}%
\pgfusepath{stroke}%
\end{pgfscope}%
\begin{pgfscope}%
\pgfpathrectangle{\pgfqpoint{0.750000in}{0.450000in}}{\pgfqpoint{4.050000in}{3.600000in}} %
\pgfusepath{clip}%
\pgfsetrectcap%
\pgfsetroundjoin%
\pgfsetlinewidth{2.007500pt}%
\definecolor{currentstroke}{rgb}{0.181818,0.181818,0.181818}%
\pgfsetstrokecolor{currentstroke}%
\pgfsetdash{}{0pt}%
\pgfpathmoveto{\pgfqpoint{0.750000in}{0.741171in}}%
\pgfpathlineto{\pgfqpoint{0.975000in}{0.741528in}}%
\pgfpathlineto{\pgfqpoint{1.200000in}{0.741532in}}%
\pgfpathlineto{\pgfqpoint{1.425000in}{0.740723in}}%
\pgfpathlineto{\pgfqpoint{1.650000in}{0.741305in}}%
\pgfpathlineto{\pgfqpoint{1.875000in}{0.742412in}}%
\pgfpathlineto{\pgfqpoint{2.100000in}{0.707823in}}%
\pgfpathlineto{\pgfqpoint{2.325000in}{0.711766in}}%
\pgfpathlineto{\pgfqpoint{2.550000in}{0.604090in}}%
\pgfpathlineto{\pgfqpoint{2.775000in}{0.796260in}}%
\pgfpathlineto{\pgfqpoint{3.000000in}{0.875907in}}%
\pgfpathlineto{\pgfqpoint{3.225000in}{0.805892in}}%
\pgfusepath{stroke}%
\end{pgfscope}%
\begin{pgfscope}%
\pgfpathrectangle{\pgfqpoint{0.750000in}{0.450000in}}{\pgfqpoint{4.050000in}{3.600000in}} %
\pgfusepath{clip}%
\pgfsetrectcap%
\pgfsetroundjoin%
\pgfsetlinewidth{2.007500pt}%
\definecolor{currentstroke}{rgb}{0.272727,0.272727,0.272727}%
\pgfsetstrokecolor{currentstroke}%
\pgfsetdash{}{0pt}%
\pgfpathmoveto{\pgfqpoint{0.750000in}{0.776771in}}%
\pgfpathlineto{\pgfqpoint{0.975000in}{0.740658in}}%
\pgfpathlineto{\pgfqpoint{1.200000in}{0.740642in}}%
\pgfpathlineto{\pgfqpoint{1.425000in}{0.739696in}}%
\pgfpathlineto{\pgfqpoint{1.650000in}{0.739986in}}%
\pgfpathlineto{\pgfqpoint{1.875000in}{0.740543in}}%
\pgfpathlineto{\pgfqpoint{2.100000in}{0.741650in}}%
\pgfpathlineto{\pgfqpoint{2.325000in}{0.707162in}}%
\pgfpathlineto{\pgfqpoint{2.550000in}{0.711115in}}%
\pgfpathlineto{\pgfqpoint{2.775000in}{0.603714in}}%
\pgfpathlineto{\pgfqpoint{3.000000in}{0.795336in}}%
\pgfpathlineto{\pgfqpoint{3.225000in}{0.874930in}}%
\pgfpathlineto{\pgfqpoint{3.450000in}{0.805075in}}%
\pgfusepath{stroke}%
\end{pgfscope}%
\begin{pgfscope}%
\pgfpathrectangle{\pgfqpoint{0.750000in}{0.450000in}}{\pgfqpoint{4.050000in}{3.600000in}} %
\pgfusepath{clip}%
\pgfsetrectcap%
\pgfsetroundjoin%
\pgfsetlinewidth{2.007500pt}%
\definecolor{currentstroke}{rgb}{0.363636,0.363636,0.363636}%
\pgfsetstrokecolor{currentstroke}%
\pgfsetdash{}{0pt}%
\pgfpathmoveto{\pgfqpoint{0.750000in}{0.776791in}}%
\pgfpathlineto{\pgfqpoint{0.975000in}{0.776899in}}%
\pgfpathlineto{\pgfqpoint{1.200000in}{0.777144in}}%
\pgfpathlineto{\pgfqpoint{1.425000in}{0.740422in}}%
\pgfpathlineto{\pgfqpoint{1.650000in}{0.741089in}}%
\pgfpathlineto{\pgfqpoint{1.875000in}{0.741420in}}%
\pgfpathlineto{\pgfqpoint{2.100000in}{0.740290in}}%
\pgfpathlineto{\pgfqpoint{2.325000in}{0.741344in}}%
\pgfpathlineto{\pgfqpoint{2.550000in}{0.706897in}}%
\pgfpathlineto{\pgfqpoint{2.775000in}{0.710787in}}%
\pgfpathlineto{\pgfqpoint{3.000000in}{0.603536in}}%
\pgfpathlineto{\pgfqpoint{3.225000in}{0.794868in}}%
\pgfpathlineto{\pgfqpoint{3.450000in}{0.874437in}}%
\pgfpathlineto{\pgfqpoint{3.675000in}{0.804653in}}%
\pgfusepath{stroke}%
\end{pgfscope}%
\begin{pgfscope}%
\pgfpathrectangle{\pgfqpoint{0.750000in}{0.450000in}}{\pgfqpoint{4.050000in}{3.600000in}} %
\pgfusepath{clip}%
\pgfsetrectcap%
\pgfsetroundjoin%
\pgfsetlinewidth{2.007500pt}%
\definecolor{currentstroke}{rgb}{0.454545,0.454545,0.454545}%
\pgfsetstrokecolor{currentstroke}%
\pgfsetdash{}{0pt}%
\pgfpathmoveto{\pgfqpoint{0.750000in}{0.737644in}}%
\pgfpathlineto{\pgfqpoint{0.975000in}{0.732640in}}%
\pgfpathlineto{\pgfqpoint{1.200000in}{0.727579in}}%
\pgfpathlineto{\pgfqpoint{1.425000in}{0.788388in}}%
\pgfpathlineto{\pgfqpoint{1.650000in}{0.885391in}}%
\pgfpathlineto{\pgfqpoint{1.875000in}{1.222294in}}%
\pgfpathlineto{\pgfqpoint{2.100000in}{1.219143in}}%
\pgfpathlineto{\pgfqpoint{2.325000in}{1.220388in}}%
\pgfpathlineto{\pgfqpoint{2.550000in}{1.222848in}}%
\pgfpathlineto{\pgfqpoint{2.775000in}{1.168021in}}%
\pgfpathlineto{\pgfqpoint{3.000000in}{1.177200in}}%
\pgfpathlineto{\pgfqpoint{3.225000in}{1.071399in}}%
\pgfpathlineto{\pgfqpoint{3.450000in}{1.253933in}}%
\pgfpathlineto{\pgfqpoint{3.675000in}{1.854144in}}%
\pgfpathlineto{\pgfqpoint{3.900000in}{0.804449in}}%
\pgfusepath{stroke}%
\end{pgfscope}%
\begin{pgfscope}%
\pgfpathrectangle{\pgfqpoint{0.750000in}{0.450000in}}{\pgfqpoint{4.050000in}{3.600000in}} %
\pgfusepath{clip}%
\pgfsetrectcap%
\pgfsetroundjoin%
\pgfsetlinewidth{2.007500pt}%
\definecolor{currentstroke}{rgb}{0.545455,0.545455,0.545455}%
\pgfsetstrokecolor{currentstroke}%
\pgfsetdash{}{0pt}%
\pgfpathmoveto{\pgfqpoint{0.750000in}{0.737906in}}%
\pgfpathlineto{\pgfqpoint{0.975000in}{0.733330in}}%
\pgfpathlineto{\pgfqpoint{1.200000in}{0.728682in}}%
\pgfpathlineto{\pgfqpoint{1.425000in}{0.789736in}}%
\pgfpathlineto{\pgfqpoint{1.650000in}{0.937827in}}%
\pgfpathlineto{\pgfqpoint{1.875000in}{1.235320in}}%
\pgfpathlineto{\pgfqpoint{2.100000in}{1.245423in}}%
\pgfpathlineto{\pgfqpoint{2.325000in}{1.335485in}}%
\pgfpathlineto{\pgfqpoint{2.550000in}{1.396107in}}%
\pgfpathlineto{\pgfqpoint{2.775000in}{1.511700in}}%
\pgfpathlineto{\pgfqpoint{3.000000in}{2.036045in}}%
\pgfpathlineto{\pgfqpoint{3.225000in}{1.177091in}}%
\pgfpathlineto{\pgfqpoint{3.450000in}{1.071258in}}%
\pgfpathlineto{\pgfqpoint{3.675000in}{2.004523in}}%
\pgfpathlineto{\pgfqpoint{3.900000in}{1.853948in}}%
\pgfpathlineto{\pgfqpoint{4.125000in}{0.804354in}}%
\pgfusepath{stroke}%
\end{pgfscope}%
\begin{pgfscope}%
\pgfpathrectangle{\pgfqpoint{0.750000in}{0.450000in}}{\pgfqpoint{4.050000in}{3.600000in}} %
\pgfusepath{clip}%
\pgfsetrectcap%
\pgfsetroundjoin%
\pgfsetlinewidth{2.007500pt}%
\definecolor{currentstroke}{rgb}{0.636364,0.636364,0.636364}%
\pgfsetstrokecolor{currentstroke}%
\pgfsetdash{}{0pt}%
\pgfpathmoveto{\pgfqpoint{0.750000in}{0.738801in}}%
\pgfpathlineto{\pgfqpoint{0.975000in}{0.735276in}}%
\pgfpathlineto{\pgfqpoint{1.200000in}{0.733080in}}%
\pgfpathlineto{\pgfqpoint{1.425000in}{0.799438in}}%
\pgfpathlineto{\pgfqpoint{1.650000in}{1.165686in}}%
\pgfpathlineto{\pgfqpoint{1.875000in}{1.716290in}}%
\pgfpathlineto{\pgfqpoint{2.100000in}{1.723532in}}%
\pgfpathlineto{\pgfqpoint{2.325000in}{1.753345in}}%
\pgfpathlineto{\pgfqpoint{2.550000in}{1.905509in}}%
\pgfpathlineto{\pgfqpoint{2.775000in}{2.018635in}}%
\pgfpathlineto{\pgfqpoint{3.000000in}{2.443464in}}%
\pgfpathlineto{\pgfqpoint{3.225000in}{1.993162in}}%
\pgfpathlineto{\pgfqpoint{3.450000in}{1.177010in}}%
\pgfpathlineto{\pgfqpoint{3.675000in}{1.902254in}}%
\pgfpathlineto{\pgfqpoint{3.900000in}{2.004451in}}%
\pgfpathlineto{\pgfqpoint{4.125000in}{1.853892in}}%
\pgfpathlineto{\pgfqpoint{4.350000in}{0.804296in}}%
\pgfusepath{stroke}%
\end{pgfscope}%
\begin{pgfscope}%
\pgfpathrectangle{\pgfqpoint{0.750000in}{0.450000in}}{\pgfqpoint{4.050000in}{3.600000in}} %
\pgfusepath{clip}%
\pgfsetrectcap%
\pgfsetroundjoin%
\pgfsetlinewidth{2.007500pt}%
\definecolor{currentstroke}{rgb}{0.727273,0.727273,0.727273}%
\pgfsetstrokecolor{currentstroke}%
\pgfsetdash{}{0pt}%
\pgfpathmoveto{\pgfqpoint{0.750000in}{0.777038in}}%
\pgfpathlineto{\pgfqpoint{0.975000in}{0.775767in}}%
\pgfpathlineto{\pgfqpoint{1.200000in}{0.788481in}}%
\pgfpathlineto{\pgfqpoint{1.425000in}{0.874904in}}%
\pgfpathlineto{\pgfqpoint{1.650000in}{1.629297in}}%
\pgfpathlineto{\pgfqpoint{1.875000in}{3.318494in}}%
\pgfpathlineto{\pgfqpoint{2.100000in}{3.320986in}}%
\pgfpathlineto{\pgfqpoint{2.325000in}{3.340296in}}%
\pgfpathlineto{\pgfqpoint{2.550000in}{3.379694in}}%
\pgfpathlineto{\pgfqpoint{2.775000in}{3.460867in}}%
\pgfpathlineto{\pgfqpoint{3.000000in}{3.650933in}}%
\pgfpathlineto{\pgfqpoint{3.225000in}{2.888222in}}%
\pgfpathlineto{\pgfqpoint{3.450000in}{2.024251in}}%
\pgfpathlineto{\pgfqpoint{3.675000in}{2.086704in}}%
\pgfpathlineto{\pgfqpoint{3.900000in}{1.861865in}}%
\pgfpathlineto{\pgfqpoint{4.125000in}{2.004322in}}%
\pgfpathlineto{\pgfqpoint{4.350000in}{2.136542in}}%
\pgfpathlineto{\pgfqpoint{4.575000in}{0.804246in}}%
\pgfusepath{stroke}%
\end{pgfscope}%
\begin{pgfscope}%
\pgfpathrectangle{\pgfqpoint{0.750000in}{0.450000in}}{\pgfqpoint{4.050000in}{3.600000in}} %
\pgfusepath{clip}%
\pgfsetrectcap%
\pgfsetroundjoin%
\pgfsetlinewidth{2.007500pt}%
\definecolor{currentstroke}{rgb}{0.818182,0.818182,0.818182}%
\pgfsetstrokecolor{currentstroke}%
\pgfsetdash{}{0pt}%
\pgfpathmoveto{\pgfqpoint{0.750000in}{0.777060in}}%
\pgfpathlineto{\pgfqpoint{0.975000in}{0.775940in}}%
\pgfpathlineto{\pgfqpoint{1.200000in}{0.788712in}}%
\pgfpathlineto{\pgfqpoint{1.425000in}{0.874444in}}%
\pgfpathlineto{\pgfqpoint{1.650000in}{1.630479in}}%
\pgfpathlineto{\pgfqpoint{1.875000in}{3.319330in}}%
\pgfpathlineto{\pgfqpoint{2.100000in}{3.322656in}}%
\pgfpathlineto{\pgfqpoint{2.325000in}{3.342830in}}%
\pgfpathlineto{\pgfqpoint{2.550000in}{3.384029in}}%
\pgfpathlineto{\pgfqpoint{2.775000in}{3.462710in}}%
\pgfpathlineto{\pgfqpoint{3.000000in}{3.637526in}}%
\pgfpathlineto{\pgfqpoint{3.225000in}{3.645763in}}%
\pgfpathlineto{\pgfqpoint{3.450000in}{2.587910in}}%
\pgfpathlineto{\pgfqpoint{3.675000in}{2.037614in}}%
\pgfpathlineto{\pgfqpoint{3.900000in}{1.962432in}}%
\pgfpathlineto{\pgfqpoint{4.125000in}{2.005848in}}%
\pgfpathlineto{\pgfqpoint{4.350000in}{2.101637in}}%
\pgfpathlineto{\pgfqpoint{4.575000in}{1.853555in}}%
\pgfpathlineto{\pgfqpoint{4.800000in}{0.804246in}}%
\pgfusepath{stroke}%
\end{pgfscope}%
\begin{pgfscope}%
\pgfpathrectangle{\pgfqpoint{0.750000in}{0.450000in}}{\pgfqpoint{4.050000in}{3.600000in}} %
\pgfusepath{clip}%
\pgfsetrectcap%
\pgfsetroundjoin%
\pgfsetlinewidth{2.007500pt}%
\definecolor{currentstroke}{rgb}{0.909091,0.909091,0.909091}%
\pgfsetstrokecolor{currentstroke}%
\pgfsetdash{}{0pt}%
\pgfpathmoveto{\pgfqpoint{0.750000in}{0.776455in}}%
\pgfpathlineto{\pgfqpoint{0.975000in}{0.739038in}}%
\pgfpathlineto{\pgfqpoint{1.200000in}{0.786591in}}%
\pgfpathlineto{\pgfqpoint{1.425000in}{0.870481in}}%
\pgfpathlineto{\pgfqpoint{1.650000in}{1.631156in}}%
\pgfpathlineto{\pgfqpoint{1.875000in}{3.320986in}}%
\pgfpathlineto{\pgfqpoint{2.100000in}{3.320986in}}%
\pgfpathlineto{\pgfqpoint{2.325000in}{3.331037in}}%
\pgfpathlineto{\pgfqpoint{2.550000in}{3.351312in}}%
\pgfpathlineto{\pgfqpoint{2.775000in}{3.399797in}}%
\pgfpathlineto{\pgfqpoint{3.000000in}{3.483029in}}%
\pgfpathlineto{\pgfqpoint{3.225000in}{3.471912in}}%
\pgfpathlineto{\pgfqpoint{3.450000in}{3.166402in}}%
\pgfusepath{stroke}%
\end{pgfscope}%
\begin{pgfscope}%
\pgfpathrectangle{\pgfqpoint{0.750000in}{0.450000in}}{\pgfqpoint{4.050000in}{3.600000in}} %
\pgfusepath{clip}%
\pgfsetbuttcap%
\pgfsetroundjoin%
\pgfsetlinewidth{0.501875pt}%
\definecolor{currentstroke}{rgb}{0.000000,0.000000,0.000000}%
\pgfsetstrokecolor{currentstroke}%
\pgfsetdash{{1.000000pt}{3.000000pt}}{0.000000pt}%
\pgfpathmoveto{\pgfqpoint{0.750000in}{0.450000in}}%
\pgfpathlineto{\pgfqpoint{0.750000in}{4.050000in}}%
\pgfusepath{stroke}%
\end{pgfscope}%
\begin{pgfscope}%
\pgfsetbuttcap%
\pgfsetroundjoin%
\definecolor{currentfill}{rgb}{0.000000,0.000000,0.000000}%
\pgfsetfillcolor{currentfill}%
\pgfsetlinewidth{0.501875pt}%
\definecolor{currentstroke}{rgb}{0.000000,0.000000,0.000000}%
\pgfsetstrokecolor{currentstroke}%
\pgfsetdash{}{0pt}%
\pgfsys@defobject{currentmarker}{\pgfqpoint{0.000000in}{0.000000in}}{\pgfqpoint{0.000000in}{0.055556in}}{%
\pgfpathmoveto{\pgfqpoint{0.000000in}{0.000000in}}%
\pgfpathlineto{\pgfqpoint{0.000000in}{0.055556in}}%
\pgfusepath{stroke,fill}%
}%
\begin{pgfscope}%
\pgfsys@transformshift{0.750000in}{0.450000in}%
\pgfsys@useobject{currentmarker}{}%
\end{pgfscope}%
\end{pgfscope}%
\begin{pgfscope}%
\pgfsetbuttcap%
\pgfsetroundjoin%
\definecolor{currentfill}{rgb}{0.000000,0.000000,0.000000}%
\pgfsetfillcolor{currentfill}%
\pgfsetlinewidth{0.501875pt}%
\definecolor{currentstroke}{rgb}{0.000000,0.000000,0.000000}%
\pgfsetstrokecolor{currentstroke}%
\pgfsetdash{}{0pt}%
\pgfsys@defobject{currentmarker}{\pgfqpoint{0.000000in}{-0.055556in}}{\pgfqpoint{0.000000in}{0.000000in}}{%
\pgfpathmoveto{\pgfqpoint{0.000000in}{0.000000in}}%
\pgfpathlineto{\pgfqpoint{0.000000in}{-0.055556in}}%
\pgfusepath{stroke,fill}%
}%
\begin{pgfscope}%
\pgfsys@transformshift{0.750000in}{4.050000in}%
\pgfsys@useobject{currentmarker}{}%
\end{pgfscope}%
\end{pgfscope}%
\begin{pgfscope}%
\pgftext[x=0.750000in,y=0.394444in,,top]{{\rmfamily\fontsize{11.000000}{13.200000}\selectfont \(\displaystyle 2^{2}\)}}%
\end{pgfscope}%
\begin{pgfscope}%
\pgfpathrectangle{\pgfqpoint{0.750000in}{0.450000in}}{\pgfqpoint{4.050000in}{3.600000in}} %
\pgfusepath{clip}%
\pgfsetbuttcap%
\pgfsetroundjoin%
\pgfsetlinewidth{0.501875pt}%
\definecolor{currentstroke}{rgb}{0.000000,0.000000,0.000000}%
\pgfsetstrokecolor{currentstroke}%
\pgfsetdash{{1.000000pt}{3.000000pt}}{0.000000pt}%
\pgfpathmoveto{\pgfqpoint{1.200000in}{0.450000in}}%
\pgfpathlineto{\pgfqpoint{1.200000in}{4.050000in}}%
\pgfusepath{stroke}%
\end{pgfscope}%
\begin{pgfscope}%
\pgfsetbuttcap%
\pgfsetroundjoin%
\definecolor{currentfill}{rgb}{0.000000,0.000000,0.000000}%
\pgfsetfillcolor{currentfill}%
\pgfsetlinewidth{0.501875pt}%
\definecolor{currentstroke}{rgb}{0.000000,0.000000,0.000000}%
\pgfsetstrokecolor{currentstroke}%
\pgfsetdash{}{0pt}%
\pgfsys@defobject{currentmarker}{\pgfqpoint{0.000000in}{0.000000in}}{\pgfqpoint{0.000000in}{0.055556in}}{%
\pgfpathmoveto{\pgfqpoint{0.000000in}{0.000000in}}%
\pgfpathlineto{\pgfqpoint{0.000000in}{0.055556in}}%
\pgfusepath{stroke,fill}%
}%
\begin{pgfscope}%
\pgfsys@transformshift{1.200000in}{0.450000in}%
\pgfsys@useobject{currentmarker}{}%
\end{pgfscope}%
\end{pgfscope}%
\begin{pgfscope}%
\pgfsetbuttcap%
\pgfsetroundjoin%
\definecolor{currentfill}{rgb}{0.000000,0.000000,0.000000}%
\pgfsetfillcolor{currentfill}%
\pgfsetlinewidth{0.501875pt}%
\definecolor{currentstroke}{rgb}{0.000000,0.000000,0.000000}%
\pgfsetstrokecolor{currentstroke}%
\pgfsetdash{}{0pt}%
\pgfsys@defobject{currentmarker}{\pgfqpoint{0.000000in}{-0.055556in}}{\pgfqpoint{0.000000in}{0.000000in}}{%
\pgfpathmoveto{\pgfqpoint{0.000000in}{0.000000in}}%
\pgfpathlineto{\pgfqpoint{0.000000in}{-0.055556in}}%
\pgfusepath{stroke,fill}%
}%
\begin{pgfscope}%
\pgfsys@transformshift{1.200000in}{4.050000in}%
\pgfsys@useobject{currentmarker}{}%
\end{pgfscope}%
\end{pgfscope}%
\begin{pgfscope}%
\pgftext[x=1.200000in,y=0.394444in,,top]{{\rmfamily\fontsize{11.000000}{13.200000}\selectfont \(\displaystyle 2^{4}\)}}%
\end{pgfscope}%
\begin{pgfscope}%
\pgfpathrectangle{\pgfqpoint{0.750000in}{0.450000in}}{\pgfqpoint{4.050000in}{3.600000in}} %
\pgfusepath{clip}%
\pgfsetbuttcap%
\pgfsetroundjoin%
\pgfsetlinewidth{0.501875pt}%
\definecolor{currentstroke}{rgb}{0.000000,0.000000,0.000000}%
\pgfsetstrokecolor{currentstroke}%
\pgfsetdash{{1.000000pt}{3.000000pt}}{0.000000pt}%
\pgfpathmoveto{\pgfqpoint{1.650000in}{0.450000in}}%
\pgfpathlineto{\pgfqpoint{1.650000in}{4.050000in}}%
\pgfusepath{stroke}%
\end{pgfscope}%
\begin{pgfscope}%
\pgfsetbuttcap%
\pgfsetroundjoin%
\definecolor{currentfill}{rgb}{0.000000,0.000000,0.000000}%
\pgfsetfillcolor{currentfill}%
\pgfsetlinewidth{0.501875pt}%
\definecolor{currentstroke}{rgb}{0.000000,0.000000,0.000000}%
\pgfsetstrokecolor{currentstroke}%
\pgfsetdash{}{0pt}%
\pgfsys@defobject{currentmarker}{\pgfqpoint{0.000000in}{0.000000in}}{\pgfqpoint{0.000000in}{0.055556in}}{%
\pgfpathmoveto{\pgfqpoint{0.000000in}{0.000000in}}%
\pgfpathlineto{\pgfqpoint{0.000000in}{0.055556in}}%
\pgfusepath{stroke,fill}%
}%
\begin{pgfscope}%
\pgfsys@transformshift{1.650000in}{0.450000in}%
\pgfsys@useobject{currentmarker}{}%
\end{pgfscope}%
\end{pgfscope}%
\begin{pgfscope}%
\pgfsetbuttcap%
\pgfsetroundjoin%
\definecolor{currentfill}{rgb}{0.000000,0.000000,0.000000}%
\pgfsetfillcolor{currentfill}%
\pgfsetlinewidth{0.501875pt}%
\definecolor{currentstroke}{rgb}{0.000000,0.000000,0.000000}%
\pgfsetstrokecolor{currentstroke}%
\pgfsetdash{}{0pt}%
\pgfsys@defobject{currentmarker}{\pgfqpoint{0.000000in}{-0.055556in}}{\pgfqpoint{0.000000in}{0.000000in}}{%
\pgfpathmoveto{\pgfqpoint{0.000000in}{0.000000in}}%
\pgfpathlineto{\pgfqpoint{0.000000in}{-0.055556in}}%
\pgfusepath{stroke,fill}%
}%
\begin{pgfscope}%
\pgfsys@transformshift{1.650000in}{4.050000in}%
\pgfsys@useobject{currentmarker}{}%
\end{pgfscope}%
\end{pgfscope}%
\begin{pgfscope}%
\pgftext[x=1.650000in,y=0.394444in,,top]{{\rmfamily\fontsize{11.000000}{13.200000}\selectfont \(\displaystyle 2^{6}\)}}%
\end{pgfscope}%
\begin{pgfscope}%
\pgfpathrectangle{\pgfqpoint{0.750000in}{0.450000in}}{\pgfqpoint{4.050000in}{3.600000in}} %
\pgfusepath{clip}%
\pgfsetbuttcap%
\pgfsetroundjoin%
\pgfsetlinewidth{0.501875pt}%
\definecolor{currentstroke}{rgb}{0.000000,0.000000,0.000000}%
\pgfsetstrokecolor{currentstroke}%
\pgfsetdash{{1.000000pt}{3.000000pt}}{0.000000pt}%
\pgfpathmoveto{\pgfqpoint{2.100000in}{0.450000in}}%
\pgfpathlineto{\pgfqpoint{2.100000in}{4.050000in}}%
\pgfusepath{stroke}%
\end{pgfscope}%
\begin{pgfscope}%
\pgfsetbuttcap%
\pgfsetroundjoin%
\definecolor{currentfill}{rgb}{0.000000,0.000000,0.000000}%
\pgfsetfillcolor{currentfill}%
\pgfsetlinewidth{0.501875pt}%
\definecolor{currentstroke}{rgb}{0.000000,0.000000,0.000000}%
\pgfsetstrokecolor{currentstroke}%
\pgfsetdash{}{0pt}%
\pgfsys@defobject{currentmarker}{\pgfqpoint{0.000000in}{0.000000in}}{\pgfqpoint{0.000000in}{0.055556in}}{%
\pgfpathmoveto{\pgfqpoint{0.000000in}{0.000000in}}%
\pgfpathlineto{\pgfqpoint{0.000000in}{0.055556in}}%
\pgfusepath{stroke,fill}%
}%
\begin{pgfscope}%
\pgfsys@transformshift{2.100000in}{0.450000in}%
\pgfsys@useobject{currentmarker}{}%
\end{pgfscope}%
\end{pgfscope}%
\begin{pgfscope}%
\pgfsetbuttcap%
\pgfsetroundjoin%
\definecolor{currentfill}{rgb}{0.000000,0.000000,0.000000}%
\pgfsetfillcolor{currentfill}%
\pgfsetlinewidth{0.501875pt}%
\definecolor{currentstroke}{rgb}{0.000000,0.000000,0.000000}%
\pgfsetstrokecolor{currentstroke}%
\pgfsetdash{}{0pt}%
\pgfsys@defobject{currentmarker}{\pgfqpoint{0.000000in}{-0.055556in}}{\pgfqpoint{0.000000in}{0.000000in}}{%
\pgfpathmoveto{\pgfqpoint{0.000000in}{0.000000in}}%
\pgfpathlineto{\pgfqpoint{0.000000in}{-0.055556in}}%
\pgfusepath{stroke,fill}%
}%
\begin{pgfscope}%
\pgfsys@transformshift{2.100000in}{4.050000in}%
\pgfsys@useobject{currentmarker}{}%
\end{pgfscope}%
\end{pgfscope}%
\begin{pgfscope}%
\pgftext[x=2.100000in,y=0.394444in,,top]{{\rmfamily\fontsize{11.000000}{13.200000}\selectfont \(\displaystyle 2^{8}\)}}%
\end{pgfscope}%
\begin{pgfscope}%
\pgfpathrectangle{\pgfqpoint{0.750000in}{0.450000in}}{\pgfqpoint{4.050000in}{3.600000in}} %
\pgfusepath{clip}%
\pgfsetbuttcap%
\pgfsetroundjoin%
\pgfsetlinewidth{0.501875pt}%
\definecolor{currentstroke}{rgb}{0.000000,0.000000,0.000000}%
\pgfsetstrokecolor{currentstroke}%
\pgfsetdash{{1.000000pt}{3.000000pt}}{0.000000pt}%
\pgfpathmoveto{\pgfqpoint{2.550000in}{0.450000in}}%
\pgfpathlineto{\pgfqpoint{2.550000in}{4.050000in}}%
\pgfusepath{stroke}%
\end{pgfscope}%
\begin{pgfscope}%
\pgfsetbuttcap%
\pgfsetroundjoin%
\definecolor{currentfill}{rgb}{0.000000,0.000000,0.000000}%
\pgfsetfillcolor{currentfill}%
\pgfsetlinewidth{0.501875pt}%
\definecolor{currentstroke}{rgb}{0.000000,0.000000,0.000000}%
\pgfsetstrokecolor{currentstroke}%
\pgfsetdash{}{0pt}%
\pgfsys@defobject{currentmarker}{\pgfqpoint{0.000000in}{0.000000in}}{\pgfqpoint{0.000000in}{0.055556in}}{%
\pgfpathmoveto{\pgfqpoint{0.000000in}{0.000000in}}%
\pgfpathlineto{\pgfqpoint{0.000000in}{0.055556in}}%
\pgfusepath{stroke,fill}%
}%
\begin{pgfscope}%
\pgfsys@transformshift{2.550000in}{0.450000in}%
\pgfsys@useobject{currentmarker}{}%
\end{pgfscope}%
\end{pgfscope}%
\begin{pgfscope}%
\pgfsetbuttcap%
\pgfsetroundjoin%
\definecolor{currentfill}{rgb}{0.000000,0.000000,0.000000}%
\pgfsetfillcolor{currentfill}%
\pgfsetlinewidth{0.501875pt}%
\definecolor{currentstroke}{rgb}{0.000000,0.000000,0.000000}%
\pgfsetstrokecolor{currentstroke}%
\pgfsetdash{}{0pt}%
\pgfsys@defobject{currentmarker}{\pgfqpoint{0.000000in}{-0.055556in}}{\pgfqpoint{0.000000in}{0.000000in}}{%
\pgfpathmoveto{\pgfqpoint{0.000000in}{0.000000in}}%
\pgfpathlineto{\pgfqpoint{0.000000in}{-0.055556in}}%
\pgfusepath{stroke,fill}%
}%
\begin{pgfscope}%
\pgfsys@transformshift{2.550000in}{4.050000in}%
\pgfsys@useobject{currentmarker}{}%
\end{pgfscope}%
\end{pgfscope}%
\begin{pgfscope}%
\pgftext[x=2.550000in,y=0.394444in,,top]{{\rmfamily\fontsize{11.000000}{13.200000}\selectfont \(\displaystyle 2^{10}\)}}%
\end{pgfscope}%
\begin{pgfscope}%
\pgfpathrectangle{\pgfqpoint{0.750000in}{0.450000in}}{\pgfqpoint{4.050000in}{3.600000in}} %
\pgfusepath{clip}%
\pgfsetbuttcap%
\pgfsetroundjoin%
\pgfsetlinewidth{0.501875pt}%
\definecolor{currentstroke}{rgb}{0.000000,0.000000,0.000000}%
\pgfsetstrokecolor{currentstroke}%
\pgfsetdash{{1.000000pt}{3.000000pt}}{0.000000pt}%
\pgfpathmoveto{\pgfqpoint{3.000000in}{0.450000in}}%
\pgfpathlineto{\pgfqpoint{3.000000in}{4.050000in}}%
\pgfusepath{stroke}%
\end{pgfscope}%
\begin{pgfscope}%
\pgfsetbuttcap%
\pgfsetroundjoin%
\definecolor{currentfill}{rgb}{0.000000,0.000000,0.000000}%
\pgfsetfillcolor{currentfill}%
\pgfsetlinewidth{0.501875pt}%
\definecolor{currentstroke}{rgb}{0.000000,0.000000,0.000000}%
\pgfsetstrokecolor{currentstroke}%
\pgfsetdash{}{0pt}%
\pgfsys@defobject{currentmarker}{\pgfqpoint{0.000000in}{0.000000in}}{\pgfqpoint{0.000000in}{0.055556in}}{%
\pgfpathmoveto{\pgfqpoint{0.000000in}{0.000000in}}%
\pgfpathlineto{\pgfqpoint{0.000000in}{0.055556in}}%
\pgfusepath{stroke,fill}%
}%
\begin{pgfscope}%
\pgfsys@transformshift{3.000000in}{0.450000in}%
\pgfsys@useobject{currentmarker}{}%
\end{pgfscope}%
\end{pgfscope}%
\begin{pgfscope}%
\pgfsetbuttcap%
\pgfsetroundjoin%
\definecolor{currentfill}{rgb}{0.000000,0.000000,0.000000}%
\pgfsetfillcolor{currentfill}%
\pgfsetlinewidth{0.501875pt}%
\definecolor{currentstroke}{rgb}{0.000000,0.000000,0.000000}%
\pgfsetstrokecolor{currentstroke}%
\pgfsetdash{}{0pt}%
\pgfsys@defobject{currentmarker}{\pgfqpoint{0.000000in}{-0.055556in}}{\pgfqpoint{0.000000in}{0.000000in}}{%
\pgfpathmoveto{\pgfqpoint{0.000000in}{0.000000in}}%
\pgfpathlineto{\pgfqpoint{0.000000in}{-0.055556in}}%
\pgfusepath{stroke,fill}%
}%
\begin{pgfscope}%
\pgfsys@transformshift{3.000000in}{4.050000in}%
\pgfsys@useobject{currentmarker}{}%
\end{pgfscope}%
\end{pgfscope}%
\begin{pgfscope}%
\pgftext[x=3.000000in,y=0.394444in,,top]{{\rmfamily\fontsize{11.000000}{13.200000}\selectfont \(\displaystyle 2^{12}\)}}%
\end{pgfscope}%
\begin{pgfscope}%
\pgfpathrectangle{\pgfqpoint{0.750000in}{0.450000in}}{\pgfqpoint{4.050000in}{3.600000in}} %
\pgfusepath{clip}%
\pgfsetbuttcap%
\pgfsetroundjoin%
\pgfsetlinewidth{0.501875pt}%
\definecolor{currentstroke}{rgb}{0.000000,0.000000,0.000000}%
\pgfsetstrokecolor{currentstroke}%
\pgfsetdash{{1.000000pt}{3.000000pt}}{0.000000pt}%
\pgfpathmoveto{\pgfqpoint{3.450000in}{0.450000in}}%
\pgfpathlineto{\pgfqpoint{3.450000in}{4.050000in}}%
\pgfusepath{stroke}%
\end{pgfscope}%
\begin{pgfscope}%
\pgfsetbuttcap%
\pgfsetroundjoin%
\definecolor{currentfill}{rgb}{0.000000,0.000000,0.000000}%
\pgfsetfillcolor{currentfill}%
\pgfsetlinewidth{0.501875pt}%
\definecolor{currentstroke}{rgb}{0.000000,0.000000,0.000000}%
\pgfsetstrokecolor{currentstroke}%
\pgfsetdash{}{0pt}%
\pgfsys@defobject{currentmarker}{\pgfqpoint{0.000000in}{0.000000in}}{\pgfqpoint{0.000000in}{0.055556in}}{%
\pgfpathmoveto{\pgfqpoint{0.000000in}{0.000000in}}%
\pgfpathlineto{\pgfqpoint{0.000000in}{0.055556in}}%
\pgfusepath{stroke,fill}%
}%
\begin{pgfscope}%
\pgfsys@transformshift{3.450000in}{0.450000in}%
\pgfsys@useobject{currentmarker}{}%
\end{pgfscope}%
\end{pgfscope}%
\begin{pgfscope}%
\pgfsetbuttcap%
\pgfsetroundjoin%
\definecolor{currentfill}{rgb}{0.000000,0.000000,0.000000}%
\pgfsetfillcolor{currentfill}%
\pgfsetlinewidth{0.501875pt}%
\definecolor{currentstroke}{rgb}{0.000000,0.000000,0.000000}%
\pgfsetstrokecolor{currentstroke}%
\pgfsetdash{}{0pt}%
\pgfsys@defobject{currentmarker}{\pgfqpoint{0.000000in}{-0.055556in}}{\pgfqpoint{0.000000in}{0.000000in}}{%
\pgfpathmoveto{\pgfqpoint{0.000000in}{0.000000in}}%
\pgfpathlineto{\pgfqpoint{0.000000in}{-0.055556in}}%
\pgfusepath{stroke,fill}%
}%
\begin{pgfscope}%
\pgfsys@transformshift{3.450000in}{4.050000in}%
\pgfsys@useobject{currentmarker}{}%
\end{pgfscope}%
\end{pgfscope}%
\begin{pgfscope}%
\pgftext[x=3.450000in,y=0.394444in,,top]{{\rmfamily\fontsize{11.000000}{13.200000}\selectfont \(\displaystyle 2^{14}\)}}%
\end{pgfscope}%
\begin{pgfscope}%
\pgfpathrectangle{\pgfqpoint{0.750000in}{0.450000in}}{\pgfqpoint{4.050000in}{3.600000in}} %
\pgfusepath{clip}%
\pgfsetbuttcap%
\pgfsetroundjoin%
\pgfsetlinewidth{0.501875pt}%
\definecolor{currentstroke}{rgb}{0.000000,0.000000,0.000000}%
\pgfsetstrokecolor{currentstroke}%
\pgfsetdash{{1.000000pt}{3.000000pt}}{0.000000pt}%
\pgfpathmoveto{\pgfqpoint{3.900000in}{0.450000in}}%
\pgfpathlineto{\pgfqpoint{3.900000in}{4.050000in}}%
\pgfusepath{stroke}%
\end{pgfscope}%
\begin{pgfscope}%
\pgfsetbuttcap%
\pgfsetroundjoin%
\definecolor{currentfill}{rgb}{0.000000,0.000000,0.000000}%
\pgfsetfillcolor{currentfill}%
\pgfsetlinewidth{0.501875pt}%
\definecolor{currentstroke}{rgb}{0.000000,0.000000,0.000000}%
\pgfsetstrokecolor{currentstroke}%
\pgfsetdash{}{0pt}%
\pgfsys@defobject{currentmarker}{\pgfqpoint{0.000000in}{0.000000in}}{\pgfqpoint{0.000000in}{0.055556in}}{%
\pgfpathmoveto{\pgfqpoint{0.000000in}{0.000000in}}%
\pgfpathlineto{\pgfqpoint{0.000000in}{0.055556in}}%
\pgfusepath{stroke,fill}%
}%
\begin{pgfscope}%
\pgfsys@transformshift{3.900000in}{0.450000in}%
\pgfsys@useobject{currentmarker}{}%
\end{pgfscope}%
\end{pgfscope}%
\begin{pgfscope}%
\pgfsetbuttcap%
\pgfsetroundjoin%
\definecolor{currentfill}{rgb}{0.000000,0.000000,0.000000}%
\pgfsetfillcolor{currentfill}%
\pgfsetlinewidth{0.501875pt}%
\definecolor{currentstroke}{rgb}{0.000000,0.000000,0.000000}%
\pgfsetstrokecolor{currentstroke}%
\pgfsetdash{}{0pt}%
\pgfsys@defobject{currentmarker}{\pgfqpoint{0.000000in}{-0.055556in}}{\pgfqpoint{0.000000in}{0.000000in}}{%
\pgfpathmoveto{\pgfqpoint{0.000000in}{0.000000in}}%
\pgfpathlineto{\pgfqpoint{0.000000in}{-0.055556in}}%
\pgfusepath{stroke,fill}%
}%
\begin{pgfscope}%
\pgfsys@transformshift{3.900000in}{4.050000in}%
\pgfsys@useobject{currentmarker}{}%
\end{pgfscope}%
\end{pgfscope}%
\begin{pgfscope}%
\pgftext[x=3.900000in,y=0.394444in,,top]{{\rmfamily\fontsize{11.000000}{13.200000}\selectfont \(\displaystyle 2^{16}\)}}%
\end{pgfscope}%
\begin{pgfscope}%
\pgfpathrectangle{\pgfqpoint{0.750000in}{0.450000in}}{\pgfqpoint{4.050000in}{3.600000in}} %
\pgfusepath{clip}%
\pgfsetbuttcap%
\pgfsetroundjoin%
\pgfsetlinewidth{0.501875pt}%
\definecolor{currentstroke}{rgb}{0.000000,0.000000,0.000000}%
\pgfsetstrokecolor{currentstroke}%
\pgfsetdash{{1.000000pt}{3.000000pt}}{0.000000pt}%
\pgfpathmoveto{\pgfqpoint{4.350000in}{0.450000in}}%
\pgfpathlineto{\pgfqpoint{4.350000in}{4.050000in}}%
\pgfusepath{stroke}%
\end{pgfscope}%
\begin{pgfscope}%
\pgfsetbuttcap%
\pgfsetroundjoin%
\definecolor{currentfill}{rgb}{0.000000,0.000000,0.000000}%
\pgfsetfillcolor{currentfill}%
\pgfsetlinewidth{0.501875pt}%
\definecolor{currentstroke}{rgb}{0.000000,0.000000,0.000000}%
\pgfsetstrokecolor{currentstroke}%
\pgfsetdash{}{0pt}%
\pgfsys@defobject{currentmarker}{\pgfqpoint{0.000000in}{0.000000in}}{\pgfqpoint{0.000000in}{0.055556in}}{%
\pgfpathmoveto{\pgfqpoint{0.000000in}{0.000000in}}%
\pgfpathlineto{\pgfqpoint{0.000000in}{0.055556in}}%
\pgfusepath{stroke,fill}%
}%
\begin{pgfscope}%
\pgfsys@transformshift{4.350000in}{0.450000in}%
\pgfsys@useobject{currentmarker}{}%
\end{pgfscope}%
\end{pgfscope}%
\begin{pgfscope}%
\pgfsetbuttcap%
\pgfsetroundjoin%
\definecolor{currentfill}{rgb}{0.000000,0.000000,0.000000}%
\pgfsetfillcolor{currentfill}%
\pgfsetlinewidth{0.501875pt}%
\definecolor{currentstroke}{rgb}{0.000000,0.000000,0.000000}%
\pgfsetstrokecolor{currentstroke}%
\pgfsetdash{}{0pt}%
\pgfsys@defobject{currentmarker}{\pgfqpoint{0.000000in}{-0.055556in}}{\pgfqpoint{0.000000in}{0.000000in}}{%
\pgfpathmoveto{\pgfqpoint{0.000000in}{0.000000in}}%
\pgfpathlineto{\pgfqpoint{0.000000in}{-0.055556in}}%
\pgfusepath{stroke,fill}%
}%
\begin{pgfscope}%
\pgfsys@transformshift{4.350000in}{4.050000in}%
\pgfsys@useobject{currentmarker}{}%
\end{pgfscope}%
\end{pgfscope}%
\begin{pgfscope}%
\pgftext[x=4.350000in,y=0.394444in,,top]{{\rmfamily\fontsize{11.000000}{13.200000}\selectfont \(\displaystyle 2^{18}\)}}%
\end{pgfscope}%
\begin{pgfscope}%
\pgfpathrectangle{\pgfqpoint{0.750000in}{0.450000in}}{\pgfqpoint{4.050000in}{3.600000in}} %
\pgfusepath{clip}%
\pgfsetbuttcap%
\pgfsetroundjoin%
\pgfsetlinewidth{0.501875pt}%
\definecolor{currentstroke}{rgb}{0.000000,0.000000,0.000000}%
\pgfsetstrokecolor{currentstroke}%
\pgfsetdash{{1.000000pt}{3.000000pt}}{0.000000pt}%
\pgfpathmoveto{\pgfqpoint{4.800000in}{0.450000in}}%
\pgfpathlineto{\pgfqpoint{4.800000in}{4.050000in}}%
\pgfusepath{stroke}%
\end{pgfscope}%
\begin{pgfscope}%
\pgfsetbuttcap%
\pgfsetroundjoin%
\definecolor{currentfill}{rgb}{0.000000,0.000000,0.000000}%
\pgfsetfillcolor{currentfill}%
\pgfsetlinewidth{0.501875pt}%
\definecolor{currentstroke}{rgb}{0.000000,0.000000,0.000000}%
\pgfsetstrokecolor{currentstroke}%
\pgfsetdash{}{0pt}%
\pgfsys@defobject{currentmarker}{\pgfqpoint{0.000000in}{0.000000in}}{\pgfqpoint{0.000000in}{0.055556in}}{%
\pgfpathmoveto{\pgfqpoint{0.000000in}{0.000000in}}%
\pgfpathlineto{\pgfqpoint{0.000000in}{0.055556in}}%
\pgfusepath{stroke,fill}%
}%
\begin{pgfscope}%
\pgfsys@transformshift{4.800000in}{0.450000in}%
\pgfsys@useobject{currentmarker}{}%
\end{pgfscope}%
\end{pgfscope}%
\begin{pgfscope}%
\pgfsetbuttcap%
\pgfsetroundjoin%
\definecolor{currentfill}{rgb}{0.000000,0.000000,0.000000}%
\pgfsetfillcolor{currentfill}%
\pgfsetlinewidth{0.501875pt}%
\definecolor{currentstroke}{rgb}{0.000000,0.000000,0.000000}%
\pgfsetstrokecolor{currentstroke}%
\pgfsetdash{}{0pt}%
\pgfsys@defobject{currentmarker}{\pgfqpoint{0.000000in}{-0.055556in}}{\pgfqpoint{0.000000in}{0.000000in}}{%
\pgfpathmoveto{\pgfqpoint{0.000000in}{0.000000in}}%
\pgfpathlineto{\pgfqpoint{0.000000in}{-0.055556in}}%
\pgfusepath{stroke,fill}%
}%
\begin{pgfscope}%
\pgfsys@transformshift{4.800000in}{4.050000in}%
\pgfsys@useobject{currentmarker}{}%
\end{pgfscope}%
\end{pgfscope}%
\begin{pgfscope}%
\pgftext[x=4.800000in,y=0.394444in,,top]{{\rmfamily\fontsize{11.000000}{13.200000}\selectfont \(\displaystyle 2^{20}\)}}%
\end{pgfscope}%
\begin{pgfscope}%
\pgfsetbuttcap%
\pgfsetroundjoin%
\definecolor{currentfill}{rgb}{0.000000,0.000000,0.000000}%
\pgfsetfillcolor{currentfill}%
\pgfsetlinewidth{0.501875pt}%
\definecolor{currentstroke}{rgb}{0.000000,0.000000,0.000000}%
\pgfsetstrokecolor{currentstroke}%
\pgfsetdash{}{0pt}%
\pgfsys@defobject{currentmarker}{\pgfqpoint{0.000000in}{0.000000in}}{\pgfqpoint{0.000000in}{0.027778in}}{%
\pgfpathmoveto{\pgfqpoint{0.000000in}{0.000000in}}%
\pgfpathlineto{\pgfqpoint{0.000000in}{0.027778in}}%
\pgfusepath{stroke,fill}%
}%
\begin{pgfscope}%
\pgfsys@transformshift{0.750000in}{0.450000in}%
\pgfsys@useobject{currentmarker}{}%
\end{pgfscope}%
\end{pgfscope}%
\begin{pgfscope}%
\pgfsetbuttcap%
\pgfsetroundjoin%
\definecolor{currentfill}{rgb}{0.000000,0.000000,0.000000}%
\pgfsetfillcolor{currentfill}%
\pgfsetlinewidth{0.501875pt}%
\definecolor{currentstroke}{rgb}{0.000000,0.000000,0.000000}%
\pgfsetstrokecolor{currentstroke}%
\pgfsetdash{}{0pt}%
\pgfsys@defobject{currentmarker}{\pgfqpoint{0.000000in}{-0.027778in}}{\pgfqpoint{0.000000in}{0.000000in}}{%
\pgfpathmoveto{\pgfqpoint{0.000000in}{0.000000in}}%
\pgfpathlineto{\pgfqpoint{0.000000in}{-0.027778in}}%
\pgfusepath{stroke,fill}%
}%
\begin{pgfscope}%
\pgfsys@transformshift{0.750000in}{4.050000in}%
\pgfsys@useobject{currentmarker}{}%
\end{pgfscope}%
\end{pgfscope}%
\begin{pgfscope}%
\pgfsetbuttcap%
\pgfsetroundjoin%
\definecolor{currentfill}{rgb}{0.000000,0.000000,0.000000}%
\pgfsetfillcolor{currentfill}%
\pgfsetlinewidth{0.501875pt}%
\definecolor{currentstroke}{rgb}{0.000000,0.000000,0.000000}%
\pgfsetstrokecolor{currentstroke}%
\pgfsetdash{}{0pt}%
\pgfsys@defobject{currentmarker}{\pgfqpoint{0.000000in}{0.000000in}}{\pgfqpoint{0.000000in}{0.027778in}}{%
\pgfpathmoveto{\pgfqpoint{0.000000in}{0.000000in}}%
\pgfpathlineto{\pgfqpoint{0.000000in}{0.027778in}}%
\pgfusepath{stroke,fill}%
}%
\begin{pgfscope}%
\pgfsys@transformshift{1.200000in}{0.450000in}%
\pgfsys@useobject{currentmarker}{}%
\end{pgfscope}%
\end{pgfscope}%
\begin{pgfscope}%
\pgfsetbuttcap%
\pgfsetroundjoin%
\definecolor{currentfill}{rgb}{0.000000,0.000000,0.000000}%
\pgfsetfillcolor{currentfill}%
\pgfsetlinewidth{0.501875pt}%
\definecolor{currentstroke}{rgb}{0.000000,0.000000,0.000000}%
\pgfsetstrokecolor{currentstroke}%
\pgfsetdash{}{0pt}%
\pgfsys@defobject{currentmarker}{\pgfqpoint{0.000000in}{-0.027778in}}{\pgfqpoint{0.000000in}{0.000000in}}{%
\pgfpathmoveto{\pgfqpoint{0.000000in}{0.000000in}}%
\pgfpathlineto{\pgfqpoint{0.000000in}{-0.027778in}}%
\pgfusepath{stroke,fill}%
}%
\begin{pgfscope}%
\pgfsys@transformshift{1.200000in}{4.050000in}%
\pgfsys@useobject{currentmarker}{}%
\end{pgfscope}%
\end{pgfscope}%
\begin{pgfscope}%
\pgfsetbuttcap%
\pgfsetroundjoin%
\definecolor{currentfill}{rgb}{0.000000,0.000000,0.000000}%
\pgfsetfillcolor{currentfill}%
\pgfsetlinewidth{0.501875pt}%
\definecolor{currentstroke}{rgb}{0.000000,0.000000,0.000000}%
\pgfsetstrokecolor{currentstroke}%
\pgfsetdash{}{0pt}%
\pgfsys@defobject{currentmarker}{\pgfqpoint{0.000000in}{0.000000in}}{\pgfqpoint{0.000000in}{0.027778in}}{%
\pgfpathmoveto{\pgfqpoint{0.000000in}{0.000000in}}%
\pgfpathlineto{\pgfqpoint{0.000000in}{0.027778in}}%
\pgfusepath{stroke,fill}%
}%
\begin{pgfscope}%
\pgfsys@transformshift{1.650000in}{0.450000in}%
\pgfsys@useobject{currentmarker}{}%
\end{pgfscope}%
\end{pgfscope}%
\begin{pgfscope}%
\pgfsetbuttcap%
\pgfsetroundjoin%
\definecolor{currentfill}{rgb}{0.000000,0.000000,0.000000}%
\pgfsetfillcolor{currentfill}%
\pgfsetlinewidth{0.501875pt}%
\definecolor{currentstroke}{rgb}{0.000000,0.000000,0.000000}%
\pgfsetstrokecolor{currentstroke}%
\pgfsetdash{}{0pt}%
\pgfsys@defobject{currentmarker}{\pgfqpoint{0.000000in}{-0.027778in}}{\pgfqpoint{0.000000in}{0.000000in}}{%
\pgfpathmoveto{\pgfqpoint{0.000000in}{0.000000in}}%
\pgfpathlineto{\pgfqpoint{0.000000in}{-0.027778in}}%
\pgfusepath{stroke,fill}%
}%
\begin{pgfscope}%
\pgfsys@transformshift{1.650000in}{4.050000in}%
\pgfsys@useobject{currentmarker}{}%
\end{pgfscope}%
\end{pgfscope}%
\begin{pgfscope}%
\pgfsetbuttcap%
\pgfsetroundjoin%
\definecolor{currentfill}{rgb}{0.000000,0.000000,0.000000}%
\pgfsetfillcolor{currentfill}%
\pgfsetlinewidth{0.501875pt}%
\definecolor{currentstroke}{rgb}{0.000000,0.000000,0.000000}%
\pgfsetstrokecolor{currentstroke}%
\pgfsetdash{}{0pt}%
\pgfsys@defobject{currentmarker}{\pgfqpoint{0.000000in}{0.000000in}}{\pgfqpoint{0.000000in}{0.027778in}}{%
\pgfpathmoveto{\pgfqpoint{0.000000in}{0.000000in}}%
\pgfpathlineto{\pgfqpoint{0.000000in}{0.027778in}}%
\pgfusepath{stroke,fill}%
}%
\begin{pgfscope}%
\pgfsys@transformshift{2.100000in}{0.450000in}%
\pgfsys@useobject{currentmarker}{}%
\end{pgfscope}%
\end{pgfscope}%
\begin{pgfscope}%
\pgfsetbuttcap%
\pgfsetroundjoin%
\definecolor{currentfill}{rgb}{0.000000,0.000000,0.000000}%
\pgfsetfillcolor{currentfill}%
\pgfsetlinewidth{0.501875pt}%
\definecolor{currentstroke}{rgb}{0.000000,0.000000,0.000000}%
\pgfsetstrokecolor{currentstroke}%
\pgfsetdash{}{0pt}%
\pgfsys@defobject{currentmarker}{\pgfqpoint{0.000000in}{-0.027778in}}{\pgfqpoint{0.000000in}{0.000000in}}{%
\pgfpathmoveto{\pgfqpoint{0.000000in}{0.000000in}}%
\pgfpathlineto{\pgfqpoint{0.000000in}{-0.027778in}}%
\pgfusepath{stroke,fill}%
}%
\begin{pgfscope}%
\pgfsys@transformshift{2.100000in}{4.050000in}%
\pgfsys@useobject{currentmarker}{}%
\end{pgfscope}%
\end{pgfscope}%
\begin{pgfscope}%
\pgfsetbuttcap%
\pgfsetroundjoin%
\definecolor{currentfill}{rgb}{0.000000,0.000000,0.000000}%
\pgfsetfillcolor{currentfill}%
\pgfsetlinewidth{0.501875pt}%
\definecolor{currentstroke}{rgb}{0.000000,0.000000,0.000000}%
\pgfsetstrokecolor{currentstroke}%
\pgfsetdash{}{0pt}%
\pgfsys@defobject{currentmarker}{\pgfqpoint{0.000000in}{0.000000in}}{\pgfqpoint{0.000000in}{0.027778in}}{%
\pgfpathmoveto{\pgfqpoint{0.000000in}{0.000000in}}%
\pgfpathlineto{\pgfqpoint{0.000000in}{0.027778in}}%
\pgfusepath{stroke,fill}%
}%
\begin{pgfscope}%
\pgfsys@transformshift{2.550000in}{0.450000in}%
\pgfsys@useobject{currentmarker}{}%
\end{pgfscope}%
\end{pgfscope}%
\begin{pgfscope}%
\pgfsetbuttcap%
\pgfsetroundjoin%
\definecolor{currentfill}{rgb}{0.000000,0.000000,0.000000}%
\pgfsetfillcolor{currentfill}%
\pgfsetlinewidth{0.501875pt}%
\definecolor{currentstroke}{rgb}{0.000000,0.000000,0.000000}%
\pgfsetstrokecolor{currentstroke}%
\pgfsetdash{}{0pt}%
\pgfsys@defobject{currentmarker}{\pgfqpoint{0.000000in}{-0.027778in}}{\pgfqpoint{0.000000in}{0.000000in}}{%
\pgfpathmoveto{\pgfqpoint{0.000000in}{0.000000in}}%
\pgfpathlineto{\pgfqpoint{0.000000in}{-0.027778in}}%
\pgfusepath{stroke,fill}%
}%
\begin{pgfscope}%
\pgfsys@transformshift{2.550000in}{4.050000in}%
\pgfsys@useobject{currentmarker}{}%
\end{pgfscope}%
\end{pgfscope}%
\begin{pgfscope}%
\pgfsetbuttcap%
\pgfsetroundjoin%
\definecolor{currentfill}{rgb}{0.000000,0.000000,0.000000}%
\pgfsetfillcolor{currentfill}%
\pgfsetlinewidth{0.501875pt}%
\definecolor{currentstroke}{rgb}{0.000000,0.000000,0.000000}%
\pgfsetstrokecolor{currentstroke}%
\pgfsetdash{}{0pt}%
\pgfsys@defobject{currentmarker}{\pgfqpoint{0.000000in}{0.000000in}}{\pgfqpoint{0.000000in}{0.027778in}}{%
\pgfpathmoveto{\pgfqpoint{0.000000in}{0.000000in}}%
\pgfpathlineto{\pgfqpoint{0.000000in}{0.027778in}}%
\pgfusepath{stroke,fill}%
}%
\begin{pgfscope}%
\pgfsys@transformshift{3.000000in}{0.450000in}%
\pgfsys@useobject{currentmarker}{}%
\end{pgfscope}%
\end{pgfscope}%
\begin{pgfscope}%
\pgfsetbuttcap%
\pgfsetroundjoin%
\definecolor{currentfill}{rgb}{0.000000,0.000000,0.000000}%
\pgfsetfillcolor{currentfill}%
\pgfsetlinewidth{0.501875pt}%
\definecolor{currentstroke}{rgb}{0.000000,0.000000,0.000000}%
\pgfsetstrokecolor{currentstroke}%
\pgfsetdash{}{0pt}%
\pgfsys@defobject{currentmarker}{\pgfqpoint{0.000000in}{-0.027778in}}{\pgfqpoint{0.000000in}{0.000000in}}{%
\pgfpathmoveto{\pgfqpoint{0.000000in}{0.000000in}}%
\pgfpathlineto{\pgfqpoint{0.000000in}{-0.027778in}}%
\pgfusepath{stroke,fill}%
}%
\begin{pgfscope}%
\pgfsys@transformshift{3.000000in}{4.050000in}%
\pgfsys@useobject{currentmarker}{}%
\end{pgfscope}%
\end{pgfscope}%
\begin{pgfscope}%
\pgfsetbuttcap%
\pgfsetroundjoin%
\definecolor{currentfill}{rgb}{0.000000,0.000000,0.000000}%
\pgfsetfillcolor{currentfill}%
\pgfsetlinewidth{0.501875pt}%
\definecolor{currentstroke}{rgb}{0.000000,0.000000,0.000000}%
\pgfsetstrokecolor{currentstroke}%
\pgfsetdash{}{0pt}%
\pgfsys@defobject{currentmarker}{\pgfqpoint{0.000000in}{0.000000in}}{\pgfqpoint{0.000000in}{0.027778in}}{%
\pgfpathmoveto{\pgfqpoint{0.000000in}{0.000000in}}%
\pgfpathlineto{\pgfqpoint{0.000000in}{0.027778in}}%
\pgfusepath{stroke,fill}%
}%
\begin{pgfscope}%
\pgfsys@transformshift{3.450000in}{0.450000in}%
\pgfsys@useobject{currentmarker}{}%
\end{pgfscope}%
\end{pgfscope}%
\begin{pgfscope}%
\pgfsetbuttcap%
\pgfsetroundjoin%
\definecolor{currentfill}{rgb}{0.000000,0.000000,0.000000}%
\pgfsetfillcolor{currentfill}%
\pgfsetlinewidth{0.501875pt}%
\definecolor{currentstroke}{rgb}{0.000000,0.000000,0.000000}%
\pgfsetstrokecolor{currentstroke}%
\pgfsetdash{}{0pt}%
\pgfsys@defobject{currentmarker}{\pgfqpoint{0.000000in}{-0.027778in}}{\pgfqpoint{0.000000in}{0.000000in}}{%
\pgfpathmoveto{\pgfqpoint{0.000000in}{0.000000in}}%
\pgfpathlineto{\pgfqpoint{0.000000in}{-0.027778in}}%
\pgfusepath{stroke,fill}%
}%
\begin{pgfscope}%
\pgfsys@transformshift{3.450000in}{4.050000in}%
\pgfsys@useobject{currentmarker}{}%
\end{pgfscope}%
\end{pgfscope}%
\begin{pgfscope}%
\pgfsetbuttcap%
\pgfsetroundjoin%
\definecolor{currentfill}{rgb}{0.000000,0.000000,0.000000}%
\pgfsetfillcolor{currentfill}%
\pgfsetlinewidth{0.501875pt}%
\definecolor{currentstroke}{rgb}{0.000000,0.000000,0.000000}%
\pgfsetstrokecolor{currentstroke}%
\pgfsetdash{}{0pt}%
\pgfsys@defobject{currentmarker}{\pgfqpoint{0.000000in}{0.000000in}}{\pgfqpoint{0.000000in}{0.027778in}}{%
\pgfpathmoveto{\pgfqpoint{0.000000in}{0.000000in}}%
\pgfpathlineto{\pgfqpoint{0.000000in}{0.027778in}}%
\pgfusepath{stroke,fill}%
}%
\begin{pgfscope}%
\pgfsys@transformshift{3.900000in}{0.450000in}%
\pgfsys@useobject{currentmarker}{}%
\end{pgfscope}%
\end{pgfscope}%
\begin{pgfscope}%
\pgfsetbuttcap%
\pgfsetroundjoin%
\definecolor{currentfill}{rgb}{0.000000,0.000000,0.000000}%
\pgfsetfillcolor{currentfill}%
\pgfsetlinewidth{0.501875pt}%
\definecolor{currentstroke}{rgb}{0.000000,0.000000,0.000000}%
\pgfsetstrokecolor{currentstroke}%
\pgfsetdash{}{0pt}%
\pgfsys@defobject{currentmarker}{\pgfqpoint{0.000000in}{-0.027778in}}{\pgfqpoint{0.000000in}{0.000000in}}{%
\pgfpathmoveto{\pgfqpoint{0.000000in}{0.000000in}}%
\pgfpathlineto{\pgfqpoint{0.000000in}{-0.027778in}}%
\pgfusepath{stroke,fill}%
}%
\begin{pgfscope}%
\pgfsys@transformshift{3.900000in}{4.050000in}%
\pgfsys@useobject{currentmarker}{}%
\end{pgfscope}%
\end{pgfscope}%
\begin{pgfscope}%
\pgfsetbuttcap%
\pgfsetroundjoin%
\definecolor{currentfill}{rgb}{0.000000,0.000000,0.000000}%
\pgfsetfillcolor{currentfill}%
\pgfsetlinewidth{0.501875pt}%
\definecolor{currentstroke}{rgb}{0.000000,0.000000,0.000000}%
\pgfsetstrokecolor{currentstroke}%
\pgfsetdash{}{0pt}%
\pgfsys@defobject{currentmarker}{\pgfqpoint{0.000000in}{0.000000in}}{\pgfqpoint{0.000000in}{0.027778in}}{%
\pgfpathmoveto{\pgfqpoint{0.000000in}{0.000000in}}%
\pgfpathlineto{\pgfqpoint{0.000000in}{0.027778in}}%
\pgfusepath{stroke,fill}%
}%
\begin{pgfscope}%
\pgfsys@transformshift{4.350000in}{0.450000in}%
\pgfsys@useobject{currentmarker}{}%
\end{pgfscope}%
\end{pgfscope}%
\begin{pgfscope}%
\pgfsetbuttcap%
\pgfsetroundjoin%
\definecolor{currentfill}{rgb}{0.000000,0.000000,0.000000}%
\pgfsetfillcolor{currentfill}%
\pgfsetlinewidth{0.501875pt}%
\definecolor{currentstroke}{rgb}{0.000000,0.000000,0.000000}%
\pgfsetstrokecolor{currentstroke}%
\pgfsetdash{}{0pt}%
\pgfsys@defobject{currentmarker}{\pgfqpoint{0.000000in}{-0.027778in}}{\pgfqpoint{0.000000in}{0.000000in}}{%
\pgfpathmoveto{\pgfqpoint{0.000000in}{0.000000in}}%
\pgfpathlineto{\pgfqpoint{0.000000in}{-0.027778in}}%
\pgfusepath{stroke,fill}%
}%
\begin{pgfscope}%
\pgfsys@transformshift{4.350000in}{4.050000in}%
\pgfsys@useobject{currentmarker}{}%
\end{pgfscope}%
\end{pgfscope}%
\begin{pgfscope}%
\pgfsetbuttcap%
\pgfsetroundjoin%
\definecolor{currentfill}{rgb}{0.000000,0.000000,0.000000}%
\pgfsetfillcolor{currentfill}%
\pgfsetlinewidth{0.501875pt}%
\definecolor{currentstroke}{rgb}{0.000000,0.000000,0.000000}%
\pgfsetstrokecolor{currentstroke}%
\pgfsetdash{}{0pt}%
\pgfsys@defobject{currentmarker}{\pgfqpoint{0.000000in}{0.000000in}}{\pgfqpoint{0.000000in}{0.027778in}}{%
\pgfpathmoveto{\pgfqpoint{0.000000in}{0.000000in}}%
\pgfpathlineto{\pgfqpoint{0.000000in}{0.027778in}}%
\pgfusepath{stroke,fill}%
}%
\begin{pgfscope}%
\pgfsys@transformshift{4.800000in}{0.450000in}%
\pgfsys@useobject{currentmarker}{}%
\end{pgfscope}%
\end{pgfscope}%
\begin{pgfscope}%
\pgfsetbuttcap%
\pgfsetroundjoin%
\definecolor{currentfill}{rgb}{0.000000,0.000000,0.000000}%
\pgfsetfillcolor{currentfill}%
\pgfsetlinewidth{0.501875pt}%
\definecolor{currentstroke}{rgb}{0.000000,0.000000,0.000000}%
\pgfsetstrokecolor{currentstroke}%
\pgfsetdash{}{0pt}%
\pgfsys@defobject{currentmarker}{\pgfqpoint{0.000000in}{-0.027778in}}{\pgfqpoint{0.000000in}{0.000000in}}{%
\pgfpathmoveto{\pgfqpoint{0.000000in}{0.000000in}}%
\pgfpathlineto{\pgfqpoint{0.000000in}{-0.027778in}}%
\pgfusepath{stroke,fill}%
}%
\begin{pgfscope}%
\pgfsys@transformshift{4.800000in}{4.050000in}%
\pgfsys@useobject{currentmarker}{}%
\end{pgfscope}%
\end{pgfscope}%
\begin{pgfscope}%
\pgftext[x=2.775000in,y=0.190049in,,top]{{\rmfamily\fontsize{11.000000}{13.200000}\selectfont stride [Bytes]}}%
\end{pgfscope}%
\begin{pgfscope}%
\pgfpathrectangle{\pgfqpoint{0.750000in}{0.450000in}}{\pgfqpoint{4.050000in}{3.600000in}} %
\pgfusepath{clip}%
\pgfsetbuttcap%
\pgfsetroundjoin%
\pgfsetlinewidth{0.501875pt}%
\definecolor{currentstroke}{rgb}{0.000000,0.000000,0.000000}%
\pgfsetstrokecolor{currentstroke}%
\pgfsetdash{{1.000000pt}{3.000000pt}}{0.000000pt}%
\pgfpathmoveto{\pgfqpoint{0.750000in}{0.450000in}}%
\pgfpathlineto{\pgfqpoint{4.800000in}{0.450000in}}%
\pgfusepath{stroke}%
\end{pgfscope}%
\begin{pgfscope}%
\pgfsetbuttcap%
\pgfsetroundjoin%
\definecolor{currentfill}{rgb}{0.000000,0.000000,0.000000}%
\pgfsetfillcolor{currentfill}%
\pgfsetlinewidth{0.501875pt}%
\definecolor{currentstroke}{rgb}{0.000000,0.000000,0.000000}%
\pgfsetstrokecolor{currentstroke}%
\pgfsetdash{}{0pt}%
\pgfsys@defobject{currentmarker}{\pgfqpoint{0.000000in}{0.000000in}}{\pgfqpoint{0.055556in}{0.000000in}}{%
\pgfpathmoveto{\pgfqpoint{0.000000in}{0.000000in}}%
\pgfpathlineto{\pgfqpoint{0.055556in}{0.000000in}}%
\pgfusepath{stroke,fill}%
}%
\begin{pgfscope}%
\pgfsys@transformshift{0.750000in}{0.450000in}%
\pgfsys@useobject{currentmarker}{}%
\end{pgfscope}%
\end{pgfscope}%
\begin{pgfscope}%
\pgfsetbuttcap%
\pgfsetroundjoin%
\definecolor{currentfill}{rgb}{0.000000,0.000000,0.000000}%
\pgfsetfillcolor{currentfill}%
\pgfsetlinewidth{0.501875pt}%
\definecolor{currentstroke}{rgb}{0.000000,0.000000,0.000000}%
\pgfsetstrokecolor{currentstroke}%
\pgfsetdash{}{0pt}%
\pgfsys@defobject{currentmarker}{\pgfqpoint{-0.055556in}{0.000000in}}{\pgfqpoint{0.000000in}{0.000000in}}{%
\pgfpathmoveto{\pgfqpoint{0.000000in}{0.000000in}}%
\pgfpathlineto{\pgfqpoint{-0.055556in}{0.000000in}}%
\pgfusepath{stroke,fill}%
}%
\begin{pgfscope}%
\pgfsys@transformshift{4.800000in}{0.450000in}%
\pgfsys@useobject{currentmarker}{}%
\end{pgfscope}%
\end{pgfscope}%
\begin{pgfscope}%
\pgftext[x=0.694444in,y=0.450000in,right,]{{\rmfamily\fontsize{11.000000}{13.200000}\selectfont \(\displaystyle 0\)}}%
\end{pgfscope}%
\begin{pgfscope}%
\pgfpathrectangle{\pgfqpoint{0.750000in}{0.450000in}}{\pgfqpoint{4.050000in}{3.600000in}} %
\pgfusepath{clip}%
\pgfsetbuttcap%
\pgfsetroundjoin%
\pgfsetlinewidth{0.501875pt}%
\definecolor{currentstroke}{rgb}{0.000000,0.000000,0.000000}%
\pgfsetstrokecolor{currentstroke}%
\pgfsetdash{{1.000000pt}{3.000000pt}}{0.000000pt}%
\pgfpathmoveto{\pgfqpoint{0.750000in}{1.170000in}}%
\pgfpathlineto{\pgfqpoint{4.800000in}{1.170000in}}%
\pgfusepath{stroke}%
\end{pgfscope}%
\begin{pgfscope}%
\pgfsetbuttcap%
\pgfsetroundjoin%
\definecolor{currentfill}{rgb}{0.000000,0.000000,0.000000}%
\pgfsetfillcolor{currentfill}%
\pgfsetlinewidth{0.501875pt}%
\definecolor{currentstroke}{rgb}{0.000000,0.000000,0.000000}%
\pgfsetstrokecolor{currentstroke}%
\pgfsetdash{}{0pt}%
\pgfsys@defobject{currentmarker}{\pgfqpoint{0.000000in}{0.000000in}}{\pgfqpoint{0.055556in}{0.000000in}}{%
\pgfpathmoveto{\pgfqpoint{0.000000in}{0.000000in}}%
\pgfpathlineto{\pgfqpoint{0.055556in}{0.000000in}}%
\pgfusepath{stroke,fill}%
}%
\begin{pgfscope}%
\pgfsys@transformshift{0.750000in}{1.170000in}%
\pgfsys@useobject{currentmarker}{}%
\end{pgfscope}%
\end{pgfscope}%
\begin{pgfscope}%
\pgfsetbuttcap%
\pgfsetroundjoin%
\definecolor{currentfill}{rgb}{0.000000,0.000000,0.000000}%
\pgfsetfillcolor{currentfill}%
\pgfsetlinewidth{0.501875pt}%
\definecolor{currentstroke}{rgb}{0.000000,0.000000,0.000000}%
\pgfsetstrokecolor{currentstroke}%
\pgfsetdash{}{0pt}%
\pgfsys@defobject{currentmarker}{\pgfqpoint{-0.055556in}{0.000000in}}{\pgfqpoint{0.000000in}{0.000000in}}{%
\pgfpathmoveto{\pgfqpoint{0.000000in}{0.000000in}}%
\pgfpathlineto{\pgfqpoint{-0.055556in}{0.000000in}}%
\pgfusepath{stroke,fill}%
}%
\begin{pgfscope}%
\pgfsys@transformshift{4.800000in}{1.170000in}%
\pgfsys@useobject{currentmarker}{}%
\end{pgfscope}%
\end{pgfscope}%
\begin{pgfscope}%
\pgftext[x=0.694444in,y=1.170000in,right,]{{\rmfamily\fontsize{11.000000}{13.200000}\selectfont \(\displaystyle 5\)}}%
\end{pgfscope}%
\begin{pgfscope}%
\pgfpathrectangle{\pgfqpoint{0.750000in}{0.450000in}}{\pgfqpoint{4.050000in}{3.600000in}} %
\pgfusepath{clip}%
\pgfsetbuttcap%
\pgfsetroundjoin%
\pgfsetlinewidth{0.501875pt}%
\definecolor{currentstroke}{rgb}{0.000000,0.000000,0.000000}%
\pgfsetstrokecolor{currentstroke}%
\pgfsetdash{{1.000000pt}{3.000000pt}}{0.000000pt}%
\pgfpathmoveto{\pgfqpoint{0.750000in}{1.890000in}}%
\pgfpathlineto{\pgfqpoint{4.800000in}{1.890000in}}%
\pgfusepath{stroke}%
\end{pgfscope}%
\begin{pgfscope}%
\pgfsetbuttcap%
\pgfsetroundjoin%
\definecolor{currentfill}{rgb}{0.000000,0.000000,0.000000}%
\pgfsetfillcolor{currentfill}%
\pgfsetlinewidth{0.501875pt}%
\definecolor{currentstroke}{rgb}{0.000000,0.000000,0.000000}%
\pgfsetstrokecolor{currentstroke}%
\pgfsetdash{}{0pt}%
\pgfsys@defobject{currentmarker}{\pgfqpoint{0.000000in}{0.000000in}}{\pgfqpoint{0.055556in}{0.000000in}}{%
\pgfpathmoveto{\pgfqpoint{0.000000in}{0.000000in}}%
\pgfpathlineto{\pgfqpoint{0.055556in}{0.000000in}}%
\pgfusepath{stroke,fill}%
}%
\begin{pgfscope}%
\pgfsys@transformshift{0.750000in}{1.890000in}%
\pgfsys@useobject{currentmarker}{}%
\end{pgfscope}%
\end{pgfscope}%
\begin{pgfscope}%
\pgfsetbuttcap%
\pgfsetroundjoin%
\definecolor{currentfill}{rgb}{0.000000,0.000000,0.000000}%
\pgfsetfillcolor{currentfill}%
\pgfsetlinewidth{0.501875pt}%
\definecolor{currentstroke}{rgb}{0.000000,0.000000,0.000000}%
\pgfsetstrokecolor{currentstroke}%
\pgfsetdash{}{0pt}%
\pgfsys@defobject{currentmarker}{\pgfqpoint{-0.055556in}{0.000000in}}{\pgfqpoint{0.000000in}{0.000000in}}{%
\pgfpathmoveto{\pgfqpoint{0.000000in}{0.000000in}}%
\pgfpathlineto{\pgfqpoint{-0.055556in}{0.000000in}}%
\pgfusepath{stroke,fill}%
}%
\begin{pgfscope}%
\pgfsys@transformshift{4.800000in}{1.890000in}%
\pgfsys@useobject{currentmarker}{}%
\end{pgfscope}%
\end{pgfscope}%
\begin{pgfscope}%
\pgftext[x=0.694444in,y=1.890000in,right,]{{\rmfamily\fontsize{11.000000}{13.200000}\selectfont \(\displaystyle 10\)}}%
\end{pgfscope}%
\begin{pgfscope}%
\pgfpathrectangle{\pgfqpoint{0.750000in}{0.450000in}}{\pgfqpoint{4.050000in}{3.600000in}} %
\pgfusepath{clip}%
\pgfsetbuttcap%
\pgfsetroundjoin%
\pgfsetlinewidth{0.501875pt}%
\definecolor{currentstroke}{rgb}{0.000000,0.000000,0.000000}%
\pgfsetstrokecolor{currentstroke}%
\pgfsetdash{{1.000000pt}{3.000000pt}}{0.000000pt}%
\pgfpathmoveto{\pgfqpoint{0.750000in}{2.610000in}}%
\pgfpathlineto{\pgfqpoint{4.800000in}{2.610000in}}%
\pgfusepath{stroke}%
\end{pgfscope}%
\begin{pgfscope}%
\pgfsetbuttcap%
\pgfsetroundjoin%
\definecolor{currentfill}{rgb}{0.000000,0.000000,0.000000}%
\pgfsetfillcolor{currentfill}%
\pgfsetlinewidth{0.501875pt}%
\definecolor{currentstroke}{rgb}{0.000000,0.000000,0.000000}%
\pgfsetstrokecolor{currentstroke}%
\pgfsetdash{}{0pt}%
\pgfsys@defobject{currentmarker}{\pgfqpoint{0.000000in}{0.000000in}}{\pgfqpoint{0.055556in}{0.000000in}}{%
\pgfpathmoveto{\pgfqpoint{0.000000in}{0.000000in}}%
\pgfpathlineto{\pgfqpoint{0.055556in}{0.000000in}}%
\pgfusepath{stroke,fill}%
}%
\begin{pgfscope}%
\pgfsys@transformshift{0.750000in}{2.610000in}%
\pgfsys@useobject{currentmarker}{}%
\end{pgfscope}%
\end{pgfscope}%
\begin{pgfscope}%
\pgfsetbuttcap%
\pgfsetroundjoin%
\definecolor{currentfill}{rgb}{0.000000,0.000000,0.000000}%
\pgfsetfillcolor{currentfill}%
\pgfsetlinewidth{0.501875pt}%
\definecolor{currentstroke}{rgb}{0.000000,0.000000,0.000000}%
\pgfsetstrokecolor{currentstroke}%
\pgfsetdash{}{0pt}%
\pgfsys@defobject{currentmarker}{\pgfqpoint{-0.055556in}{0.000000in}}{\pgfqpoint{0.000000in}{0.000000in}}{%
\pgfpathmoveto{\pgfqpoint{0.000000in}{0.000000in}}%
\pgfpathlineto{\pgfqpoint{-0.055556in}{0.000000in}}%
\pgfusepath{stroke,fill}%
}%
\begin{pgfscope}%
\pgfsys@transformshift{4.800000in}{2.610000in}%
\pgfsys@useobject{currentmarker}{}%
\end{pgfscope}%
\end{pgfscope}%
\begin{pgfscope}%
\pgftext[x=0.694444in,y=2.610000in,right,]{{\rmfamily\fontsize{11.000000}{13.200000}\selectfont \(\displaystyle 15\)}}%
\end{pgfscope}%
\begin{pgfscope}%
\pgfpathrectangle{\pgfqpoint{0.750000in}{0.450000in}}{\pgfqpoint{4.050000in}{3.600000in}} %
\pgfusepath{clip}%
\pgfsetbuttcap%
\pgfsetroundjoin%
\pgfsetlinewidth{0.501875pt}%
\definecolor{currentstroke}{rgb}{0.000000,0.000000,0.000000}%
\pgfsetstrokecolor{currentstroke}%
\pgfsetdash{{1.000000pt}{3.000000pt}}{0.000000pt}%
\pgfpathmoveto{\pgfqpoint{0.750000in}{3.330000in}}%
\pgfpathlineto{\pgfqpoint{4.800000in}{3.330000in}}%
\pgfusepath{stroke}%
\end{pgfscope}%
\begin{pgfscope}%
\pgfsetbuttcap%
\pgfsetroundjoin%
\definecolor{currentfill}{rgb}{0.000000,0.000000,0.000000}%
\pgfsetfillcolor{currentfill}%
\pgfsetlinewidth{0.501875pt}%
\definecolor{currentstroke}{rgb}{0.000000,0.000000,0.000000}%
\pgfsetstrokecolor{currentstroke}%
\pgfsetdash{}{0pt}%
\pgfsys@defobject{currentmarker}{\pgfqpoint{0.000000in}{0.000000in}}{\pgfqpoint{0.055556in}{0.000000in}}{%
\pgfpathmoveto{\pgfqpoint{0.000000in}{0.000000in}}%
\pgfpathlineto{\pgfqpoint{0.055556in}{0.000000in}}%
\pgfusepath{stroke,fill}%
}%
\begin{pgfscope}%
\pgfsys@transformshift{0.750000in}{3.330000in}%
\pgfsys@useobject{currentmarker}{}%
\end{pgfscope}%
\end{pgfscope}%
\begin{pgfscope}%
\pgfsetbuttcap%
\pgfsetroundjoin%
\definecolor{currentfill}{rgb}{0.000000,0.000000,0.000000}%
\pgfsetfillcolor{currentfill}%
\pgfsetlinewidth{0.501875pt}%
\definecolor{currentstroke}{rgb}{0.000000,0.000000,0.000000}%
\pgfsetstrokecolor{currentstroke}%
\pgfsetdash{}{0pt}%
\pgfsys@defobject{currentmarker}{\pgfqpoint{-0.055556in}{0.000000in}}{\pgfqpoint{0.000000in}{0.000000in}}{%
\pgfpathmoveto{\pgfqpoint{0.000000in}{0.000000in}}%
\pgfpathlineto{\pgfqpoint{-0.055556in}{0.000000in}}%
\pgfusepath{stroke,fill}%
}%
\begin{pgfscope}%
\pgfsys@transformshift{4.800000in}{3.330000in}%
\pgfsys@useobject{currentmarker}{}%
\end{pgfscope}%
\end{pgfscope}%
\begin{pgfscope}%
\pgftext[x=0.694444in,y=3.330000in,right,]{{\rmfamily\fontsize{11.000000}{13.200000}\selectfont \(\displaystyle 20\)}}%
\end{pgfscope}%
\begin{pgfscope}%
\pgfpathrectangle{\pgfqpoint{0.750000in}{0.450000in}}{\pgfqpoint{4.050000in}{3.600000in}} %
\pgfusepath{clip}%
\pgfsetbuttcap%
\pgfsetroundjoin%
\pgfsetlinewidth{0.501875pt}%
\definecolor{currentstroke}{rgb}{0.000000,0.000000,0.000000}%
\pgfsetstrokecolor{currentstroke}%
\pgfsetdash{{1.000000pt}{3.000000pt}}{0.000000pt}%
\pgfpathmoveto{\pgfqpoint{0.750000in}{4.050000in}}%
\pgfpathlineto{\pgfqpoint{4.800000in}{4.050000in}}%
\pgfusepath{stroke}%
\end{pgfscope}%
\begin{pgfscope}%
\pgfsetbuttcap%
\pgfsetroundjoin%
\definecolor{currentfill}{rgb}{0.000000,0.000000,0.000000}%
\pgfsetfillcolor{currentfill}%
\pgfsetlinewidth{0.501875pt}%
\definecolor{currentstroke}{rgb}{0.000000,0.000000,0.000000}%
\pgfsetstrokecolor{currentstroke}%
\pgfsetdash{}{0pt}%
\pgfsys@defobject{currentmarker}{\pgfqpoint{0.000000in}{0.000000in}}{\pgfqpoint{0.055556in}{0.000000in}}{%
\pgfpathmoveto{\pgfqpoint{0.000000in}{0.000000in}}%
\pgfpathlineto{\pgfqpoint{0.055556in}{0.000000in}}%
\pgfusepath{stroke,fill}%
}%
\begin{pgfscope}%
\pgfsys@transformshift{0.750000in}{4.050000in}%
\pgfsys@useobject{currentmarker}{}%
\end{pgfscope}%
\end{pgfscope}%
\begin{pgfscope}%
\pgfsetbuttcap%
\pgfsetroundjoin%
\definecolor{currentfill}{rgb}{0.000000,0.000000,0.000000}%
\pgfsetfillcolor{currentfill}%
\pgfsetlinewidth{0.501875pt}%
\definecolor{currentstroke}{rgb}{0.000000,0.000000,0.000000}%
\pgfsetstrokecolor{currentstroke}%
\pgfsetdash{}{0pt}%
\pgfsys@defobject{currentmarker}{\pgfqpoint{-0.055556in}{0.000000in}}{\pgfqpoint{0.000000in}{0.000000in}}{%
\pgfpathmoveto{\pgfqpoint{0.000000in}{0.000000in}}%
\pgfpathlineto{\pgfqpoint{-0.055556in}{0.000000in}}%
\pgfusepath{stroke,fill}%
}%
\begin{pgfscope}%
\pgfsys@transformshift{4.800000in}{4.050000in}%
\pgfsys@useobject{currentmarker}{}%
\end{pgfscope}%
\end{pgfscope}%
\begin{pgfscope}%
\pgftext[x=0.694444in,y=4.050000in,right,]{{\rmfamily\fontsize{11.000000}{13.200000}\selectfont \(\displaystyle 25\)}}%
\end{pgfscope}%
\begin{pgfscope}%
\pgftext[x=0.475405in,y=2.250000in,,bottom,rotate=90.000000]{{\rmfamily\fontsize{11.000000}{13.200000}\selectfont loadtime [ns]}}%
\end{pgfscope}%
\begin{pgfscope}%
\pgfsetbuttcap%
\pgfsetroundjoin%
\pgfsetlinewidth{1.003750pt}%
\definecolor{currentstroke}{rgb}{0.000000,0.000000,0.000000}%
\pgfsetstrokecolor{currentstroke}%
\pgfsetdash{}{0pt}%
\pgfpathmoveto{\pgfqpoint{0.750000in}{4.050000in}}%
\pgfpathlineto{\pgfqpoint{4.800000in}{4.050000in}}%
\pgfusepath{stroke}%
\end{pgfscope}%
\begin{pgfscope}%
\pgfsetbuttcap%
\pgfsetroundjoin%
\pgfsetlinewidth{1.003750pt}%
\definecolor{currentstroke}{rgb}{0.000000,0.000000,0.000000}%
\pgfsetstrokecolor{currentstroke}%
\pgfsetdash{}{0pt}%
\pgfpathmoveto{\pgfqpoint{4.800000in}{0.450000in}}%
\pgfpathlineto{\pgfqpoint{4.800000in}{4.050000in}}%
\pgfusepath{stroke}%
\end{pgfscope}%
\begin{pgfscope}%
\pgfsetbuttcap%
\pgfsetroundjoin%
\pgfsetlinewidth{1.003750pt}%
\definecolor{currentstroke}{rgb}{0.000000,0.000000,0.000000}%
\pgfsetstrokecolor{currentstroke}%
\pgfsetdash{}{0pt}%
\pgfpathmoveto{\pgfqpoint{0.750000in}{0.450000in}}%
\pgfpathlineto{\pgfqpoint{4.800000in}{0.450000in}}%
\pgfusepath{stroke}%
\end{pgfscope}%
\begin{pgfscope}%
\pgfsetbuttcap%
\pgfsetroundjoin%
\pgfsetlinewidth{1.003750pt}%
\definecolor{currentstroke}{rgb}{0.000000,0.000000,0.000000}%
\pgfsetstrokecolor{currentstroke}%
\pgfsetdash{}{0pt}%
\pgfpathmoveto{\pgfqpoint{0.750000in}{0.450000in}}%
\pgfpathlineto{\pgfqpoint{0.750000in}{4.050000in}}%
\pgfusepath{stroke}%
\end{pgfscope}%
\begin{pgfscope}%
\pgfsetbuttcap%
\pgfsetroundjoin%
\definecolor{currentfill}{rgb}{1.000000,1.000000,1.000000}%
\pgfsetfillcolor{currentfill}%
\pgfsetlinewidth{1.003750pt}%
\definecolor{currentstroke}{rgb}{0.000000,0.000000,0.000000}%
\pgfsetstrokecolor{currentstroke}%
\pgfsetdash{}{0pt}%
\pgfpathmoveto{\pgfqpoint{5.002500in}{1.228688in}}%
\pgfpathlineto{\pgfqpoint{6.393204in}{1.228688in}}%
\pgfpathlineto{\pgfqpoint{6.393204in}{4.410000in}}%
\pgfpathlineto{\pgfqpoint{5.002500in}{4.410000in}}%
\pgfpathlineto{\pgfqpoint{5.002500in}{1.228688in}}%
\pgfpathclose%
\pgfusepath{stroke,fill}%
\end{pgfscope}%
\begin{pgfscope}%
\pgfsetrectcap%
\pgfsetroundjoin%
\pgfsetlinewidth{2.007500pt}%
\definecolor{currentstroke}{rgb}{0.000000,0.000000,0.000000}%
\pgfsetstrokecolor{currentstroke}%
\pgfsetdash{}{0pt}%
\pgfpathmoveto{\pgfqpoint{5.130833in}{4.243938in}}%
\pgfpathlineto{\pgfqpoint{5.387500in}{4.243938in}}%
\pgfusepath{stroke}%
\end{pgfscope}%
\begin{pgfscope}%
\pgftext[x=5.589167in,y=4.179771in,left,base]{{\rmfamily\fontsize{13.200000}{15.840000}\selectfont \(\displaystyle 2^{12}\) Bytes}}%
\end{pgfscope}%
\begin{pgfscope}%
\pgfsetrectcap%
\pgfsetroundjoin%
\pgfsetlinewidth{2.007500pt}%
\definecolor{currentstroke}{rgb}{0.090909,0.090909,0.090909}%
\pgfsetstrokecolor{currentstroke}%
\pgfsetdash{}{0pt}%
\pgfpathmoveto{\pgfqpoint{5.130833in}{3.959728in}}%
\pgfpathlineto{\pgfqpoint{5.387500in}{3.959728in}}%
\pgfusepath{stroke}%
\end{pgfscope}%
\begin{pgfscope}%
\pgftext[x=5.589167in,y=3.895561in,left,base]{{\rmfamily\fontsize{13.200000}{15.840000}\selectfont \(\displaystyle 2^{13}\) Bytes}}%
\end{pgfscope}%
\begin{pgfscope}%
\pgfsetrectcap%
\pgfsetroundjoin%
\pgfsetlinewidth{2.007500pt}%
\definecolor{currentstroke}{rgb}{0.181818,0.181818,0.181818}%
\pgfsetstrokecolor{currentstroke}%
\pgfsetdash{}{0pt}%
\pgfpathmoveto{\pgfqpoint{5.130833in}{3.675518in}}%
\pgfpathlineto{\pgfqpoint{5.387500in}{3.675518in}}%
\pgfusepath{stroke}%
\end{pgfscope}%
\begin{pgfscope}%
\pgftext[x=5.589167in,y=3.611351in,left,base]{{\rmfamily\fontsize{13.200000}{15.840000}\selectfont \(\displaystyle 2^{14}\) Bytes}}%
\end{pgfscope}%
\begin{pgfscope}%
\pgfsetrectcap%
\pgfsetroundjoin%
\pgfsetlinewidth{2.007500pt}%
\definecolor{currentstroke}{rgb}{0.272727,0.272727,0.272727}%
\pgfsetstrokecolor{currentstroke}%
\pgfsetdash{}{0pt}%
\pgfpathmoveto{\pgfqpoint{5.130833in}{3.391307in}}%
\pgfpathlineto{\pgfqpoint{5.387500in}{3.391307in}}%
\pgfusepath{stroke}%
\end{pgfscope}%
\begin{pgfscope}%
\pgftext[x=5.589167in,y=3.327141in,left,base]{{\rmfamily\fontsize{13.200000}{15.840000}\selectfont \(\displaystyle 2^{15}\) Bytes}}%
\end{pgfscope}%
\begin{pgfscope}%
\pgfsetrectcap%
\pgfsetroundjoin%
\pgfsetlinewidth{2.007500pt}%
\definecolor{currentstroke}{rgb}{0.363636,0.363636,0.363636}%
\pgfsetstrokecolor{currentstroke}%
\pgfsetdash{}{0pt}%
\pgfpathmoveto{\pgfqpoint{5.130833in}{3.107097in}}%
\pgfpathlineto{\pgfqpoint{5.387500in}{3.107097in}}%
\pgfusepath{stroke}%
\end{pgfscope}%
\begin{pgfscope}%
\pgftext[x=5.589167in,y=3.042931in,left,base]{{\rmfamily\fontsize{13.200000}{15.840000}\selectfont \(\displaystyle 2^{16}\) Bytes}}%
\end{pgfscope}%
\begin{pgfscope}%
\pgfsetrectcap%
\pgfsetroundjoin%
\pgfsetlinewidth{2.007500pt}%
\definecolor{currentstroke}{rgb}{0.454545,0.454545,0.454545}%
\pgfsetstrokecolor{currentstroke}%
\pgfsetdash{}{0pt}%
\pgfpathmoveto{\pgfqpoint{5.130833in}{2.822887in}}%
\pgfpathlineto{\pgfqpoint{5.387500in}{2.822887in}}%
\pgfusepath{stroke}%
\end{pgfscope}%
\begin{pgfscope}%
\pgftext[x=5.589167in,y=2.758720in,left,base]{{\rmfamily\fontsize{13.200000}{15.840000}\selectfont \(\displaystyle 2^{17}\) Bytes}}%
\end{pgfscope}%
\begin{pgfscope}%
\pgfsetrectcap%
\pgfsetroundjoin%
\pgfsetlinewidth{2.007500pt}%
\definecolor{currentstroke}{rgb}{0.545455,0.545455,0.545455}%
\pgfsetstrokecolor{currentstroke}%
\pgfsetdash{}{0pt}%
\pgfpathmoveto{\pgfqpoint{5.130833in}{2.538677in}}%
\pgfpathlineto{\pgfqpoint{5.387500in}{2.538677in}}%
\pgfusepath{stroke}%
\end{pgfscope}%
\begin{pgfscope}%
\pgftext[x=5.589167in,y=2.474510in,left,base]{{\rmfamily\fontsize{13.200000}{15.840000}\selectfont \(\displaystyle 2^{18}\) Bytes}}%
\end{pgfscope}%
\begin{pgfscope}%
\pgfsetrectcap%
\pgfsetroundjoin%
\pgfsetlinewidth{2.007500pt}%
\definecolor{currentstroke}{rgb}{0.636364,0.636364,0.636364}%
\pgfsetstrokecolor{currentstroke}%
\pgfsetdash{}{0pt}%
\pgfpathmoveto{\pgfqpoint{5.130833in}{2.254467in}}%
\pgfpathlineto{\pgfqpoint{5.387500in}{2.254467in}}%
\pgfusepath{stroke}%
\end{pgfscope}%
\begin{pgfscope}%
\pgftext[x=5.589167in,y=2.190300in,left,base]{{\rmfamily\fontsize{13.200000}{15.840000}\selectfont \(\displaystyle 2^{19}\) Bytes}}%
\end{pgfscope}%
\begin{pgfscope}%
\pgfsetrectcap%
\pgfsetroundjoin%
\pgfsetlinewidth{2.007500pt}%
\definecolor{currentstroke}{rgb}{0.727273,0.727273,0.727273}%
\pgfsetstrokecolor{currentstroke}%
\pgfsetdash{}{0pt}%
\pgfpathmoveto{\pgfqpoint{5.130833in}{1.970257in}}%
\pgfpathlineto{\pgfqpoint{5.387500in}{1.970257in}}%
\pgfusepath{stroke}%
\end{pgfscope}%
\begin{pgfscope}%
\pgftext[x=5.589167in,y=1.906090in,left,base]{{\rmfamily\fontsize{13.200000}{15.840000}\selectfont \(\displaystyle 2^{20}\) Bytes}}%
\end{pgfscope}%
\begin{pgfscope}%
\pgfsetrectcap%
\pgfsetroundjoin%
\pgfsetlinewidth{2.007500pt}%
\definecolor{currentstroke}{rgb}{0.818182,0.818182,0.818182}%
\pgfsetstrokecolor{currentstroke}%
\pgfsetdash{}{0pt}%
\pgfpathmoveto{\pgfqpoint{5.130833in}{1.686047in}}%
\pgfpathlineto{\pgfqpoint{5.387500in}{1.686047in}}%
\pgfusepath{stroke}%
\end{pgfscope}%
\begin{pgfscope}%
\pgftext[x=5.589167in,y=1.621880in,left,base]{{\rmfamily\fontsize{13.200000}{15.840000}\selectfont \(\displaystyle 2^{21}\) Bytes}}%
\end{pgfscope}%
\begin{pgfscope}%
\pgfsetrectcap%
\pgfsetroundjoin%
\pgfsetlinewidth{2.007500pt}%
\definecolor{currentstroke}{rgb}{0.909091,0.909091,0.909091}%
\pgfsetstrokecolor{currentstroke}%
\pgfsetdash{}{0pt}%
\pgfpathmoveto{\pgfqpoint{5.130833in}{1.401836in}}%
\pgfpathlineto{\pgfqpoint{5.387500in}{1.401836in}}%
\pgfusepath{stroke}%
\end{pgfscope}%
\begin{pgfscope}%
\pgftext[x=5.589167in,y=1.337670in,left,base]{{\rmfamily\fontsize{13.200000}{15.840000}\selectfont \(\displaystyle 2^{22}\) Bytes}}%
\end{pgfscope}%
\end{pgfpicture}%
\makeatother%
\endgroup%

	\label{pointer_chasing_remote}
	\caption{Pointer chasing Benchmark on \emph{moore}. The lines' colors indicate the
	         used arrays' sizes in Bytes. Some lines are of an unequal length as
			 the \texttt{ssh} connection to \emph{moore} broke during the experiment.} 
\end{figure}
%%%%%%%
% PART 3
%%%%%%%%

\section{Experiment: Mutli-threaded Load Bandwidth}

In figure \ref{threads_local} and \ref{threads_remote} we see 
an increase in bandwidth at low thread counts. But for very high
thread counts we spectate a decreasing bandwidth, which is due to
thread overhead. At a certain number of threads we reach the 
maximum bandwidth of the system, thus by increasing the 
thread count further, we gain an instruction overhead.

\begin{figure}
	%% Creator: Matplotlib, PGF backend
%%
%% To include the figure in your LaTeX document, write
%%   \input{<filename>.pgf}
%%
%% Make sure the required packages are loaded in your preamble
%%   \usepackage{pgf}
%%
%% Figures using additional raster images can only be included by \input if
%% they are in the same directory as the main LaTeX file. For loading figures
%% from other directories you can use the `import` package
%%   \usepackage{import}
%% and then include the figures with
%%   \import{<path to file>}{<filename>.pgf}
%%
%% Matplotlib used the following preamble
%%   \usepackage[T1]{fontenc}
%%   \usepackage{lmodern}
%%
\begingroup%
\makeatletter%
\begin{pgfpicture}%
\pgfpathrectangle{\pgfpointorigin}{\pgfqpoint{6.000000in}{5.000000in}}%
\pgfusepath{use as bounding box}%
\begin{pgfscope}%
\pgfsetbuttcap%
\pgfsetroundjoin%
\definecolor{currentfill}{rgb}{1.000000,1.000000,1.000000}%
\pgfsetfillcolor{currentfill}%
\pgfsetlinewidth{0.000000pt}%
\definecolor{currentstroke}{rgb}{1.000000,1.000000,1.000000}%
\pgfsetstrokecolor{currentstroke}%
\pgfsetdash{}{0pt}%
\pgfpathmoveto{\pgfqpoint{0.000000in}{0.000000in}}%
\pgfpathlineto{\pgfqpoint{6.000000in}{0.000000in}}%
\pgfpathlineto{\pgfqpoint{6.000000in}{5.000000in}}%
\pgfpathlineto{\pgfqpoint{0.000000in}{5.000000in}}%
\pgfpathclose%
\pgfusepath{fill}%
\end{pgfscope}%
\begin{pgfscope}%
\pgfsetbuttcap%
\pgfsetroundjoin%
\definecolor{currentfill}{rgb}{1.000000,1.000000,1.000000}%
\pgfsetfillcolor{currentfill}%
\pgfsetlinewidth{0.000000pt}%
\definecolor{currentstroke}{rgb}{0.000000,0.000000,0.000000}%
\pgfsetstrokecolor{currentstroke}%
\pgfsetstrokeopacity{0.000000}%
\pgfsetdash{}{0pt}%
\pgfpathmoveto{\pgfqpoint{0.750000in}{0.500000in}}%
\pgfpathlineto{\pgfqpoint{4.800000in}{0.500000in}}%
\pgfpathlineto{\pgfqpoint{4.800000in}{4.500000in}}%
\pgfpathlineto{\pgfqpoint{0.750000in}{4.500000in}}%
\pgfpathclose%
\pgfusepath{fill}%
\end{pgfscope}%
\begin{pgfscope}%
\pgfpathrectangle{\pgfqpoint{0.750000in}{0.500000in}}{\pgfqpoint{4.050000in}{4.000000in}} %
\pgfusepath{clip}%
\pgfsetrectcap%
\pgfsetroundjoin%
\pgfsetlinewidth{2.007500pt}%
\definecolor{currentstroke}{rgb}{0.000000,0.000000,0.000000}%
\pgfsetstrokecolor{currentstroke}%
\pgfsetdash{}{0pt}%
\pgfpathmoveto{\pgfqpoint{0.952500in}{0.564996in}}%
\pgfpathlineto{\pgfqpoint{1.155000in}{0.862314in}}%
\pgfpathlineto{\pgfqpoint{1.357500in}{1.120536in}}%
\pgfpathlineto{\pgfqpoint{1.560000in}{0.877990in}}%
\pgfpathlineto{\pgfqpoint{1.762500in}{0.766561in}}%
\pgfpathlineto{\pgfqpoint{1.965000in}{0.705290in}}%
\pgfpathlineto{\pgfqpoint{2.167500in}{0.795848in}}%
\pgfpathlineto{\pgfqpoint{2.370000in}{0.626979in}}%
\pgfpathlineto{\pgfqpoint{2.572500in}{0.642713in}}%
\pgfpathlineto{\pgfqpoint{2.775000in}{0.657300in}}%
\pgfpathlineto{\pgfqpoint{2.977500in}{0.649622in}}%
\pgfpathlineto{\pgfqpoint{3.180000in}{0.706577in}}%
\pgfpathlineto{\pgfqpoint{3.382500in}{0.567502in}}%
\pgfpathlineto{\pgfqpoint{3.585000in}{0.698915in}}%
\pgfpathlineto{\pgfqpoint{3.787500in}{0.634316in}}%
\pgfpathlineto{\pgfqpoint{3.990000in}{0.712259in}}%
\pgfpathlineto{\pgfqpoint{4.192500in}{0.640626in}}%
\pgfpathlineto{\pgfqpoint{4.395000in}{0.644723in}}%
\pgfpathlineto{\pgfqpoint{4.597500in}{0.605503in}}%
\pgfusepath{stroke}%
\end{pgfscope}%
\begin{pgfscope}%
\pgfpathrectangle{\pgfqpoint{0.750000in}{0.500000in}}{\pgfqpoint{4.050000in}{4.000000in}} %
\pgfusepath{clip}%
\pgfsetrectcap%
\pgfsetroundjoin%
\pgfsetlinewidth{2.007500pt}%
\definecolor{currentstroke}{rgb}{0.111111,0.111111,0.111111}%
\pgfsetstrokecolor{currentstroke}%
\pgfsetdash{}{0pt}%
\pgfpathmoveto{\pgfqpoint{0.952500in}{0.598105in}}%
\pgfpathlineto{\pgfqpoint{1.155000in}{0.947968in}}%
\pgfpathlineto{\pgfqpoint{1.357500in}{1.123132in}}%
\pgfpathlineto{\pgfqpoint{1.560000in}{1.033158in}}%
\pgfpathlineto{\pgfqpoint{1.762500in}{0.786766in}}%
\pgfpathlineto{\pgfqpoint{1.965000in}{0.926737in}}%
\pgfpathlineto{\pgfqpoint{2.167500in}{0.943370in}}%
\pgfpathlineto{\pgfqpoint{2.370000in}{0.920655in}}%
\pgfpathlineto{\pgfqpoint{2.572500in}{0.923347in}}%
\pgfpathlineto{\pgfqpoint{2.775000in}{0.817404in}}%
\pgfpathlineto{\pgfqpoint{2.977500in}{0.790954in}}%
\pgfpathlineto{\pgfqpoint{3.180000in}{0.763855in}}%
\pgfpathlineto{\pgfqpoint{3.382500in}{0.849004in}}%
\pgfpathlineto{\pgfqpoint{3.585000in}{0.804415in}}%
\pgfpathlineto{\pgfqpoint{3.787500in}{0.966886in}}%
\pgfpathlineto{\pgfqpoint{3.990000in}{0.812987in}}%
\pgfpathlineto{\pgfqpoint{4.192500in}{0.867075in}}%
\pgfpathlineto{\pgfqpoint{4.395000in}{0.909651in}}%
\pgfpathlineto{\pgfqpoint{4.597500in}{0.953592in}}%
\pgfusepath{stroke}%
\end{pgfscope}%
\begin{pgfscope}%
\pgfpathrectangle{\pgfqpoint{0.750000in}{0.500000in}}{\pgfqpoint{4.050000in}{4.000000in}} %
\pgfusepath{clip}%
\pgfsetrectcap%
\pgfsetroundjoin%
\pgfsetlinewidth{2.007500pt}%
\definecolor{currentstroke}{rgb}{0.222222,0.222222,0.222222}%
\pgfsetstrokecolor{currentstroke}%
\pgfsetdash{}{0pt}%
\pgfpathmoveto{\pgfqpoint{0.952500in}{0.618152in}}%
\pgfpathlineto{\pgfqpoint{1.155000in}{0.977839in}}%
\pgfpathlineto{\pgfqpoint{1.357500in}{1.199306in}}%
\pgfpathlineto{\pgfqpoint{1.560000in}{1.070210in}}%
\pgfpathlineto{\pgfqpoint{1.762500in}{1.170703in}}%
\pgfpathlineto{\pgfqpoint{1.965000in}{1.075196in}}%
\pgfpathlineto{\pgfqpoint{2.167500in}{1.077958in}}%
\pgfpathlineto{\pgfqpoint{2.370000in}{1.054960in}}%
\pgfpathlineto{\pgfqpoint{2.572500in}{1.209764in}}%
\pgfpathlineto{\pgfqpoint{2.775000in}{1.173162in}}%
\pgfpathlineto{\pgfqpoint{2.977500in}{1.232947in}}%
\pgfpathlineto{\pgfqpoint{3.180000in}{1.140875in}}%
\pgfpathlineto{\pgfqpoint{3.382500in}{1.218457in}}%
\pgfpathlineto{\pgfqpoint{3.585000in}{1.178126in}}%
\pgfpathlineto{\pgfqpoint{3.787500in}{1.422625in}}%
\pgfpathlineto{\pgfqpoint{3.990000in}{1.118968in}}%
\pgfpathlineto{\pgfqpoint{4.192500in}{1.208981in}}%
\pgfpathlineto{\pgfqpoint{4.395000in}{1.211013in}}%
\pgfpathlineto{\pgfqpoint{4.597500in}{1.136149in}}%
\pgfusepath{stroke}%
\end{pgfscope}%
\begin{pgfscope}%
\pgfpathrectangle{\pgfqpoint{0.750000in}{0.500000in}}{\pgfqpoint{4.050000in}{4.000000in}} %
\pgfusepath{clip}%
\pgfsetrectcap%
\pgfsetroundjoin%
\pgfsetlinewidth{2.007500pt}%
\definecolor{currentstroke}{rgb}{0.333333,0.333333,0.333333}%
\pgfsetstrokecolor{currentstroke}%
\pgfsetdash{}{0pt}%
\pgfpathmoveto{\pgfqpoint{0.952500in}{0.690984in}}%
\pgfpathlineto{\pgfqpoint{1.155000in}{1.034101in}}%
\pgfpathlineto{\pgfqpoint{1.357500in}{1.280311in}}%
\pgfpathlineto{\pgfqpoint{1.560000in}{1.448179in}}%
\pgfpathlineto{\pgfqpoint{1.762500in}{1.599237in}}%
\pgfpathlineto{\pgfqpoint{1.965000in}{1.614828in}}%
\pgfpathlineto{\pgfqpoint{2.167500in}{1.468924in}}%
\pgfpathlineto{\pgfqpoint{2.370000in}{1.463602in}}%
\pgfpathlineto{\pgfqpoint{2.572500in}{1.605136in}}%
\pgfpathlineto{\pgfqpoint{2.775000in}{1.606946in}}%
\pgfpathlineto{\pgfqpoint{2.977500in}{1.511685in}}%
\pgfpathlineto{\pgfqpoint{3.180000in}{1.710853in}}%
\pgfpathlineto{\pgfqpoint{3.382500in}{1.574646in}}%
\pgfpathlineto{\pgfqpoint{3.585000in}{1.440086in}}%
\pgfpathlineto{\pgfqpoint{3.787500in}{1.292009in}}%
\pgfpathlineto{\pgfqpoint{3.990000in}{1.382539in}}%
\pgfpathlineto{\pgfqpoint{4.192500in}{1.574799in}}%
\pgfpathlineto{\pgfqpoint{4.395000in}{1.520277in}}%
\pgfpathlineto{\pgfqpoint{4.597500in}{1.555288in}}%
\pgfusepath{stroke}%
\end{pgfscope}%
\begin{pgfscope}%
\pgfpathrectangle{\pgfqpoint{0.750000in}{0.500000in}}{\pgfqpoint{4.050000in}{4.000000in}} %
\pgfusepath{clip}%
\pgfsetrectcap%
\pgfsetroundjoin%
\pgfsetlinewidth{2.007500pt}%
\definecolor{currentstroke}{rgb}{0.444444,0.444444,0.444444}%
\pgfsetstrokecolor{currentstroke}%
\pgfsetdash{}{0pt}%
\pgfpathmoveto{\pgfqpoint{0.952500in}{0.702117in}}%
\pgfpathlineto{\pgfqpoint{1.155000in}{1.268213in}}%
\pgfpathlineto{\pgfqpoint{1.357500in}{1.445284in}}%
\pgfpathlineto{\pgfqpoint{1.560000in}{1.809231in}}%
\pgfpathlineto{\pgfqpoint{1.762500in}{1.765729in}}%
\pgfpathlineto{\pgfqpoint{1.965000in}{1.562831in}}%
\pgfpathlineto{\pgfqpoint{2.167500in}{1.804185in}}%
\pgfpathlineto{\pgfqpoint{2.370000in}{1.854660in}}%
\pgfpathlineto{\pgfqpoint{2.572500in}{1.865909in}}%
\pgfpathlineto{\pgfqpoint{2.775000in}{1.930219in}}%
\pgfpathlineto{\pgfqpoint{2.977500in}{1.766294in}}%
\pgfpathlineto{\pgfqpoint{3.180000in}{2.047517in}}%
\pgfpathlineto{\pgfqpoint{3.382500in}{2.094095in}}%
\pgfpathlineto{\pgfqpoint{3.585000in}{2.113304in}}%
\pgfpathlineto{\pgfqpoint{3.787500in}{1.991979in}}%
\pgfpathlineto{\pgfqpoint{3.990000in}{2.128116in}}%
\pgfpathlineto{\pgfqpoint{4.192500in}{2.163047in}}%
\pgfpathlineto{\pgfqpoint{4.395000in}{1.937534in}}%
\pgfpathlineto{\pgfqpoint{4.597500in}{2.152160in}}%
\pgfusepath{stroke}%
\end{pgfscope}%
\begin{pgfscope}%
\pgfpathrectangle{\pgfqpoint{0.750000in}{0.500000in}}{\pgfqpoint{4.050000in}{4.000000in}} %
\pgfusepath{clip}%
\pgfsetrectcap%
\pgfsetroundjoin%
\pgfsetlinewidth{2.007500pt}%
\definecolor{currentstroke}{rgb}{0.555556,0.555556,0.555556}%
\pgfsetstrokecolor{currentstroke}%
\pgfsetdash{}{0pt}%
\pgfpathmoveto{\pgfqpoint{0.952500in}{0.872996in}}%
\pgfpathlineto{\pgfqpoint{1.155000in}{1.482953in}}%
\pgfpathlineto{\pgfqpoint{1.357500in}{1.846394in}}%
\pgfpathlineto{\pgfqpoint{1.560000in}{1.989054in}}%
\pgfpathlineto{\pgfqpoint{1.762500in}{2.010344in}}%
\pgfpathlineto{\pgfqpoint{1.965000in}{2.099252in}}%
\pgfpathlineto{\pgfqpoint{2.167500in}{2.146196in}}%
\pgfpathlineto{\pgfqpoint{2.370000in}{2.355160in}}%
\pgfpathlineto{\pgfqpoint{2.572500in}{2.172819in}}%
\pgfpathlineto{\pgfqpoint{2.775000in}{2.396778in}}%
\pgfpathlineto{\pgfqpoint{2.977500in}{2.582965in}}%
\pgfpathlineto{\pgfqpoint{3.180000in}{2.533057in}}%
\pgfpathlineto{\pgfqpoint{3.382500in}{2.509672in}}%
\pgfpathlineto{\pgfqpoint{3.585000in}{2.555938in}}%
\pgfpathlineto{\pgfqpoint{3.787500in}{2.541555in}}%
\pgfpathlineto{\pgfqpoint{3.990000in}{2.565524in}}%
\pgfpathlineto{\pgfqpoint{4.192500in}{2.552685in}}%
\pgfpathlineto{\pgfqpoint{4.395000in}{2.696340in}}%
\pgfpathlineto{\pgfqpoint{4.597500in}{2.630296in}}%
\pgfusepath{stroke}%
\end{pgfscope}%
\begin{pgfscope}%
\pgfpathrectangle{\pgfqpoint{0.750000in}{0.500000in}}{\pgfqpoint{4.050000in}{4.000000in}} %
\pgfusepath{clip}%
\pgfsetrectcap%
\pgfsetroundjoin%
\pgfsetlinewidth{2.007500pt}%
\definecolor{currentstroke}{rgb}{0.666667,0.666667,0.666667}%
\pgfsetstrokecolor{currentstroke}%
\pgfsetdash{}{0pt}%
\pgfpathmoveto{\pgfqpoint{0.952500in}{1.094539in}}%
\pgfpathlineto{\pgfqpoint{1.155000in}{1.855663in}}%
\pgfpathlineto{\pgfqpoint{1.357500in}{2.248933in}}%
\pgfpathlineto{\pgfqpoint{1.560000in}{2.370316in}}%
\pgfpathlineto{\pgfqpoint{1.762500in}{2.507528in}}%
\pgfpathlineto{\pgfqpoint{1.965000in}{2.432899in}}%
\pgfpathlineto{\pgfqpoint{2.167500in}{2.660459in}}%
\pgfpathlineto{\pgfqpoint{2.370000in}{2.791276in}}%
\pgfpathlineto{\pgfqpoint{2.572500in}{2.884965in}}%
\pgfpathlineto{\pgfqpoint{2.775000in}{2.874673in}}%
\pgfpathlineto{\pgfqpoint{2.977500in}{2.795746in}}%
\pgfpathlineto{\pgfqpoint{3.180000in}{2.807636in}}%
\pgfpathlineto{\pgfqpoint{3.382500in}{2.880281in}}%
\pgfpathlineto{\pgfqpoint{3.585000in}{2.906974in}}%
\pgfpathlineto{\pgfqpoint{3.787500in}{3.089376in}}%
\pgfpathlineto{\pgfqpoint{3.990000in}{3.051907in}}%
\pgfpathlineto{\pgfqpoint{4.192500in}{3.008498in}}%
\pgfpathlineto{\pgfqpoint{4.395000in}{2.996537in}}%
\pgfpathlineto{\pgfqpoint{4.597500in}{2.952187in}}%
\pgfusepath{stroke}%
\end{pgfscope}%
\begin{pgfscope}%
\pgfpathrectangle{\pgfqpoint{0.750000in}{0.500000in}}{\pgfqpoint{4.050000in}{4.000000in}} %
\pgfusepath{clip}%
\pgfsetrectcap%
\pgfsetroundjoin%
\pgfsetlinewidth{2.007500pt}%
\definecolor{currentstroke}{rgb}{0.777778,0.777778,0.777778}%
\pgfsetstrokecolor{currentstroke}%
\pgfsetdash{}{0pt}%
\pgfpathmoveto{\pgfqpoint{0.952500in}{1.275559in}}%
\pgfpathlineto{\pgfqpoint{1.155000in}{2.260186in}}%
\pgfpathlineto{\pgfqpoint{1.357500in}{2.957692in}}%
\pgfpathlineto{\pgfqpoint{1.560000in}{3.273285in}}%
\pgfpathlineto{\pgfqpoint{1.762500in}{3.298958in}}%
\pgfpathlineto{\pgfqpoint{1.965000in}{3.238989in}}%
\pgfpathlineto{\pgfqpoint{2.167500in}{3.375442in}}%
\pgfpathlineto{\pgfqpoint{2.370000in}{3.284511in}}%
\pgfpathlineto{\pgfqpoint{2.572500in}{3.495248in}}%
\pgfpathlineto{\pgfqpoint{2.775000in}{3.492029in}}%
\pgfpathlineto{\pgfqpoint{2.977500in}{3.486212in}}%
\pgfpathlineto{\pgfqpoint{3.180000in}{3.568839in}}%
\pgfpathlineto{\pgfqpoint{3.382500in}{3.639741in}}%
\pgfpathlineto{\pgfqpoint{3.585000in}{3.464066in}}%
\pgfpathlineto{\pgfqpoint{3.787500in}{3.435760in}}%
\pgfpathlineto{\pgfqpoint{3.990000in}{3.507119in}}%
\pgfpathlineto{\pgfqpoint{4.192500in}{3.490398in}}%
\pgfpathlineto{\pgfqpoint{4.395000in}{3.571158in}}%
\pgfpathlineto{\pgfqpoint{4.597500in}{3.443185in}}%
\pgfusepath{stroke}%
\end{pgfscope}%
\begin{pgfscope}%
\pgfpathrectangle{\pgfqpoint{0.750000in}{0.500000in}}{\pgfqpoint{4.050000in}{4.000000in}} %
\pgfusepath{clip}%
\pgfsetrectcap%
\pgfsetroundjoin%
\pgfsetlinewidth{2.007500pt}%
\definecolor{currentstroke}{rgb}{0.888889,0.888889,0.888889}%
\pgfsetstrokecolor{currentstroke}%
\pgfsetdash{}{0pt}%
\pgfpathmoveto{\pgfqpoint{0.952500in}{1.351100in}}%
\pgfpathlineto{\pgfqpoint{1.155000in}{2.555652in}}%
\pgfpathlineto{\pgfqpoint{1.357500in}{3.441262in}}%
\pgfpathlineto{\pgfqpoint{1.560000in}{3.744380in}}%
\pgfpathlineto{\pgfqpoint{1.762500in}{3.634694in}}%
\pgfpathlineto{\pgfqpoint{1.965000in}{3.631808in}}%
\pgfpathlineto{\pgfqpoint{2.167500in}{3.833720in}}%
\pgfpathlineto{\pgfqpoint{2.370000in}{3.943128in}}%
\pgfpathlineto{\pgfqpoint{2.572500in}{3.883048in}}%
\pgfpathlineto{\pgfqpoint{2.775000in}{3.842005in}}%
\pgfpathlineto{\pgfqpoint{2.977500in}{3.928192in}}%
\pgfpathlineto{\pgfqpoint{3.180000in}{3.928833in}}%
\pgfpathlineto{\pgfqpoint{3.382500in}{3.955870in}}%
\pgfpathlineto{\pgfqpoint{3.585000in}{3.967281in}}%
\pgfpathlineto{\pgfqpoint{3.787500in}{3.825070in}}%
\pgfpathlineto{\pgfqpoint{3.990000in}{3.939078in}}%
\pgfpathlineto{\pgfqpoint{4.192500in}{3.954434in}}%
\pgfpathlineto{\pgfqpoint{4.395000in}{3.934589in}}%
\pgfpathlineto{\pgfqpoint{4.597500in}{4.012404in}}%
\pgfusepath{stroke}%
\end{pgfscope}%
\begin{pgfscope}%
\pgfpathrectangle{\pgfqpoint{0.750000in}{0.500000in}}{\pgfqpoint{4.050000in}{4.000000in}} %
\pgfusepath{clip}%
\pgfsetbuttcap%
\pgfsetroundjoin%
\pgfsetlinewidth{0.501875pt}%
\definecolor{currentstroke}{rgb}{0.000000,0.000000,0.000000}%
\pgfsetstrokecolor{currentstroke}%
\pgfsetdash{{1.000000pt}{3.000000pt}}{0.000000pt}%
\pgfpathmoveto{\pgfqpoint{0.750000in}{0.500000in}}%
\pgfpathlineto{\pgfqpoint{0.750000in}{4.500000in}}%
\pgfusepath{stroke}%
\end{pgfscope}%
\begin{pgfscope}%
\pgfsetbuttcap%
\pgfsetroundjoin%
\definecolor{currentfill}{rgb}{0.000000,0.000000,0.000000}%
\pgfsetfillcolor{currentfill}%
\pgfsetlinewidth{0.501875pt}%
\definecolor{currentstroke}{rgb}{0.000000,0.000000,0.000000}%
\pgfsetstrokecolor{currentstroke}%
\pgfsetdash{}{0pt}%
\pgfsys@defobject{currentmarker}{\pgfqpoint{0.000000in}{0.000000in}}{\pgfqpoint{0.000000in}{0.055556in}}{%
\pgfpathmoveto{\pgfqpoint{0.000000in}{0.000000in}}%
\pgfpathlineto{\pgfqpoint{0.000000in}{0.055556in}}%
\pgfusepath{stroke,fill}%
}%
\begin{pgfscope}%
\pgfsys@transformshift{0.750000in}{0.500000in}%
\pgfsys@useobject{currentmarker}{}%
\end{pgfscope}%
\end{pgfscope}%
\begin{pgfscope}%
\pgfsetbuttcap%
\pgfsetroundjoin%
\definecolor{currentfill}{rgb}{0.000000,0.000000,0.000000}%
\pgfsetfillcolor{currentfill}%
\pgfsetlinewidth{0.501875pt}%
\definecolor{currentstroke}{rgb}{0.000000,0.000000,0.000000}%
\pgfsetstrokecolor{currentstroke}%
\pgfsetdash{}{0pt}%
\pgfsys@defobject{currentmarker}{\pgfqpoint{0.000000in}{-0.055556in}}{\pgfqpoint{0.000000in}{0.000000in}}{%
\pgfpathmoveto{\pgfqpoint{0.000000in}{0.000000in}}%
\pgfpathlineto{\pgfqpoint{0.000000in}{-0.055556in}}%
\pgfusepath{stroke,fill}%
}%
\begin{pgfscope}%
\pgfsys@transformshift{0.750000in}{4.500000in}%
\pgfsys@useobject{currentmarker}{}%
\end{pgfscope}%
\end{pgfscope}%
\begin{pgfscope}%
\pgftext[x=0.750000in,y=0.444444in,,top]{{\rmfamily\fontsize{11.000000}{13.200000}\selectfont \(\displaystyle 0\)}}%
\end{pgfscope}%
\begin{pgfscope}%
\pgfpathrectangle{\pgfqpoint{0.750000in}{0.500000in}}{\pgfqpoint{4.050000in}{4.000000in}} %
\pgfusepath{clip}%
\pgfsetbuttcap%
\pgfsetroundjoin%
\pgfsetlinewidth{0.501875pt}%
\definecolor{currentstroke}{rgb}{0.000000,0.000000,0.000000}%
\pgfsetstrokecolor{currentstroke}%
\pgfsetdash{{1.000000pt}{3.000000pt}}{0.000000pt}%
\pgfpathmoveto{\pgfqpoint{1.762500in}{0.500000in}}%
\pgfpathlineto{\pgfqpoint{1.762500in}{4.500000in}}%
\pgfusepath{stroke}%
\end{pgfscope}%
\begin{pgfscope}%
\pgfsetbuttcap%
\pgfsetroundjoin%
\definecolor{currentfill}{rgb}{0.000000,0.000000,0.000000}%
\pgfsetfillcolor{currentfill}%
\pgfsetlinewidth{0.501875pt}%
\definecolor{currentstroke}{rgb}{0.000000,0.000000,0.000000}%
\pgfsetstrokecolor{currentstroke}%
\pgfsetdash{}{0pt}%
\pgfsys@defobject{currentmarker}{\pgfqpoint{0.000000in}{0.000000in}}{\pgfqpoint{0.000000in}{0.055556in}}{%
\pgfpathmoveto{\pgfqpoint{0.000000in}{0.000000in}}%
\pgfpathlineto{\pgfqpoint{0.000000in}{0.055556in}}%
\pgfusepath{stroke,fill}%
}%
\begin{pgfscope}%
\pgfsys@transformshift{1.762500in}{0.500000in}%
\pgfsys@useobject{currentmarker}{}%
\end{pgfscope}%
\end{pgfscope}%
\begin{pgfscope}%
\pgfsetbuttcap%
\pgfsetroundjoin%
\definecolor{currentfill}{rgb}{0.000000,0.000000,0.000000}%
\pgfsetfillcolor{currentfill}%
\pgfsetlinewidth{0.501875pt}%
\definecolor{currentstroke}{rgb}{0.000000,0.000000,0.000000}%
\pgfsetstrokecolor{currentstroke}%
\pgfsetdash{}{0pt}%
\pgfsys@defobject{currentmarker}{\pgfqpoint{0.000000in}{-0.055556in}}{\pgfqpoint{0.000000in}{0.000000in}}{%
\pgfpathmoveto{\pgfqpoint{0.000000in}{0.000000in}}%
\pgfpathlineto{\pgfqpoint{0.000000in}{-0.055556in}}%
\pgfusepath{stroke,fill}%
}%
\begin{pgfscope}%
\pgfsys@transformshift{1.762500in}{4.500000in}%
\pgfsys@useobject{currentmarker}{}%
\end{pgfscope}%
\end{pgfscope}%
\begin{pgfscope}%
\pgftext[x=1.762500in,y=0.444444in,,top]{{\rmfamily\fontsize{11.000000}{13.200000}\selectfont \(\displaystyle 5\)}}%
\end{pgfscope}%
\begin{pgfscope}%
\pgfpathrectangle{\pgfqpoint{0.750000in}{0.500000in}}{\pgfqpoint{4.050000in}{4.000000in}} %
\pgfusepath{clip}%
\pgfsetbuttcap%
\pgfsetroundjoin%
\pgfsetlinewidth{0.501875pt}%
\definecolor{currentstroke}{rgb}{0.000000,0.000000,0.000000}%
\pgfsetstrokecolor{currentstroke}%
\pgfsetdash{{1.000000pt}{3.000000pt}}{0.000000pt}%
\pgfpathmoveto{\pgfqpoint{2.775000in}{0.500000in}}%
\pgfpathlineto{\pgfqpoint{2.775000in}{4.500000in}}%
\pgfusepath{stroke}%
\end{pgfscope}%
\begin{pgfscope}%
\pgfsetbuttcap%
\pgfsetroundjoin%
\definecolor{currentfill}{rgb}{0.000000,0.000000,0.000000}%
\pgfsetfillcolor{currentfill}%
\pgfsetlinewidth{0.501875pt}%
\definecolor{currentstroke}{rgb}{0.000000,0.000000,0.000000}%
\pgfsetstrokecolor{currentstroke}%
\pgfsetdash{}{0pt}%
\pgfsys@defobject{currentmarker}{\pgfqpoint{0.000000in}{0.000000in}}{\pgfqpoint{0.000000in}{0.055556in}}{%
\pgfpathmoveto{\pgfqpoint{0.000000in}{0.000000in}}%
\pgfpathlineto{\pgfqpoint{0.000000in}{0.055556in}}%
\pgfusepath{stroke,fill}%
}%
\begin{pgfscope}%
\pgfsys@transformshift{2.775000in}{0.500000in}%
\pgfsys@useobject{currentmarker}{}%
\end{pgfscope}%
\end{pgfscope}%
\begin{pgfscope}%
\pgfsetbuttcap%
\pgfsetroundjoin%
\definecolor{currentfill}{rgb}{0.000000,0.000000,0.000000}%
\pgfsetfillcolor{currentfill}%
\pgfsetlinewidth{0.501875pt}%
\definecolor{currentstroke}{rgb}{0.000000,0.000000,0.000000}%
\pgfsetstrokecolor{currentstroke}%
\pgfsetdash{}{0pt}%
\pgfsys@defobject{currentmarker}{\pgfqpoint{0.000000in}{-0.055556in}}{\pgfqpoint{0.000000in}{0.000000in}}{%
\pgfpathmoveto{\pgfqpoint{0.000000in}{0.000000in}}%
\pgfpathlineto{\pgfqpoint{0.000000in}{-0.055556in}}%
\pgfusepath{stroke,fill}%
}%
\begin{pgfscope}%
\pgfsys@transformshift{2.775000in}{4.500000in}%
\pgfsys@useobject{currentmarker}{}%
\end{pgfscope}%
\end{pgfscope}%
\begin{pgfscope}%
\pgftext[x=2.775000in,y=0.444444in,,top]{{\rmfamily\fontsize{11.000000}{13.200000}\selectfont \(\displaystyle 10\)}}%
\end{pgfscope}%
\begin{pgfscope}%
\pgfpathrectangle{\pgfqpoint{0.750000in}{0.500000in}}{\pgfqpoint{4.050000in}{4.000000in}} %
\pgfusepath{clip}%
\pgfsetbuttcap%
\pgfsetroundjoin%
\pgfsetlinewidth{0.501875pt}%
\definecolor{currentstroke}{rgb}{0.000000,0.000000,0.000000}%
\pgfsetstrokecolor{currentstroke}%
\pgfsetdash{{1.000000pt}{3.000000pt}}{0.000000pt}%
\pgfpathmoveto{\pgfqpoint{3.787500in}{0.500000in}}%
\pgfpathlineto{\pgfqpoint{3.787500in}{4.500000in}}%
\pgfusepath{stroke}%
\end{pgfscope}%
\begin{pgfscope}%
\pgfsetbuttcap%
\pgfsetroundjoin%
\definecolor{currentfill}{rgb}{0.000000,0.000000,0.000000}%
\pgfsetfillcolor{currentfill}%
\pgfsetlinewidth{0.501875pt}%
\definecolor{currentstroke}{rgb}{0.000000,0.000000,0.000000}%
\pgfsetstrokecolor{currentstroke}%
\pgfsetdash{}{0pt}%
\pgfsys@defobject{currentmarker}{\pgfqpoint{0.000000in}{0.000000in}}{\pgfqpoint{0.000000in}{0.055556in}}{%
\pgfpathmoveto{\pgfqpoint{0.000000in}{0.000000in}}%
\pgfpathlineto{\pgfqpoint{0.000000in}{0.055556in}}%
\pgfusepath{stroke,fill}%
}%
\begin{pgfscope}%
\pgfsys@transformshift{3.787500in}{0.500000in}%
\pgfsys@useobject{currentmarker}{}%
\end{pgfscope}%
\end{pgfscope}%
\begin{pgfscope}%
\pgfsetbuttcap%
\pgfsetroundjoin%
\definecolor{currentfill}{rgb}{0.000000,0.000000,0.000000}%
\pgfsetfillcolor{currentfill}%
\pgfsetlinewidth{0.501875pt}%
\definecolor{currentstroke}{rgb}{0.000000,0.000000,0.000000}%
\pgfsetstrokecolor{currentstroke}%
\pgfsetdash{}{0pt}%
\pgfsys@defobject{currentmarker}{\pgfqpoint{0.000000in}{-0.055556in}}{\pgfqpoint{0.000000in}{0.000000in}}{%
\pgfpathmoveto{\pgfqpoint{0.000000in}{0.000000in}}%
\pgfpathlineto{\pgfqpoint{0.000000in}{-0.055556in}}%
\pgfusepath{stroke,fill}%
}%
\begin{pgfscope}%
\pgfsys@transformshift{3.787500in}{4.500000in}%
\pgfsys@useobject{currentmarker}{}%
\end{pgfscope}%
\end{pgfscope}%
\begin{pgfscope}%
\pgftext[x=3.787500in,y=0.444444in,,top]{{\rmfamily\fontsize{11.000000}{13.200000}\selectfont \(\displaystyle 15\)}}%
\end{pgfscope}%
\begin{pgfscope}%
\pgfpathrectangle{\pgfqpoint{0.750000in}{0.500000in}}{\pgfqpoint{4.050000in}{4.000000in}} %
\pgfusepath{clip}%
\pgfsetbuttcap%
\pgfsetroundjoin%
\pgfsetlinewidth{0.501875pt}%
\definecolor{currentstroke}{rgb}{0.000000,0.000000,0.000000}%
\pgfsetstrokecolor{currentstroke}%
\pgfsetdash{{1.000000pt}{3.000000pt}}{0.000000pt}%
\pgfpathmoveto{\pgfqpoint{4.800000in}{0.500000in}}%
\pgfpathlineto{\pgfqpoint{4.800000in}{4.500000in}}%
\pgfusepath{stroke}%
\end{pgfscope}%
\begin{pgfscope}%
\pgfsetbuttcap%
\pgfsetroundjoin%
\definecolor{currentfill}{rgb}{0.000000,0.000000,0.000000}%
\pgfsetfillcolor{currentfill}%
\pgfsetlinewidth{0.501875pt}%
\definecolor{currentstroke}{rgb}{0.000000,0.000000,0.000000}%
\pgfsetstrokecolor{currentstroke}%
\pgfsetdash{}{0pt}%
\pgfsys@defobject{currentmarker}{\pgfqpoint{0.000000in}{0.000000in}}{\pgfqpoint{0.000000in}{0.055556in}}{%
\pgfpathmoveto{\pgfqpoint{0.000000in}{0.000000in}}%
\pgfpathlineto{\pgfqpoint{0.000000in}{0.055556in}}%
\pgfusepath{stroke,fill}%
}%
\begin{pgfscope}%
\pgfsys@transformshift{4.800000in}{0.500000in}%
\pgfsys@useobject{currentmarker}{}%
\end{pgfscope}%
\end{pgfscope}%
\begin{pgfscope}%
\pgfsetbuttcap%
\pgfsetroundjoin%
\definecolor{currentfill}{rgb}{0.000000,0.000000,0.000000}%
\pgfsetfillcolor{currentfill}%
\pgfsetlinewidth{0.501875pt}%
\definecolor{currentstroke}{rgb}{0.000000,0.000000,0.000000}%
\pgfsetstrokecolor{currentstroke}%
\pgfsetdash{}{0pt}%
\pgfsys@defobject{currentmarker}{\pgfqpoint{0.000000in}{-0.055556in}}{\pgfqpoint{0.000000in}{0.000000in}}{%
\pgfpathmoveto{\pgfqpoint{0.000000in}{0.000000in}}%
\pgfpathlineto{\pgfqpoint{0.000000in}{-0.055556in}}%
\pgfusepath{stroke,fill}%
}%
\begin{pgfscope}%
\pgfsys@transformshift{4.800000in}{4.500000in}%
\pgfsys@useobject{currentmarker}{}%
\end{pgfscope}%
\end{pgfscope}%
\begin{pgfscope}%
\pgftext[x=4.800000in,y=0.444444in,,top]{{\rmfamily\fontsize{11.000000}{13.200000}\selectfont \(\displaystyle 20\)}}%
\end{pgfscope}%
\begin{pgfscope}%
\pgftext[x=2.775000in,y=0.240049in,,top]{{\rmfamily\fontsize{11.000000}{13.200000}\selectfont Thread Count}}%
\end{pgfscope}%
\begin{pgfscope}%
\pgfpathrectangle{\pgfqpoint{0.750000in}{0.500000in}}{\pgfqpoint{4.050000in}{4.000000in}} %
\pgfusepath{clip}%
\pgfsetbuttcap%
\pgfsetroundjoin%
\pgfsetlinewidth{0.501875pt}%
\definecolor{currentstroke}{rgb}{0.000000,0.000000,0.000000}%
\pgfsetstrokecolor{currentstroke}%
\pgfsetdash{{1.000000pt}{3.000000pt}}{0.000000pt}%
\pgfpathmoveto{\pgfqpoint{0.750000in}{0.500000in}}%
\pgfpathlineto{\pgfqpoint{4.800000in}{0.500000in}}%
\pgfusepath{stroke}%
\end{pgfscope}%
\begin{pgfscope}%
\pgfsetbuttcap%
\pgfsetroundjoin%
\definecolor{currentfill}{rgb}{0.000000,0.000000,0.000000}%
\pgfsetfillcolor{currentfill}%
\pgfsetlinewidth{0.501875pt}%
\definecolor{currentstroke}{rgb}{0.000000,0.000000,0.000000}%
\pgfsetstrokecolor{currentstroke}%
\pgfsetdash{}{0pt}%
\pgfsys@defobject{currentmarker}{\pgfqpoint{0.000000in}{0.000000in}}{\pgfqpoint{0.055556in}{0.000000in}}{%
\pgfpathmoveto{\pgfqpoint{0.000000in}{0.000000in}}%
\pgfpathlineto{\pgfqpoint{0.055556in}{0.000000in}}%
\pgfusepath{stroke,fill}%
}%
\begin{pgfscope}%
\pgfsys@transformshift{0.750000in}{0.500000in}%
\pgfsys@useobject{currentmarker}{}%
\end{pgfscope}%
\end{pgfscope}%
\begin{pgfscope}%
\pgfsetbuttcap%
\pgfsetroundjoin%
\definecolor{currentfill}{rgb}{0.000000,0.000000,0.000000}%
\pgfsetfillcolor{currentfill}%
\pgfsetlinewidth{0.501875pt}%
\definecolor{currentstroke}{rgb}{0.000000,0.000000,0.000000}%
\pgfsetstrokecolor{currentstroke}%
\pgfsetdash{}{0pt}%
\pgfsys@defobject{currentmarker}{\pgfqpoint{-0.055556in}{0.000000in}}{\pgfqpoint{0.000000in}{0.000000in}}{%
\pgfpathmoveto{\pgfqpoint{0.000000in}{0.000000in}}%
\pgfpathlineto{\pgfqpoint{-0.055556in}{0.000000in}}%
\pgfusepath{stroke,fill}%
}%
\begin{pgfscope}%
\pgfsys@transformshift{4.800000in}{0.500000in}%
\pgfsys@useobject{currentmarker}{}%
\end{pgfscope}%
\end{pgfscope}%
\begin{pgfscope}%
\pgftext[x=0.694444in,y=0.500000in,right,]{{\rmfamily\fontsize{11.000000}{13.200000}\selectfont \(\displaystyle 0.5\)}}%
\end{pgfscope}%
\begin{pgfscope}%
\pgfpathrectangle{\pgfqpoint{0.750000in}{0.500000in}}{\pgfqpoint{4.050000in}{4.000000in}} %
\pgfusepath{clip}%
\pgfsetbuttcap%
\pgfsetroundjoin%
\pgfsetlinewidth{0.501875pt}%
\definecolor{currentstroke}{rgb}{0.000000,0.000000,0.000000}%
\pgfsetstrokecolor{currentstroke}%
\pgfsetdash{{1.000000pt}{3.000000pt}}{0.000000pt}%
\pgfpathmoveto{\pgfqpoint{0.750000in}{1.000000in}}%
\pgfpathlineto{\pgfqpoint{4.800000in}{1.000000in}}%
\pgfusepath{stroke}%
\end{pgfscope}%
\begin{pgfscope}%
\pgfsetbuttcap%
\pgfsetroundjoin%
\definecolor{currentfill}{rgb}{0.000000,0.000000,0.000000}%
\pgfsetfillcolor{currentfill}%
\pgfsetlinewidth{0.501875pt}%
\definecolor{currentstroke}{rgb}{0.000000,0.000000,0.000000}%
\pgfsetstrokecolor{currentstroke}%
\pgfsetdash{}{0pt}%
\pgfsys@defobject{currentmarker}{\pgfqpoint{0.000000in}{0.000000in}}{\pgfqpoint{0.055556in}{0.000000in}}{%
\pgfpathmoveto{\pgfqpoint{0.000000in}{0.000000in}}%
\pgfpathlineto{\pgfqpoint{0.055556in}{0.000000in}}%
\pgfusepath{stroke,fill}%
}%
\begin{pgfscope}%
\pgfsys@transformshift{0.750000in}{1.000000in}%
\pgfsys@useobject{currentmarker}{}%
\end{pgfscope}%
\end{pgfscope}%
\begin{pgfscope}%
\pgfsetbuttcap%
\pgfsetroundjoin%
\definecolor{currentfill}{rgb}{0.000000,0.000000,0.000000}%
\pgfsetfillcolor{currentfill}%
\pgfsetlinewidth{0.501875pt}%
\definecolor{currentstroke}{rgb}{0.000000,0.000000,0.000000}%
\pgfsetstrokecolor{currentstroke}%
\pgfsetdash{}{0pt}%
\pgfsys@defobject{currentmarker}{\pgfqpoint{-0.055556in}{0.000000in}}{\pgfqpoint{0.000000in}{0.000000in}}{%
\pgfpathmoveto{\pgfqpoint{0.000000in}{0.000000in}}%
\pgfpathlineto{\pgfqpoint{-0.055556in}{0.000000in}}%
\pgfusepath{stroke,fill}%
}%
\begin{pgfscope}%
\pgfsys@transformshift{4.800000in}{1.000000in}%
\pgfsys@useobject{currentmarker}{}%
\end{pgfscope}%
\end{pgfscope}%
\begin{pgfscope}%
\pgftext[x=0.694444in,y=1.000000in,right,]{{\rmfamily\fontsize{11.000000}{13.200000}\selectfont \(\displaystyle 1.0\)}}%
\end{pgfscope}%
\begin{pgfscope}%
\pgfpathrectangle{\pgfqpoint{0.750000in}{0.500000in}}{\pgfqpoint{4.050000in}{4.000000in}} %
\pgfusepath{clip}%
\pgfsetbuttcap%
\pgfsetroundjoin%
\pgfsetlinewidth{0.501875pt}%
\definecolor{currentstroke}{rgb}{0.000000,0.000000,0.000000}%
\pgfsetstrokecolor{currentstroke}%
\pgfsetdash{{1.000000pt}{3.000000pt}}{0.000000pt}%
\pgfpathmoveto{\pgfqpoint{0.750000in}{1.500000in}}%
\pgfpathlineto{\pgfqpoint{4.800000in}{1.500000in}}%
\pgfusepath{stroke}%
\end{pgfscope}%
\begin{pgfscope}%
\pgfsetbuttcap%
\pgfsetroundjoin%
\definecolor{currentfill}{rgb}{0.000000,0.000000,0.000000}%
\pgfsetfillcolor{currentfill}%
\pgfsetlinewidth{0.501875pt}%
\definecolor{currentstroke}{rgb}{0.000000,0.000000,0.000000}%
\pgfsetstrokecolor{currentstroke}%
\pgfsetdash{}{0pt}%
\pgfsys@defobject{currentmarker}{\pgfqpoint{0.000000in}{0.000000in}}{\pgfqpoint{0.055556in}{0.000000in}}{%
\pgfpathmoveto{\pgfqpoint{0.000000in}{0.000000in}}%
\pgfpathlineto{\pgfqpoint{0.055556in}{0.000000in}}%
\pgfusepath{stroke,fill}%
}%
\begin{pgfscope}%
\pgfsys@transformshift{0.750000in}{1.500000in}%
\pgfsys@useobject{currentmarker}{}%
\end{pgfscope}%
\end{pgfscope}%
\begin{pgfscope}%
\pgfsetbuttcap%
\pgfsetroundjoin%
\definecolor{currentfill}{rgb}{0.000000,0.000000,0.000000}%
\pgfsetfillcolor{currentfill}%
\pgfsetlinewidth{0.501875pt}%
\definecolor{currentstroke}{rgb}{0.000000,0.000000,0.000000}%
\pgfsetstrokecolor{currentstroke}%
\pgfsetdash{}{0pt}%
\pgfsys@defobject{currentmarker}{\pgfqpoint{-0.055556in}{0.000000in}}{\pgfqpoint{0.000000in}{0.000000in}}{%
\pgfpathmoveto{\pgfqpoint{0.000000in}{0.000000in}}%
\pgfpathlineto{\pgfqpoint{-0.055556in}{0.000000in}}%
\pgfusepath{stroke,fill}%
}%
\begin{pgfscope}%
\pgfsys@transformshift{4.800000in}{1.500000in}%
\pgfsys@useobject{currentmarker}{}%
\end{pgfscope}%
\end{pgfscope}%
\begin{pgfscope}%
\pgftext[x=0.694444in,y=1.500000in,right,]{{\rmfamily\fontsize{11.000000}{13.200000}\selectfont \(\displaystyle 1.5\)}}%
\end{pgfscope}%
\begin{pgfscope}%
\pgfpathrectangle{\pgfqpoint{0.750000in}{0.500000in}}{\pgfqpoint{4.050000in}{4.000000in}} %
\pgfusepath{clip}%
\pgfsetbuttcap%
\pgfsetroundjoin%
\pgfsetlinewidth{0.501875pt}%
\definecolor{currentstroke}{rgb}{0.000000,0.000000,0.000000}%
\pgfsetstrokecolor{currentstroke}%
\pgfsetdash{{1.000000pt}{3.000000pt}}{0.000000pt}%
\pgfpathmoveto{\pgfqpoint{0.750000in}{2.000000in}}%
\pgfpathlineto{\pgfqpoint{4.800000in}{2.000000in}}%
\pgfusepath{stroke}%
\end{pgfscope}%
\begin{pgfscope}%
\pgfsetbuttcap%
\pgfsetroundjoin%
\definecolor{currentfill}{rgb}{0.000000,0.000000,0.000000}%
\pgfsetfillcolor{currentfill}%
\pgfsetlinewidth{0.501875pt}%
\definecolor{currentstroke}{rgb}{0.000000,0.000000,0.000000}%
\pgfsetstrokecolor{currentstroke}%
\pgfsetdash{}{0pt}%
\pgfsys@defobject{currentmarker}{\pgfqpoint{0.000000in}{0.000000in}}{\pgfqpoint{0.055556in}{0.000000in}}{%
\pgfpathmoveto{\pgfqpoint{0.000000in}{0.000000in}}%
\pgfpathlineto{\pgfqpoint{0.055556in}{0.000000in}}%
\pgfusepath{stroke,fill}%
}%
\begin{pgfscope}%
\pgfsys@transformshift{0.750000in}{2.000000in}%
\pgfsys@useobject{currentmarker}{}%
\end{pgfscope}%
\end{pgfscope}%
\begin{pgfscope}%
\pgfsetbuttcap%
\pgfsetroundjoin%
\definecolor{currentfill}{rgb}{0.000000,0.000000,0.000000}%
\pgfsetfillcolor{currentfill}%
\pgfsetlinewidth{0.501875pt}%
\definecolor{currentstroke}{rgb}{0.000000,0.000000,0.000000}%
\pgfsetstrokecolor{currentstroke}%
\pgfsetdash{}{0pt}%
\pgfsys@defobject{currentmarker}{\pgfqpoint{-0.055556in}{0.000000in}}{\pgfqpoint{0.000000in}{0.000000in}}{%
\pgfpathmoveto{\pgfqpoint{0.000000in}{0.000000in}}%
\pgfpathlineto{\pgfqpoint{-0.055556in}{0.000000in}}%
\pgfusepath{stroke,fill}%
}%
\begin{pgfscope}%
\pgfsys@transformshift{4.800000in}{2.000000in}%
\pgfsys@useobject{currentmarker}{}%
\end{pgfscope}%
\end{pgfscope}%
\begin{pgfscope}%
\pgftext[x=0.694444in,y=2.000000in,right,]{{\rmfamily\fontsize{11.000000}{13.200000}\selectfont \(\displaystyle 2.0\)}}%
\end{pgfscope}%
\begin{pgfscope}%
\pgfpathrectangle{\pgfqpoint{0.750000in}{0.500000in}}{\pgfqpoint{4.050000in}{4.000000in}} %
\pgfusepath{clip}%
\pgfsetbuttcap%
\pgfsetroundjoin%
\pgfsetlinewidth{0.501875pt}%
\definecolor{currentstroke}{rgb}{0.000000,0.000000,0.000000}%
\pgfsetstrokecolor{currentstroke}%
\pgfsetdash{{1.000000pt}{3.000000pt}}{0.000000pt}%
\pgfpathmoveto{\pgfqpoint{0.750000in}{2.500000in}}%
\pgfpathlineto{\pgfqpoint{4.800000in}{2.500000in}}%
\pgfusepath{stroke}%
\end{pgfscope}%
\begin{pgfscope}%
\pgfsetbuttcap%
\pgfsetroundjoin%
\definecolor{currentfill}{rgb}{0.000000,0.000000,0.000000}%
\pgfsetfillcolor{currentfill}%
\pgfsetlinewidth{0.501875pt}%
\definecolor{currentstroke}{rgb}{0.000000,0.000000,0.000000}%
\pgfsetstrokecolor{currentstroke}%
\pgfsetdash{}{0pt}%
\pgfsys@defobject{currentmarker}{\pgfqpoint{0.000000in}{0.000000in}}{\pgfqpoint{0.055556in}{0.000000in}}{%
\pgfpathmoveto{\pgfqpoint{0.000000in}{0.000000in}}%
\pgfpathlineto{\pgfqpoint{0.055556in}{0.000000in}}%
\pgfusepath{stroke,fill}%
}%
\begin{pgfscope}%
\pgfsys@transformshift{0.750000in}{2.500000in}%
\pgfsys@useobject{currentmarker}{}%
\end{pgfscope}%
\end{pgfscope}%
\begin{pgfscope}%
\pgfsetbuttcap%
\pgfsetroundjoin%
\definecolor{currentfill}{rgb}{0.000000,0.000000,0.000000}%
\pgfsetfillcolor{currentfill}%
\pgfsetlinewidth{0.501875pt}%
\definecolor{currentstroke}{rgb}{0.000000,0.000000,0.000000}%
\pgfsetstrokecolor{currentstroke}%
\pgfsetdash{}{0pt}%
\pgfsys@defobject{currentmarker}{\pgfqpoint{-0.055556in}{0.000000in}}{\pgfqpoint{0.000000in}{0.000000in}}{%
\pgfpathmoveto{\pgfqpoint{0.000000in}{0.000000in}}%
\pgfpathlineto{\pgfqpoint{-0.055556in}{0.000000in}}%
\pgfusepath{stroke,fill}%
}%
\begin{pgfscope}%
\pgfsys@transformshift{4.800000in}{2.500000in}%
\pgfsys@useobject{currentmarker}{}%
\end{pgfscope}%
\end{pgfscope}%
\begin{pgfscope}%
\pgftext[x=0.694444in,y=2.500000in,right,]{{\rmfamily\fontsize{11.000000}{13.200000}\selectfont \(\displaystyle 2.5\)}}%
\end{pgfscope}%
\begin{pgfscope}%
\pgfpathrectangle{\pgfqpoint{0.750000in}{0.500000in}}{\pgfqpoint{4.050000in}{4.000000in}} %
\pgfusepath{clip}%
\pgfsetbuttcap%
\pgfsetroundjoin%
\pgfsetlinewidth{0.501875pt}%
\definecolor{currentstroke}{rgb}{0.000000,0.000000,0.000000}%
\pgfsetstrokecolor{currentstroke}%
\pgfsetdash{{1.000000pt}{3.000000pt}}{0.000000pt}%
\pgfpathmoveto{\pgfqpoint{0.750000in}{3.000000in}}%
\pgfpathlineto{\pgfqpoint{4.800000in}{3.000000in}}%
\pgfusepath{stroke}%
\end{pgfscope}%
\begin{pgfscope}%
\pgfsetbuttcap%
\pgfsetroundjoin%
\definecolor{currentfill}{rgb}{0.000000,0.000000,0.000000}%
\pgfsetfillcolor{currentfill}%
\pgfsetlinewidth{0.501875pt}%
\definecolor{currentstroke}{rgb}{0.000000,0.000000,0.000000}%
\pgfsetstrokecolor{currentstroke}%
\pgfsetdash{}{0pt}%
\pgfsys@defobject{currentmarker}{\pgfqpoint{0.000000in}{0.000000in}}{\pgfqpoint{0.055556in}{0.000000in}}{%
\pgfpathmoveto{\pgfqpoint{0.000000in}{0.000000in}}%
\pgfpathlineto{\pgfqpoint{0.055556in}{0.000000in}}%
\pgfusepath{stroke,fill}%
}%
\begin{pgfscope}%
\pgfsys@transformshift{0.750000in}{3.000000in}%
\pgfsys@useobject{currentmarker}{}%
\end{pgfscope}%
\end{pgfscope}%
\begin{pgfscope}%
\pgfsetbuttcap%
\pgfsetroundjoin%
\definecolor{currentfill}{rgb}{0.000000,0.000000,0.000000}%
\pgfsetfillcolor{currentfill}%
\pgfsetlinewidth{0.501875pt}%
\definecolor{currentstroke}{rgb}{0.000000,0.000000,0.000000}%
\pgfsetstrokecolor{currentstroke}%
\pgfsetdash{}{0pt}%
\pgfsys@defobject{currentmarker}{\pgfqpoint{-0.055556in}{0.000000in}}{\pgfqpoint{0.000000in}{0.000000in}}{%
\pgfpathmoveto{\pgfqpoint{0.000000in}{0.000000in}}%
\pgfpathlineto{\pgfqpoint{-0.055556in}{0.000000in}}%
\pgfusepath{stroke,fill}%
}%
\begin{pgfscope}%
\pgfsys@transformshift{4.800000in}{3.000000in}%
\pgfsys@useobject{currentmarker}{}%
\end{pgfscope}%
\end{pgfscope}%
\begin{pgfscope}%
\pgftext[x=0.694444in,y=3.000000in,right,]{{\rmfamily\fontsize{11.000000}{13.200000}\selectfont \(\displaystyle 3.0\)}}%
\end{pgfscope}%
\begin{pgfscope}%
\pgfpathrectangle{\pgfqpoint{0.750000in}{0.500000in}}{\pgfqpoint{4.050000in}{4.000000in}} %
\pgfusepath{clip}%
\pgfsetbuttcap%
\pgfsetroundjoin%
\pgfsetlinewidth{0.501875pt}%
\definecolor{currentstroke}{rgb}{0.000000,0.000000,0.000000}%
\pgfsetstrokecolor{currentstroke}%
\pgfsetdash{{1.000000pt}{3.000000pt}}{0.000000pt}%
\pgfpathmoveto{\pgfqpoint{0.750000in}{3.500000in}}%
\pgfpathlineto{\pgfqpoint{4.800000in}{3.500000in}}%
\pgfusepath{stroke}%
\end{pgfscope}%
\begin{pgfscope}%
\pgfsetbuttcap%
\pgfsetroundjoin%
\definecolor{currentfill}{rgb}{0.000000,0.000000,0.000000}%
\pgfsetfillcolor{currentfill}%
\pgfsetlinewidth{0.501875pt}%
\definecolor{currentstroke}{rgb}{0.000000,0.000000,0.000000}%
\pgfsetstrokecolor{currentstroke}%
\pgfsetdash{}{0pt}%
\pgfsys@defobject{currentmarker}{\pgfqpoint{0.000000in}{0.000000in}}{\pgfqpoint{0.055556in}{0.000000in}}{%
\pgfpathmoveto{\pgfqpoint{0.000000in}{0.000000in}}%
\pgfpathlineto{\pgfqpoint{0.055556in}{0.000000in}}%
\pgfusepath{stroke,fill}%
}%
\begin{pgfscope}%
\pgfsys@transformshift{0.750000in}{3.500000in}%
\pgfsys@useobject{currentmarker}{}%
\end{pgfscope}%
\end{pgfscope}%
\begin{pgfscope}%
\pgfsetbuttcap%
\pgfsetroundjoin%
\definecolor{currentfill}{rgb}{0.000000,0.000000,0.000000}%
\pgfsetfillcolor{currentfill}%
\pgfsetlinewidth{0.501875pt}%
\definecolor{currentstroke}{rgb}{0.000000,0.000000,0.000000}%
\pgfsetstrokecolor{currentstroke}%
\pgfsetdash{}{0pt}%
\pgfsys@defobject{currentmarker}{\pgfqpoint{-0.055556in}{0.000000in}}{\pgfqpoint{0.000000in}{0.000000in}}{%
\pgfpathmoveto{\pgfqpoint{0.000000in}{0.000000in}}%
\pgfpathlineto{\pgfqpoint{-0.055556in}{0.000000in}}%
\pgfusepath{stroke,fill}%
}%
\begin{pgfscope}%
\pgfsys@transformshift{4.800000in}{3.500000in}%
\pgfsys@useobject{currentmarker}{}%
\end{pgfscope}%
\end{pgfscope}%
\begin{pgfscope}%
\pgftext[x=0.694444in,y=3.500000in,right,]{{\rmfamily\fontsize{11.000000}{13.200000}\selectfont \(\displaystyle 3.5\)}}%
\end{pgfscope}%
\begin{pgfscope}%
\pgfpathrectangle{\pgfqpoint{0.750000in}{0.500000in}}{\pgfqpoint{4.050000in}{4.000000in}} %
\pgfusepath{clip}%
\pgfsetbuttcap%
\pgfsetroundjoin%
\pgfsetlinewidth{0.501875pt}%
\definecolor{currentstroke}{rgb}{0.000000,0.000000,0.000000}%
\pgfsetstrokecolor{currentstroke}%
\pgfsetdash{{1.000000pt}{3.000000pt}}{0.000000pt}%
\pgfpathmoveto{\pgfqpoint{0.750000in}{4.000000in}}%
\pgfpathlineto{\pgfqpoint{4.800000in}{4.000000in}}%
\pgfusepath{stroke}%
\end{pgfscope}%
\begin{pgfscope}%
\pgfsetbuttcap%
\pgfsetroundjoin%
\definecolor{currentfill}{rgb}{0.000000,0.000000,0.000000}%
\pgfsetfillcolor{currentfill}%
\pgfsetlinewidth{0.501875pt}%
\definecolor{currentstroke}{rgb}{0.000000,0.000000,0.000000}%
\pgfsetstrokecolor{currentstroke}%
\pgfsetdash{}{0pt}%
\pgfsys@defobject{currentmarker}{\pgfqpoint{0.000000in}{0.000000in}}{\pgfqpoint{0.055556in}{0.000000in}}{%
\pgfpathmoveto{\pgfqpoint{0.000000in}{0.000000in}}%
\pgfpathlineto{\pgfqpoint{0.055556in}{0.000000in}}%
\pgfusepath{stroke,fill}%
}%
\begin{pgfscope}%
\pgfsys@transformshift{0.750000in}{4.000000in}%
\pgfsys@useobject{currentmarker}{}%
\end{pgfscope}%
\end{pgfscope}%
\begin{pgfscope}%
\pgfsetbuttcap%
\pgfsetroundjoin%
\definecolor{currentfill}{rgb}{0.000000,0.000000,0.000000}%
\pgfsetfillcolor{currentfill}%
\pgfsetlinewidth{0.501875pt}%
\definecolor{currentstroke}{rgb}{0.000000,0.000000,0.000000}%
\pgfsetstrokecolor{currentstroke}%
\pgfsetdash{}{0pt}%
\pgfsys@defobject{currentmarker}{\pgfqpoint{-0.055556in}{0.000000in}}{\pgfqpoint{0.000000in}{0.000000in}}{%
\pgfpathmoveto{\pgfqpoint{0.000000in}{0.000000in}}%
\pgfpathlineto{\pgfqpoint{-0.055556in}{0.000000in}}%
\pgfusepath{stroke,fill}%
}%
\begin{pgfscope}%
\pgfsys@transformshift{4.800000in}{4.000000in}%
\pgfsys@useobject{currentmarker}{}%
\end{pgfscope}%
\end{pgfscope}%
\begin{pgfscope}%
\pgftext[x=0.694444in,y=4.000000in,right,]{{\rmfamily\fontsize{11.000000}{13.200000}\selectfont \(\displaystyle 4.0\)}}%
\end{pgfscope}%
\begin{pgfscope}%
\pgfpathrectangle{\pgfqpoint{0.750000in}{0.500000in}}{\pgfqpoint{4.050000in}{4.000000in}} %
\pgfusepath{clip}%
\pgfsetbuttcap%
\pgfsetroundjoin%
\pgfsetlinewidth{0.501875pt}%
\definecolor{currentstroke}{rgb}{0.000000,0.000000,0.000000}%
\pgfsetstrokecolor{currentstroke}%
\pgfsetdash{{1.000000pt}{3.000000pt}}{0.000000pt}%
\pgfpathmoveto{\pgfqpoint{0.750000in}{4.500000in}}%
\pgfpathlineto{\pgfqpoint{4.800000in}{4.500000in}}%
\pgfusepath{stroke}%
\end{pgfscope}%
\begin{pgfscope}%
\pgfsetbuttcap%
\pgfsetroundjoin%
\definecolor{currentfill}{rgb}{0.000000,0.000000,0.000000}%
\pgfsetfillcolor{currentfill}%
\pgfsetlinewidth{0.501875pt}%
\definecolor{currentstroke}{rgb}{0.000000,0.000000,0.000000}%
\pgfsetstrokecolor{currentstroke}%
\pgfsetdash{}{0pt}%
\pgfsys@defobject{currentmarker}{\pgfqpoint{0.000000in}{0.000000in}}{\pgfqpoint{0.055556in}{0.000000in}}{%
\pgfpathmoveto{\pgfqpoint{0.000000in}{0.000000in}}%
\pgfpathlineto{\pgfqpoint{0.055556in}{0.000000in}}%
\pgfusepath{stroke,fill}%
}%
\begin{pgfscope}%
\pgfsys@transformshift{0.750000in}{4.500000in}%
\pgfsys@useobject{currentmarker}{}%
\end{pgfscope}%
\end{pgfscope}%
\begin{pgfscope}%
\pgfsetbuttcap%
\pgfsetroundjoin%
\definecolor{currentfill}{rgb}{0.000000,0.000000,0.000000}%
\pgfsetfillcolor{currentfill}%
\pgfsetlinewidth{0.501875pt}%
\definecolor{currentstroke}{rgb}{0.000000,0.000000,0.000000}%
\pgfsetstrokecolor{currentstroke}%
\pgfsetdash{}{0pt}%
\pgfsys@defobject{currentmarker}{\pgfqpoint{-0.055556in}{0.000000in}}{\pgfqpoint{0.000000in}{0.000000in}}{%
\pgfpathmoveto{\pgfqpoint{0.000000in}{0.000000in}}%
\pgfpathlineto{\pgfqpoint{-0.055556in}{0.000000in}}%
\pgfusepath{stroke,fill}%
}%
\begin{pgfscope}%
\pgfsys@transformshift{4.800000in}{4.500000in}%
\pgfsys@useobject{currentmarker}{}%
\end{pgfscope}%
\end{pgfscope}%
\begin{pgfscope}%
\pgftext[x=0.694444in,y=4.500000in,right,]{{\rmfamily\fontsize{11.000000}{13.200000}\selectfont \(\displaystyle 4.5\)}}%
\end{pgfscope}%
\begin{pgfscope}%
\pgftext[x=0.433852in,y=2.500000in,,bottom,rotate=90.000000]{{\rmfamily\fontsize{11.000000}{13.200000}\selectfont Bandwidth [Bytes/s]}}%
\end{pgfscope}%
\begin{pgfscope}%
\pgftext[x=0.750000in,y=4.541667in,left,base]{{\rmfamily\fontsize{11.000000}{13.200000}\selectfont \(\displaystyle \times10^{9}\)}}%
\end{pgfscope}%
\begin{pgfscope}%
\pgfsetbuttcap%
\pgfsetroundjoin%
\pgfsetlinewidth{1.003750pt}%
\definecolor{currentstroke}{rgb}{0.000000,0.000000,0.000000}%
\pgfsetstrokecolor{currentstroke}%
\pgfsetdash{}{0pt}%
\pgfpathmoveto{\pgfqpoint{0.750000in}{4.500000in}}%
\pgfpathlineto{\pgfqpoint{4.800000in}{4.500000in}}%
\pgfusepath{stroke}%
\end{pgfscope}%
\begin{pgfscope}%
\pgfsetbuttcap%
\pgfsetroundjoin%
\pgfsetlinewidth{1.003750pt}%
\definecolor{currentstroke}{rgb}{0.000000,0.000000,0.000000}%
\pgfsetstrokecolor{currentstroke}%
\pgfsetdash{}{0pt}%
\pgfpathmoveto{\pgfqpoint{4.800000in}{0.500000in}}%
\pgfpathlineto{\pgfqpoint{4.800000in}{4.500000in}}%
\pgfusepath{stroke}%
\end{pgfscope}%
\begin{pgfscope}%
\pgfsetbuttcap%
\pgfsetroundjoin%
\pgfsetlinewidth{1.003750pt}%
\definecolor{currentstroke}{rgb}{0.000000,0.000000,0.000000}%
\pgfsetstrokecolor{currentstroke}%
\pgfsetdash{}{0pt}%
\pgfpathmoveto{\pgfqpoint{0.750000in}{0.500000in}}%
\pgfpathlineto{\pgfqpoint{4.800000in}{0.500000in}}%
\pgfusepath{stroke}%
\end{pgfscope}%
\begin{pgfscope}%
\pgfsetbuttcap%
\pgfsetroundjoin%
\pgfsetlinewidth{1.003750pt}%
\definecolor{currentstroke}{rgb}{0.000000,0.000000,0.000000}%
\pgfsetstrokecolor{currentstroke}%
\pgfsetdash{}{0pt}%
\pgfpathmoveto{\pgfqpoint{0.750000in}{0.500000in}}%
\pgfpathlineto{\pgfqpoint{0.750000in}{4.500000in}}%
\pgfusepath{stroke}%
\end{pgfscope}%
\begin{pgfscope}%
\pgfsetbuttcap%
\pgfsetroundjoin%
\definecolor{currentfill}{rgb}{1.000000,1.000000,1.000000}%
\pgfsetfillcolor{currentfill}%
\pgfsetlinewidth{1.003750pt}%
\definecolor{currentstroke}{rgb}{0.000000,0.000000,0.000000}%
\pgfsetstrokecolor{currentstroke}%
\pgfsetdash{}{0pt}%
\pgfpathmoveto{\pgfqpoint{5.002500in}{2.144169in}}%
\pgfpathlineto{\pgfqpoint{6.667464in}{2.144169in}}%
\pgfpathlineto{\pgfqpoint{6.667464in}{4.500000in}}%
\pgfpathlineto{\pgfqpoint{5.002500in}{4.500000in}}%
\pgfpathlineto{\pgfqpoint{5.002500in}{2.144169in}}%
\pgfpathclose%
\pgfusepath{stroke,fill}%
\end{pgfscope}%
\begin{pgfscope}%
\pgfsetrectcap%
\pgfsetroundjoin%
\pgfsetlinewidth{2.007500pt}%
\definecolor{currentstroke}{rgb}{0.000000,0.000000,0.000000}%
\pgfsetstrokecolor{currentstroke}%
\pgfsetdash{}{0pt}%
\pgfpathmoveto{\pgfqpoint{5.130833in}{4.362500in}}%
\pgfpathlineto{\pgfqpoint{5.387500in}{4.362500in}}%
\pgfusepath{stroke}%
\end{pgfscope}%
\begin{pgfscope}%
\pgftext[x=5.589167in,y=4.298333in,left,base]{{\rmfamily\fontsize{13.200000}{15.840000}\selectfont \(\displaystyle 2^{18}\) Elements}}%
\end{pgfscope}%
\begin{pgfscope}%
\pgfsetrectcap%
\pgfsetroundjoin%
\pgfsetlinewidth{2.007500pt}%
\definecolor{currentstroke}{rgb}{0.111111,0.111111,0.111111}%
\pgfsetstrokecolor{currentstroke}%
\pgfsetdash{}{0pt}%
\pgfpathmoveto{\pgfqpoint{5.130833in}{4.106852in}}%
\pgfpathlineto{\pgfqpoint{5.387500in}{4.106852in}}%
\pgfusepath{stroke}%
\end{pgfscope}%
\begin{pgfscope}%
\pgftext[x=5.589167in,y=4.042685in,left,base]{{\rmfamily\fontsize{13.200000}{15.840000}\selectfont \(\displaystyle 2^{19}\) Elements}}%
\end{pgfscope}%
\begin{pgfscope}%
\pgfsetrectcap%
\pgfsetroundjoin%
\pgfsetlinewidth{2.007500pt}%
\definecolor{currentstroke}{rgb}{0.222222,0.222222,0.222222}%
\pgfsetstrokecolor{currentstroke}%
\pgfsetdash{}{0pt}%
\pgfpathmoveto{\pgfqpoint{5.130833in}{3.851204in}}%
\pgfpathlineto{\pgfqpoint{5.387500in}{3.851204in}}%
\pgfusepath{stroke}%
\end{pgfscope}%
\begin{pgfscope}%
\pgftext[x=5.589167in,y=3.787038in,left,base]{{\rmfamily\fontsize{13.200000}{15.840000}\selectfont \(\displaystyle 2^{20}\) Elements}}%
\end{pgfscope}%
\begin{pgfscope}%
\pgfsetrectcap%
\pgfsetroundjoin%
\pgfsetlinewidth{2.007500pt}%
\definecolor{currentstroke}{rgb}{0.333333,0.333333,0.333333}%
\pgfsetstrokecolor{currentstroke}%
\pgfsetdash{}{0pt}%
\pgfpathmoveto{\pgfqpoint{5.130833in}{3.595556in}}%
\pgfpathlineto{\pgfqpoint{5.387500in}{3.595556in}}%
\pgfusepath{stroke}%
\end{pgfscope}%
\begin{pgfscope}%
\pgftext[x=5.589167in,y=3.531390in,left,base]{{\rmfamily\fontsize{13.200000}{15.840000}\selectfont \(\displaystyle 2^{21}\) Elements}}%
\end{pgfscope}%
\begin{pgfscope}%
\pgfsetrectcap%
\pgfsetroundjoin%
\pgfsetlinewidth{2.007500pt}%
\definecolor{currentstroke}{rgb}{0.444444,0.444444,0.444444}%
\pgfsetstrokecolor{currentstroke}%
\pgfsetdash{}{0pt}%
\pgfpathmoveto{\pgfqpoint{5.130833in}{3.339908in}}%
\pgfpathlineto{\pgfqpoint{5.387500in}{3.339908in}}%
\pgfusepath{stroke}%
\end{pgfscope}%
\begin{pgfscope}%
\pgftext[x=5.589167in,y=3.275742in,left,base]{{\rmfamily\fontsize{13.200000}{15.840000}\selectfont \(\displaystyle 2^{22}\) Elements}}%
\end{pgfscope}%
\begin{pgfscope}%
\pgfsetrectcap%
\pgfsetroundjoin%
\pgfsetlinewidth{2.007500pt}%
\definecolor{currentstroke}{rgb}{0.555556,0.555556,0.555556}%
\pgfsetstrokecolor{currentstroke}%
\pgfsetdash{}{0pt}%
\pgfpathmoveto{\pgfqpoint{5.130833in}{3.084260in}}%
\pgfpathlineto{\pgfqpoint{5.387500in}{3.084260in}}%
\pgfusepath{stroke}%
\end{pgfscope}%
\begin{pgfscope}%
\pgftext[x=5.589167in,y=3.020094in,left,base]{{\rmfamily\fontsize{13.200000}{15.840000}\selectfont \(\displaystyle 2^{23}\) Elements}}%
\end{pgfscope}%
\begin{pgfscope}%
\pgfsetrectcap%
\pgfsetroundjoin%
\pgfsetlinewidth{2.007500pt}%
\definecolor{currentstroke}{rgb}{0.666667,0.666667,0.666667}%
\pgfsetstrokecolor{currentstroke}%
\pgfsetdash{}{0pt}%
\pgfpathmoveto{\pgfqpoint{5.130833in}{2.828613in}}%
\pgfpathlineto{\pgfqpoint{5.387500in}{2.828613in}}%
\pgfusepath{stroke}%
\end{pgfscope}%
\begin{pgfscope}%
\pgftext[x=5.589167in,y=2.764446in,left,base]{{\rmfamily\fontsize{13.200000}{15.840000}\selectfont \(\displaystyle 2^{24}\) Elements}}%
\end{pgfscope}%
\begin{pgfscope}%
\pgfsetrectcap%
\pgfsetroundjoin%
\pgfsetlinewidth{2.007500pt}%
\definecolor{currentstroke}{rgb}{0.777778,0.777778,0.777778}%
\pgfsetstrokecolor{currentstroke}%
\pgfsetdash{}{0pt}%
\pgfpathmoveto{\pgfqpoint{5.130833in}{2.572965in}}%
\pgfpathlineto{\pgfqpoint{5.387500in}{2.572965in}}%
\pgfusepath{stroke}%
\end{pgfscope}%
\begin{pgfscope}%
\pgftext[x=5.589167in,y=2.508798in,left,base]{{\rmfamily\fontsize{13.200000}{15.840000}\selectfont \(\displaystyle 2^{25}\) Elements}}%
\end{pgfscope}%
\begin{pgfscope}%
\pgfsetrectcap%
\pgfsetroundjoin%
\pgfsetlinewidth{2.007500pt}%
\definecolor{currentstroke}{rgb}{0.888889,0.888889,0.888889}%
\pgfsetstrokecolor{currentstroke}%
\pgfsetdash{}{0pt}%
\pgfpathmoveto{\pgfqpoint{5.130833in}{2.317317in}}%
\pgfpathlineto{\pgfqpoint{5.387500in}{2.317317in}}%
\pgfusepath{stroke}%
\end{pgfscope}%
\begin{pgfscope}%
\pgftext[x=5.589167in,y=2.253150in,left,base]{{\rmfamily\fontsize{13.200000}{15.840000}\selectfont \(\displaystyle 2^{26}\) Elements}}%
\end{pgfscope}%
\end{pgfpicture}%
\makeatother%
\endgroup%

	\label{threads_local}
	\caption{Bandwidth with increasing number of threads on home pc.}
\end{figure}

\begin{figure}
	%% Creator: Matplotlib, PGF backend
%%
%% To include the figure in your LaTeX document, write
%%   \input{<filename>.pgf}
%%
%% Make sure the required packages are loaded in your preamble
%%   \usepackage{pgf}
%%
%% Figures using additional raster images can only be included by \input if
%% they are in the same directory as the main LaTeX file. For loading figures
%% from other directories you can use the `import` package
%%   \usepackage{import}
%% and then include the figures with
%%   \import{<path to file>}{<filename>.pgf}
%%
%% Matplotlib used the following preamble
%%   \usepackage[T1]{fontenc}
%%   \usepackage{lmodern}
%%
\begingroup%
\makeatletter%
\begin{pgfpicture}%
\pgfpathrectangle{\pgfpointorigin}{\pgfqpoint{6.000000in}{5.000000in}}%
\pgfusepath{use as bounding box}%
\begin{pgfscope}%
\pgfsetbuttcap%
\pgfsetroundjoin%
\definecolor{currentfill}{rgb}{1.000000,1.000000,1.000000}%
\pgfsetfillcolor{currentfill}%
\pgfsetlinewidth{0.000000pt}%
\definecolor{currentstroke}{rgb}{1.000000,1.000000,1.000000}%
\pgfsetstrokecolor{currentstroke}%
\pgfsetdash{}{0pt}%
\pgfpathmoveto{\pgfqpoint{0.000000in}{0.000000in}}%
\pgfpathlineto{\pgfqpoint{6.000000in}{0.000000in}}%
\pgfpathlineto{\pgfqpoint{6.000000in}{5.000000in}}%
\pgfpathlineto{\pgfqpoint{0.000000in}{5.000000in}}%
\pgfpathclose%
\pgfusepath{fill}%
\end{pgfscope}%
\begin{pgfscope}%
\pgfsetbuttcap%
\pgfsetroundjoin%
\definecolor{currentfill}{rgb}{1.000000,1.000000,1.000000}%
\pgfsetfillcolor{currentfill}%
\pgfsetlinewidth{0.000000pt}%
\definecolor{currentstroke}{rgb}{0.000000,0.000000,0.000000}%
\pgfsetstrokecolor{currentstroke}%
\pgfsetstrokeopacity{0.000000}%
\pgfsetdash{}{0pt}%
\pgfpathmoveto{\pgfqpoint{0.750000in}{0.500000in}}%
\pgfpathlineto{\pgfqpoint{4.800000in}{0.500000in}}%
\pgfpathlineto{\pgfqpoint{4.800000in}{4.500000in}}%
\pgfpathlineto{\pgfqpoint{0.750000in}{4.500000in}}%
\pgfpathclose%
\pgfusepath{fill}%
\end{pgfscope}%
\begin{pgfscope}%
\pgfpathrectangle{\pgfqpoint{0.750000in}{0.500000in}}{\pgfqpoint{4.050000in}{4.000000in}} %
\pgfusepath{clip}%
\pgfsetrectcap%
\pgfsetroundjoin%
\pgfsetlinewidth{2.007500pt}%
\definecolor{currentstroke}{rgb}{0.000000,0.000000,0.000000}%
\pgfsetstrokecolor{currentstroke}%
\pgfsetdash{}{0pt}%
\pgfpathmoveto{\pgfqpoint{0.807857in}{0.941880in}}%
\pgfpathlineto{\pgfqpoint{0.865714in}{1.288428in}}%
\pgfpathlineto{\pgfqpoint{0.923571in}{1.393608in}}%
\pgfpathlineto{\pgfqpoint{0.981429in}{1.418357in}}%
\pgfpathlineto{\pgfqpoint{1.039286in}{1.395241in}}%
\pgfpathlineto{\pgfqpoint{1.097143in}{1.323159in}}%
\pgfpathlineto{\pgfqpoint{1.155000in}{1.260392in}}%
\pgfpathlineto{\pgfqpoint{1.212857in}{1.177262in}}%
\pgfpathlineto{\pgfqpoint{1.270714in}{1.131204in}}%
\pgfpathlineto{\pgfqpoint{1.328571in}{1.037633in}}%
\pgfpathlineto{\pgfqpoint{1.386429in}{1.034924in}}%
\pgfpathlineto{\pgfqpoint{1.444286in}{0.982215in}}%
\pgfpathlineto{\pgfqpoint{1.502143in}{0.945335in}}%
\pgfpathlineto{\pgfqpoint{1.560000in}{0.914254in}}%
\pgfpathlineto{\pgfqpoint{1.617857in}{0.908784in}}%
\pgfpathlineto{\pgfqpoint{1.675714in}{0.873725in}}%
\pgfpathlineto{\pgfqpoint{1.733571in}{0.855777in}}%
\pgfpathlineto{\pgfqpoint{1.791429in}{0.835143in}}%
\pgfpathlineto{\pgfqpoint{1.849286in}{0.852176in}}%
\pgfpathlineto{\pgfqpoint{1.907143in}{0.820170in}}%
\pgfpathlineto{\pgfqpoint{1.965000in}{0.794125in}}%
\pgfpathlineto{\pgfqpoint{2.022857in}{0.792433in}}%
\pgfpathlineto{\pgfqpoint{2.080714in}{0.778335in}}%
\pgfpathlineto{\pgfqpoint{2.138571in}{0.759628in}}%
\pgfpathlineto{\pgfqpoint{2.196429in}{0.749763in}}%
\pgfpathlineto{\pgfqpoint{2.254286in}{0.747396in}}%
\pgfpathlineto{\pgfqpoint{2.312143in}{0.733624in}}%
\pgfpathlineto{\pgfqpoint{2.370000in}{0.715707in}}%
\pgfpathlineto{\pgfqpoint{2.427857in}{0.711751in}}%
\pgfpathlineto{\pgfqpoint{2.485714in}{0.701018in}}%
\pgfpathlineto{\pgfqpoint{2.543571in}{0.695066in}}%
\pgfpathlineto{\pgfqpoint{2.601429in}{0.688380in}}%
\pgfpathlineto{\pgfqpoint{2.659286in}{0.678613in}}%
\pgfpathlineto{\pgfqpoint{2.717143in}{0.678655in}}%
\pgfpathlineto{\pgfqpoint{2.775000in}{0.672014in}}%
\pgfpathlineto{\pgfqpoint{2.832857in}{0.678870in}}%
\pgfpathlineto{\pgfqpoint{2.890714in}{0.679870in}}%
\pgfpathlineto{\pgfqpoint{2.948571in}{0.663243in}}%
\pgfpathlineto{\pgfqpoint{3.006429in}{0.661931in}}%
\pgfpathlineto{\pgfqpoint{3.064286in}{0.658668in}}%
\pgfpathlineto{\pgfqpoint{3.122143in}{0.651277in}}%
\pgfpathlineto{\pgfqpoint{3.180000in}{0.655025in}}%
\pgfpathlineto{\pgfqpoint{3.237857in}{0.647351in}}%
\pgfpathlineto{\pgfqpoint{3.295714in}{0.644717in}}%
\pgfpathlineto{\pgfqpoint{3.353571in}{0.632975in}}%
\pgfpathlineto{\pgfqpoint{3.411429in}{0.634435in}}%
\pgfpathlineto{\pgfqpoint{3.469286in}{0.643267in}}%
\pgfpathlineto{\pgfqpoint{3.527143in}{0.632706in}}%
\pgfpathlineto{\pgfqpoint{3.585000in}{0.635061in}}%
\pgfpathlineto{\pgfqpoint{3.642857in}{0.631560in}}%
\pgfpathlineto{\pgfqpoint{3.700714in}{0.629157in}}%
\pgfpathlineto{\pgfqpoint{3.758571in}{0.632400in}}%
\pgfpathlineto{\pgfqpoint{3.816429in}{0.617609in}}%
\pgfpathlineto{\pgfqpoint{3.874286in}{0.624834in}}%
\pgfpathlineto{\pgfqpoint{3.932143in}{0.619774in}}%
\pgfpathlineto{\pgfqpoint{3.990000in}{0.617546in}}%
\pgfpathlineto{\pgfqpoint{4.047857in}{0.621003in}}%
\pgfpathlineto{\pgfqpoint{4.105714in}{0.609821in}}%
\pgfpathlineto{\pgfqpoint{4.163571in}{0.610373in}}%
\pgfpathlineto{\pgfqpoint{4.221429in}{0.607851in}}%
\pgfpathlineto{\pgfqpoint{4.279286in}{0.608121in}}%
\pgfpathlineto{\pgfqpoint{4.337143in}{0.606885in}}%
\pgfpathlineto{\pgfqpoint{4.395000in}{0.608191in}}%
\pgfpathlineto{\pgfqpoint{4.452857in}{0.609471in}}%
\pgfpathlineto{\pgfqpoint{4.510714in}{0.602439in}}%
\pgfpathlineto{\pgfqpoint{4.568571in}{0.602774in}}%
\pgfpathlineto{\pgfqpoint{4.626429in}{0.600248in}}%
\pgfpathlineto{\pgfqpoint{4.684286in}{0.595893in}}%
\pgfpathlineto{\pgfqpoint{4.742143in}{0.595101in}}%
\pgfusepath{stroke}%
\end{pgfscope}%
\begin{pgfscope}%
\pgfpathrectangle{\pgfqpoint{0.750000in}{0.500000in}}{\pgfqpoint{4.050000in}{4.000000in}} %
\pgfusepath{clip}%
\pgfsetrectcap%
\pgfsetroundjoin%
\pgfsetlinewidth{2.007500pt}%
\definecolor{currentstroke}{rgb}{0.111111,0.111111,0.111111}%
\pgfsetstrokecolor{currentstroke}%
\pgfsetdash{}{0pt}%
\pgfpathmoveto{\pgfqpoint{0.807857in}{1.023209in}}%
\pgfpathlineto{\pgfqpoint{0.865714in}{1.414512in}}%
\pgfpathlineto{\pgfqpoint{0.923571in}{1.647191in}}%
\pgfpathlineto{\pgfqpoint{0.981429in}{1.806945in}}%
\pgfpathlineto{\pgfqpoint{1.039286in}{1.892153in}}%
\pgfpathlineto{\pgfqpoint{1.097143in}{1.872454in}}%
\pgfpathlineto{\pgfqpoint{1.155000in}{1.801201in}}%
\pgfpathlineto{\pgfqpoint{1.212857in}{1.792699in}}%
\pgfpathlineto{\pgfqpoint{1.270714in}{1.767607in}}%
\pgfpathlineto{\pgfqpoint{1.328571in}{1.671346in}}%
\pgfpathlineto{\pgfqpoint{1.386429in}{1.505847in}}%
\pgfpathlineto{\pgfqpoint{1.444286in}{1.536450in}}%
\pgfpathlineto{\pgfqpoint{1.502143in}{1.454060in}}%
\pgfpathlineto{\pgfqpoint{1.560000in}{1.429915in}}%
\pgfpathlineto{\pgfqpoint{1.617857in}{1.327482in}}%
\pgfpathlineto{\pgfqpoint{1.675714in}{1.360142in}}%
\pgfpathlineto{\pgfqpoint{1.733571in}{1.268760in}}%
\pgfpathlineto{\pgfqpoint{1.791429in}{1.268673in}}%
\pgfpathlineto{\pgfqpoint{1.849286in}{1.223073in}}%
\pgfpathlineto{\pgfqpoint{1.907143in}{1.172093in}}%
\pgfpathlineto{\pgfqpoint{1.965000in}{1.166185in}}%
\pgfpathlineto{\pgfqpoint{2.022857in}{1.162931in}}%
\pgfpathlineto{\pgfqpoint{2.080714in}{1.164673in}}%
\pgfpathlineto{\pgfqpoint{2.138571in}{1.132018in}}%
\pgfpathlineto{\pgfqpoint{2.196429in}{1.150717in}}%
\pgfpathlineto{\pgfqpoint{2.254286in}{1.073012in}}%
\pgfpathlineto{\pgfqpoint{2.312143in}{1.098617in}}%
\pgfpathlineto{\pgfqpoint{2.370000in}{1.100354in}}%
\pgfpathlineto{\pgfqpoint{2.427857in}{1.007547in}}%
\pgfpathlineto{\pgfqpoint{2.485714in}{0.990571in}}%
\pgfpathlineto{\pgfqpoint{2.543571in}{1.011613in}}%
\pgfpathlineto{\pgfqpoint{2.601429in}{0.974719in}}%
\pgfpathlineto{\pgfqpoint{2.659286in}{0.969479in}}%
\pgfpathlineto{\pgfqpoint{2.717143in}{0.948632in}}%
\pgfpathlineto{\pgfqpoint{2.775000in}{0.936025in}}%
\pgfpathlineto{\pgfqpoint{2.832857in}{0.928763in}}%
\pgfpathlineto{\pgfqpoint{2.890714in}{0.914719in}}%
\pgfpathlineto{\pgfqpoint{2.948571in}{0.913558in}}%
\pgfpathlineto{\pgfqpoint{3.006429in}{0.863562in}}%
\pgfpathlineto{\pgfqpoint{3.064286in}{0.875983in}}%
\pgfpathlineto{\pgfqpoint{3.122143in}{0.876859in}}%
\pgfpathlineto{\pgfqpoint{3.180000in}{0.842328in}}%
\pgfpathlineto{\pgfqpoint{3.237857in}{0.823118in}}%
\pgfpathlineto{\pgfqpoint{3.295714in}{0.856747in}}%
\pgfpathlineto{\pgfqpoint{3.353571in}{0.840294in}}%
\pgfpathlineto{\pgfqpoint{3.411429in}{0.824175in}}%
\pgfpathlineto{\pgfqpoint{3.469286in}{0.830754in}}%
\pgfpathlineto{\pgfqpoint{3.527143in}{0.818122in}}%
\pgfpathlineto{\pgfqpoint{3.585000in}{0.815674in}}%
\pgfpathlineto{\pgfqpoint{3.642857in}{0.798225in}}%
\pgfpathlineto{\pgfqpoint{3.700714in}{0.796190in}}%
\pgfpathlineto{\pgfqpoint{3.758571in}{0.787982in}}%
\pgfpathlineto{\pgfqpoint{3.816429in}{0.765682in}}%
\pgfpathlineto{\pgfqpoint{3.874286in}{0.781196in}}%
\pgfpathlineto{\pgfqpoint{3.932143in}{0.778461in}}%
\pgfpathlineto{\pgfqpoint{3.990000in}{0.764774in}}%
\pgfpathlineto{\pgfqpoint{4.047857in}{0.770455in}}%
\pgfpathlineto{\pgfqpoint{4.105714in}{0.763218in}}%
\pgfpathlineto{\pgfqpoint{4.163571in}{0.760045in}}%
\pgfpathlineto{\pgfqpoint{4.221429in}{0.762723in}}%
\pgfpathlineto{\pgfqpoint{4.279286in}{0.743713in}}%
\pgfpathlineto{\pgfqpoint{4.337143in}{0.735467in}}%
\pgfpathlineto{\pgfqpoint{4.395000in}{0.736770in}}%
\pgfpathlineto{\pgfqpoint{4.452857in}{0.733651in}}%
\pgfpathlineto{\pgfqpoint{4.510714in}{0.729870in}}%
\pgfpathlineto{\pgfqpoint{4.568571in}{0.716020in}}%
\pgfpathlineto{\pgfqpoint{4.626429in}{0.713612in}}%
\pgfpathlineto{\pgfqpoint{4.684286in}{0.709333in}}%
\pgfpathlineto{\pgfqpoint{4.742143in}{0.715517in}}%
\pgfusepath{stroke}%
\end{pgfscope}%
\begin{pgfscope}%
\pgfpathrectangle{\pgfqpoint{0.750000in}{0.500000in}}{\pgfqpoint{4.050000in}{4.000000in}} %
\pgfusepath{clip}%
\pgfsetrectcap%
\pgfsetroundjoin%
\pgfsetlinewidth{2.007500pt}%
\definecolor{currentstroke}{rgb}{0.222222,0.222222,0.222222}%
\pgfsetstrokecolor{currentstroke}%
\pgfsetdash{}{0pt}%
\pgfpathmoveto{\pgfqpoint{0.807857in}{1.158703in}}%
\pgfpathlineto{\pgfqpoint{0.865714in}{1.482759in}}%
\pgfpathlineto{\pgfqpoint{0.923571in}{1.837834in}}%
\pgfpathlineto{\pgfqpoint{0.981429in}{2.172891in}}%
\pgfpathlineto{\pgfqpoint{1.039286in}{2.481938in}}%
\pgfpathlineto{\pgfqpoint{1.097143in}{2.632859in}}%
\pgfpathlineto{\pgfqpoint{1.155000in}{2.598364in}}%
\pgfpathlineto{\pgfqpoint{1.212857in}{2.787257in}}%
\pgfpathlineto{\pgfqpoint{1.270714in}{2.715370in}}%
\pgfpathlineto{\pgfqpoint{1.328571in}{2.693520in}}%
\pgfpathlineto{\pgfqpoint{1.386429in}{2.646071in}}%
\pgfpathlineto{\pgfqpoint{1.444286in}{2.658060in}}%
\pgfpathlineto{\pgfqpoint{1.502143in}{2.565205in}}%
\pgfpathlineto{\pgfqpoint{1.560000in}{2.588796in}}%
\pgfpathlineto{\pgfqpoint{1.617857in}{2.491092in}}%
\pgfpathlineto{\pgfqpoint{1.675714in}{2.442429in}}%
\pgfpathlineto{\pgfqpoint{1.733571in}{2.318566in}}%
\pgfpathlineto{\pgfqpoint{1.791429in}{2.171755in}}%
\pgfpathlineto{\pgfqpoint{1.849286in}{2.238234in}}%
\pgfpathlineto{\pgfqpoint{1.907143in}{2.116055in}}%
\pgfpathlineto{\pgfqpoint{1.965000in}{2.134314in}}%
\pgfpathlineto{\pgfqpoint{2.022857in}{2.102848in}}%
\pgfpathlineto{\pgfqpoint{2.080714in}{2.004674in}}%
\pgfpathlineto{\pgfqpoint{2.138571in}{1.916282in}}%
\pgfpathlineto{\pgfqpoint{2.196429in}{1.856771in}}%
\pgfpathlineto{\pgfqpoint{2.254286in}{1.782754in}}%
\pgfpathlineto{\pgfqpoint{2.312143in}{1.800602in}}%
\pgfpathlineto{\pgfqpoint{2.370000in}{1.746311in}}%
\pgfpathlineto{\pgfqpoint{2.427857in}{1.644459in}}%
\pgfpathlineto{\pgfqpoint{2.485714in}{1.568984in}}%
\pgfpathlineto{\pgfqpoint{2.543571in}{1.613133in}}%
\pgfpathlineto{\pgfqpoint{2.601429in}{1.556670in}}%
\pgfpathlineto{\pgfqpoint{2.659286in}{1.580039in}}%
\pgfpathlineto{\pgfqpoint{2.717143in}{1.541171in}}%
\pgfpathlineto{\pgfqpoint{2.775000in}{1.492370in}}%
\pgfpathlineto{\pgfqpoint{2.832857in}{1.478982in}}%
\pgfpathlineto{\pgfqpoint{2.890714in}{1.482026in}}%
\pgfpathlineto{\pgfqpoint{2.948571in}{1.408957in}}%
\pgfpathlineto{\pgfqpoint{3.006429in}{1.375197in}}%
\pgfpathlineto{\pgfqpoint{3.064286in}{1.372576in}}%
\pgfpathlineto{\pgfqpoint{3.122143in}{1.365979in}}%
\pgfpathlineto{\pgfqpoint{3.180000in}{1.326811in}}%
\pgfpathlineto{\pgfqpoint{3.237857in}{1.312413in}}%
\pgfpathlineto{\pgfqpoint{3.295714in}{1.281344in}}%
\pgfpathlineto{\pgfqpoint{3.353571in}{1.308255in}}%
\pgfpathlineto{\pgfqpoint{3.411429in}{1.273340in}}%
\pgfpathlineto{\pgfqpoint{3.469286in}{1.250765in}}%
\pgfpathlineto{\pgfqpoint{3.527143in}{1.220450in}}%
\pgfpathlineto{\pgfqpoint{3.585000in}{1.227273in}}%
\pgfpathlineto{\pgfqpoint{3.642857in}{1.159403in}}%
\pgfpathlineto{\pgfqpoint{3.700714in}{1.181312in}}%
\pgfpathlineto{\pgfqpoint{3.758571in}{1.157601in}}%
\pgfpathlineto{\pgfqpoint{3.816429in}{1.167164in}}%
\pgfpathlineto{\pgfqpoint{3.874286in}{1.149579in}}%
\pgfpathlineto{\pgfqpoint{3.932143in}{1.127460in}}%
\pgfpathlineto{\pgfqpoint{3.990000in}{1.146299in}}%
\pgfpathlineto{\pgfqpoint{4.047857in}{1.132271in}}%
\pgfpathlineto{\pgfqpoint{4.105714in}{1.085238in}}%
\pgfpathlineto{\pgfqpoint{4.163571in}{1.062070in}}%
\pgfpathlineto{\pgfqpoint{4.221429in}{1.074897in}}%
\pgfpathlineto{\pgfqpoint{4.279286in}{1.066613in}}%
\pgfpathlineto{\pgfqpoint{4.337143in}{1.041584in}}%
\pgfpathlineto{\pgfqpoint{4.395000in}{1.032091in}}%
\pgfpathlineto{\pgfqpoint{4.452857in}{1.025398in}}%
\pgfpathlineto{\pgfqpoint{4.510714in}{1.008911in}}%
\pgfpathlineto{\pgfqpoint{4.568571in}{1.025384in}}%
\pgfpathlineto{\pgfqpoint{4.626429in}{0.985937in}}%
\pgfpathlineto{\pgfqpoint{4.684286in}{1.002989in}}%
\pgfpathlineto{\pgfqpoint{4.742143in}{0.998666in}}%
\pgfusepath{stroke}%
\end{pgfscope}%
\begin{pgfscope}%
\pgfpathrectangle{\pgfqpoint{0.750000in}{0.500000in}}{\pgfqpoint{4.050000in}{4.000000in}} %
\pgfusepath{clip}%
\pgfsetrectcap%
\pgfsetroundjoin%
\pgfsetlinewidth{2.007500pt}%
\definecolor{currentstroke}{rgb}{0.333333,0.333333,0.333333}%
\pgfsetstrokecolor{currentstroke}%
\pgfsetdash{}{0pt}%
\pgfpathmoveto{\pgfqpoint{0.807857in}{1.465158in}}%
\pgfpathlineto{\pgfqpoint{0.865714in}{1.656396in}}%
\pgfpathlineto{\pgfqpoint{0.923571in}{1.990250in}}%
\pgfpathlineto{\pgfqpoint{0.981429in}{2.384323in}}%
\pgfpathlineto{\pgfqpoint{1.039286in}{2.715695in}}%
\pgfpathlineto{\pgfqpoint{1.097143in}{2.971780in}}%
\pgfpathlineto{\pgfqpoint{1.155000in}{3.184607in}}%
\pgfpathlineto{\pgfqpoint{1.212857in}{3.403005in}}%
\pgfpathlineto{\pgfqpoint{1.270714in}{3.564067in}}%
\pgfpathlineto{\pgfqpoint{1.328571in}{3.432640in}}%
\pgfpathlineto{\pgfqpoint{1.386429in}{3.365647in}}%
\pgfpathlineto{\pgfqpoint{1.444286in}{3.442335in}}%
\pgfpathlineto{\pgfqpoint{1.502143in}{3.371489in}}%
\pgfpathlineto{\pgfqpoint{1.560000in}{3.395894in}}%
\pgfpathlineto{\pgfqpoint{1.617857in}{3.335307in}}%
\pgfpathlineto{\pgfqpoint{1.675714in}{3.464857in}}%
\pgfpathlineto{\pgfqpoint{1.733571in}{3.394309in}}%
\pgfpathlineto{\pgfqpoint{1.791429in}{3.212708in}}%
\pgfpathlineto{\pgfqpoint{1.849286in}{3.345706in}}%
\pgfpathlineto{\pgfqpoint{1.907143in}{3.343225in}}%
\pgfpathlineto{\pgfqpoint{1.965000in}{3.290385in}}%
\pgfpathlineto{\pgfqpoint{2.022857in}{3.163534in}}%
\pgfpathlineto{\pgfqpoint{2.080714in}{3.084883in}}%
\pgfpathlineto{\pgfqpoint{2.138571in}{2.976461in}}%
\pgfpathlineto{\pgfqpoint{2.196429in}{2.942384in}}%
\pgfpathlineto{\pgfqpoint{2.254286in}{2.886532in}}%
\pgfpathlineto{\pgfqpoint{2.312143in}{2.818610in}}%
\pgfpathlineto{\pgfqpoint{2.370000in}{2.863143in}}%
\pgfpathlineto{\pgfqpoint{2.427857in}{2.730975in}}%
\pgfpathlineto{\pgfqpoint{2.485714in}{2.663083in}}%
\pgfpathlineto{\pgfqpoint{2.543571in}{2.726671in}}%
\pgfpathlineto{\pgfqpoint{2.601429in}{2.559498in}}%
\pgfpathlineto{\pgfqpoint{2.659286in}{2.450314in}}%
\pgfpathlineto{\pgfqpoint{2.717143in}{2.424021in}}%
\pgfpathlineto{\pgfqpoint{2.775000in}{2.392857in}}%
\pgfpathlineto{\pgfqpoint{2.832857in}{2.362701in}}%
\pgfpathlineto{\pgfqpoint{2.890714in}{2.323080in}}%
\pgfpathlineto{\pgfqpoint{2.948571in}{2.279541in}}%
\pgfpathlineto{\pgfqpoint{3.006429in}{2.274707in}}%
\pgfpathlineto{\pgfqpoint{3.064286in}{2.157576in}}%
\pgfpathlineto{\pgfqpoint{3.122143in}{2.188733in}}%
\pgfpathlineto{\pgfqpoint{3.180000in}{2.119837in}}%
\pgfpathlineto{\pgfqpoint{3.237857in}{2.085377in}}%
\pgfpathlineto{\pgfqpoint{3.295714in}{2.042750in}}%
\pgfpathlineto{\pgfqpoint{3.353571in}{2.061341in}}%
\pgfpathlineto{\pgfqpoint{3.411429in}{1.935405in}}%
\pgfpathlineto{\pgfqpoint{3.469286in}{1.953339in}}%
\pgfpathlineto{\pgfqpoint{3.527143in}{1.882771in}}%
\pgfpathlineto{\pgfqpoint{3.585000in}{1.846369in}}%
\pgfpathlineto{\pgfqpoint{3.642857in}{1.914759in}}%
\pgfpathlineto{\pgfqpoint{3.700714in}{1.842903in}}%
\pgfpathlineto{\pgfqpoint{3.758571in}{1.782326in}}%
\pgfpathlineto{\pgfqpoint{3.816429in}{1.761771in}}%
\pgfpathlineto{\pgfqpoint{3.874286in}{1.802686in}}%
\pgfpathlineto{\pgfqpoint{3.932143in}{1.713552in}}%
\pgfpathlineto{\pgfqpoint{3.990000in}{1.671650in}}%
\pgfpathlineto{\pgfqpoint{4.047857in}{1.620559in}}%
\pgfpathlineto{\pgfqpoint{4.105714in}{1.703858in}}%
\pgfpathlineto{\pgfqpoint{4.163571in}{1.630565in}}%
\pgfpathlineto{\pgfqpoint{4.221429in}{1.599857in}}%
\pgfpathlineto{\pgfqpoint{4.279286in}{1.551913in}}%
\pgfpathlineto{\pgfqpoint{4.337143in}{1.568536in}}%
\pgfpathlineto{\pgfqpoint{4.395000in}{1.564933in}}%
\pgfpathlineto{\pgfqpoint{4.452857in}{1.575862in}}%
\pgfpathlineto{\pgfqpoint{4.510714in}{1.537880in}}%
\pgfpathlineto{\pgfqpoint{4.568571in}{1.498549in}}%
\pgfpathlineto{\pgfqpoint{4.626429in}{1.487309in}}%
\pgfpathlineto{\pgfqpoint{4.684286in}{1.443501in}}%
\pgfpathlineto{\pgfqpoint{4.742143in}{1.484202in}}%
\pgfusepath{stroke}%
\end{pgfscope}%
\begin{pgfscope}%
\pgfpathrectangle{\pgfqpoint{0.750000in}{0.500000in}}{\pgfqpoint{4.050000in}{4.000000in}} %
\pgfusepath{clip}%
\pgfsetrectcap%
\pgfsetroundjoin%
\pgfsetlinewidth{2.007500pt}%
\definecolor{currentstroke}{rgb}{0.444444,0.444444,0.444444}%
\pgfsetstrokecolor{currentstroke}%
\pgfsetdash{}{0pt}%
\pgfpathmoveto{\pgfqpoint{0.807857in}{1.493791in}}%
\pgfpathlineto{\pgfqpoint{0.865714in}{1.864117in}}%
\pgfpathlineto{\pgfqpoint{0.923571in}{2.082213in}}%
\pgfpathlineto{\pgfqpoint{0.981429in}{2.516372in}}%
\pgfpathlineto{\pgfqpoint{1.039286in}{2.902082in}}%
\pgfpathlineto{\pgfqpoint{1.097143in}{3.253062in}}%
\pgfpathlineto{\pgfqpoint{1.155000in}{3.605411in}}%
\pgfpathlineto{\pgfqpoint{1.212857in}{3.943332in}}%
\pgfpathlineto{\pgfqpoint{1.270714in}{4.124347in}}%
\pgfpathlineto{\pgfqpoint{1.328571in}{3.838256in}}%
\pgfpathlineto{\pgfqpoint{1.386429in}{3.699273in}}%
\pgfpathlineto{\pgfqpoint{1.444286in}{3.852100in}}%
\pgfpathlineto{\pgfqpoint{1.502143in}{3.891135in}}%
\pgfpathlineto{\pgfqpoint{1.560000in}{3.917219in}}%
\pgfpathlineto{\pgfqpoint{1.617857in}{4.031343in}}%
\pgfpathlineto{\pgfqpoint{1.675714in}{4.153137in}}%
\pgfpathlineto{\pgfqpoint{1.733571in}{4.128110in}}%
\pgfpathlineto{\pgfqpoint{1.791429in}{4.058816in}}%
\pgfpathlineto{\pgfqpoint{1.849286in}{3.891447in}}%
\pgfpathlineto{\pgfqpoint{1.907143in}{3.923428in}}%
\pgfpathlineto{\pgfqpoint{1.965000in}{3.952538in}}%
\pgfpathlineto{\pgfqpoint{2.022857in}{3.883845in}}%
\pgfpathlineto{\pgfqpoint{2.080714in}{3.870654in}}%
\pgfpathlineto{\pgfqpoint{2.138571in}{4.021998in}}%
\pgfpathlineto{\pgfqpoint{2.196429in}{3.886486in}}%
\pgfpathlineto{\pgfqpoint{2.254286in}{3.872522in}}%
\pgfpathlineto{\pgfqpoint{2.312143in}{3.907162in}}%
\pgfpathlineto{\pgfqpoint{2.370000in}{3.823535in}}%
\pgfpathlineto{\pgfqpoint{2.427857in}{3.827417in}}%
\pgfpathlineto{\pgfqpoint{2.485714in}{3.919845in}}%
\pgfpathlineto{\pgfqpoint{2.543571in}{3.932779in}}%
\pgfpathlineto{\pgfqpoint{2.601429in}{3.905497in}}%
\pgfpathlineto{\pgfqpoint{2.659286in}{3.765273in}}%
\pgfpathlineto{\pgfqpoint{2.717143in}{3.666407in}}%
\pgfpathlineto{\pgfqpoint{2.775000in}{3.643607in}}%
\pgfpathlineto{\pgfqpoint{2.832857in}{3.672747in}}%
\pgfpathlineto{\pgfqpoint{2.890714in}{3.638640in}}%
\pgfpathlineto{\pgfqpoint{2.948571in}{3.458842in}}%
\pgfpathlineto{\pgfqpoint{3.006429in}{3.528181in}}%
\pgfpathlineto{\pgfqpoint{3.064286in}{3.401479in}}%
\pgfpathlineto{\pgfqpoint{3.122143in}{3.327846in}}%
\pgfpathlineto{\pgfqpoint{3.180000in}{3.360589in}}%
\pgfpathlineto{\pgfqpoint{3.237857in}{3.317151in}}%
\pgfpathlineto{\pgfqpoint{3.295714in}{3.203894in}}%
\pgfpathlineto{\pgfqpoint{3.353571in}{3.212613in}}%
\pgfpathlineto{\pgfqpoint{3.411429in}{3.181465in}}%
\pgfpathlineto{\pgfqpoint{3.469286in}{3.087069in}}%
\pgfpathlineto{\pgfqpoint{3.527143in}{3.071857in}}%
\pgfpathlineto{\pgfqpoint{3.585000in}{3.070061in}}%
\pgfpathlineto{\pgfqpoint{3.642857in}{3.061807in}}%
\pgfpathlineto{\pgfqpoint{3.700714in}{2.951646in}}%
\pgfpathlineto{\pgfqpoint{3.758571in}{2.948903in}}%
\pgfpathlineto{\pgfqpoint{3.816429in}{2.876857in}}%
\pgfpathlineto{\pgfqpoint{3.874286in}{2.831681in}}%
\pgfpathlineto{\pgfqpoint{3.932143in}{2.852200in}}%
\pgfpathlineto{\pgfqpoint{3.990000in}{2.680004in}}%
\pgfpathlineto{\pgfqpoint{4.047857in}{2.739122in}}%
\pgfpathlineto{\pgfqpoint{4.105714in}{2.647707in}}%
\pgfpathlineto{\pgfqpoint{4.163571in}{2.542172in}}%
\pgfpathlineto{\pgfqpoint{4.221429in}{2.591972in}}%
\pgfpathlineto{\pgfqpoint{4.279286in}{2.620159in}}%
\pgfpathlineto{\pgfqpoint{4.337143in}{2.582941in}}%
\pgfpathlineto{\pgfqpoint{4.395000in}{2.516950in}}%
\pgfpathlineto{\pgfqpoint{4.452857in}{2.509512in}}%
\pgfpathlineto{\pgfqpoint{4.510714in}{2.501322in}}%
\pgfpathlineto{\pgfqpoint{4.568571in}{2.486025in}}%
\pgfpathlineto{\pgfqpoint{4.626429in}{2.468565in}}%
\pgfpathlineto{\pgfqpoint{4.684286in}{2.392295in}}%
\pgfpathlineto{\pgfqpoint{4.742143in}{2.337951in}}%
\pgfusepath{stroke}%
\end{pgfscope}%
\begin{pgfscope}%
\pgfpathrectangle{\pgfqpoint{0.750000in}{0.500000in}}{\pgfqpoint{4.050000in}{4.000000in}} %
\pgfusepath{clip}%
\pgfsetrectcap%
\pgfsetroundjoin%
\pgfsetlinewidth{2.007500pt}%
\definecolor{currentstroke}{rgb}{0.555556,0.555556,0.555556}%
\pgfsetstrokecolor{currentstroke}%
\pgfsetdash{}{0pt}%
\pgfpathmoveto{\pgfqpoint{0.807857in}{1.581743in}}%
\pgfpathlineto{\pgfqpoint{0.865714in}{2.048551in}}%
\pgfpathlineto{\pgfqpoint{0.923571in}{2.401641in}}%
\pgfpathlineto{\pgfqpoint{0.981429in}{2.757106in}}%
\pgfpathlineto{\pgfqpoint{1.039286in}{3.177299in}}%
\pgfpathlineto{\pgfqpoint{1.097143in}{3.558352in}}%
\pgfpathlineto{\pgfqpoint{1.155000in}{3.873022in}}%
\pgfpathlineto{\pgfqpoint{1.212857in}{4.162775in}}%
\pgfpathlineto{\pgfqpoint{1.270714in}{4.163231in}}%
\pgfpathlineto{\pgfqpoint{1.328571in}{4.004081in}}%
\pgfpathlineto{\pgfqpoint{1.386429in}{3.839452in}}%
\pgfpathlineto{\pgfqpoint{1.444286in}{3.915312in}}%
\pgfpathlineto{\pgfqpoint{1.502143in}{4.008272in}}%
\pgfpathlineto{\pgfqpoint{1.560000in}{4.103380in}}%
\pgfpathlineto{\pgfqpoint{1.617857in}{4.200826in}}%
\pgfpathlineto{\pgfqpoint{1.675714in}{4.265010in}}%
\pgfpathlineto{\pgfqpoint{1.733571in}{4.262252in}}%
\pgfpathlineto{\pgfqpoint{1.791429in}{4.111308in}}%
\pgfpathlineto{\pgfqpoint{1.849286in}{4.053629in}}%
\pgfpathlineto{\pgfqpoint{1.907143in}{4.131388in}}%
\pgfpathlineto{\pgfqpoint{1.965000in}{4.095047in}}%
\pgfpathlineto{\pgfqpoint{2.022857in}{4.199279in}}%
\pgfpathlineto{\pgfqpoint{2.080714in}{4.251472in}}%
\pgfpathlineto{\pgfqpoint{2.138571in}{4.227515in}}%
\pgfpathlineto{\pgfqpoint{2.196429in}{4.182031in}}%
\pgfpathlineto{\pgfqpoint{2.254286in}{4.068292in}}%
\pgfpathlineto{\pgfqpoint{2.312143in}{3.991825in}}%
\pgfpathlineto{\pgfqpoint{2.370000in}{4.012507in}}%
\pgfpathlineto{\pgfqpoint{2.427857in}{4.073734in}}%
\pgfpathlineto{\pgfqpoint{2.485714in}{4.144483in}}%
\pgfpathlineto{\pgfqpoint{2.543571in}{4.099355in}}%
\pgfpathlineto{\pgfqpoint{2.601429in}{4.074200in}}%
\pgfpathlineto{\pgfqpoint{2.659286in}{4.061836in}}%
\pgfpathlineto{\pgfqpoint{2.717143in}{3.959333in}}%
\pgfpathlineto{\pgfqpoint{2.775000in}{3.876981in}}%
\pgfpathlineto{\pgfqpoint{2.832857in}{3.977439in}}%
\pgfpathlineto{\pgfqpoint{2.890714in}{3.863085in}}%
\pgfpathlineto{\pgfqpoint{2.948571in}{3.848695in}}%
\pgfpathlineto{\pgfqpoint{3.006429in}{3.991118in}}%
\pgfpathlineto{\pgfqpoint{3.064286in}{3.912804in}}%
\pgfpathlineto{\pgfqpoint{3.122143in}{3.941852in}}%
\pgfpathlineto{\pgfqpoint{3.180000in}{3.957423in}}%
\pgfpathlineto{\pgfqpoint{3.237857in}{3.926206in}}%
\pgfpathlineto{\pgfqpoint{3.295714in}{3.913442in}}%
\pgfpathlineto{\pgfqpoint{3.353571in}{3.961795in}}%
\pgfpathlineto{\pgfqpoint{3.411429in}{4.003137in}}%
\pgfpathlineto{\pgfqpoint{3.469286in}{3.963175in}}%
\pgfpathlineto{\pgfqpoint{3.527143in}{4.024139in}}%
\pgfpathlineto{\pgfqpoint{3.585000in}{3.917364in}}%
\pgfpathlineto{\pgfqpoint{3.642857in}{3.908932in}}%
\pgfpathlineto{\pgfqpoint{3.700714in}{3.974504in}}%
\pgfpathlineto{\pgfqpoint{3.758571in}{3.854019in}}%
\pgfpathlineto{\pgfqpoint{3.816429in}{3.896746in}}%
\pgfpathlineto{\pgfqpoint{3.874286in}{3.888182in}}%
\pgfpathlineto{\pgfqpoint{3.932143in}{3.742108in}}%
\pgfpathlineto{\pgfqpoint{3.990000in}{3.823156in}}%
\pgfpathlineto{\pgfqpoint{4.047857in}{3.846812in}}%
\pgfpathlineto{\pgfqpoint{4.105714in}{3.705004in}}%
\pgfpathlineto{\pgfqpoint{4.163571in}{3.640413in}}%
\pgfpathlineto{\pgfqpoint{4.221429in}{3.740489in}}%
\pgfpathlineto{\pgfqpoint{4.279286in}{3.650536in}}%
\pgfpathlineto{\pgfqpoint{4.337143in}{3.551939in}}%
\pgfpathlineto{\pgfqpoint{4.395000in}{3.679909in}}%
\pgfpathlineto{\pgfqpoint{4.452857in}{3.704520in}}%
\pgfpathlineto{\pgfqpoint{4.510714in}{3.515043in}}%
\pgfpathlineto{\pgfqpoint{4.568571in}{3.604987in}}%
\pgfpathlineto{\pgfqpoint{4.626429in}{3.492651in}}%
\pgfpathlineto{\pgfqpoint{4.684286in}{3.616793in}}%
\pgfpathlineto{\pgfqpoint{4.742143in}{3.575832in}}%
\pgfusepath{stroke}%
\end{pgfscope}%
\begin{pgfscope}%
\pgfpathrectangle{\pgfqpoint{0.750000in}{0.500000in}}{\pgfqpoint{4.050000in}{4.000000in}} %
\pgfusepath{clip}%
\pgfsetrectcap%
\pgfsetroundjoin%
\pgfsetlinewidth{2.007500pt}%
\definecolor{currentstroke}{rgb}{0.666667,0.666667,0.666667}%
\pgfsetstrokecolor{currentstroke}%
\pgfsetdash{}{0pt}%
\pgfpathmoveto{\pgfqpoint{0.807857in}{1.666935in}}%
\pgfpathlineto{\pgfqpoint{0.865714in}{2.186416in}}%
\pgfpathlineto{\pgfqpoint{0.923571in}{2.437474in}}%
\pgfpathlineto{\pgfqpoint{0.981429in}{2.954279in}}%
\pgfpathlineto{\pgfqpoint{1.039286in}{3.451447in}}%
\pgfpathlineto{\pgfqpoint{1.097143in}{3.833509in}}%
\pgfpathlineto{\pgfqpoint{1.155000in}{4.072581in}}%
\pgfpathlineto{\pgfqpoint{1.212857in}{4.308982in}}%
\pgfpathlineto{\pgfqpoint{1.270714in}{4.198633in}}%
\pgfpathlineto{\pgfqpoint{1.328571in}{3.984954in}}%
\pgfpathlineto{\pgfqpoint{1.386429in}{3.842421in}}%
\pgfpathlineto{\pgfqpoint{1.444286in}{3.970905in}}%
\pgfpathlineto{\pgfqpoint{1.502143in}{4.088878in}}%
\pgfpathlineto{\pgfqpoint{1.560000in}{4.183631in}}%
\pgfpathlineto{\pgfqpoint{1.617857in}{4.266556in}}%
\pgfpathlineto{\pgfqpoint{1.675714in}{4.339099in}}%
\pgfpathlineto{\pgfqpoint{1.733571in}{4.199689in}}%
\pgfpathlineto{\pgfqpoint{1.791429in}{4.003006in}}%
\pgfpathlineto{\pgfqpoint{1.849286in}{3.926811in}}%
\pgfpathlineto{\pgfqpoint{1.907143in}{4.015308in}}%
\pgfpathlineto{\pgfqpoint{1.965000in}{4.102521in}}%
\pgfpathlineto{\pgfqpoint{2.022857in}{4.167441in}}%
\pgfpathlineto{\pgfqpoint{2.080714in}{4.189512in}}%
\pgfpathlineto{\pgfqpoint{2.138571in}{4.228581in}}%
\pgfpathlineto{\pgfqpoint{2.196429in}{4.148973in}}%
\pgfpathlineto{\pgfqpoint{2.254286in}{3.984518in}}%
\pgfpathlineto{\pgfqpoint{2.312143in}{3.928827in}}%
\pgfpathlineto{\pgfqpoint{2.370000in}{4.001561in}}%
\pgfpathlineto{\pgfqpoint{2.427857in}{4.040120in}}%
\pgfpathlineto{\pgfqpoint{2.485714in}{4.094178in}}%
\pgfpathlineto{\pgfqpoint{2.543571in}{4.101908in}}%
\pgfpathlineto{\pgfqpoint{2.601429in}{4.117479in}}%
\pgfpathlineto{\pgfqpoint{2.659286in}{4.020621in}}%
\pgfpathlineto{\pgfqpoint{2.717143in}{3.962618in}}%
\pgfpathlineto{\pgfqpoint{2.775000in}{3.935106in}}%
\pgfpathlineto{\pgfqpoint{2.832857in}{3.959865in}}%
\pgfpathlineto{\pgfqpoint{2.890714in}{3.969247in}}%
\pgfpathlineto{\pgfqpoint{2.948571in}{3.983092in}}%
\pgfpathlineto{\pgfqpoint{3.006429in}{4.003665in}}%
\pgfpathlineto{\pgfqpoint{3.064286in}{4.002952in}}%
\pgfpathlineto{\pgfqpoint{3.122143in}{3.936895in}}%
\pgfpathlineto{\pgfqpoint{3.180000in}{3.882960in}}%
\pgfpathlineto{\pgfqpoint{3.237857in}{3.889912in}}%
\pgfpathlineto{\pgfqpoint{3.295714in}{3.894188in}}%
\pgfpathlineto{\pgfqpoint{3.353571in}{3.892203in}}%
\pgfpathlineto{\pgfqpoint{3.411429in}{3.905225in}}%
\pgfpathlineto{\pgfqpoint{3.469286in}{3.896510in}}%
\pgfpathlineto{\pgfqpoint{3.527143in}{3.920953in}}%
\pgfpathlineto{\pgfqpoint{3.585000in}{3.934056in}}%
\pgfpathlineto{\pgfqpoint{3.642857in}{3.927426in}}%
\pgfpathlineto{\pgfqpoint{3.700714in}{3.908447in}}%
\pgfpathlineto{\pgfqpoint{3.758571in}{3.911067in}}%
\pgfpathlineto{\pgfqpoint{3.816429in}{3.941534in}}%
\pgfpathlineto{\pgfqpoint{3.874286in}{3.929958in}}%
\pgfpathlineto{\pgfqpoint{3.932143in}{3.922763in}}%
\pgfpathlineto{\pgfqpoint{3.990000in}{3.966885in}}%
\pgfpathlineto{\pgfqpoint{4.047857in}{3.923297in}}%
\pgfpathlineto{\pgfqpoint{4.105714in}{3.931081in}}%
\pgfpathlineto{\pgfqpoint{4.163571in}{3.895837in}}%
\pgfpathlineto{\pgfqpoint{4.221429in}{3.927717in}}%
\pgfpathlineto{\pgfqpoint{4.279286in}{3.888107in}}%
\pgfpathlineto{\pgfqpoint{4.337143in}{3.900332in}}%
\pgfpathlineto{\pgfqpoint{4.395000in}{3.910297in}}%
\pgfpathlineto{\pgfqpoint{4.452857in}{3.867335in}}%
\pgfpathlineto{\pgfqpoint{4.510714in}{3.900316in}}%
\pgfpathlineto{\pgfqpoint{4.568571in}{3.904007in}}%
\pgfpathlineto{\pgfqpoint{4.626429in}{3.907124in}}%
\pgfpathlineto{\pgfqpoint{4.684286in}{3.940155in}}%
\pgfpathlineto{\pgfqpoint{4.742143in}{3.921708in}}%
\pgfusepath{stroke}%
\end{pgfscope}%
\begin{pgfscope}%
\pgfpathrectangle{\pgfqpoint{0.750000in}{0.500000in}}{\pgfqpoint{4.050000in}{4.000000in}} %
\pgfusepath{clip}%
\pgfsetrectcap%
\pgfsetroundjoin%
\pgfsetlinewidth{2.007500pt}%
\definecolor{currentstroke}{rgb}{0.777778,0.777778,0.777778}%
\pgfsetstrokecolor{currentstroke}%
\pgfsetdash{}{0pt}%
\pgfpathmoveto{\pgfqpoint{0.807857in}{1.736363in}}%
\pgfpathlineto{\pgfqpoint{0.865714in}{2.314772in}}%
\pgfpathlineto{\pgfqpoint{0.923571in}{2.487515in}}%
\pgfpathlineto{\pgfqpoint{0.981429in}{3.065746in}}%
\pgfpathlineto{\pgfqpoint{1.039286in}{3.615833in}}%
\pgfpathlineto{\pgfqpoint{1.097143in}{3.989226in}}%
\pgfpathlineto{\pgfqpoint{1.155000in}{4.148721in}}%
\pgfpathlineto{\pgfqpoint{1.212857in}{4.339089in}}%
\pgfpathlineto{\pgfqpoint{1.270714in}{4.181041in}}%
\pgfpathlineto{\pgfqpoint{1.328571in}{3.953299in}}%
\pgfpathlineto{\pgfqpoint{1.386429in}{3.827532in}}%
\pgfpathlineto{\pgfqpoint{1.444286in}{3.982401in}}%
\pgfpathlineto{\pgfqpoint{1.502143in}{4.139977in}}%
\pgfpathlineto{\pgfqpoint{1.560000in}{4.210510in}}%
\pgfpathlineto{\pgfqpoint{1.617857in}{4.286067in}}%
\pgfpathlineto{\pgfqpoint{1.675714in}{4.349458in}}%
\pgfpathlineto{\pgfqpoint{1.733571in}{4.209195in}}%
\pgfpathlineto{\pgfqpoint{1.791429in}{4.009223in}}%
\pgfpathlineto{\pgfqpoint{1.849286in}{3.923253in}}%
\pgfpathlineto{\pgfqpoint{1.907143in}{4.050892in}}%
\pgfpathlineto{\pgfqpoint{1.965000in}{4.154291in}}%
\pgfpathlineto{\pgfqpoint{2.022857in}{4.206711in}}%
\pgfpathlineto{\pgfqpoint{2.080714in}{4.251070in}}%
\pgfpathlineto{\pgfqpoint{2.138571in}{4.289984in}}%
\pgfpathlineto{\pgfqpoint{2.196429in}{4.187285in}}%
\pgfpathlineto{\pgfqpoint{2.254286in}{4.014997in}}%
\pgfpathlineto{\pgfqpoint{2.312143in}{3.962450in}}%
\pgfpathlineto{\pgfqpoint{2.370000in}{4.062218in}}%
\pgfpathlineto{\pgfqpoint{2.427857in}{4.097108in}}%
\pgfpathlineto{\pgfqpoint{2.485714in}{4.160067in}}%
\pgfpathlineto{\pgfqpoint{2.543571in}{4.178932in}}%
\pgfpathlineto{\pgfqpoint{2.601429in}{4.234185in}}%
\pgfpathlineto{\pgfqpoint{2.659286in}{4.130600in}}%
\pgfpathlineto{\pgfqpoint{2.717143in}{4.025531in}}%
\pgfpathlineto{\pgfqpoint{2.775000in}{3.991046in}}%
\pgfpathlineto{\pgfqpoint{2.832857in}{4.024996in}}%
\pgfpathlineto{\pgfqpoint{2.890714in}{4.059852in}}%
\pgfpathlineto{\pgfqpoint{2.948571in}{4.090550in}}%
\pgfpathlineto{\pgfqpoint{3.006429in}{4.115016in}}%
\pgfpathlineto{\pgfqpoint{3.064286in}{4.095913in}}%
\pgfpathlineto{\pgfqpoint{3.122143in}{4.048565in}}%
\pgfpathlineto{\pgfqpoint{3.180000in}{3.976693in}}%
\pgfpathlineto{\pgfqpoint{3.237857in}{3.965513in}}%
\pgfpathlineto{\pgfqpoint{3.295714in}{3.990375in}}%
\pgfpathlineto{\pgfqpoint{3.353571in}{3.998042in}}%
\pgfpathlineto{\pgfqpoint{3.411429in}{4.021594in}}%
\pgfpathlineto{\pgfqpoint{3.469286in}{4.034005in}}%
\pgfpathlineto{\pgfqpoint{3.527143in}{4.055708in}}%
\pgfpathlineto{\pgfqpoint{3.585000in}{4.066828in}}%
\pgfpathlineto{\pgfqpoint{3.642857in}{4.071389in}}%
\pgfpathlineto{\pgfqpoint{3.700714in}{4.098962in}}%
\pgfpathlineto{\pgfqpoint{3.758571in}{4.078249in}}%
\pgfpathlineto{\pgfqpoint{3.816429in}{4.070105in}}%
\pgfpathlineto{\pgfqpoint{3.874286in}{4.098149in}}%
\pgfpathlineto{\pgfqpoint{3.932143in}{4.109166in}}%
\pgfpathlineto{\pgfqpoint{3.990000in}{4.150674in}}%
\pgfpathlineto{\pgfqpoint{4.047857in}{4.143579in}}%
\pgfpathlineto{\pgfqpoint{4.105714in}{4.143245in}}%
\pgfpathlineto{\pgfqpoint{4.163571in}{4.123992in}}%
\pgfpathlineto{\pgfqpoint{4.221429in}{4.155834in}}%
\pgfpathlineto{\pgfqpoint{4.279286in}{4.194946in}}%
\pgfpathlineto{\pgfqpoint{4.337143in}{4.152171in}}%
\pgfpathlineto{\pgfqpoint{4.395000in}{4.142131in}}%
\pgfpathlineto{\pgfqpoint{4.452857in}{4.208735in}}%
\pgfpathlineto{\pgfqpoint{4.510714in}{4.202798in}}%
\pgfpathlineto{\pgfqpoint{4.568571in}{4.199945in}}%
\pgfpathlineto{\pgfqpoint{4.626429in}{4.234579in}}%
\pgfpathlineto{\pgfqpoint{4.684286in}{4.182625in}}%
\pgfpathlineto{\pgfqpoint{4.742143in}{4.177390in}}%
\pgfusepath{stroke}%
\end{pgfscope}%
\begin{pgfscope}%
\pgfpathrectangle{\pgfqpoint{0.750000in}{0.500000in}}{\pgfqpoint{4.050000in}{4.000000in}} %
\pgfusepath{clip}%
\pgfsetrectcap%
\pgfsetroundjoin%
\pgfsetlinewidth{2.007500pt}%
\definecolor{currentstroke}{rgb}{0.888889,0.888889,0.888889}%
\pgfsetstrokecolor{currentstroke}%
\pgfsetdash{}{0pt}%
\pgfpathmoveto{\pgfqpoint{0.807857in}{1.757189in}}%
\pgfpathlineto{\pgfqpoint{0.865714in}{2.337184in}}%
\pgfpathlineto{\pgfqpoint{0.923571in}{2.522721in}}%
\pgfpathlineto{\pgfqpoint{0.981429in}{3.128177in}}%
\pgfpathlineto{\pgfqpoint{1.039286in}{3.707663in}}%
\pgfpathlineto{\pgfqpoint{1.097143in}{4.063875in}}%
\pgfpathlineto{\pgfqpoint{1.155000in}{4.192396in}}%
\pgfpathlineto{\pgfqpoint{1.212857in}{4.358639in}}%
\pgfpathlineto{\pgfqpoint{1.270714in}{4.200107in}}%
\pgfpathlineto{\pgfqpoint{1.328571in}{3.923426in}}%
\pgfpathlineto{\pgfqpoint{1.386429in}{3.844132in}}%
\pgfpathlineto{\pgfqpoint{1.444286in}{3.999856in}}%
\pgfpathlineto{\pgfqpoint{1.502143in}{4.172497in}}%
\pgfpathlineto{\pgfqpoint{1.560000in}{4.236095in}}%
\pgfpathlineto{\pgfqpoint{1.617857in}{4.329281in}}%
\pgfpathlineto{\pgfqpoint{1.675714in}{4.386498in}}%
\pgfpathlineto{\pgfqpoint{1.733571in}{4.229005in}}%
\pgfpathlineto{\pgfqpoint{1.791429in}{3.989922in}}%
\pgfpathlineto{\pgfqpoint{1.849286in}{3.938081in}}%
\pgfpathlineto{\pgfqpoint{1.907143in}{4.073158in}}%
\pgfpathlineto{\pgfqpoint{1.965000in}{4.173960in}}%
\pgfpathlineto{\pgfqpoint{2.022857in}{4.249221in}}%
\pgfpathlineto{\pgfqpoint{2.080714in}{4.296494in}}%
\pgfpathlineto{\pgfqpoint{2.138571in}{4.320653in}}%
\pgfpathlineto{\pgfqpoint{2.196429in}{4.222643in}}%
\pgfpathlineto{\pgfqpoint{2.254286in}{4.060900in}}%
\pgfpathlineto{\pgfqpoint{2.312143in}{4.003013in}}%
\pgfpathlineto{\pgfqpoint{2.370000in}{4.114857in}}%
\pgfpathlineto{\pgfqpoint{2.427857in}{4.173851in}}%
\pgfpathlineto{\pgfqpoint{2.485714in}{4.232444in}}%
\pgfpathlineto{\pgfqpoint{2.543571in}{4.264960in}}%
\pgfpathlineto{\pgfqpoint{2.601429in}{4.283110in}}%
\pgfpathlineto{\pgfqpoint{2.659286in}{4.204408in}}%
\pgfpathlineto{\pgfqpoint{2.717143in}{4.105193in}}%
\pgfpathlineto{\pgfqpoint{2.775000in}{4.071629in}}%
\pgfpathlineto{\pgfqpoint{2.832857in}{4.095021in}}%
\pgfpathlineto{\pgfqpoint{2.890714in}{4.135825in}}%
\pgfpathlineto{\pgfqpoint{2.948571in}{4.199747in}}%
\pgfpathlineto{\pgfqpoint{3.006429in}{4.206214in}}%
\pgfpathlineto{\pgfqpoint{3.064286in}{4.220986in}}%
\pgfpathlineto{\pgfqpoint{3.122143in}{4.158568in}}%
\pgfpathlineto{\pgfqpoint{3.180000in}{4.074305in}}%
\pgfpathlineto{\pgfqpoint{3.237857in}{4.059695in}}%
\pgfpathlineto{\pgfqpoint{3.295714in}{4.078210in}}%
\pgfpathlineto{\pgfqpoint{3.353571in}{4.115209in}}%
\pgfpathlineto{\pgfqpoint{3.411429in}{4.129886in}}%
\pgfpathlineto{\pgfqpoint{3.469286in}{4.181408in}}%
\pgfpathlineto{\pgfqpoint{3.527143in}{4.208555in}}%
\pgfpathlineto{\pgfqpoint{3.585000in}{4.212176in}}%
\pgfpathlineto{\pgfqpoint{3.642857in}{4.226079in}}%
\pgfpathlineto{\pgfqpoint{3.700714in}{4.232984in}}%
\pgfpathlineto{\pgfqpoint{3.758571in}{4.260251in}}%
\pgfpathlineto{\pgfqpoint{3.816429in}{4.280130in}}%
\pgfpathlineto{\pgfqpoint{3.874286in}{4.330825in}}%
\pgfpathlineto{\pgfqpoint{3.932143in}{4.304612in}}%
\pgfpathlineto{\pgfqpoint{3.990000in}{4.326878in}}%
\pgfpathlineto{\pgfqpoint{4.047857in}{4.323285in}}%
\pgfpathlineto{\pgfqpoint{4.105714in}{4.332729in}}%
\pgfpathlineto{\pgfqpoint{4.163571in}{4.361961in}}%
\pgfpathlineto{\pgfqpoint{4.221429in}{4.353266in}}%
\pgfpathlineto{\pgfqpoint{4.279286in}{4.370790in}}%
\pgfpathlineto{\pgfqpoint{4.337143in}{4.370737in}}%
\pgfpathlineto{\pgfqpoint{4.395000in}{4.389971in}}%
\pgfpathlineto{\pgfqpoint{4.452857in}{4.403991in}}%
\pgfpathlineto{\pgfqpoint{4.510714in}{4.403508in}}%
\pgfpathlineto{\pgfqpoint{4.568571in}{4.393850in}}%
\pgfpathlineto{\pgfqpoint{4.626429in}{4.400624in}}%
\pgfpathlineto{\pgfqpoint{4.684286in}{4.390231in}}%
\pgfpathlineto{\pgfqpoint{4.742143in}{4.408131in}}%
\pgfusepath{stroke}%
\end{pgfscope}%
\begin{pgfscope}%
\pgfpathrectangle{\pgfqpoint{0.750000in}{0.500000in}}{\pgfqpoint{4.050000in}{4.000000in}} %
\pgfusepath{clip}%
\pgfsetbuttcap%
\pgfsetroundjoin%
\pgfsetlinewidth{0.501875pt}%
\definecolor{currentstroke}{rgb}{0.000000,0.000000,0.000000}%
\pgfsetstrokecolor{currentstroke}%
\pgfsetdash{{1.000000pt}{3.000000pt}}{0.000000pt}%
\pgfpathmoveto{\pgfqpoint{0.750000in}{0.500000in}}%
\pgfpathlineto{\pgfqpoint{0.750000in}{4.500000in}}%
\pgfusepath{stroke}%
\end{pgfscope}%
\begin{pgfscope}%
\pgfsetbuttcap%
\pgfsetroundjoin%
\definecolor{currentfill}{rgb}{0.000000,0.000000,0.000000}%
\pgfsetfillcolor{currentfill}%
\pgfsetlinewidth{0.501875pt}%
\definecolor{currentstroke}{rgb}{0.000000,0.000000,0.000000}%
\pgfsetstrokecolor{currentstroke}%
\pgfsetdash{}{0pt}%
\pgfsys@defobject{currentmarker}{\pgfqpoint{0.000000in}{0.000000in}}{\pgfqpoint{0.000000in}{0.055556in}}{%
\pgfpathmoveto{\pgfqpoint{0.000000in}{0.000000in}}%
\pgfpathlineto{\pgfqpoint{0.000000in}{0.055556in}}%
\pgfusepath{stroke,fill}%
}%
\begin{pgfscope}%
\pgfsys@transformshift{0.750000in}{0.500000in}%
\pgfsys@useobject{currentmarker}{}%
\end{pgfscope}%
\end{pgfscope}%
\begin{pgfscope}%
\pgfsetbuttcap%
\pgfsetroundjoin%
\definecolor{currentfill}{rgb}{0.000000,0.000000,0.000000}%
\pgfsetfillcolor{currentfill}%
\pgfsetlinewidth{0.501875pt}%
\definecolor{currentstroke}{rgb}{0.000000,0.000000,0.000000}%
\pgfsetstrokecolor{currentstroke}%
\pgfsetdash{}{0pt}%
\pgfsys@defobject{currentmarker}{\pgfqpoint{0.000000in}{-0.055556in}}{\pgfqpoint{0.000000in}{0.000000in}}{%
\pgfpathmoveto{\pgfqpoint{0.000000in}{0.000000in}}%
\pgfpathlineto{\pgfqpoint{0.000000in}{-0.055556in}}%
\pgfusepath{stroke,fill}%
}%
\begin{pgfscope}%
\pgfsys@transformshift{0.750000in}{4.500000in}%
\pgfsys@useobject{currentmarker}{}%
\end{pgfscope}%
\end{pgfscope}%
\begin{pgfscope}%
\pgftext[x=0.750000in,y=0.444444in,,top]{{\rmfamily\fontsize{11.000000}{13.200000}\selectfont \(\displaystyle 0\)}}%
\end{pgfscope}%
\begin{pgfscope}%
\pgfpathrectangle{\pgfqpoint{0.750000in}{0.500000in}}{\pgfqpoint{4.050000in}{4.000000in}} %
\pgfusepath{clip}%
\pgfsetbuttcap%
\pgfsetroundjoin%
\pgfsetlinewidth{0.501875pt}%
\definecolor{currentstroke}{rgb}{0.000000,0.000000,0.000000}%
\pgfsetstrokecolor{currentstroke}%
\pgfsetdash{{1.000000pt}{3.000000pt}}{0.000000pt}%
\pgfpathmoveto{\pgfqpoint{1.328571in}{0.500000in}}%
\pgfpathlineto{\pgfqpoint{1.328571in}{4.500000in}}%
\pgfusepath{stroke}%
\end{pgfscope}%
\begin{pgfscope}%
\pgfsetbuttcap%
\pgfsetroundjoin%
\definecolor{currentfill}{rgb}{0.000000,0.000000,0.000000}%
\pgfsetfillcolor{currentfill}%
\pgfsetlinewidth{0.501875pt}%
\definecolor{currentstroke}{rgb}{0.000000,0.000000,0.000000}%
\pgfsetstrokecolor{currentstroke}%
\pgfsetdash{}{0pt}%
\pgfsys@defobject{currentmarker}{\pgfqpoint{0.000000in}{0.000000in}}{\pgfqpoint{0.000000in}{0.055556in}}{%
\pgfpathmoveto{\pgfqpoint{0.000000in}{0.000000in}}%
\pgfpathlineto{\pgfqpoint{0.000000in}{0.055556in}}%
\pgfusepath{stroke,fill}%
}%
\begin{pgfscope}%
\pgfsys@transformshift{1.328571in}{0.500000in}%
\pgfsys@useobject{currentmarker}{}%
\end{pgfscope}%
\end{pgfscope}%
\begin{pgfscope}%
\pgfsetbuttcap%
\pgfsetroundjoin%
\definecolor{currentfill}{rgb}{0.000000,0.000000,0.000000}%
\pgfsetfillcolor{currentfill}%
\pgfsetlinewidth{0.501875pt}%
\definecolor{currentstroke}{rgb}{0.000000,0.000000,0.000000}%
\pgfsetstrokecolor{currentstroke}%
\pgfsetdash{}{0pt}%
\pgfsys@defobject{currentmarker}{\pgfqpoint{0.000000in}{-0.055556in}}{\pgfqpoint{0.000000in}{0.000000in}}{%
\pgfpathmoveto{\pgfqpoint{0.000000in}{0.000000in}}%
\pgfpathlineto{\pgfqpoint{0.000000in}{-0.055556in}}%
\pgfusepath{stroke,fill}%
}%
\begin{pgfscope}%
\pgfsys@transformshift{1.328571in}{4.500000in}%
\pgfsys@useobject{currentmarker}{}%
\end{pgfscope}%
\end{pgfscope}%
\begin{pgfscope}%
\pgftext[x=1.328571in,y=0.444444in,,top]{{\rmfamily\fontsize{11.000000}{13.200000}\selectfont \(\displaystyle 10\)}}%
\end{pgfscope}%
\begin{pgfscope}%
\pgfpathrectangle{\pgfqpoint{0.750000in}{0.500000in}}{\pgfqpoint{4.050000in}{4.000000in}} %
\pgfusepath{clip}%
\pgfsetbuttcap%
\pgfsetroundjoin%
\pgfsetlinewidth{0.501875pt}%
\definecolor{currentstroke}{rgb}{0.000000,0.000000,0.000000}%
\pgfsetstrokecolor{currentstroke}%
\pgfsetdash{{1.000000pt}{3.000000pt}}{0.000000pt}%
\pgfpathmoveto{\pgfqpoint{1.907143in}{0.500000in}}%
\pgfpathlineto{\pgfqpoint{1.907143in}{4.500000in}}%
\pgfusepath{stroke}%
\end{pgfscope}%
\begin{pgfscope}%
\pgfsetbuttcap%
\pgfsetroundjoin%
\definecolor{currentfill}{rgb}{0.000000,0.000000,0.000000}%
\pgfsetfillcolor{currentfill}%
\pgfsetlinewidth{0.501875pt}%
\definecolor{currentstroke}{rgb}{0.000000,0.000000,0.000000}%
\pgfsetstrokecolor{currentstroke}%
\pgfsetdash{}{0pt}%
\pgfsys@defobject{currentmarker}{\pgfqpoint{0.000000in}{0.000000in}}{\pgfqpoint{0.000000in}{0.055556in}}{%
\pgfpathmoveto{\pgfqpoint{0.000000in}{0.000000in}}%
\pgfpathlineto{\pgfqpoint{0.000000in}{0.055556in}}%
\pgfusepath{stroke,fill}%
}%
\begin{pgfscope}%
\pgfsys@transformshift{1.907143in}{0.500000in}%
\pgfsys@useobject{currentmarker}{}%
\end{pgfscope}%
\end{pgfscope}%
\begin{pgfscope}%
\pgfsetbuttcap%
\pgfsetroundjoin%
\definecolor{currentfill}{rgb}{0.000000,0.000000,0.000000}%
\pgfsetfillcolor{currentfill}%
\pgfsetlinewidth{0.501875pt}%
\definecolor{currentstroke}{rgb}{0.000000,0.000000,0.000000}%
\pgfsetstrokecolor{currentstroke}%
\pgfsetdash{}{0pt}%
\pgfsys@defobject{currentmarker}{\pgfqpoint{0.000000in}{-0.055556in}}{\pgfqpoint{0.000000in}{0.000000in}}{%
\pgfpathmoveto{\pgfqpoint{0.000000in}{0.000000in}}%
\pgfpathlineto{\pgfqpoint{0.000000in}{-0.055556in}}%
\pgfusepath{stroke,fill}%
}%
\begin{pgfscope}%
\pgfsys@transformshift{1.907143in}{4.500000in}%
\pgfsys@useobject{currentmarker}{}%
\end{pgfscope}%
\end{pgfscope}%
\begin{pgfscope}%
\pgftext[x=1.907143in,y=0.444444in,,top]{{\rmfamily\fontsize{11.000000}{13.200000}\selectfont \(\displaystyle 20\)}}%
\end{pgfscope}%
\begin{pgfscope}%
\pgfpathrectangle{\pgfqpoint{0.750000in}{0.500000in}}{\pgfqpoint{4.050000in}{4.000000in}} %
\pgfusepath{clip}%
\pgfsetbuttcap%
\pgfsetroundjoin%
\pgfsetlinewidth{0.501875pt}%
\definecolor{currentstroke}{rgb}{0.000000,0.000000,0.000000}%
\pgfsetstrokecolor{currentstroke}%
\pgfsetdash{{1.000000pt}{3.000000pt}}{0.000000pt}%
\pgfpathmoveto{\pgfqpoint{2.485714in}{0.500000in}}%
\pgfpathlineto{\pgfqpoint{2.485714in}{4.500000in}}%
\pgfusepath{stroke}%
\end{pgfscope}%
\begin{pgfscope}%
\pgfsetbuttcap%
\pgfsetroundjoin%
\definecolor{currentfill}{rgb}{0.000000,0.000000,0.000000}%
\pgfsetfillcolor{currentfill}%
\pgfsetlinewidth{0.501875pt}%
\definecolor{currentstroke}{rgb}{0.000000,0.000000,0.000000}%
\pgfsetstrokecolor{currentstroke}%
\pgfsetdash{}{0pt}%
\pgfsys@defobject{currentmarker}{\pgfqpoint{0.000000in}{0.000000in}}{\pgfqpoint{0.000000in}{0.055556in}}{%
\pgfpathmoveto{\pgfqpoint{0.000000in}{0.000000in}}%
\pgfpathlineto{\pgfqpoint{0.000000in}{0.055556in}}%
\pgfusepath{stroke,fill}%
}%
\begin{pgfscope}%
\pgfsys@transformshift{2.485714in}{0.500000in}%
\pgfsys@useobject{currentmarker}{}%
\end{pgfscope}%
\end{pgfscope}%
\begin{pgfscope}%
\pgfsetbuttcap%
\pgfsetroundjoin%
\definecolor{currentfill}{rgb}{0.000000,0.000000,0.000000}%
\pgfsetfillcolor{currentfill}%
\pgfsetlinewidth{0.501875pt}%
\definecolor{currentstroke}{rgb}{0.000000,0.000000,0.000000}%
\pgfsetstrokecolor{currentstroke}%
\pgfsetdash{}{0pt}%
\pgfsys@defobject{currentmarker}{\pgfqpoint{0.000000in}{-0.055556in}}{\pgfqpoint{0.000000in}{0.000000in}}{%
\pgfpathmoveto{\pgfqpoint{0.000000in}{0.000000in}}%
\pgfpathlineto{\pgfqpoint{0.000000in}{-0.055556in}}%
\pgfusepath{stroke,fill}%
}%
\begin{pgfscope}%
\pgfsys@transformshift{2.485714in}{4.500000in}%
\pgfsys@useobject{currentmarker}{}%
\end{pgfscope}%
\end{pgfscope}%
\begin{pgfscope}%
\pgftext[x=2.485714in,y=0.444444in,,top]{{\rmfamily\fontsize{11.000000}{13.200000}\selectfont \(\displaystyle 30\)}}%
\end{pgfscope}%
\begin{pgfscope}%
\pgfpathrectangle{\pgfqpoint{0.750000in}{0.500000in}}{\pgfqpoint{4.050000in}{4.000000in}} %
\pgfusepath{clip}%
\pgfsetbuttcap%
\pgfsetroundjoin%
\pgfsetlinewidth{0.501875pt}%
\definecolor{currentstroke}{rgb}{0.000000,0.000000,0.000000}%
\pgfsetstrokecolor{currentstroke}%
\pgfsetdash{{1.000000pt}{3.000000pt}}{0.000000pt}%
\pgfpathmoveto{\pgfqpoint{3.064286in}{0.500000in}}%
\pgfpathlineto{\pgfqpoint{3.064286in}{4.500000in}}%
\pgfusepath{stroke}%
\end{pgfscope}%
\begin{pgfscope}%
\pgfsetbuttcap%
\pgfsetroundjoin%
\definecolor{currentfill}{rgb}{0.000000,0.000000,0.000000}%
\pgfsetfillcolor{currentfill}%
\pgfsetlinewidth{0.501875pt}%
\definecolor{currentstroke}{rgb}{0.000000,0.000000,0.000000}%
\pgfsetstrokecolor{currentstroke}%
\pgfsetdash{}{0pt}%
\pgfsys@defobject{currentmarker}{\pgfqpoint{0.000000in}{0.000000in}}{\pgfqpoint{0.000000in}{0.055556in}}{%
\pgfpathmoveto{\pgfqpoint{0.000000in}{0.000000in}}%
\pgfpathlineto{\pgfqpoint{0.000000in}{0.055556in}}%
\pgfusepath{stroke,fill}%
}%
\begin{pgfscope}%
\pgfsys@transformshift{3.064286in}{0.500000in}%
\pgfsys@useobject{currentmarker}{}%
\end{pgfscope}%
\end{pgfscope}%
\begin{pgfscope}%
\pgfsetbuttcap%
\pgfsetroundjoin%
\definecolor{currentfill}{rgb}{0.000000,0.000000,0.000000}%
\pgfsetfillcolor{currentfill}%
\pgfsetlinewidth{0.501875pt}%
\definecolor{currentstroke}{rgb}{0.000000,0.000000,0.000000}%
\pgfsetstrokecolor{currentstroke}%
\pgfsetdash{}{0pt}%
\pgfsys@defobject{currentmarker}{\pgfqpoint{0.000000in}{-0.055556in}}{\pgfqpoint{0.000000in}{0.000000in}}{%
\pgfpathmoveto{\pgfqpoint{0.000000in}{0.000000in}}%
\pgfpathlineto{\pgfqpoint{0.000000in}{-0.055556in}}%
\pgfusepath{stroke,fill}%
}%
\begin{pgfscope}%
\pgfsys@transformshift{3.064286in}{4.500000in}%
\pgfsys@useobject{currentmarker}{}%
\end{pgfscope}%
\end{pgfscope}%
\begin{pgfscope}%
\pgftext[x=3.064286in,y=0.444444in,,top]{{\rmfamily\fontsize{11.000000}{13.200000}\selectfont \(\displaystyle 40\)}}%
\end{pgfscope}%
\begin{pgfscope}%
\pgfpathrectangle{\pgfqpoint{0.750000in}{0.500000in}}{\pgfqpoint{4.050000in}{4.000000in}} %
\pgfusepath{clip}%
\pgfsetbuttcap%
\pgfsetroundjoin%
\pgfsetlinewidth{0.501875pt}%
\definecolor{currentstroke}{rgb}{0.000000,0.000000,0.000000}%
\pgfsetstrokecolor{currentstroke}%
\pgfsetdash{{1.000000pt}{3.000000pt}}{0.000000pt}%
\pgfpathmoveto{\pgfqpoint{3.642857in}{0.500000in}}%
\pgfpathlineto{\pgfqpoint{3.642857in}{4.500000in}}%
\pgfusepath{stroke}%
\end{pgfscope}%
\begin{pgfscope}%
\pgfsetbuttcap%
\pgfsetroundjoin%
\definecolor{currentfill}{rgb}{0.000000,0.000000,0.000000}%
\pgfsetfillcolor{currentfill}%
\pgfsetlinewidth{0.501875pt}%
\definecolor{currentstroke}{rgb}{0.000000,0.000000,0.000000}%
\pgfsetstrokecolor{currentstroke}%
\pgfsetdash{}{0pt}%
\pgfsys@defobject{currentmarker}{\pgfqpoint{0.000000in}{0.000000in}}{\pgfqpoint{0.000000in}{0.055556in}}{%
\pgfpathmoveto{\pgfqpoint{0.000000in}{0.000000in}}%
\pgfpathlineto{\pgfqpoint{0.000000in}{0.055556in}}%
\pgfusepath{stroke,fill}%
}%
\begin{pgfscope}%
\pgfsys@transformshift{3.642857in}{0.500000in}%
\pgfsys@useobject{currentmarker}{}%
\end{pgfscope}%
\end{pgfscope}%
\begin{pgfscope}%
\pgfsetbuttcap%
\pgfsetroundjoin%
\definecolor{currentfill}{rgb}{0.000000,0.000000,0.000000}%
\pgfsetfillcolor{currentfill}%
\pgfsetlinewidth{0.501875pt}%
\definecolor{currentstroke}{rgb}{0.000000,0.000000,0.000000}%
\pgfsetstrokecolor{currentstroke}%
\pgfsetdash{}{0pt}%
\pgfsys@defobject{currentmarker}{\pgfqpoint{0.000000in}{-0.055556in}}{\pgfqpoint{0.000000in}{0.000000in}}{%
\pgfpathmoveto{\pgfqpoint{0.000000in}{0.000000in}}%
\pgfpathlineto{\pgfqpoint{0.000000in}{-0.055556in}}%
\pgfusepath{stroke,fill}%
}%
\begin{pgfscope}%
\pgfsys@transformshift{3.642857in}{4.500000in}%
\pgfsys@useobject{currentmarker}{}%
\end{pgfscope}%
\end{pgfscope}%
\begin{pgfscope}%
\pgftext[x=3.642857in,y=0.444444in,,top]{{\rmfamily\fontsize{11.000000}{13.200000}\selectfont \(\displaystyle 50\)}}%
\end{pgfscope}%
\begin{pgfscope}%
\pgfpathrectangle{\pgfqpoint{0.750000in}{0.500000in}}{\pgfqpoint{4.050000in}{4.000000in}} %
\pgfusepath{clip}%
\pgfsetbuttcap%
\pgfsetroundjoin%
\pgfsetlinewidth{0.501875pt}%
\definecolor{currentstroke}{rgb}{0.000000,0.000000,0.000000}%
\pgfsetstrokecolor{currentstroke}%
\pgfsetdash{{1.000000pt}{3.000000pt}}{0.000000pt}%
\pgfpathmoveto{\pgfqpoint{4.221429in}{0.500000in}}%
\pgfpathlineto{\pgfqpoint{4.221429in}{4.500000in}}%
\pgfusepath{stroke}%
\end{pgfscope}%
\begin{pgfscope}%
\pgfsetbuttcap%
\pgfsetroundjoin%
\definecolor{currentfill}{rgb}{0.000000,0.000000,0.000000}%
\pgfsetfillcolor{currentfill}%
\pgfsetlinewidth{0.501875pt}%
\definecolor{currentstroke}{rgb}{0.000000,0.000000,0.000000}%
\pgfsetstrokecolor{currentstroke}%
\pgfsetdash{}{0pt}%
\pgfsys@defobject{currentmarker}{\pgfqpoint{0.000000in}{0.000000in}}{\pgfqpoint{0.000000in}{0.055556in}}{%
\pgfpathmoveto{\pgfqpoint{0.000000in}{0.000000in}}%
\pgfpathlineto{\pgfqpoint{0.000000in}{0.055556in}}%
\pgfusepath{stroke,fill}%
}%
\begin{pgfscope}%
\pgfsys@transformshift{4.221429in}{0.500000in}%
\pgfsys@useobject{currentmarker}{}%
\end{pgfscope}%
\end{pgfscope}%
\begin{pgfscope}%
\pgfsetbuttcap%
\pgfsetroundjoin%
\definecolor{currentfill}{rgb}{0.000000,0.000000,0.000000}%
\pgfsetfillcolor{currentfill}%
\pgfsetlinewidth{0.501875pt}%
\definecolor{currentstroke}{rgb}{0.000000,0.000000,0.000000}%
\pgfsetstrokecolor{currentstroke}%
\pgfsetdash{}{0pt}%
\pgfsys@defobject{currentmarker}{\pgfqpoint{0.000000in}{-0.055556in}}{\pgfqpoint{0.000000in}{0.000000in}}{%
\pgfpathmoveto{\pgfqpoint{0.000000in}{0.000000in}}%
\pgfpathlineto{\pgfqpoint{0.000000in}{-0.055556in}}%
\pgfusepath{stroke,fill}%
}%
\begin{pgfscope}%
\pgfsys@transformshift{4.221429in}{4.500000in}%
\pgfsys@useobject{currentmarker}{}%
\end{pgfscope}%
\end{pgfscope}%
\begin{pgfscope}%
\pgftext[x=4.221429in,y=0.444444in,,top]{{\rmfamily\fontsize{11.000000}{13.200000}\selectfont \(\displaystyle 60\)}}%
\end{pgfscope}%
\begin{pgfscope}%
\pgfpathrectangle{\pgfqpoint{0.750000in}{0.500000in}}{\pgfqpoint{4.050000in}{4.000000in}} %
\pgfusepath{clip}%
\pgfsetbuttcap%
\pgfsetroundjoin%
\pgfsetlinewidth{0.501875pt}%
\definecolor{currentstroke}{rgb}{0.000000,0.000000,0.000000}%
\pgfsetstrokecolor{currentstroke}%
\pgfsetdash{{1.000000pt}{3.000000pt}}{0.000000pt}%
\pgfpathmoveto{\pgfqpoint{4.800000in}{0.500000in}}%
\pgfpathlineto{\pgfqpoint{4.800000in}{4.500000in}}%
\pgfusepath{stroke}%
\end{pgfscope}%
\begin{pgfscope}%
\pgfsetbuttcap%
\pgfsetroundjoin%
\definecolor{currentfill}{rgb}{0.000000,0.000000,0.000000}%
\pgfsetfillcolor{currentfill}%
\pgfsetlinewidth{0.501875pt}%
\definecolor{currentstroke}{rgb}{0.000000,0.000000,0.000000}%
\pgfsetstrokecolor{currentstroke}%
\pgfsetdash{}{0pt}%
\pgfsys@defobject{currentmarker}{\pgfqpoint{0.000000in}{0.000000in}}{\pgfqpoint{0.000000in}{0.055556in}}{%
\pgfpathmoveto{\pgfqpoint{0.000000in}{0.000000in}}%
\pgfpathlineto{\pgfqpoint{0.000000in}{0.055556in}}%
\pgfusepath{stroke,fill}%
}%
\begin{pgfscope}%
\pgfsys@transformshift{4.800000in}{0.500000in}%
\pgfsys@useobject{currentmarker}{}%
\end{pgfscope}%
\end{pgfscope}%
\begin{pgfscope}%
\pgfsetbuttcap%
\pgfsetroundjoin%
\definecolor{currentfill}{rgb}{0.000000,0.000000,0.000000}%
\pgfsetfillcolor{currentfill}%
\pgfsetlinewidth{0.501875pt}%
\definecolor{currentstroke}{rgb}{0.000000,0.000000,0.000000}%
\pgfsetstrokecolor{currentstroke}%
\pgfsetdash{}{0pt}%
\pgfsys@defobject{currentmarker}{\pgfqpoint{0.000000in}{-0.055556in}}{\pgfqpoint{0.000000in}{0.000000in}}{%
\pgfpathmoveto{\pgfqpoint{0.000000in}{0.000000in}}%
\pgfpathlineto{\pgfqpoint{0.000000in}{-0.055556in}}%
\pgfusepath{stroke,fill}%
}%
\begin{pgfscope}%
\pgfsys@transformshift{4.800000in}{4.500000in}%
\pgfsys@useobject{currentmarker}{}%
\end{pgfscope}%
\end{pgfscope}%
\begin{pgfscope}%
\pgftext[x=4.800000in,y=0.444444in,,top]{{\rmfamily\fontsize{11.000000}{13.200000}\selectfont \(\displaystyle 70\)}}%
\end{pgfscope}%
\begin{pgfscope}%
\pgftext[x=2.775000in,y=0.240049in,,top]{{\rmfamily\fontsize{11.000000}{13.200000}\selectfont Thread Count}}%
\end{pgfscope}%
\begin{pgfscope}%
\pgfpathrectangle{\pgfqpoint{0.750000in}{0.500000in}}{\pgfqpoint{4.050000in}{4.000000in}} %
\pgfusepath{clip}%
\pgfsetbuttcap%
\pgfsetroundjoin%
\pgfsetlinewidth{0.501875pt}%
\definecolor{currentstroke}{rgb}{0.000000,0.000000,0.000000}%
\pgfsetstrokecolor{currentstroke}%
\pgfsetdash{{1.000000pt}{3.000000pt}}{0.000000pt}%
\pgfpathmoveto{\pgfqpoint{0.750000in}{0.500000in}}%
\pgfpathlineto{\pgfqpoint{4.800000in}{0.500000in}}%
\pgfusepath{stroke}%
\end{pgfscope}%
\begin{pgfscope}%
\pgfsetbuttcap%
\pgfsetroundjoin%
\definecolor{currentfill}{rgb}{0.000000,0.000000,0.000000}%
\pgfsetfillcolor{currentfill}%
\pgfsetlinewidth{0.501875pt}%
\definecolor{currentstroke}{rgb}{0.000000,0.000000,0.000000}%
\pgfsetstrokecolor{currentstroke}%
\pgfsetdash{}{0pt}%
\pgfsys@defobject{currentmarker}{\pgfqpoint{0.000000in}{0.000000in}}{\pgfqpoint{0.055556in}{0.000000in}}{%
\pgfpathmoveto{\pgfqpoint{0.000000in}{0.000000in}}%
\pgfpathlineto{\pgfqpoint{0.055556in}{0.000000in}}%
\pgfusepath{stroke,fill}%
}%
\begin{pgfscope}%
\pgfsys@transformshift{0.750000in}{0.500000in}%
\pgfsys@useobject{currentmarker}{}%
\end{pgfscope}%
\end{pgfscope}%
\begin{pgfscope}%
\pgfsetbuttcap%
\pgfsetroundjoin%
\definecolor{currentfill}{rgb}{0.000000,0.000000,0.000000}%
\pgfsetfillcolor{currentfill}%
\pgfsetlinewidth{0.501875pt}%
\definecolor{currentstroke}{rgb}{0.000000,0.000000,0.000000}%
\pgfsetstrokecolor{currentstroke}%
\pgfsetdash{}{0pt}%
\pgfsys@defobject{currentmarker}{\pgfqpoint{-0.055556in}{0.000000in}}{\pgfqpoint{0.000000in}{0.000000in}}{%
\pgfpathmoveto{\pgfqpoint{0.000000in}{0.000000in}}%
\pgfpathlineto{\pgfqpoint{-0.055556in}{0.000000in}}%
\pgfusepath{stroke,fill}%
}%
\begin{pgfscope}%
\pgfsys@transformshift{4.800000in}{0.500000in}%
\pgfsys@useobject{currentmarker}{}%
\end{pgfscope}%
\end{pgfscope}%
\begin{pgfscope}%
\pgftext[x=0.694444in,y=0.500000in,right,]{{\rmfamily\fontsize{11.000000}{13.200000}\selectfont \(\displaystyle 0.0\)}}%
\end{pgfscope}%
\begin{pgfscope}%
\pgfpathrectangle{\pgfqpoint{0.750000in}{0.500000in}}{\pgfqpoint{4.050000in}{4.000000in}} %
\pgfusepath{clip}%
\pgfsetbuttcap%
\pgfsetroundjoin%
\pgfsetlinewidth{0.501875pt}%
\definecolor{currentstroke}{rgb}{0.000000,0.000000,0.000000}%
\pgfsetstrokecolor{currentstroke}%
\pgfsetdash{{1.000000pt}{3.000000pt}}{0.000000pt}%
\pgfpathmoveto{\pgfqpoint{0.750000in}{1.166667in}}%
\pgfpathlineto{\pgfqpoint{4.800000in}{1.166667in}}%
\pgfusepath{stroke}%
\end{pgfscope}%
\begin{pgfscope}%
\pgfsetbuttcap%
\pgfsetroundjoin%
\definecolor{currentfill}{rgb}{0.000000,0.000000,0.000000}%
\pgfsetfillcolor{currentfill}%
\pgfsetlinewidth{0.501875pt}%
\definecolor{currentstroke}{rgb}{0.000000,0.000000,0.000000}%
\pgfsetstrokecolor{currentstroke}%
\pgfsetdash{}{0pt}%
\pgfsys@defobject{currentmarker}{\pgfqpoint{0.000000in}{0.000000in}}{\pgfqpoint{0.055556in}{0.000000in}}{%
\pgfpathmoveto{\pgfqpoint{0.000000in}{0.000000in}}%
\pgfpathlineto{\pgfqpoint{0.055556in}{0.000000in}}%
\pgfusepath{stroke,fill}%
}%
\begin{pgfscope}%
\pgfsys@transformshift{0.750000in}{1.166667in}%
\pgfsys@useobject{currentmarker}{}%
\end{pgfscope}%
\end{pgfscope}%
\begin{pgfscope}%
\pgfsetbuttcap%
\pgfsetroundjoin%
\definecolor{currentfill}{rgb}{0.000000,0.000000,0.000000}%
\pgfsetfillcolor{currentfill}%
\pgfsetlinewidth{0.501875pt}%
\definecolor{currentstroke}{rgb}{0.000000,0.000000,0.000000}%
\pgfsetstrokecolor{currentstroke}%
\pgfsetdash{}{0pt}%
\pgfsys@defobject{currentmarker}{\pgfqpoint{-0.055556in}{0.000000in}}{\pgfqpoint{0.000000in}{0.000000in}}{%
\pgfpathmoveto{\pgfqpoint{0.000000in}{0.000000in}}%
\pgfpathlineto{\pgfqpoint{-0.055556in}{0.000000in}}%
\pgfusepath{stroke,fill}%
}%
\begin{pgfscope}%
\pgfsys@transformshift{4.800000in}{1.166667in}%
\pgfsys@useobject{currentmarker}{}%
\end{pgfscope}%
\end{pgfscope}%
\begin{pgfscope}%
\pgftext[x=0.694444in,y=1.166667in,right,]{{\rmfamily\fontsize{11.000000}{13.200000}\selectfont \(\displaystyle 0.5\)}}%
\end{pgfscope}%
\begin{pgfscope}%
\pgfpathrectangle{\pgfqpoint{0.750000in}{0.500000in}}{\pgfqpoint{4.050000in}{4.000000in}} %
\pgfusepath{clip}%
\pgfsetbuttcap%
\pgfsetroundjoin%
\pgfsetlinewidth{0.501875pt}%
\definecolor{currentstroke}{rgb}{0.000000,0.000000,0.000000}%
\pgfsetstrokecolor{currentstroke}%
\pgfsetdash{{1.000000pt}{3.000000pt}}{0.000000pt}%
\pgfpathmoveto{\pgfqpoint{0.750000in}{1.833333in}}%
\pgfpathlineto{\pgfqpoint{4.800000in}{1.833333in}}%
\pgfusepath{stroke}%
\end{pgfscope}%
\begin{pgfscope}%
\pgfsetbuttcap%
\pgfsetroundjoin%
\definecolor{currentfill}{rgb}{0.000000,0.000000,0.000000}%
\pgfsetfillcolor{currentfill}%
\pgfsetlinewidth{0.501875pt}%
\definecolor{currentstroke}{rgb}{0.000000,0.000000,0.000000}%
\pgfsetstrokecolor{currentstroke}%
\pgfsetdash{}{0pt}%
\pgfsys@defobject{currentmarker}{\pgfqpoint{0.000000in}{0.000000in}}{\pgfqpoint{0.055556in}{0.000000in}}{%
\pgfpathmoveto{\pgfqpoint{0.000000in}{0.000000in}}%
\pgfpathlineto{\pgfqpoint{0.055556in}{0.000000in}}%
\pgfusepath{stroke,fill}%
}%
\begin{pgfscope}%
\pgfsys@transformshift{0.750000in}{1.833333in}%
\pgfsys@useobject{currentmarker}{}%
\end{pgfscope}%
\end{pgfscope}%
\begin{pgfscope}%
\pgfsetbuttcap%
\pgfsetroundjoin%
\definecolor{currentfill}{rgb}{0.000000,0.000000,0.000000}%
\pgfsetfillcolor{currentfill}%
\pgfsetlinewidth{0.501875pt}%
\definecolor{currentstroke}{rgb}{0.000000,0.000000,0.000000}%
\pgfsetstrokecolor{currentstroke}%
\pgfsetdash{}{0pt}%
\pgfsys@defobject{currentmarker}{\pgfqpoint{-0.055556in}{0.000000in}}{\pgfqpoint{0.000000in}{0.000000in}}{%
\pgfpathmoveto{\pgfqpoint{0.000000in}{0.000000in}}%
\pgfpathlineto{\pgfqpoint{-0.055556in}{0.000000in}}%
\pgfusepath{stroke,fill}%
}%
\begin{pgfscope}%
\pgfsys@transformshift{4.800000in}{1.833333in}%
\pgfsys@useobject{currentmarker}{}%
\end{pgfscope}%
\end{pgfscope}%
\begin{pgfscope}%
\pgftext[x=0.694444in,y=1.833333in,right,]{{\rmfamily\fontsize{11.000000}{13.200000}\selectfont \(\displaystyle 1.0\)}}%
\end{pgfscope}%
\begin{pgfscope}%
\pgfpathrectangle{\pgfqpoint{0.750000in}{0.500000in}}{\pgfqpoint{4.050000in}{4.000000in}} %
\pgfusepath{clip}%
\pgfsetbuttcap%
\pgfsetroundjoin%
\pgfsetlinewidth{0.501875pt}%
\definecolor{currentstroke}{rgb}{0.000000,0.000000,0.000000}%
\pgfsetstrokecolor{currentstroke}%
\pgfsetdash{{1.000000pt}{3.000000pt}}{0.000000pt}%
\pgfpathmoveto{\pgfqpoint{0.750000in}{2.500000in}}%
\pgfpathlineto{\pgfqpoint{4.800000in}{2.500000in}}%
\pgfusepath{stroke}%
\end{pgfscope}%
\begin{pgfscope}%
\pgfsetbuttcap%
\pgfsetroundjoin%
\definecolor{currentfill}{rgb}{0.000000,0.000000,0.000000}%
\pgfsetfillcolor{currentfill}%
\pgfsetlinewidth{0.501875pt}%
\definecolor{currentstroke}{rgb}{0.000000,0.000000,0.000000}%
\pgfsetstrokecolor{currentstroke}%
\pgfsetdash{}{0pt}%
\pgfsys@defobject{currentmarker}{\pgfqpoint{0.000000in}{0.000000in}}{\pgfqpoint{0.055556in}{0.000000in}}{%
\pgfpathmoveto{\pgfqpoint{0.000000in}{0.000000in}}%
\pgfpathlineto{\pgfqpoint{0.055556in}{0.000000in}}%
\pgfusepath{stroke,fill}%
}%
\begin{pgfscope}%
\pgfsys@transformshift{0.750000in}{2.500000in}%
\pgfsys@useobject{currentmarker}{}%
\end{pgfscope}%
\end{pgfscope}%
\begin{pgfscope}%
\pgfsetbuttcap%
\pgfsetroundjoin%
\definecolor{currentfill}{rgb}{0.000000,0.000000,0.000000}%
\pgfsetfillcolor{currentfill}%
\pgfsetlinewidth{0.501875pt}%
\definecolor{currentstroke}{rgb}{0.000000,0.000000,0.000000}%
\pgfsetstrokecolor{currentstroke}%
\pgfsetdash{}{0pt}%
\pgfsys@defobject{currentmarker}{\pgfqpoint{-0.055556in}{0.000000in}}{\pgfqpoint{0.000000in}{0.000000in}}{%
\pgfpathmoveto{\pgfqpoint{0.000000in}{0.000000in}}%
\pgfpathlineto{\pgfqpoint{-0.055556in}{0.000000in}}%
\pgfusepath{stroke,fill}%
}%
\begin{pgfscope}%
\pgfsys@transformshift{4.800000in}{2.500000in}%
\pgfsys@useobject{currentmarker}{}%
\end{pgfscope}%
\end{pgfscope}%
\begin{pgfscope}%
\pgftext[x=0.694444in,y=2.500000in,right,]{{\rmfamily\fontsize{11.000000}{13.200000}\selectfont \(\displaystyle 1.5\)}}%
\end{pgfscope}%
\begin{pgfscope}%
\pgfpathrectangle{\pgfqpoint{0.750000in}{0.500000in}}{\pgfqpoint{4.050000in}{4.000000in}} %
\pgfusepath{clip}%
\pgfsetbuttcap%
\pgfsetroundjoin%
\pgfsetlinewidth{0.501875pt}%
\definecolor{currentstroke}{rgb}{0.000000,0.000000,0.000000}%
\pgfsetstrokecolor{currentstroke}%
\pgfsetdash{{1.000000pt}{3.000000pt}}{0.000000pt}%
\pgfpathmoveto{\pgfqpoint{0.750000in}{3.166667in}}%
\pgfpathlineto{\pgfqpoint{4.800000in}{3.166667in}}%
\pgfusepath{stroke}%
\end{pgfscope}%
\begin{pgfscope}%
\pgfsetbuttcap%
\pgfsetroundjoin%
\definecolor{currentfill}{rgb}{0.000000,0.000000,0.000000}%
\pgfsetfillcolor{currentfill}%
\pgfsetlinewidth{0.501875pt}%
\definecolor{currentstroke}{rgb}{0.000000,0.000000,0.000000}%
\pgfsetstrokecolor{currentstroke}%
\pgfsetdash{}{0pt}%
\pgfsys@defobject{currentmarker}{\pgfqpoint{0.000000in}{0.000000in}}{\pgfqpoint{0.055556in}{0.000000in}}{%
\pgfpathmoveto{\pgfqpoint{0.000000in}{0.000000in}}%
\pgfpathlineto{\pgfqpoint{0.055556in}{0.000000in}}%
\pgfusepath{stroke,fill}%
}%
\begin{pgfscope}%
\pgfsys@transformshift{0.750000in}{3.166667in}%
\pgfsys@useobject{currentmarker}{}%
\end{pgfscope}%
\end{pgfscope}%
\begin{pgfscope}%
\pgfsetbuttcap%
\pgfsetroundjoin%
\definecolor{currentfill}{rgb}{0.000000,0.000000,0.000000}%
\pgfsetfillcolor{currentfill}%
\pgfsetlinewidth{0.501875pt}%
\definecolor{currentstroke}{rgb}{0.000000,0.000000,0.000000}%
\pgfsetstrokecolor{currentstroke}%
\pgfsetdash{}{0pt}%
\pgfsys@defobject{currentmarker}{\pgfqpoint{-0.055556in}{0.000000in}}{\pgfqpoint{0.000000in}{0.000000in}}{%
\pgfpathmoveto{\pgfqpoint{0.000000in}{0.000000in}}%
\pgfpathlineto{\pgfqpoint{-0.055556in}{0.000000in}}%
\pgfusepath{stroke,fill}%
}%
\begin{pgfscope}%
\pgfsys@transformshift{4.800000in}{3.166667in}%
\pgfsys@useobject{currentmarker}{}%
\end{pgfscope}%
\end{pgfscope}%
\begin{pgfscope}%
\pgftext[x=0.694444in,y=3.166667in,right,]{{\rmfamily\fontsize{11.000000}{13.200000}\selectfont \(\displaystyle 2.0\)}}%
\end{pgfscope}%
\begin{pgfscope}%
\pgfpathrectangle{\pgfqpoint{0.750000in}{0.500000in}}{\pgfqpoint{4.050000in}{4.000000in}} %
\pgfusepath{clip}%
\pgfsetbuttcap%
\pgfsetroundjoin%
\pgfsetlinewidth{0.501875pt}%
\definecolor{currentstroke}{rgb}{0.000000,0.000000,0.000000}%
\pgfsetstrokecolor{currentstroke}%
\pgfsetdash{{1.000000pt}{3.000000pt}}{0.000000pt}%
\pgfpathmoveto{\pgfqpoint{0.750000in}{3.833333in}}%
\pgfpathlineto{\pgfqpoint{4.800000in}{3.833333in}}%
\pgfusepath{stroke}%
\end{pgfscope}%
\begin{pgfscope}%
\pgfsetbuttcap%
\pgfsetroundjoin%
\definecolor{currentfill}{rgb}{0.000000,0.000000,0.000000}%
\pgfsetfillcolor{currentfill}%
\pgfsetlinewidth{0.501875pt}%
\definecolor{currentstroke}{rgb}{0.000000,0.000000,0.000000}%
\pgfsetstrokecolor{currentstroke}%
\pgfsetdash{}{0pt}%
\pgfsys@defobject{currentmarker}{\pgfqpoint{0.000000in}{0.000000in}}{\pgfqpoint{0.055556in}{0.000000in}}{%
\pgfpathmoveto{\pgfqpoint{0.000000in}{0.000000in}}%
\pgfpathlineto{\pgfqpoint{0.055556in}{0.000000in}}%
\pgfusepath{stroke,fill}%
}%
\begin{pgfscope}%
\pgfsys@transformshift{0.750000in}{3.833333in}%
\pgfsys@useobject{currentmarker}{}%
\end{pgfscope}%
\end{pgfscope}%
\begin{pgfscope}%
\pgfsetbuttcap%
\pgfsetroundjoin%
\definecolor{currentfill}{rgb}{0.000000,0.000000,0.000000}%
\pgfsetfillcolor{currentfill}%
\pgfsetlinewidth{0.501875pt}%
\definecolor{currentstroke}{rgb}{0.000000,0.000000,0.000000}%
\pgfsetstrokecolor{currentstroke}%
\pgfsetdash{}{0pt}%
\pgfsys@defobject{currentmarker}{\pgfqpoint{-0.055556in}{0.000000in}}{\pgfqpoint{0.000000in}{0.000000in}}{%
\pgfpathmoveto{\pgfqpoint{0.000000in}{0.000000in}}%
\pgfpathlineto{\pgfqpoint{-0.055556in}{0.000000in}}%
\pgfusepath{stroke,fill}%
}%
\begin{pgfscope}%
\pgfsys@transformshift{4.800000in}{3.833333in}%
\pgfsys@useobject{currentmarker}{}%
\end{pgfscope}%
\end{pgfscope}%
\begin{pgfscope}%
\pgftext[x=0.694444in,y=3.833333in,right,]{{\rmfamily\fontsize{11.000000}{13.200000}\selectfont \(\displaystyle 2.5\)}}%
\end{pgfscope}%
\begin{pgfscope}%
\pgfpathrectangle{\pgfqpoint{0.750000in}{0.500000in}}{\pgfqpoint{4.050000in}{4.000000in}} %
\pgfusepath{clip}%
\pgfsetbuttcap%
\pgfsetroundjoin%
\pgfsetlinewidth{0.501875pt}%
\definecolor{currentstroke}{rgb}{0.000000,0.000000,0.000000}%
\pgfsetstrokecolor{currentstroke}%
\pgfsetdash{{1.000000pt}{3.000000pt}}{0.000000pt}%
\pgfpathmoveto{\pgfqpoint{0.750000in}{4.500000in}}%
\pgfpathlineto{\pgfqpoint{4.800000in}{4.500000in}}%
\pgfusepath{stroke}%
\end{pgfscope}%
\begin{pgfscope}%
\pgfsetbuttcap%
\pgfsetroundjoin%
\definecolor{currentfill}{rgb}{0.000000,0.000000,0.000000}%
\pgfsetfillcolor{currentfill}%
\pgfsetlinewidth{0.501875pt}%
\definecolor{currentstroke}{rgb}{0.000000,0.000000,0.000000}%
\pgfsetstrokecolor{currentstroke}%
\pgfsetdash{}{0pt}%
\pgfsys@defobject{currentmarker}{\pgfqpoint{0.000000in}{0.000000in}}{\pgfqpoint{0.055556in}{0.000000in}}{%
\pgfpathmoveto{\pgfqpoint{0.000000in}{0.000000in}}%
\pgfpathlineto{\pgfqpoint{0.055556in}{0.000000in}}%
\pgfusepath{stroke,fill}%
}%
\begin{pgfscope}%
\pgfsys@transformshift{0.750000in}{4.500000in}%
\pgfsys@useobject{currentmarker}{}%
\end{pgfscope}%
\end{pgfscope}%
\begin{pgfscope}%
\pgfsetbuttcap%
\pgfsetroundjoin%
\definecolor{currentfill}{rgb}{0.000000,0.000000,0.000000}%
\pgfsetfillcolor{currentfill}%
\pgfsetlinewidth{0.501875pt}%
\definecolor{currentstroke}{rgb}{0.000000,0.000000,0.000000}%
\pgfsetstrokecolor{currentstroke}%
\pgfsetdash{}{0pt}%
\pgfsys@defobject{currentmarker}{\pgfqpoint{-0.055556in}{0.000000in}}{\pgfqpoint{0.000000in}{0.000000in}}{%
\pgfpathmoveto{\pgfqpoint{0.000000in}{0.000000in}}%
\pgfpathlineto{\pgfqpoint{-0.055556in}{0.000000in}}%
\pgfusepath{stroke,fill}%
}%
\begin{pgfscope}%
\pgfsys@transformshift{4.800000in}{4.500000in}%
\pgfsys@useobject{currentmarker}{}%
\end{pgfscope}%
\end{pgfscope}%
\begin{pgfscope}%
\pgftext[x=0.694444in,y=4.500000in,right,]{{\rmfamily\fontsize{11.000000}{13.200000}\selectfont \(\displaystyle 3.0\)}}%
\end{pgfscope}%
\begin{pgfscope}%
\pgftext[x=0.433852in,y=2.500000in,,bottom,rotate=90.000000]{{\rmfamily\fontsize{11.000000}{13.200000}\selectfont Bandwidth [Bytes/s]}}%
\end{pgfscope}%
\begin{pgfscope}%
\pgftext[x=0.750000in,y=4.541667in,left,base]{{\rmfamily\fontsize{11.000000}{13.200000}\selectfont \(\displaystyle \times10^{9}\)}}%
\end{pgfscope}%
\begin{pgfscope}%
\pgfsetbuttcap%
\pgfsetroundjoin%
\pgfsetlinewidth{1.003750pt}%
\definecolor{currentstroke}{rgb}{0.000000,0.000000,0.000000}%
\pgfsetstrokecolor{currentstroke}%
\pgfsetdash{}{0pt}%
\pgfpathmoveto{\pgfqpoint{0.750000in}{4.500000in}}%
\pgfpathlineto{\pgfqpoint{4.800000in}{4.500000in}}%
\pgfusepath{stroke}%
\end{pgfscope}%
\begin{pgfscope}%
\pgfsetbuttcap%
\pgfsetroundjoin%
\pgfsetlinewidth{1.003750pt}%
\definecolor{currentstroke}{rgb}{0.000000,0.000000,0.000000}%
\pgfsetstrokecolor{currentstroke}%
\pgfsetdash{}{0pt}%
\pgfpathmoveto{\pgfqpoint{4.800000in}{0.500000in}}%
\pgfpathlineto{\pgfqpoint{4.800000in}{4.500000in}}%
\pgfusepath{stroke}%
\end{pgfscope}%
\begin{pgfscope}%
\pgfsetbuttcap%
\pgfsetroundjoin%
\pgfsetlinewidth{1.003750pt}%
\definecolor{currentstroke}{rgb}{0.000000,0.000000,0.000000}%
\pgfsetstrokecolor{currentstroke}%
\pgfsetdash{}{0pt}%
\pgfpathmoveto{\pgfqpoint{0.750000in}{0.500000in}}%
\pgfpathlineto{\pgfqpoint{4.800000in}{0.500000in}}%
\pgfusepath{stroke}%
\end{pgfscope}%
\begin{pgfscope}%
\pgfsetbuttcap%
\pgfsetroundjoin%
\pgfsetlinewidth{1.003750pt}%
\definecolor{currentstroke}{rgb}{0.000000,0.000000,0.000000}%
\pgfsetstrokecolor{currentstroke}%
\pgfsetdash{}{0pt}%
\pgfpathmoveto{\pgfqpoint{0.750000in}{0.500000in}}%
\pgfpathlineto{\pgfqpoint{0.750000in}{4.500000in}}%
\pgfusepath{stroke}%
\end{pgfscope}%
\begin{pgfscope}%
\pgfsetbuttcap%
\pgfsetroundjoin%
\definecolor{currentfill}{rgb}{1.000000,1.000000,1.000000}%
\pgfsetfillcolor{currentfill}%
\pgfsetlinewidth{1.003750pt}%
\definecolor{currentstroke}{rgb}{0.000000,0.000000,0.000000}%
\pgfsetstrokecolor{currentstroke}%
\pgfsetdash{}{0pt}%
\pgfpathmoveto{\pgfqpoint{5.002500in}{2.144169in}}%
\pgfpathlineto{\pgfqpoint{6.667464in}{2.144169in}}%
\pgfpathlineto{\pgfqpoint{6.667464in}{4.500000in}}%
\pgfpathlineto{\pgfqpoint{5.002500in}{4.500000in}}%
\pgfpathlineto{\pgfqpoint{5.002500in}{2.144169in}}%
\pgfpathclose%
\pgfusepath{stroke,fill}%
\end{pgfscope}%
\begin{pgfscope}%
\pgfsetrectcap%
\pgfsetroundjoin%
\pgfsetlinewidth{2.007500pt}%
\definecolor{currentstroke}{rgb}{0.000000,0.000000,0.000000}%
\pgfsetstrokecolor{currentstroke}%
\pgfsetdash{}{0pt}%
\pgfpathmoveto{\pgfqpoint{5.130833in}{4.362500in}}%
\pgfpathlineto{\pgfqpoint{5.387500in}{4.362500in}}%
\pgfusepath{stroke}%
\end{pgfscope}%
\begin{pgfscope}%
\pgftext[x=5.589167in,y=4.298333in,left,base]{{\rmfamily\fontsize{13.200000}{15.840000}\selectfont \(\displaystyle 2^{18}\) Elements}}%
\end{pgfscope}%
\begin{pgfscope}%
\pgfsetrectcap%
\pgfsetroundjoin%
\pgfsetlinewidth{2.007500pt}%
\definecolor{currentstroke}{rgb}{0.111111,0.111111,0.111111}%
\pgfsetstrokecolor{currentstroke}%
\pgfsetdash{}{0pt}%
\pgfpathmoveto{\pgfqpoint{5.130833in}{4.106852in}}%
\pgfpathlineto{\pgfqpoint{5.387500in}{4.106852in}}%
\pgfusepath{stroke}%
\end{pgfscope}%
\begin{pgfscope}%
\pgftext[x=5.589167in,y=4.042685in,left,base]{{\rmfamily\fontsize{13.200000}{15.840000}\selectfont \(\displaystyle 2^{19}\) Elements}}%
\end{pgfscope}%
\begin{pgfscope}%
\pgfsetrectcap%
\pgfsetroundjoin%
\pgfsetlinewidth{2.007500pt}%
\definecolor{currentstroke}{rgb}{0.222222,0.222222,0.222222}%
\pgfsetstrokecolor{currentstroke}%
\pgfsetdash{}{0pt}%
\pgfpathmoveto{\pgfqpoint{5.130833in}{3.851204in}}%
\pgfpathlineto{\pgfqpoint{5.387500in}{3.851204in}}%
\pgfusepath{stroke}%
\end{pgfscope}%
\begin{pgfscope}%
\pgftext[x=5.589167in,y=3.787038in,left,base]{{\rmfamily\fontsize{13.200000}{15.840000}\selectfont \(\displaystyle 2^{20}\) Elements}}%
\end{pgfscope}%
\begin{pgfscope}%
\pgfsetrectcap%
\pgfsetroundjoin%
\pgfsetlinewidth{2.007500pt}%
\definecolor{currentstroke}{rgb}{0.333333,0.333333,0.333333}%
\pgfsetstrokecolor{currentstroke}%
\pgfsetdash{}{0pt}%
\pgfpathmoveto{\pgfqpoint{5.130833in}{3.595556in}}%
\pgfpathlineto{\pgfqpoint{5.387500in}{3.595556in}}%
\pgfusepath{stroke}%
\end{pgfscope}%
\begin{pgfscope}%
\pgftext[x=5.589167in,y=3.531390in,left,base]{{\rmfamily\fontsize{13.200000}{15.840000}\selectfont \(\displaystyle 2^{21}\) Elements}}%
\end{pgfscope}%
\begin{pgfscope}%
\pgfsetrectcap%
\pgfsetroundjoin%
\pgfsetlinewidth{2.007500pt}%
\definecolor{currentstroke}{rgb}{0.444444,0.444444,0.444444}%
\pgfsetstrokecolor{currentstroke}%
\pgfsetdash{}{0pt}%
\pgfpathmoveto{\pgfqpoint{5.130833in}{3.339908in}}%
\pgfpathlineto{\pgfqpoint{5.387500in}{3.339908in}}%
\pgfusepath{stroke}%
\end{pgfscope}%
\begin{pgfscope}%
\pgftext[x=5.589167in,y=3.275742in,left,base]{{\rmfamily\fontsize{13.200000}{15.840000}\selectfont \(\displaystyle 2^{22}\) Elements}}%
\end{pgfscope}%
\begin{pgfscope}%
\pgfsetrectcap%
\pgfsetroundjoin%
\pgfsetlinewidth{2.007500pt}%
\definecolor{currentstroke}{rgb}{0.555556,0.555556,0.555556}%
\pgfsetstrokecolor{currentstroke}%
\pgfsetdash{}{0pt}%
\pgfpathmoveto{\pgfqpoint{5.130833in}{3.084260in}}%
\pgfpathlineto{\pgfqpoint{5.387500in}{3.084260in}}%
\pgfusepath{stroke}%
\end{pgfscope}%
\begin{pgfscope}%
\pgftext[x=5.589167in,y=3.020094in,left,base]{{\rmfamily\fontsize{13.200000}{15.840000}\selectfont \(\displaystyle 2^{23}\) Elements}}%
\end{pgfscope}%
\begin{pgfscope}%
\pgfsetrectcap%
\pgfsetroundjoin%
\pgfsetlinewidth{2.007500pt}%
\definecolor{currentstroke}{rgb}{0.666667,0.666667,0.666667}%
\pgfsetstrokecolor{currentstroke}%
\pgfsetdash{}{0pt}%
\pgfpathmoveto{\pgfqpoint{5.130833in}{2.828613in}}%
\pgfpathlineto{\pgfqpoint{5.387500in}{2.828613in}}%
\pgfusepath{stroke}%
\end{pgfscope}%
\begin{pgfscope}%
\pgftext[x=5.589167in,y=2.764446in,left,base]{{\rmfamily\fontsize{13.200000}{15.840000}\selectfont \(\displaystyle 2^{24}\) Elements}}%
\end{pgfscope}%
\begin{pgfscope}%
\pgfsetrectcap%
\pgfsetroundjoin%
\pgfsetlinewidth{2.007500pt}%
\definecolor{currentstroke}{rgb}{0.777778,0.777778,0.777778}%
\pgfsetstrokecolor{currentstroke}%
\pgfsetdash{}{0pt}%
\pgfpathmoveto{\pgfqpoint{5.130833in}{2.572965in}}%
\pgfpathlineto{\pgfqpoint{5.387500in}{2.572965in}}%
\pgfusepath{stroke}%
\end{pgfscope}%
\begin{pgfscope}%
\pgftext[x=5.589167in,y=2.508798in,left,base]{{\rmfamily\fontsize{13.200000}{15.840000}\selectfont \(\displaystyle 2^{25}\) Elements}}%
\end{pgfscope}%
\begin{pgfscope}%
\pgfsetrectcap%
\pgfsetroundjoin%
\pgfsetlinewidth{2.007500pt}%
\definecolor{currentstroke}{rgb}{0.888889,0.888889,0.888889}%
\pgfsetstrokecolor{currentstroke}%
\pgfsetdash{}{0pt}%
\pgfpathmoveto{\pgfqpoint{5.130833in}{2.317317in}}%
\pgfpathlineto{\pgfqpoint{5.387500in}{2.317317in}}%
\pgfusepath{stroke}%
\end{pgfscope}%
\begin{pgfscope}%
\pgftext[x=5.589167in,y=2.253150in,left,base]{{\rmfamily\fontsize{13.200000}{15.840000}\selectfont \(\displaystyle 2^{26}\) Elements}}%
\end{pgfscope}%
\end{pgfpicture}%
\makeatother%
\endgroup%

	\label{threads_remote}
	\caption{Bandwidth with increasing number of threads on \emph{moore}.}

\end{figure}

\end{document}
