% PREAMBLE
%%%%%%%%%%
%%%%%%%%%%

\documentclass[DIV=12,oneside,a4paper]{scrartcl}

% PACKAGES
%%%%%%%%%%
\usepackage[english]{babel}
\usepackage{graphicx}
\usepackage{pgf}
\usepackage{placeins}
\usepackage{listings}


% DOCUMENT
%%%%%%%%%%
%%%%%%%%%%

\begin{document}

%%%%%%%
% TITLE
%%%%%%%

\title{Exercise Sheet II}
\subject{Advanced Parallel Computing}
\author{Klaus Naumann \& Christoph Klein}
\maketitle

%%%%%%%
% TABLE OF CONTENTS
%%%%%%%

%\newpage
%\tableofcontents
%\newpage

%%%%%%%
% PART 1
%%%%%%%%

\section{Review: The Future of Microprocessors}

\section{Review: Software and the Concurrency Revolution}
In their 2005 released paper "Software and the Concurrency Revolution" Sutter 
and Larus focus on concurrency and its importance for programmers and 
programming languages alike. Since the step in computer architecture from 
uniprocessors to multicore processors concurrency turned into a key to boost 
application performance on parallel architectures. Though the benefits of 
concurrency in software development are foreseeable it demands an advanced 
knowledge of the underlying hardware architecture and programmers to think in 
an unusual way.

Sutter and Luras outline the importance of separating applications in hundreds 
of tasks to gain performance in applications. The industry has to create new 
parallel-focused constructs as languages and tools to exploit the parallel 
hardware and make parallel applications understandable and transparent. 
Furthermore concurrency can increase stability and functionality of software.

We highly accept the opinion of the papers authors. Concurrency can highly 
increase the performance and stability of applications on a parallel architecture
as the difficulty without tools supporting the programmer in parallel application 
development.
      
%%%%%%%
% PART 2
%%%%%%%%

\section{Experiment: Pointer Chasing Benchmark}

%%%%%%%
% PART 3
%%%%%%%%

%\section{Introduction: Pthreads}

%%%%%%%
% PART 4
%%%%%%%%

\section{Experiment: Mutli-threaded Load Bandwidth}


\end{document}
