% PREAMBLE
%%%%%%%%%%
%%%%%%%%%%

\documentclass[DIV=12,oneside,a4paper]{scrartcl}

% PACKAGES
%%%%%%%%%%
\usepackage[english]{babel}
\usepackage{graphicx}
\usepackage{pgf}
\usepackage{placeins}
\usepackage{listings}


% DOCUMENT
%%%%%%%%%%
%%%%%%%%%%

\begin{document}

%%%%%%%
% TITLE
%%%%%%%

\title{Exercise Sheet III}
\subject{Advanced Parallel Computing}
\author{Klaus Naumann \& Christoph Klein}
\maketitle

%%%%%%%
% TABLE OF CONTENTS
%%%%%%%

%\newpage
%\tableofcontents
%\newpage

%%%%%%%
% PART 1 - Reading
%%%%%%%

\section{Review: Cost-Effective Parallel Computing}
The paper "Cost-Effective Parallel Computing" from David A. Wood and 
Mark D. Hill published in February 1995 deals with the authors thesis
that parallel computing can be cost-effective even when speed-ups are
non-linear in comparison to uniprocessor systems. The prerequisite of 
the authors claim is the significant fraction of memory cost compared
to system cost.
\\
In order to take advantage of the efficiency, capacity and bandwidth 
of a large and expensive memory unit several processors are needed.
Cost-effectivity increase especially in cases when speedups exceed 
costups ($\frac{costs(parallel)}{costs(uniprocessor)}$). To further
maximize the benefits of parallel systems software development costs
for parallalizing applications should be taken under control.
\\
In our view the result of the authors work is correct. Especially for
large computational problems parallel systems are far more cost-effective
than uniprocessor systems.

\section{Review: A Survey of Cache Coherence Schemes for Multiprocessor}
The paper "A Survey of Cache Coherence Schemes for Multiprocessor" from
Per Stenström released in June 1990 outlines three problems of a shared-
memory multiprocessor: the memory module handling only one request at 
a time, communication contention and latency time which is long because 
of the complexity of the interconnection network. As these problems 
increase memory access time and therefore slow-down the execution, cache 
memory served as a good way to reduce memory acces time. To maintain 
cache coherence in memory the author inspects different possibile 
implemetations: software based, hardware based or a combination of both.
\\
The survey outlines the following hardware based solutions: the 
write-invalidate scheme (when a processor updates a block, all other 
copies set to invalid) and the write-update policy (all copies get updated). 
As software based solutions cache coherence can be realized by indiscriminate 
or selective invalidation or based on timestamps with the aim to avoid 
complex hardware mechanisms. Unfortunately apart from the write-invalidate 
and write-update snoopy protocols none of the other solutions were 
implemented. 
\\
Cache coherence is still a hot topic in parallel systems. Though snoopy
or directoy-based protocols are the most common solution for cache coherence, 
the Dekker's Algorithm is a software based attempt to maintain cache coherence.
    
%%%%%%%
% PART 2 - Experiments
%%%%%%%

\section{Experiment: Shared Counter Performance Analysis}

\end{document}
