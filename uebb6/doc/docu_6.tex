% PREAMBLE
%%%%%%%%%%
%%%%%%%%%%

\documentclass[oneside,a4paper]{scrartcl}

% PACKAGES
%%%%%%%%%%
\usepackage[english]{babel}
\usepackage{graphicx}
\usepackage{pgf}
\usepackage{placeins}
\usepackage{listings}


% DOCUMENT
%%%%%%%%%%
%%%%%%%%%%

\begin{document}

%%%%%%%
% TITLE
%%%%%%%

\title{Exercise Sheet VI}
\subject{Advanced Parallel Computing}
\author{Klaus Naumann \& Christoph Klein}
\maketitle

%%%%%%%
% PART 1 - Reading
%%%%%%%

%%%%%%%
% PART 2 - Experiments
%%%%%%%

\section{Parallel Prefix Sum -- Development}

We provide the code in different files, which we will explain shortly:
\begin{description}
	\item[./src/main.cc] This conains our \texttt{main} function and handles
	command line arguments. You can choose the thread count \texttt{-t},
	the array size \texttt{-s}, the number of measurements to average
	time measurements \texttt{-n}. Furthermore you can skip the 
	sequential calculation, which verifies the parallel results by
	\texttt{-sq} and you can enable array size correction, as the
	program only accepepts array sizes $s$ and thread counts $t$, which
	satisfy
	\[s \;\textnormal{mod}\; t = 0\:.\]
	This means if $s$ does not satisfy this condition the program
	takes another value $s'$ as array size, which is determined by
	\[ s' = s + (t - s\;\textnormal{mod}\; t)\:.\]
	You can enable this option by \texttt{-correct-size}. Furthermore
	our code is templated to enable various numeric types. Especially
	you will run quickly into overflow problems with large arrays, if you
	consider high random numbers and \texttt{int} as the numeric type.
	\item[./inc/thread\_handler.h] This is a generic written
	pthread handling class. You can see it as syntactic sugar.
	\item[./inc/chCommandLine.h] Command line argument handling.
	\item[.time\_measurement.h] This file provides an easy class based way
	to make time measurements in your program.
	\item[./src/thread\_arg.h] Here you can find the class definition of
	our objects, which are given to the threads on creation, thus the
	Thread-Argument-Class.
	\item[./src/thread\_routines.h] provides the prefix scan routine,
	the function, which will be executed by threads on their creation,
	and the random array initialization function.

\section{Parallel Prefix Sum -- Analysis}
mean = 10

\end{description}

\end{document}
