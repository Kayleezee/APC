% PREAMBLE
\documentclass[oneside,a4paper]{scrartcl}

% PACKAGES
%%%%%%%%%%
\usepackage[english]{babel}
\usepackage{graphicx}
\usepackage{pgf}
\usepackage{placeins}
\usepackage{listings}


% DOCUMENT
%%%%%%%%%%
%%%%%%%%%%

\begin{document}

%%%%%%%
% TITLE
%%%%%%%

\title{Exercise Sheet VIII}
\subject{Advanced Parallel Computing}
\author{Klaus Naumann \& Christoph Klein}
\maketitle

%%%%%%%
% PART 1 - Reading
%%%%%%%
\section*{Reading}
\subsection*{First Paper}
The paper 'The AMD Opteron Northbridge Architecture' from Pat Conway and Bill Hughes
published in 2007 presents an improved Northbridge architecture realized by AMD's
Direct Connect architecture, which uses an industry-standard HyperTransport technology
to link the processors with each other.

The main advantages of a HyperTransport interconnection are scalability, a high
bandwidth and wol latency. Furthermore the Direct Connect architecture provides
on chip integrated memory controller, thus reducing the front side bus bottlenecks.
The implemented HyperTransport protocol (HTP) ensures a coherent shared memory address
space. The HTP uses a broadcast-based coherence protocol, thus supports a good scaling
up to an eight-socket. Moreover this architecture has a lower power consumption
in comparison to the front side bus architecture.

To our mind this paper presents a detailed technical view on the new Northbridge
architecture, thus it provides useful information for the system architect, but not for
the system user. We would accept this paper, because sharing system architecture information
supports/accelerates system architecture developement.

\section*{Red-Black Tree - Coarse-Grain Lock -- Development}
\label{dev}
We provide the code in different files, which we will explain shortly:
\begin{description}
    \item[./src/rbtreemain.c] This file contains our \texttt{main} function, the updatestream and 
				time-measurement functions and handles command line arguments. 
				You can choose the thread count, the search and insert operations.
    \item[./src/main.c] This file contains the \texttt{main} function as described in 
			\texttt{./src/rbtreemain.c} which is modified to handle the testing part of 
			our implementation of the red-black tree for various threads and ratios of 
			insert/search operations which we modified by hand after each ratio test routine. 
\end{description}
The files \texttt{rbtree.c} and \texttt{rbtree.h} contain functions for several operations of the red-black
tree and were already provided in exercise 7.
As a lock we used \emph{pthread\_rwlock} which provides a shared read lock and an exclusive write lock.

\section*{Red-Black Tree - Coarse-Grain Lock -- Analysis}
We executed our program using various amount of threads and insert/search ratios. As you can see for each 
thread amount and ratio the executed operations per second are on a nearly constant level.  With increasing
insert operations the operations per second decrease. \\
 In figure \ref{plot_1090} the algorithm can execute between 1.00e+05 to nearly 3.00e+05 operations per second 
depending on the amount of threads. Here we used a ratio of 10/90 (insert/search). With a ratio of 90/10 (insert/search)
the possible operations per second decrease to a rate of 0.75e+05 to nearly 1.50e+05 as you can see in figure 
\ref{plot_9010}.
    
\begin{figure}
	\centering
	\includegraphics[width=1.0\textwidth]{results1090.png}
	\caption{Red-Black Tree test on \emph{moore} with a ratio of 10/90 (Insert/Search) for 1 to 48 threads.}
	\label{plot_1090}
\end{figure}

\begin{figure}
	\centering
	\includegraphics[width=1.0\textwidth]{results5050.png}
	\caption{Red-Black Tree test on \emph{moore} with a ratio of 50/50 (Insert/Search) for 1 to 48 threads.}
	\label{plot_5050}
\end{figure}

\begin{figure}
	\centering
	\includegraphics[width=1.0\textwidth]{results9010.png}
	\caption{Red-Black Tree test on \emph{moore} with a ratio of 90/10 (Insert/Search) for 1 to 48 threads.}
	\label{plot_9010}
\end{figure}
\end{document}
